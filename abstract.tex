\begin{abstract}[language=ukrainian, header=false]

\textbf{Ахмедов~Р.~Д. Поля iмпульсних антен в лiнiйному i нелiнiйному 
середовищi.} -- Квалiфiкацiйна наукова праця на правах рукопису.

Дисертацiя на здобуття наукового ступеня кандидата 
фiзико-математичних наук за спецiальнiстю 01.04.03 -- 
Радiофiзика i електронiка (фiзико-математичні науки). -- 
ХНУ~iменi~В.~Н.~Каразiна, Харкiв, 2020.

Дисертаційну роботу присвячено теоретичному дослідженню властивостей 
імпульсного електромагнітного  пікосекундного та наносекундного 
випромінювання в ближній та проміжній зонах. Ідеологію всієї роботи можна 
окреслити єдиним підходом до розв'язання задач випромінювання та приймання 
електромагнітних імпульсів з урахуванням ефектів ближньої зони, що 
полягає у відмові від спектральних перетворень та роботі в часовій області, 
що дозволяє уникнути появи ближньої зони, як особливого 
випадку розв'язання. Задачі, яким присвячена робота, здебільшого 
розглянуті з апертурними антенами імпульсного випромінювання в якості 
джерела поля. Увага до ближньої зони обумовлена декількома факторами:
ефектом концентрації енергії апертурними імпульсними 
антенам у ближній зоні, що спричиняє прояв слабконелінійних ефектів та
викривлення фронту імпульсу в зоні формування електромагнітної хвилі.

В огляді літератури проаналізовано спеціально наукові методи, актуальні 
для предмету дослідження. Огляд містить результати сучасних досліджень 
як закордонних, так і вітчизняних авторів. Детально проаналізовано
метод еволюційних рівнянь та особливості його застосування. Приведено 
найбільш поширені методи виокремлення інформації з часової послідовності,
що застосовуються в задачах комунікації та локації, а також представлено 
сучасні та перспективні для галузі методи науки про дані. Розглянуто 
можливість застосування теорії збурень для лінеаризації задачі поширення 
імпульсної наносекундної електромагнітної хвилі у середовищі з нелінійними 
індуктивними властивостями, які представлено у вигляді розкладу по малому 
параметру в матеріальних рівняннях середовища.

Автором вперше отримано аналітичний розв'язок задачі випромінювання 
поодинокого наносекундного електромагнітного імпульсу в нелінійне 
середовище з урахуванням процесу формування нестаціонарних 
електромагнітних хвиль у ближній зоні джерела. В якості джерела поля
розглянуто  однонапрямний рівномірний розподіл електричного 
струму у формі плаского диска. Представлений нелінійний розв'язок 
отримано для збуджувального імпульсу з часовою залежністю у вигляді 
ступеневої функції Хевісайда. Отриманий аналітичний вираз для 
напруженості електричного поля містить кратний інтеграл над 
швидкоосцилюючою функцією. Інтеграл було чисельно проаховано з 
застосуванням квадратурних формул Сімпсона-Рунге для багатовимірних
невласних інтегралів. Застосована методика дозволила отримати 
розв'язок з заданою точністю.

В роботі продемонстровано, що позитивною особливістю застосованого 
алгоритму є можливість узагальнення розв'язку з часовою залежністю 
струму збудження у вигляді функції Хевісайда до розв'язку нелінійної задачі 
випромінювання з довільною часовою залежністю струму.

Хоча кінцевий вираз для компонентів поля містить інтеграл, залежність 
від азимутального кута присутня в явному вигляді. Такий вираз з явною 
амплітудною та кутовою залежністю дозволив спостерігати деякі відомі 
ефекти взаємодії гармонійного поля з середовищем у випадку поширення 
наносекундних імпульсів. Також проведено аналіз отриманих графіків 
поля-поправки лінійного розв'язку. Проведено узагальнення розв'язку 
задачі випромінювання у Керрівське середовище  до задачі з 
поліноміальним нелінійним вектором поляризації. Для побудови нелінійного 
розв'язку знехтувано дисперсійними властивостями середовища, а також 
втратами провідності середовища.

Нелінійний розв'язок для Керрівського середовища отримано з 
лінійного наближення з використанням елементів теорії збурень та методу 
еволюційних рівнянь. Лінійну задачу випромінювання розв'язано для лінзової 
антени імпульсного випромінювання у наближенні плаского кругового 
синхронного однонапрямленого розподілу електричного струму. В роботі 
вперше представлено перехідну функцію для такої антени з явною 
залежністю від часу та просторових координат, яка справедлива в 
довільній точці спостереження.

Окрім розв'язку задач випромінювання явному виді проведено аналіз 
характеристик напрямленості лінзових антен імпульсного випромінювання при 
збуджені струмами з різними часовими формами як для ближньої так і дальньої 
зони випромінювання. Також проаналізовано влив точки спостереження на 
форму електромагнітного імпульсу з різними часовими залежностями 
збуджувального струму.

В роботі описано ефект інтерференції мінімумів та максимумів гармонійно 
промодульованого імпульсу вздовж напрямку поширення. При даному ефекті 
формується картина піків інтенсивності, споріднена до кілець Фур'є, що 
формуються внаслідок дифракції гармонійної хвилі на круглому отворі.

Отриманий аналітичний розв'язок задачі випромінювання у часовій області 
використано для моделювання процесу прийому-передачі електромагнітного 
імпульсу та виокремлення з нього корисної інформації. Аналіз отриманих результатів
та огляд літератури виявили ряд недоліків у сучасному способі виокремлення 
корисної інформації з імпульсної електромагнітної хвилі. В роботі 
представлено авторську методику виокремлення корисної інформації з 
надширокостугового імпульсу з урахуванням залежності його форми від 
точки спостереження. В основі запропонованої методики -- топологічно розділена 
на енодер та декорер фізична нейронна мережа з тривалою короткочасною 
пам'яттю у якості структурного елементу.

Моделювання процесу бездротової передачі інформації проведено з урахуванням 
зашумленості каналу. В якості передавальної антени використано лінзову антену
імпульсного випромінювання, а в якості приймальної детектор електричного поля.
Для моделювання багатокористувацького середовища використано імпульси різної 
форми і показана стійкість запропонованої системи до накладання імпульсів та до
розпізнавання сторонніх та власних імпульсів. Задля можливості відтворення 
отриманих результатів в дисертації представлено статистичні характеристики 
тренувальних даних, архітектуру нейронної мережі та візуалізацію процесу 
мінімалізації цільової функції. В роботі запропоновано методику впровадження 
отриманої моделі для розв'язання практичних задач зондування та 
телекомунікації шляхом застосування методики переносну нявчання.

Моделювання проведено з повнозв'язним та рекурентним енкодером. 
Проаналізовано обмеження та недоліки застосовання згорткових нейронних 
мереж в задачах аналізу часових послідовностей в реальному часі. За
результатами моделювання фізична нейронна мережа з застосуванням 
тривалої короткочасної пам'ятті дозволяє класифікувати надширокосмугові
наносекундні імпульси в реальному часі без оцифрування та з урахуванням 
взаємонакладання імпульсів. Отримані результати дозволяють вважати 
запропоновану методику розв'язанням задачі класифікації 
надширокосмугових нано- та пікосекундних імпульсів довільної 
форми при рівні амплітуди сигналу меншим за рівень шуму.
Моделювання показало, що особливо перспективно запропонована 
методика виглядає при аналізі імпульсів складної форми, наприклад 
великої кількості пелюсток чи фрактальної природи сигналу, коли досягнення 
критерію Найквіста при аналогово-цифровому перетворенні проблематичне 
через високу верхню частоту спектру імпульсу.

В роботі використано лише спеціально наукові та загальнонаукові методи
теоретичного аналізу, що знайшли експериментальне підтвердження.
Текст дисертації складають опубліковані та апробовані матеріали наукових 
досліджень. Матеріал, що викладено у роботі пов'язаний з декількома 
науково-дослідницькими проектами та програмами. Нові наукові результати,
що отримано в роботі не суперечать сучасним науковим знанням. На 
запропоновану автором методику виокремлення корисної інформації подано 
заяву на отримання винахідницького патенту.

Практично значним результатом кваліфікаційної роботи є методика обробки 
прийнятого надширокосмугового сигналу. Вирази для нелінійного поля, 
доповнені енергетичними діаграми, дозволяють теоретично оцінювати 
необхідність врахування нелінійних ефектів в практичних задачах, де 
застосовуються антени імпульсного випромінювання. Отримана модель 
поля лінзової антени імпульсного випромінювання у лінійному наближені 
може бути застосована для моделювання в реальному часі процесу 
поширення імпульсної надширокосмугової електромагнітної хвилі з 
урахуванням ефектів ближньої зони.

\keywords{часова область, електромагнітний імпульс, надширокосмугова 
електродинаміка, Керрівська нелінійність, слабка нелінійність, 
метод еволюційних рівнянь, машинне нявчання, рекурентні нейронні мережі,
тривала короткочасна пам'ять}

\end{abstract}

%%%%%%%%%%%%%%%%%%%%%%%%%%%%%%%%%%%%%%%%%%

 \begin{abstract}[language=english, header=false]
 	
\textbf{Akhmedov.~R. Field of Impuple Radiating Antennas in Linear and 
Nonlinear Medium} -- Qualifying scientific work on the rights of the manuscript.

Dissertation for the degree of a candidate of physical and 
mathematical sciences in specialty 01.04.03 -- radiophysics. 
V.~N.~Karazin Kharkiv National University, 
the Ministry of Education and Science of Ukraine, Kharkiv, 2020.

 In the thesis, we consider a problem of ...
 
 \keywords{time domain, electromagnetic pulse, ultrawidband, Kerr nonlinerity, 
 	weak nonlinerity, evalutionary approach, machine learning, 
 	recurent neural networks, long-short term memory}
 
\end{abstract}

% \nocite{Bar98fasp1,Bar98fasp2,PrB01umc}

% \begin{bibset}% [a]
%   {Список публікацій здобувача за~темою~дисертації}
%   % {Список публікацій здобувача}
%   \bibliographystyle{gost2008}
%   %
%   % Якщо не треба нумерація з крапкою, можна закоментувати наступні три рядки.
%   \makeatletter
%   \renewcommand\@biblabel[1]{#1.}
%   \makeatother
%   \bibliography{../my}
% \end{bibset}

\newpage

\begin{center} 
	\underline{\textbf{Список публікацій здобувача за темою дисертації}}
\end{center}

\vspace{1cm}

\mybibappendix
