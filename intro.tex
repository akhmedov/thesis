\chapter*{Вступ}

%%%%%%%%%%%%%%%%%%%%%%%%%%%%%%%%%%%%%%%%%%%%%%%%%%%%%%%%%%%%%%%%%%%%%%%%%%%%%%%
\paragraph{Актуальність теми}

Зміна форми імпульсу від нелінійних факторів впливає на якість розв'язку 
електродинамічних задач. Визначення впливу слабкої нелінійності на 
розповсюдження нестаціонарних сигналів.

Теорія розповсюдження імпульсів може бути використана для визначення 
стану середовища в нелінійних, нестаціонарних та неоднорідних випадках
випромінювання. Мова іде про такі параметри оточующого середовища, як 
проникність, температура, густина, іонізованість, тощо.

Наступне покоління бездротових локальних систем передачі даних для 
квантових комп'ютерів. Велика множина форм імпульсів для радіозв'язку 
дозволяє побудувати не бінарні протокол передачі даних канального рівня,
що підвищують кількість інформації що передається, навіть у зашумованому або 
нелінійному просторі. Особливий приріст швидкості варто очікувати на 
квантових обчислюваних системах, за рахунок основи кодування більше двійки.

Час та частота - це абстракції, що описують одне явище природи - зміна
енергетичних взаємодій в системі. Для макроскопічної електродинаміки, зокрема,
існують підходи, що застосовують обидві абстракції. Вибір абстракції, тобто 
підходу, визначаються характером задачі, що вирішується. Історично склалось, що
більш широке розповсюдження отримала частотно-орієнтована методологія, яка в 
повній мірі виправдала себе. Однак новітні технології все частіше потребують 
розв'язків, які методи частотної області надати не зможуть. Наприклад, 
розв'язання задач розповсюдження та випромінювання в нестаціонарних 
неоднорідних середовищах, що характеризуються нелінійними та анізотропними 
ефектами. Це відновило інтерес до методів часової області.

%%%%%%%%%%%%%%%%%%%%%%%%%%%%%%%%%%%%%%%%%%%%%%%%%%%%%%%%%%%%%%%%%%%%%%%%%%%%%%%
\paragraph{Зв'язок роботи з науковими програмами}

Erasmus+

%%%%%%%%%%%%%%%%%%%%%%%%%%%%%%%%%%%%%%%%%%%%%%%%%%%%%%%%%%%%%%%%%%%%%%%%%%%%%%%
\paragraph{Мета та задача дослідження}

%%%%%%%%%%%%%%%%%%%%%%%%%%%%%%%%%%%%%%%%%%%%%%%%%%%%%%%%%%%%%%%%%%%%%%%%%%%%%%%
\paragraph{Методи дослідження}

З огляду на те, що обернене перетворення Фур'є не гарантує сходимості при 
обмеженому частотному спектрі, використання частотних методів неможливе в 
середовищах, що перераховані вище. Мотивуючись цим, вибираємо часовий метод 
для аналізу імпульсного випромінювання, а саме метод еволюційних рівнянь.
В якості основи для метода є вилучення поперечних компонент поля методом 
Рімана-Вольтера, також відомого у вітчизняній літературі як метод еволюційних 
рівнянь. Це дозволяє звести задачу розв'язання системи рівнянь Максвела до 
розв'язання диференціального рівняння другого порядку в часних похідних.
У випадку вакуумного середовища це рівняння зводиться до рівняння 
Клейна-Гордона.

\textcolor{red}{Рекурентні нейронні мережі}

\textcolor{red}{Метод перехідної функції}

%%%%%%%%%%%%%%%%%%%%%%%%%%%%%%%%%%%%%%%%%%%%%%%%%%%%%%%%%%%%%%%%%%%%%%%%%%%%%%%
\paragraph{Наукова новизна отриманих результатів}

\textcolor{red}{Розв'язок декількох задач випромінювання в часовій області для 
лінійного та нелінійного простору}

\textcolor{red}{Адаптація методу еволюційних рівнянь для чисельного розв'язку
та побудова відповідного програмного комплексу, що розповсюджується під 
ліцензією GPL, як один з проектів GNU спів-товариства}

\textcolor{red}{Авторський метод синтезу протоколів передачі інформації}

%%%%%%%%%%%%%%%%%%%%%%%%%%%%%%%%%%%%%%%%%%%%%%%%%%%%%%%%%%%%%%%%%%%%%%%%%%%%%%%
\paragraph{Практичне значення отриманих результатів}

\begin{enumerate} 
	\item Шведський проект по аналізу властивостей над провідникових матеріалів
	\item Передача інформації імпульсами на велику відстань
\end{enumerate} 

%%%%%%%%%%%%%%%%%%%%%%%%%%%%%%%%%%%%%%%%%%%%%%%%%%%%%%%%%%%%%%%%%%%%%%%%%%%%%%%
\paragraph{Особистий вклад дисертанта}

Розв'язок задачі плаского диску слабкого нелінійного простору
Авторський метод синтезу протоколів передачі інформації

%%%%%%%%%%%%%%%%%%%%%%%%%%%%%%%%%%%%%%%%%%%%%%%%%%%%%%%%%%%%%%%%%%%%%%%%%%%%%%%
\paragraph{Апробація результатів дослідження}

\begin{enumerate} 
	\item Erasmus+
	\item Upsala Univ.
\end{enumerate} 

%%%%%%%%%%%%%%%%%%%%%%%%%%%%%%%%%%%%%%%%%%%%%%%%%%%%%%%%%%%%%%%%%%%%%%%%%%%%%%%
\paragraph{Публікації}

%%%%%%%%%%%%%%%%%%%%%%%%%%%%%%%%%%%%%%%%%%%%%%%%%%%%%%%%%%%%%%%%%%%%%%%%%%%%%%%
\paragraph{Зміст роботи}

%%%%%%%%%%%%%%%%%%%%%%%%%%%%%%%%%%%%%%%%%%%%%%%%%%%%%%%%%%%%%%%%%%%%%%%%%%%%%%%
\paragraph{Подяка}
