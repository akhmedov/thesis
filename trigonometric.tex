\chapter{Деякі властивості тригонометричних функції}
\label{ch:trigonometric}

В цьому розділі представлено деякі маловідомі властивості тригонометричних 
функцій для зручності споживання викладеного в кваліфікаційній роботі 
матеріалу.

%\textcolor{blue}{
%\begin{equation*}
%\derivat{}{\varphi} \arccos \varphi = - \frac{1}{ \sqrt{1 - \varphi^2} }
%\end{equation*}
%%
%\begin{equation*}
%\derivat{}{\varphi} \arctan \varphi = \frac{1}{1 + \varphi^2}
%\end{equation*}
%%
%\begin{equation*}
%\cos \alpha \cos \beta = \frac{1}{2} 
%\left(  \cos (\alpha + \beta) + \cos (\alpha - \beta) \right)
%\end{equation*}
%%
%\begin{equation*}
%\sin \alpha \cos \beta = \frac{1}{2} 
%\left( \sin (\alpha + \beta) + \sin (\alpha - \beta) \right)
%\end{equation*}
%%
%\begin{equation*}
%\sin \alpha \sin \beta = \frac{1}{2} 
%\left( \cos (\alpha - \beta) - \cos (\alpha + \beta) \right)
%\end{equation*}
%%
%\begin{equation*}
%e^{im \varphi} = \cos m \varphi + i \sin m \varphi
%\end{equation*}
%%
%\begin{equation*}
%e^{-im \varphi} = \cos m \varphi - i \sin m \varphi
%\end{equation*}
%%
%\begin{equation*}
%\sin \varphi = \frac{e^{i \varphi} - e^{- i \varphi}}{2i}
%\end{equation*}
%%
%\begin{equation*}
%\cos \varphi = \frac{e^{i \varphi} + e^{- i \varphi}}{2}
%\end{equation*}
%%
%\begin{equation*}
%\arctan \frac{1}{x} = \frac{\pi}{2} - \arctan x
%\end{equation*}
%%
%\begin{equation*}
%\pi - \arccos x = \arccos (-x)
%\end{equation*}
%} % textcolor blue
%
\begin{equation}
\arccos x - \arccos y = \mp \arccos \left( 
xy + \sqrt{(1-x^2)(1-y^2)} \right),
\left\{ \begin{array}{c} x \ge y \\ x < y  \end{array} \right\}
\end{equation}
%
\begin{equation}
\arctan x - \arctan y = 
\arctan \frac{x-y}{1+xy}, xy > -1 
\end{equation}
%
\begin{equation}
\arctan x - \arctan y = \pm \pi + \arctan \frac{x-y}{1+xy}, 
\left\{ \begin{array}{c} x > 0 \\ x < 0  \end{array} \right\}, xy < -1 
\end{equation}

\section{Визначення символу Кронекера, через інтеграли}

В роботі часто застосовується визначення символу Кронекера через 
інтеграл на комплексній площині над експонентою з уявним показником.
Тут зібрані деякі не табличні інтеграли, що застосовані в роботі. 

\begin{equation} \begin{aligned} \label{eq:int_exp0}
\int_{0}^{2\pi} e^{\pm i (m-n) \varphi} d \varphi = 2 \pi \delta_{m,n} 
\end{aligned} \end{equation}
%
\begin{equation} \begin{aligned} \label{eq:int_exp1}
\int \limits_{0}^{2\pi} d \varphi \sin \varphi 
\left( \cos m \varphi - i \sin m \varphi \right) = 
i \pi \left( \delta_{m,-1} - \delta_{m,1} \right)
\end{aligned} \end{equation}
%
%\textcolor{blue}{ \begin{equation*} \begin{aligned}
%\int_{0}^{2\pi} d \varphi \sin \varphi 
%\left( \cos m \varphi - i \sin m \varphi \right) = \int_{0}^{2\pi} d \varphi
%\left( \sin \varphi \cos m \varphi - i \sin \varphi \sin m \varphi \right) = \\
%= \frac{1}{2} \int_{0}^{2\pi} d \varphi \left( \sin (\varphi + m \varphi) + 
%\sin (\varphi - m \varphi) - i \cos (\varphi - m \varphi) + 
%i \cos (\varphi + m \varphi) \right) = \\
%= \frac{i}{2} \int_{0}^{2\pi} d \varphi \left( -i \sin (\varphi + m \varphi) -
%i \sin (\varphi - m \varphi) - \cos (\varphi - m \varphi) + 
%\cos (\varphi + m \varphi) \right) = \\
%= \frac{i}{2} \int_{0}^{2\pi} d \varphi \left( e^{-i (\varphi + m \varphi)} - 
%e^{i (\varphi - m \varphi)} \right) = 
%i \pi \left( \delta_{m,-1} - \delta_{m,1} \right)
%\end{aligned} \end{equation*} }
%
\begin{equation} \begin{aligned} \label{eq:int_exp2}
\int \limits_{0}^{2\pi} d \varphi \cos \varphi 
( \cos m \varphi - i \sin m \varphi) = \pi ( \delta_{m,-1} + \delta_{m,1} )
\end{aligned} \end{equation}
%
%\textcolor{blue}{ \begin{equation*} \begin{aligned}
%\int_{0}^{2\pi} d \varphi \cos \varphi 
%\left( \cos m \varphi - i \sin m \varphi \right) = \int_{0}^{2\pi} d \varphi
%\left( \cos \varphi \cos m \varphi - i \cos \varphi \sin m \varphi \right) = \\
%= \frac{1}{2} \int_{0}^{2\pi} d \varphi \left( 
%\cos (\varphi + m \varphi) + \cos (\varphi - m \varphi) - 
%i \sin (m \varphi + \varphi) - i \sin (m \varphi - \varphi) \right) = \\
%= \frac{1}{2} \int_{0}^{2\pi} d \varphi \left( 
%\cos (\varphi + m \varphi) + \cos (\varphi - m \varphi) - 
%i \sin (m \varphi + \varphi) + i \sin (\varphi - m \varphi) \right) = \\
%= \frac{1}{2} \int_{0}^{2\pi} d \varphi 
%\left( e^{-i (1 + m) \varphi} - e^{i (1 - m) \varphi} \right) = 
%\pi \left( \delta_{m,-1} + \delta_{m,1} \right)
%\end{aligned} \end{equation*} }
%
\begin{equation} \begin{aligned} \label{eq:int_exp3}
\int_0^{2\pi} e^{-i m \varphi} \cos \varphi \sin^2 \varphi d \varphi = 
\frac{\pi \delta_{m,1} }{4} + \frac{\pi \delta_{m,-1} }{4} - 
\frac{\pi \delta_{m,-3} }{4} - \frac{\pi \delta_{m,3} }{4}
\end{aligned} \end{equation}
%
%\textcolor{blue}{ \begin{equation*} \begin{aligned}
%e^{-i m \varphi} \cos \varphi \sin^2 \varphi = e^{-i m \varphi} 
%\frac{e^{i\varphi} + e^{-i\varphi}}{2} \frac{1 - \cos 2\varphi}{2} = \\
%\frac{2e^{-i(m-1)\varphi} + 2e^{-i(m+1)\varphi}}{8} - 
%\frac{e^{2i\varphi} + e^{-2i\varphi}}{2} 
%\frac{2e^{-i(m-1)\varphi} + 2e^{-i(m+1)\varphi}}{8} = \\
%\frac{2e^{-i(m-1)\varphi} + 2e^{-i(m+1)\varphi}}{8} - 
%\frac{e^{-i(m-3)\varphi} + e^{-i(m+1)\varphi} + 
%e^{-i(m-1)\varphi} + e^{-i(m+3)\varphi}}{8} = \\
%= \frac{e^{-i(m-1)\varphi}}{8} + \frac{e^{-i(m+1)\varphi}}{8} -
%\frac{e^{-i(m-3)\varphi}}{8} - \frac{e^{-i(m+3)\varphi}}{8}
%\end{aligned} \end{equation*} }
%
\begin{equation} \begin{aligned} \label{eq:int_exp4}
\int_{0}^{2\pi} e^{-i m \varphi} \sin^3 \varphi d \varphi = 
\frac{3 \pi i}{4} \delta_{m,-1} - \frac{3 \pi i}{4} \delta_{m,1} - 
\frac{\pi i}{4} \delta_{m,-3} + \frac{\pi i}{4} \delta_{m,3}
\end{aligned} \end{equation}
%
%\textcolor{blue}{ \begin{equation*} \begin{aligned}
%e^{-i m \varphi} \sin^3 \varphi = e^{-i m \varphi} 
%\frac{1 - \cos 2\varphi}{2} \frac{e^{i\varphi} - e^{-i\varphi}}{2i} = \\
%= \frac{e^{-i(m-1)\varphi} - e^{-i(m+1)\varphi}}{4i} - 
%\frac{e^{-i(m-3)\varphi} + e^{-i(m+1)\varphi} -
%e^{-i(m-1)\varphi} - e^{-i(m+3)\varphi}}{8i} = \\
%= \frac{3 e^{-i(m-1)\varphi}}{8i} - \frac{3 e^{-i(m+1)\varphi}}{8i} - 
%\frac{e^{-i(m-3)\varphi}}{8i} + \frac{e^{-i(m+3)\varphi}}{8i} = \\
%= \frac{3i e^{-i(m+1)\varphi}}{8} - \frac{3i e^{-i(m-1)\varphi}}{8} + 
%\frac{i e^{-i(m-3)\varphi}}{8} - \frac{i e^{-i(m+3)\varphi}}{8} 
%\end{aligned} \end{equation*} }
%
%\textcolor{blue}{ \begin{equation*} \begin{aligned}
%\int_{0}^{2\pi} e^{-i m \varphi} \sin^3 \varphi d \varphi = 
%\frac{i\pi}{4} \left( 3 \delta_{m,-1} - 3 \delta_{m,1} + 
%\delta_{m,3} - \delta_{m,-3} \right)
%\end{aligned} \end{equation*} }
%
\begin{equation} \begin{aligned} \label{eq:int_exp5}
\int_0^{2\pi} e^{-i m \varphi} \sin \varphi \cos^2 \varphi d \varphi = 
\frac{\pi i }{4} \delta_{m,-1} - \frac{\pi i }{4} \delta_{m,1} -
\frac{\pi i }{4} \delta_{m,3} + \frac{\pi i }{4} \delta_{m,-3}
\end{aligned} \end{equation}
%
%\textcolor{blue}{ \begin{equation*} \begin{aligned}
%e^{-i m \varphi} \sin \varphi \cos^2 \varphi = 
%\cos^2 \varphi e^{-i m \varphi} \frac{e^{i\varphi} - e^{-i\varphi}}{2i} = \\
%= \frac{1 + \cos 2\varphi}{2} 
%\frac{e^{i(1-m)\varphi} - e^{-i(1+m)\varphi}}{2i} = \\
%= \frac{e^{i(1-m)\varphi} - e^{-i(1+m)\varphi}}{4i} + 
%\frac{e^{2i\varphi} + e^{-2i\varphi}}{2} 
%\frac{e^{i(1-m)\varphi} - e^{-i(1+m)\varphi}}{4i} = \\
%\frac{ 2 e^{i(1-m)\varphi} - 2 e^{-i(1+m)\varphi}}{8i} +
%\frac{e^{i(3-m)\varphi} + e^{-i(1+m)\varphi} - 
%e^{-i(m-1)\varphi} - e^{-i(3+m)\varphi}}{8i} = \\
%= -\frac{ i e^{-i (m-1) \varphi} }{8} + \frac{ i e^{-i (m+1) \varphi} }{8} -
%\frac{ i e^{-i (m-3) \varphi} }{8} + \frac{ i e^{-i (m+3) \varphi} }{8}
%\end{aligned} \end{equation*} }
%
\begin{equation} \begin{aligned} \label{eq:int_exp6}
\int_{0}^{2\pi} e^{-i m \varphi} \cos^3 \varphi d \varphi = 
\frac{\pi}{4} \delta_{m,-3} + \frac{\pi}{4} \delta_{m,3} + 
\frac{3 \pi}{4} \delta_{m,-1} + \frac{3 \pi}{4} \delta_{m,1}
\end{aligned} \end{equation}
%
%\textcolor{blue}{ \begin{equation*} \begin{aligned}
%e^{-i m \varphi} \cos^3 \varphi = 
%\cos^2 \varphi e^{-i m \varphi} \frac{e^{i \varphi} + e^{-i \varphi}}{2} =
%\frac{\cos^2 \varphi}{2} 
%\left( e^{-i (1+m) \varphi} + e^{i (1-m) \varphi} \right) = \\
%= \frac{ 1 + \cos 2 \varphi } { 4 } 
%\left( e^{-i (1+m) \varphi} + e^{i (1-m) \varphi} \right) = \\
%= \frac{e^{-i(1+m) \varphi} + e^{i(1-m) \varphi}}{4} + 
%\frac{e^{-i(1+m) \varphi} + e^{i(1-m) \varphi}}{4}
%\frac{e^{2i\varphi} + e^{-2i\varphi}}{2} = \\
%= \frac{e^{-i(1+m) \varphi} + e^{i(1-m) \varphi}}{4} +
%\frac{ e^{i(1-m) \varphi} + e^{-i(3+m) \varphi} + 
%e^{i(3-m) \varphi} + e^{-i(1+m) \varphi} }{8} = \\
%= \frac{3 e^{i(1-m) \varphi}}{8} + \frac{e^{-i(3+m) \varphi}}{8} +
%\frac{e^{i(3-m) \varphi}}{8} + \frac{ 3 e^{-i(1+m) \varphi} }{8}
%\end{aligned} \end{equation*} }
%
%\textcolor{blue}{ \begin{equation*} \begin{aligned}
%\int_{0}^{2\pi} d \varphi \left( \frac{3 e^{i(1-m) \varphi}}{8} + 
%\frac{e^{-i(3+m) \varphi}}{8} + \frac{e^{i(3-m) \varphi}}{8} + 
%\frac{ 3 e^{-i(1+m) \varphi} }{8} \right) = \\
%= \frac{\pi}{4} \delta_{m,-3} + \frac{\pi}{4} \delta_{m,3} + 
%\frac{3 \pi}{4} \delta_{m,-1} + \frac{3 \pi}{4} \delta_{m,1}
%\end{aligned} \end{equation*} }
