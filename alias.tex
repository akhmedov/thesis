\DeclareMathOperator{\Rea}{Re}
\DeclareMathOperator{\Ima}{Im}

\newcommand{\N}{\mathbb{N}}
\newcommand{\Z}{\mathbb{Z}}
\newcommand{\Q}{\mathbb{Q}}
\newcommand{\R}{\mathbb{R}}

\newcommand{\crossprod}[2]{ \left[  #1 \times #2 \right] } % вектор произвед
\newcommand{\dotprod}[2]{ \left<  #1 \cdot #2 \right> } % скалрное произвед
\newcommand{\triple}[3]{ \left<  #1 , #2 , #3 \right> } % скалрное произвед

\newcommand{\vect}[1]{ \overrightarrow{\mathbf #1} } % bold and with arrow
\newcommand{\func}[2]{ #1 \left( #2 \right) } % функциональная зависимость

\newcommand{\partder}[2]{ \frac{\partial #1}{\partial #2}}
\newcommand{\derivat}[2]{ \frac{d #1}{d #2}}

\newcommand{\rot}{\mathop{\mathrm{rot}}\nolimits} % rot{A}
