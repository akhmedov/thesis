\chapter{Комплексні функції Ломмеля двох змінних}
\label{ch:lommel}

Функція Ломмеля -- особлива гіпергеометрична функція, що в деяких випадках 
розкладається в нескінченний ряд по циліндричним функціям Бесселя першого роду.
Цікавим є те, що функція Ломмеля зустрічається в багатьох роботах присвячених
нестаціонарному випромінюванню TEM рупора: хоча отримання моделей проводилось 
різними способами та при різних спрощеннях всі вони зводяться до лінійних 
комбінацій даної функції. Можливо це є свідченням на користь того, що вона 
точно формалізує деякі фундаментальні особливості нестаціонарного 
електромагнітного поля. В даному розділі наведено деякі особливості 
комплексної функції Ломмеля двох змінних одна з яких уявна, а також деякі 
табличні та не табличні інтеграли що містять її. 

%%%%%%%%%%%%%%%%%%%%%%%%%%%%%%%%%%%%%%%%%%%%%%%%%%%%%%%%%%%%%%%%%%%%%%%%%%%%%%%
\section{Визначення та лінійні властивості}

В \cite{imp:Boersma1961} приводиться визначення через функцію Бесселя.
%
\begin{equation}
U_n \left[ W, Z \right] = \sum \limits_{m = 0}^{\infty} (-1)^m
\left( \frac{W}{Z} \right)^{n + 2m} J_{n + 2m} (Z)
\end{equation}

В \cite{imp:Boersma1961} також можна знайти наступну властивість.
%
\begin{equation}
U_n \left[ W, Z \right] + U_{n+2} \left[ W, Z \right] = 
\left( \frac{W}{Z} \right)^n J_n (Z)
\end{equation}
%
%\textcolor{blue} { \begin{equation*} \begin{aligned}
%W_\pm = \pm i (\nu ct - \nu z) \\
%Z = \sqrt{\nu^2 c^2t^2 - \nu^2 z^2}
%\end{aligned} \end{equation*} }
%
%\textcolor{blue}{ \begin{equation*}
%U_0 \left[ W, Z \right] = \sum \limits_{m = 0}^{\infty} (-1)^m
%\left( \frac{W}{Z} \right)^{2m} J_{2m} (Z) = J_0 (Z) - \frac{W^2}{Z^2} J_2 (Z) +
%\frac{W^4}{Z^4} J_4 (Z) - ...
%\end{equation*} }
%
%\textcolor{blue}{ \begin{equation*}
%U_2 \left[ W, Z \right] = \sum \limits_{m = 0}^{\infty} (-1)^m
%\left( \frac{W}{Z} \right)^{2 + 2m} J_{2 + 2m} (Z) = 
%\frac{W^2}{Z^2} J_2 (Z) - \frac{W^4}{Z^4} J_4 (Z) + \frac{W^6}{Z^6} J_6 (Z) - ...
%\end{equation*} }
%
%\textcolor{blue}{ \begin{equation*} \begin{aligned}
%U_0 [W, Z] - U_2 [W, Z] = J_0(Z) - 2 \left( \frac{W^2}{Z^2} J_2 (Z) - 
%\frac{W^4}{Z^4} J_4 (Z) + \frac{W^6}{Z^6} J_6 (Z) - ... \right)
%\end{aligned} \end{equation*} }
%
%\textcolor{blue}{ \begin{equation*} \begin{aligned}
%U_0 [W, Z] - U_2 [W, Z] = J_0(Z) - 2 U_2(W,Z)
%\end{aligned} \end{equation*} }
%
%\textcolor{blue}{ \begin{equation*} \begin{aligned}
%\left( \frac{W}{Z} \right)^{2n} = \left( 
%\frac{- i \nu (\mathit{V}t - z)}
%{\nu \sqrt{\mathit{V}^2t^2 - z^2}} \right)^{2n} = 
%(-i)^{2n} \nu^{2n} \frac{(\mathit{V}t - z)^{2n}}
%{(\mathit{V}t - z)^n (\mathit{V}t + z)^n} = \\
%= (-i)^{2n} \left( \frac{\mathit{V}t - z}{\mathit{V}t + z} \right)^n = 
%(-1)^{n} \left( \frac{\mathit{V}t - z}{\mathit{V}t + z} \right)^n = 
%\left( - \frac{\mathit{V}t - z}{\mathit{V}t + z} \right)^n
%\end{aligned} \end{equation*} }
%
%\textcolor{blue}{ \begin{equation*} \begin{aligned}
%U_0 [W, Z] - U_2 [W, Z] = J_0(Z) + 2 \left[ 
%\frac{\mathit{V}t - z}{\mathit{V}t + z} J_2(Z) + \left( 
%\frac{\mathit{V}t - z}{\mathit{V}t + z} \right)^2 J_4(Z) + ... \right]
%\end{aligned} \end{equation*} }
%
\begin{equation} \begin{aligned}
U_0 [W, Z] - U_2 [W, Z] = J_0(Z) + 2 \sum_{m=1}^{\infty} \left( 
\frac{\mathit{V}t - z}{\mathit{V}t + z} \right)^m J_{2m} (Z)
\end{aligned} \end{equation}

%%%%%%%%%%%%%%%%%%%%%%%%%%%%%%%%%%%%%%%%%%%%%%%%%%%%%%%%%%%%%%%%%%%%%%%%%%%%%%%
\section{Інтегродиференціальні властивості}

Функція Ломмеля типова для нестаціонарних задач. В \cite[ст. 41]{imp:Borisov1991} 
приведено корисні інтегродиференціальні.
%
\begin{equation} \begin{aligned}
\int \limits_{\xi}^{\tau} ds e^{-i \gamma s} J_0(\sqrt{s^2 - \xi^2 }) = 
\frac{e^{-i \gamma \tau}}{\sqrt{\gamma^2 - 1}} \left( U_1(W_+,Z) + \right. \\ 
\left. + i U_2(W_+,Z) - U_1(W_-,Z) - i U_2(W_-,Z) \right)
\end{aligned} \end{equation}

Тут $ W_\pm = (\gamma \pm \sqrt{\gamma^2 - 1}) (\tau - \xi) $ a 
$ Z = \sqrt{\tau^2 - \xi^2} $. Також для використання цієї формули повинна
виконуватись умова $ \tau - \xi > 0 $.
%
%\textcolor{red}{ \begin{equation}
%\left. \begin{array}{c}
%U_{2n} (W_+, Z) = U_{2n} (W_-, Z) \\
%U_{2n+1} (W_+, Z) = - U_{2n+1} (W_-, Z)
%\end{array} \right| n \in \Z
%\end{equation} }
%
\begin{equation} 
\partder{}{Z} U_n (W,Z) = - \frac{Z}{W} U_{n+1} (W,Z)
\end{equation}
%
\begin{equation}
2 \partder{}{W} U_n (W,Z) = U_{n-1} (W,Z) + 
\left( \frac{Z}{W} \right)^2 U_{n+1} (W,Z)
\end{equation}

%%%%%%%%%%%%%%%%%%%%%%%%%%%%%%%%%%%%%%%%%%%%%%%%%%%%%%%%%%%%%%%%%%%%%%%%%%%%%%%
\section{Отримання інтегралу $ I_3 $ в явному виді}

\begin{equation}
I_3 = R \int \limits_{0}^{\infty} \frac{d \nu}{\rho \nu} 
J_1(\nu \rho) J_1(\nu R) (U_0[ W_-, Z ] - U_2[ W_-, Z ])
\end{equation}
%
\begin{equation} \label{eq:intergal3}
I_3 = I_1 - 2 R \int_{0}^{\infty} \frac{d \nu}{\nu \rho} 
J_1(\nu \rho) J_1(\nu R) U_2[ W_-, Z ]
\end{equation}
%
На осі випромінювання, тобто при $ \rho = 0 $. 
%
%\textcolor{blue}{ \begin{equation*} \begin{aligned}
%\left. 2 R \int_{0}^{\infty} \frac{d \nu}{\nu \rho} 
%J_1(\nu \rho) J_1(\nu R) \sum_{m=1}^{\infty} \left( 
%\frac{ct - z}{ct + z} \right)^m J_{2m} (Z) 
%\right|_{\rho = 0} = \\ = R \frac{ct - z}{ct + z} \int_{0}^{\infty} 
%d \nu J_1(\nu R) J_2 (\nu \sqrt{c^2t^2 + z^2}) + \\ 
%+ R \left( \frac{ct - z}{ct + z} \right)^2 
%\int_{0}^{\infty} d \nu J_1(\nu R) J_4 (\nu \sqrt{c^2t^2 + z^2}) + ...
%\end{aligned} \end{equation*} }
%
\begin{equation} \begin{aligned} \label{eq:i3_pol_int}
I_3 = I_1 + R \sum_{m=1}^{\infty} \left( \frac{ct - z}{ct + z} \right)^m 
\int_{0}^{\infty} d \nu J_1(\nu R) J_{2m} (\nu \sqrt{c^2t^2 + z^2})
\end{aligned} \end{equation}
%
За допомогою табличного інтегралу 6.512.4 з 
\cite[ст. 681]{imp:GradshtejnInt}, що має вид
%
\begin{equation*} \begin{aligned}
\int_0^\infty J_{\nu+2n+1} (ax) J_\nu (bx) dx = 
\begin{cases} \frac{b^\nu}{a^{\nu+1}}
P_n^{(\nu,0)} \left( 1 - \frac{2b^2}{a^2} \right) , 0 < b < a \\
0, 0 < a < b \end{cases}
\end{aligned} \end{equation*}
%
знайдемо значення $ I_3 $ в явному виді. Параметри мають задовольняти умові  
$ \Re \nu > - 1 - n $, яка виконується для інтегралу \eqref{eq:intergal3}. 
Через $ P_n^{(\nu,0)} (x) $ позначено поліном Якобі:
%
\begin{equation*} \begin{aligned}
P_n^{(\nu,0)} (x) = \frac{1}{2^n} \sum_{m=0}^{n} C_{n+\nu}^{m} C_{n}^{n-m} 
\left( x - 1 \right)^{n-m} \left( x + 1 \right)^m 
\end{aligned} \end{equation*}
%
Таким чином, для всіх подій $ R > \sqrt{c^2 t^2 - z^2} $ тобто області 
електромагнитного снаряду на осі симетрії випромінювача 
$ \left. I_3 \right|_{\rho = 0}^{R > \sqrt{c^2 t^2 - z^2}} = 1/2 $. Розглянемо 
значення $ I_3 $ на осі випромінювання для $ R < \sqrt{c^2 t^2 - z^2} $.
%
%\textcolor{blue}{ \begin{equation*} \begin{aligned}
%\left. I_3 \right|_{\rho = 0}^{R < \sqrt{c^2 t^2 - z^2}} = 
%\frac{R^2}{c^2 t^2 - z^2} \sum_{m=1}^{\infty} 
%\left( \frac{ct - z}{ct + z} \right)^m P_{m-1}^{(1,0)} 
%\left( 1 - \frac{2R^2}{c^2 t^2 - z^2} \right)
%\end{aligned} \end{equation*} }
%
\begin{equation*} \begin{aligned}
\left. I_3 \right|_{\rho = 0}^{R < \sqrt{c^2 t^2 - z^2}} = 
\frac{R^2}{c^2 t^2 - z^2} \sum_{m=0}^{\infty} 
\left( \frac{ct - z}{ct + z} \right)^{m+1} P_{m}^{(1,0)} 
\left( 1 - \frac{2R^2}{c^2 t^2 - z^2} \right)
\end{aligned} \end{equation*}
%
\begin{equation} \begin{aligned} \label{eq:i3onaxis}
\left. I_3 \right|^{\rho = 0} = \begin{cases} 1/2, R^2 > c^2 t^2 - z^2 \\
\frac{R^2}{c^2 t^2 - z^2} \sum_{m=0}^{\infty} 
\left( \frac{ct - z}{ct + z} \right)^{m+1} P_{m}^{(1,0)} 
\left( 1 - \frac{2R^2}{c^2 t^2 - z^2} \right), 
R^2 < c^2 t^2 - z^2 \end{cases}
\end{aligned} \end{equation}

%%%%%%%%%%%%%%%%%%%%%%%%%%%%%%%%%%%%%%%%%%%%%%%%%%%%%%%%%%%%%%%%%%%%%%%%%%%%%%%
\section{Отримання інтегралу $ I_4 $ в явному виді}

\begin{equation}
I_4 = R \int \limits_{0}^{\infty} d \nu J_0(\nu \rho) J_1(\nu R) 
(U_0[ W_-, Z ] - U_2[ W_-, Z ])
\end{equation}
%
\begin{equation} \label{eq:i4_pol_int}
I_4 = I_2 + 2 R \sum_{m=1}^{\infty} \left( \frac{ct - z}{ct + z} \right)^m 
\int_{0}^{\infty} d \nu J_1(\nu R) J_{2m} (\nu \sqrt{c^2t^2 + z^2})
\end{equation}
%
По аналогії з \eqref{eq:intergal3}, використаємо формулу 2.12.31.1 з 
\cite[ст. 209]{imp:SpecFunc1983} для пошуку значення інтегралу при $ \rho = 0 $.
%
\begin{equation*} \begin{aligned}
\left. I_4 \right|_{\rho = 0}^{R < \sqrt{c^2 t^2 - z^2}} = 
\frac{2 R^2}{c^2 t^2 - z^2} \sum_{m=0}^{\infty} 
\left( \frac{ct - z}{ct + z} \right)^{m+1} P_{m}^{(1,0)} 
\left( 1 - \frac{2R^2}{c^2 t^2 - z^2} \right)
\end{aligned} \end{equation*}
%
\begin{equation} \begin{aligned} \label{eq:i4onaxis}
\left. I_4 \right|^{\rho = 0} = 2 \left. I_3 \right|^{\rho = 0}
\end{aligned} \end{equation}

%%%%%%%%%%%%%%%%%%%%%%%%%%%%%%%%%%%%%%%%%%%%%%%%%%%%%%%%%%%%%%%%%%%%%%%%%%%%%%%
\section{Отримання інтегралу $ I_5 $ в явному виді}

\begin{equation}
I_5 = 2 i R \int \limits_{0}^{\infty} d \nu 
J_1 \left( \nu R \right) J_1 \left( \nu \rho \right)
U_1 \left[ - i \nu \left( ct - z \right), \nu \sqrt{c^2t^2 - z^2} \right]
\end{equation}

%\textcolor{blue}{ \begin{equation*} \begin{aligned}
%I_5 = 2 i R \sum_{m=0}^{\infty} (-1)^m (-i)^{2m+1}
%\left( \frac{ct-z}{\sqrt{c^2t^2-z^2}} \right)^{2m+1} \cdot \\ 
%\cdot \int_0^\infty d \nu J_1(\nu \rho) J_1(\nu R) 
%J_{2m+1} \left( \nu \sqrt{c^2t^2 - z^2} \right)
%\end{aligned}  \end{equation*} }
%
%\textcolor{blue}{ \begin{equation*} \begin{aligned}
%i \cdot (-i)^{2m+1} \cdot (-1)^m = (-1) \cdot (-i)^{2m+2} \cdot (-1)^m = \\
%= \left( \frac{1}{i} \right)^{2m+2} \cdot (-1)^{m+1} = 
%(-1)^{m+1} \cdot (-1)^{m+1} = 1
%\end{aligned}  \end{equation*} }
%
%\textcolor{blue}{ \begin{equation*} \begin{aligned}
%I_5 = 2 R \sum_{m=0}^{\infty}
%\left( \frac{ct-z}{\sqrt{c^2t^2-z^2}} \right)^{2m+1}
%\int_0^\infty d \nu J_1(\nu \rho) J_1(\nu R) 
%J_{2m+1} \left( \nu \sqrt{c^2t^2 - z^2} \right)
%\end{aligned}  \end{equation*} }
%
%\textcolor{blue}{ \begin{equation*} \begin{aligned}
%\left( \frac{ct-z}{\sqrt{c^2t^2-z^2}} \right)^{2m+1} = 
%\left( \frac{ct-z}{ct+z} \right)^{m+1/2}
%\end{aligned}  \end{equation*} }

\begin{equation} \begin{aligned} \label{eq:i5series}
I_5 = 2 R \sum_{m=0}^{\infty}
\sqrt[2m+1]{\frac{ct-z}{ct+z}}
\int_0^\infty d \nu J_1(\nu \rho) J_1(\nu R) 
J_{2m+1} \left( \nu \sqrt{c^2t^2 - z^2} \right)
\end{aligned}  \end{equation}

Користуючись формулою 2.12.42.15 з \cite{imp:SpecFunc1983} можна отримати 
аналітичний розвязок для першого доданку інтегралу \eqref{eq:i5series}.

%\textcolor{blue}{ \begin{equation*} \begin{aligned}
%I_5 = \frac{R}{\pi}
%\sqrt{\frac{ct-z}{ct+z}}
%\frac{\sqrt{ (c^2t^2 - z^2) - (\rho - R)^2} 
%\sqrt{(\rho + R)^2 - (c^2t^2 - z^2)}}
%{\rho R \sqrt{c^2t^2 - z^2}}
%\end{aligned}  \end{equation*} }
%
%\textcolor{blue}{ \begin{equation*} \begin{aligned}
%I_5 = \frac{1}{\pi}
%\sqrt{\frac{c^2t^2-z^2}{(ct+z)^2}}
%\frac{\sqrt{ (c^2t^2 - z^2) - (\rho - R)^2} 
%\sqrt{(\rho + R)^2 - (c^2t^2 - z^2)}}
%{\rho \sqrt{c^2t^2 - z^2}}
%\end{aligned}  \end{equation*} }

\begin{equation} \begin{aligned} \label{eq:i50}
\left. I_5 \right|^{m=0} = \frac
{\sqrt{(c^2t^2 - z^2) - (\rho-R)^2}\sqrt{(\rho+R)^2 - (c^2t^2 - z^2)}}
{\pi \rho (ct+z)}
\end{aligned}  \end{equation}
