\chapter*{Список умовних скорочень}

% a b c d e f g h i j k l m n o p q r s t u v w x y z
% а б в г ґ д е є ж з и і ї й к л м н о п р с т у ф х ц ч ш щ ь ю я

BER -- (від англ. bit error rate/ratio) коефіцієнт бітових помилок;

GPU -- (від англ. graphics processing unit) графічний процесор;

GRU -- (від англ. gated recurrent unit) вентильний рекурентний вузол;

IPS -- (від англ. indor position system) система внутрішнього позиціювання;

IRA -- (від англ. impulse radiating antenna) антена імпульсного випромінювання;

LIRA -- (від англ. lens impulse radiating antenna) лінзова антена імпульсного
випромінювання;

LSTM -- (від англ. long short-term memory) штучна нейронна мережа тривалої 
короткочасної пам'яті;

RIRA -- (від англ. reflector impulse radiating antenna) рефлекторна антена 
імпульсного випромінювання;

RNN -- (від англ. recurrent neural network) рекурентна нейронна мережа;

TE -- (від англ. transvers electric) поперечна магнітна, тобто хвиля без 
$ E_z $ компоненти;

TM -- (від англ. transvers magnetic) поперечна електрична, тобто хвиля без 
$ H_z $ компоненти;

TEM -- (від англ. transvers electromagnetic) поперечна електромагнітна, тобто 
хвиля  без $ E_z $ та $ H_z $ компонент;

АШНМ -- апаратна штучна нейронна мережа;

СТО -- спеціальна теорія відносності;

НШС -- надширока смуга (англ. UWB);

ШНМ -- штучна нейронна мережа.
