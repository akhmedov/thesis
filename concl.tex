\chapter*{Висновки}

\begin{enumerate}
%
\item З \eqref{eq:FraunhoferDistance} видно, що дальня зона (зона Фраунгофера) не 
наступає ніколи для лінійного випадку нестаціонарного режиму випромінювання.
%
\item Події простору часу, коли спостерігач ще не отримав сигналу з граничного кільця
диска, що випромінює відповідають умові $ (\rho - R)^2 > c^2 t^2 - z^2 $. 
Ефект електромагнітного снаряду спостерігається саме у цій області. Для плаского 
диску з електричним така умова наступає тільки в прожекторній зоні диску.
%
\item Події простору-часу, коли спостерігач отримав сигнал з найвіддаленішої точки 
диску відповідають умові $ (\rho + R)^2 < c^2 t^2 - z^2 $. Якщо поле породжене 
статичним джерелом то і саме поле у цій області може бути лише часово-незалежним.
%
\item \textcolor{red}{ Плаский диск формує квазі пласку хвилю при достатньому 
віддаленні точки спостерігання. }
%
\end{enumerate}
%

