\begin{abstract}[language=ukrainian, header=false]

\textbf{Ахмедов~Р.~Д. Поля iмпульсних антен у лiнiйному та нелiнiйному 
середовищах.} -- Квалiфiкацiйна наукова праця на правах рукопису.

Дисертація на здобуття наукового ступеня кандидата 
фiзико-матема-\\ тичних наук за спецiальнiстю 01.04.03 -- 
радiофiзика (фiзико-математичні науки). -- 
Харківський національний університет iменi~В.~Н.~Каразiна
Міністерства освіти і науки України, Харкiв, 2020.

Дисертаційну роботу присвячено теоретичному дослідженню властивостей 
імпульсного електромагнітного  пікосекундного та наносекундного 
випромінювання в ближній та проміжній зонах. Тематику всієї роботи можна 
окреслити єдиним підходом до розв'язання задач випромінювання та приймання 
електромагнітних імпульсів з урахуванням ефектів ближньої зони, який  
полягає у відмові від спектральних перетворень та роботі в часовій області, 
що дозволяє уникнути появи ближньої зони, як особливого 
випадку розв'язання. Задачі, яким присвячена робота, здебільшого 
розглянуто з апертурними антенами імпульсного випромінювання як 
джерело поля. Увага до ближньої зони обумовлена декількома факторами:
ефектом концентрації енергії апертурними імпульсними 
антенами у ближній зоні, що спричиняє прояв слабконелінійних ефектів та
викривлення фронту імпульсу в зоні формування електромагнітної хвилі.

В огляді літератури проаналізовано спеціально наукові методи, актуальні 
для предмета дослідження. Огляд зачіпає результати сучасних досліджень 
як закордонних, так і вітчизняних авторів. Детально проаналізовано
метод еволюційних рівнянь та особливості його застосування. Приведено 
найбільш поширені методи виокремлення інформації з часової послідовності,
що застосовують в задачах комунікації та локації, а також представлено 
сучасні та перспективні для галузі методи науки про дані. Розглянуто 
можливість застосування теорії збурень для лінеаризації задачі поширення 
імпульсної наносекундної електромагнітної хвилі в середовищі з нелінійними 
індуктивними властивостями, які представлено у вигляді розкладу за малим 
параметру в матеріальних рівняннях середовища.

Уперше отримано аналітичний розв'язок задачі випромінювання 
поодинокого наносекундного електромагнітного імпульсу в нелінійне 
середовище з урахуванням процесу формування нестаціонарних 
електромагнітних хвиль у ближній зоні джерела. У функції джерела поля
розглянуто  однонапрямний рівномірний розподіл електричного 
струму у формі плаского диска. Представлений нелінійний розв'язок 
отримано для збуджувального імпульсу з часовою залежністю у вигляді 
ступеневої функції Хевісайда. Отриманий аналітичний вираз для 
напруженості електричного поля містить кратний інтеграл над 
швидкоосцилюючою функцією. Інтеграл було чисельно прораховано із 
застосуванням квадратурних формул Сімпсона-Рунге для багатовимірних
невласних інтегралів. Застосована методика дозволила отримати 
розв'язок із заданою точністю.

У роботі продемонстровано, що позитивною особливістю застосованого 
алгоритму є можливість узагальнення розв'язку з часовою залежністю 
струму збудження у вигляді функції Хевісайда до розв'язку нелінійної задачі 
випромінювання з довільною часовою залежністю струму.

Хоча кінцевий вираз для компонентів поля містить інтеграл, залежність 
від азимутального кута присутня в явному вигляді. Такий вираз з явною 
амплітудною та кутовою залежністю дозволив спостерігати деякі відомі 
ефекти взаємодії гармонійного поля з середовищем у випадку поширення 
наносекундних імпульсів. Також проведено аналіз отриманих графіків 
поля-поправки лінійного розв'язку. Проведено узагальнення розв'язку 
задачі випромінювання в керрівське середовище  до задачі з 
поліноміальним нелінійним вектором поляризації. Для побудови нелінійного 
розв'язку знехтувано дисперсійними властивостями середовища, а також 
втратами провідності середовища.

Нелінійний розв'язок для керрівського середовища отримано з 
лінійного наближення з використанням елементів теорії збурень та методу 
еволюційних рівнянь. Лінійну задачу випромінювання розв'язано для лінзової 
антени імпульсного випромінювання в наближенні плаского кругового 
синхронного однонапрямленого розподілу електричного струму. У роботі 
вперше представлено перехідну функцію для такої антени з явною 
залежністю від часу та просторових координат, яка справедлива в 
довільній точці спостереження.

Окрім розв'язку задач випромінювання в явному виді, проведено аналіз 
характеристик напрямленості лінзових антен імпульсного випромінювання 
збудженні струмами з різними часовими формами як для ближньої, так і дальньої 
зони випромінювання. Також проаналізовано влив точки спостереження на 
форму електромагнітного імпульсу з різними часовими залежностями 
збуджувального струму.

У роботі описано ефект інтерференції мінімумів та максимумів гармонійно 
промодульованого імпульсу вздовж напрямку поширення. При цьому ефекті 
формується картина піків інтенсивності, споріднена до кілець Фур'є, що 
формуються внаслідок дифракції гармонійної хвилі на круглому отворі.

Отриманий аналітичний розв'язок задачі випромінювання в часовій області 
використано для моделювання процесу приймання-передавання електромагнітного 
імпульсу та виокремлення з нього корисної інформації. Аналіз отриманих результатів
та огляд літератури виявили низку недоліків у сучасному способі виокремлення 
корисної інформації з імпульсної електромагнітної хвилі. У роботі 
представлено авторську методику виокремлення корисної інформації з 
надширокостугового імпульсу з урахуванням залежності його форми від 
точки спостереження. В основі запропонованої методики -- топологічно розділена 
на енодер та декорер фізична нейронна мережа з тривалою короткочасною 
пам'яттю як структурний елемент.

Моделювання процесу бездротового передання інформації проведено з урахуванням 
зашумленості каналу. У ролі передавальної антени використано лінзову антену
імпульсного випромінювання, а у ролі приймальної -- детектор електричного поля.
Для моделювання багатокористувацького середовища використано імпульси різної 
форми й продемонстровано стійкість запропонованої системи до накладання 
імпульсів та до розпізнавання сторонніх та власних імпульсів. Задля можливості 
відтворення отриманих результатів у дисертації представлено статистичні 
характеристики тренувальних даних, архітектуру нейронної мережі та 
візуалізацію процесу мінімізації цільової функції. У роботі запропоновано 
методику впровадження отриманої моделі для розв'язання практичних 
задач зондування та телекомунікації шляхом застосування методики 
перенесення нявчання.

Моделювання проведено з повнозв'язним та рекурентним енкодером. 
Проаналізовано обмеження та недоліки застосовання згорткових нейронних 
мереж у задачах аналізу часових послідовностей у реальному часі. За
результатами моделювання фізична нейронна мережа із застосуванням 
тривалої короткочасної пам'яті дозволяє класифікувати надширокосмугові
наносекундні імпульси в реальному часі без оцифрування та з урахуванням 
взаємонакладання імпульсів. Отримані результати дозволяють вважати 
запропоновану методику розв'язанням задачі класифікації 
надширокосмугових нано- та пікосекундних імпульсів довільної 
форми при рівні амплітуди сигналу меншим за рівень шуму.
Моделювання показало, що особливо перспективно запропонована 
методика виглядає при аналізі імпульсів складної форми, наприклад, 
великої кількості пелюсток чи фрактальної природи сигналу, коли досягнення 
критерію Найквіста при аналогово-цифровому перетворенні проблематичне 
через високу верхню частоту спектра імпульсу.

У роботі використано лише спеціально наукові та загальнонаукові методи
теоретичного аналізу, що знайшли експериментальне підтвердження.
Текст дисертації складають опубліковані та апробовані матеріали наукових 
досліджень. Матеріал, що викладено в роботі, пов'язаний з декількома 
науково-дослідними проектами та програмами. Нові наукові результати,
що отримано в роботі, не суперечать сучасним науковим знанням. На 
запропоновану автором методику виокремлення корисної інформації подано 
заяву на отримання винахідницького патенту.

Практично значним результатом кваліфікаційної роботи є методика об-\\
робки 
прийнятого надширокосмугового сигналу. Вирази для нелінійного поля, 
доповнені енергетичними діаграми, дозволяють теоретично оцінювати 
необхідність врахування нелінійних ефектів у практичних задачах, де 
застосовують антени імпульсного випромінювання. Отримана модель 
поля лінзової антени імпульсного випромінювання в лінійному наближенні 
може бути застосована для моделювання в реальному часі процесу 
поширення імпульсної надширокосмугової електромагнітної хвилі з 
урахуванням ефектів ближньої зони.

\keywords{часова область, електромагнітний імпульс, надширокосмугова 
електродинаміка, керрівська нелінійність, слабка нелінійність, 
метод еволюційних рівнянь, машинне нявчання, рекурентні нейронні мережі,
тривала короткочасна пам'ять}

\end{abstract}

%%%%%%%%%%%%%%%%%%%%%%%%%%%%%%%%%%%%%%%%%%

 \begin{abstract}[language=english, header=false]
 	
\textbf{Akhmedov.~R. Field of Impuple Radiating Antennas in Linear and 
Nonlinear Medium} -- Qualifying scientific work on the rights of the manuscript.

Dissertation for the degree of a candidate of physical and 
mathematical sciences in specialty 01.04.03 -- radiophysics. --
V.~N.~Karazin Kharkiv National University, 
the Ministry of Education and Science of Ukraine, Kharkiv, 2020.

The manuscript is a theoretical investigation of impulse ultrawideband 
electromagnetic field properties in near and far radiation zones. The whole thesis 
topic is united by the same approach of radiation and reception problems solving 
with an attention to near zone effects. The idea of the approach is to avoid furrier 
transform and provide a solution in time domain. As the result the solution will not 
carry near radiation zone as a spatial case of radiation. Most of problems considered 
in the manuscript have impulse radiating antenna as a source of electromagnetic filed. 
The attention to near radiation zone is caused by energy concentration with nonlinear 
interactions and impulse shape distortion.

The literature review contains the analysis of special scientific methods that are 
reliable for the subject of study. The review touches the results of the modern 
research on the topic as for foreign studies as well for national one. The 
evolutionary approach to time domain electromagnetics and its corner cases 
are particularly analyzed. Widely used methods of information extraction from 
transient electromagnetic wave, that are applied to telecommunication and 
remote sensing problems are discussed in the section. Moreover, cutting edge 
technologies in the domain are discussed too. Perturbation theory applicability 
in task of linearization of nanosecond electromagnetic pulse radiation problem 
into Kerr nonlinear medium that is introduced by Tailor extension of the 
nonlinear polarization vector.

The analytical solution for the nanosecond electromagnetic pulse radiation into 
nonlinear medium with near zone accounting was obtained for the first time. 
The circular aperture is considered as a source of a single nonstationary 
electromagnetic pulse with Heaviside like time dependency.  The obtained 
analytical statement for e-filed intensity contains multiple integral over fast 
oscillation function with Bessel function core. The integral was numerically 
calculated with explicit accuracy by multidimensional quadrature rules of 
Simpson-Runge.

The obtained statement of nonlinear filed intensity contains integral coefficient, 
but the dependency from azimuthal angle and maximum magnitude are presented 
in closed form. The statement allowed to observe time domain analogs of known 
nonlinear effects. Also, graphical analysis of the statement performed. Moreover, 
the cubic nonlinearity was generalized to the case of weak polynomial nonlinearity. 
The nonlinear propagation modeling does not include the influence of medium 
dispersion.

Nonlinear solution for Kerr medium problem is obtained from linear approximation 
solution with usage of elements of perturbation theory and evolutionary approach. 
Linear radiation problem with lens impulse radiation antenna as an emitter is solved 
in appreciation of circular equally distributed electrical current density. The manuscript 
contains the transient response for the antenna in closed form of time-space. 
Moreover, the transient response is applicable to any observer location.

The solution for the radiation problem is not the only result. The manu-\\script contains 
the result of the analysis of directivity of impulse radiating antennas that are exited 
with electromagnetic pulses with various shapes as for near as well for far radiation 
zones. Also the influence of observer location to the shape of electromagnetic pulse 
was discussed with attention to the shape of the excitation current.

The manuscript contains description of the minimum-maximum interference of 
harmonically modulated electromagnetic pulse along radiation axis. The effect 
faces the local maximums-minimums interference in time domain that is familiar to 
Fourier circus that appears due to narrowband wave diffraction on the pinhole.

The obtained liner solution in closed form was applied to information transmitting 
and reception modeling. The analysis and literature review faces out the near-far 
communication problem. The manuscript contains the patented methodic of 
information extraction with near zone accounting. The methodic is based on usage 
of topolectal separated on encoder and decoder physical neural network with 
long-short term memory as a structural unit.

Modeling of wireless information transferring is carried with taking into account 
of white gaussian noise channel. Lens impulse radiation antenna is considered 
as transition device and lossless electromagnetic field detector is considered as 
reception one. Multiuser environment simulation is provided throw usage of 
different shapes of excitation current. The manuscript contains statistical 
properties of the dataset, end-to-end machine learning pipeline architecture of 
and visualization of training process. The optimized methodic of implementation 
of presented technology in communication and remote sensing problems is 
proposed with attention to transfer learning.

The modeling is done on fully-connected and recurrent encoders. The 
restrictions and disadvantages of fully connected encoders applicability in time
domain electromagnetics are analyzed. It was shown that usage of long-short 
term memory allows to label ultrawideband signal in real time without 
analog-to-digital transformation. Moreover, it works even in case of impulse 
signal overlaying or in case when signal level is less then noise level. The 
modeling faces that the methodic gives promising accuracy in analysis of 
impulses of a difficult shape for example: big number of oscillations, signal 
of fractal nature, unreachable Nyquist criteria.

The thesis uses only domain-specialized and general-scientific methods of 
theoretical analysis that has experimental proof. The text of the dissertation 
consists of materials that are published and approbated on international 
conferences. The dissertation content is included to several research and 
development programs and projects. The obtained results are not in conflict 
with existing science vision on this question.

The thesis is the theoretical research nevertheless several practical use cases 
of the manuscript materials are discussed. One if them is the proposed 
methodic of information extraction from receipted transient ultrawideband 
electromagnetic wave. The obtained nonlinear filed statements supplemented 
with energy diagrams admit a theoretical estimation of influence of nonlinear 
effects for the problems of emission of lens impulse radiating antennas. The
received model of linear radiation for impulse radiation antennas can be 
applied for real time modeling of filed radiation and propagation in free space 
with a respect to near radiation zone effects.

 \keywords{time domain, electromagnetic pulse, ultrawideband, Kerr nonlinearity, 
 	weak nonlinearity, evolutionary approach, machine learning, 
 	recurrent neural networks, long-short term memory}
 
\end{abstract}

% \nocite{Bar98fasp1,Bar98fasp2,PrB01umc}

% \begin{bibset}% [a]
%   {Список публікацій здобувача за~темою~дисертації}
%   % {Список публікацій здобувача}
%   \bibliographystyle{gost2008}
%   %
%   % Якщо не треба нумерація з крапкою, можна закоментувати наступні три рядки.
%   \makeatletter
%   \renewcommand\@biblabel[1]{#1.}
%   \makeatother
%   \bibliography{../my}
% \end{bibset}

\newpage

\begin{center} 
	\underline{\textbf{Список публікацій здобувача за темою дисертації}}
\end{center}

\vspace{1cm}

\mybibappendix
