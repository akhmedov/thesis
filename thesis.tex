\documentclass[type=phd,guide=mon2017dev.karazin]{mon2017dev}[2019/11/16]

\usepackage[T2A]{fontenc}
\usepackage[utf8]{inputenc}
\usepackage[english,russian,ukrainian]{babel}

\usepackage[intlimits]{amsmath}
\allowdisplaybreaks
\usepackage{amsthm}
\usepackage{amssymb}

% Таблиці зі стовпчиками, що розтягуються
\usepackage{tabularx}
\usepackage{geometry}
\geometry{hmargin={30mm,15mm},lines=29,vcentering}

% експорт зображень
\usepackage{graphicx}
\graphicspath{ {Application/} }

\usepackage{eqparbox} 
\usepackage[x11names]{xcolor}

% нумкрация теорем/лемм
\theoremstyle{plain}
\newtheorem{theorem}{Теорема}[chapter]
\newtheorem{lemma}{Лема}[chapter]
\newtheorem{corollary}{Наслідок}[chapter]
\theoremstyle{definition}
\newtheorem{definition}{Означення}[chapter]
\newtheorem{example}{Приклад}[chapter]
\theoremstyle{remark}
\newtheorem{remark}{Зауваження}[chapter]

\DeclareMathOperator{\Rea}{Re}
\DeclareMathOperator{\Ima}{Im}
\DeclareMathOperator{\sinc}{sinc} % \sinc t = \frac{\sin t}{t}
% \DeclareMathOperator{\min}{min}

\newcommand{\N}{\mathbb{N}}
\newcommand{\Z}{\mathbb{Z}}
\newcommand{\Q}{\mathbb{Q}}
\newcommand{\R}{\mathbb{R}}

\newcommand{\rot}{\mathop{\mathrm{rot}}\nolimits} % rot{A}

\newcommand{\crossprod}[2]{ \left[  #1 \times #2 \right] } % вектор произвед
\newcommand{\dotprod}[2]{ \left<  #1 \cdot #2 \right> } % скалрное произвед
\newcommand{\triple}[3]{ \left<  #1 , #2 , #3 \right> } % скалрное произвед

\newcommand{\vect}[1]{ \overrightarrow{\mathbf #1} } % bold and with arrow
\newcommand{\func}[2]{ #1 \left( #2 \right) } % функциональная зависимость

\newcommand{\partder}[2]{ \frac{\partial #1}{\partial #2}}
\newcommand{\derivat}[2]{ \frac{d #1}{d #2}}

\newcommand{\mybibappendix}{
	
	\begin{center} 
		\textit{\textbf{Наукові праці у наукових фахових виданнях України:}}
	\end{center}
	
	\newcounter{ItemsInMyWriting}
	
	\begin{enumerate}
		
		\item Dumin O.M., Tretyakov O.A., \textbf{Akhmedov R.D.}, Dumina O.O. 
		Evolutionary Approach for the Problem of Electromagnetic Fiead 
		Propagation Through Nonlinear Medium // Вісник Харківського національного 
		університету імені В.Н. Каразіна, Серія ``Радіофізика та електроніка''. 
		2015. випуск 24. С. 23--28.
		
		\textit{Внесок здобувача: Аналітична робота по доказу та виведенню 
			математичних співвідношень. Аналіз отриманих результатів.}
		
		\item Думін О. М., \textbf{Ахмедов Р. Д.}, Міжмодове перетворення нестаціонарного 
		електромагнітного поля в нелінійному необмеженому середовищі // Вісник 
		Харківського національного університету імені В.Н. Каразіна, Серія 
		``Радіофізика та електроніка''. 2017, випуск 26. С. 42--47.
		
		\textit{Внесок здобувача: Отримання модового розкладу поля у відкритому 
			просторі методом еволюційних рівнянь. Статистична обробка отриманих результатів. }
		
		\item Думін О. М., \textbf{Ахмедов Р. Д.}, Випромінювання та розповсюдження 
		електромагнітного снаряду в нелінійному середовищі // Вісник 
		Харківського національного університету імені В.Н. Каразіна, Серія 
		``Радіофізика та електроніка''. 2017, випуск 27. С. 37--42.
		
		\textit{Внесок здобувача: Застосування теорії збурень для врахування 
			нелінійних складових поляризації}
		
		\item Думін О., \textbf{Ахмедов Р.}, Черкасов Д., Імпульсне випромінювання 
		антени з круговою апертурою в ближній зоні // Вісник Харківського 
		національного університету імені В. Н. Каразіна. Серія ``Радіофізика та 
		електроніка''. 2018. випуск 28. C. 30--33.
		
		\textit{Внесок здобувача: Підготовка графічних матеріалів до публікації. 
			Аналітична робота над математичним апаратом методу еволюційних рівнянь.}
		
		\item Думін О.М., \textbf{Ахмедов Р.Д.}, Черкасов Д.В., Поширення імпульсної 
		електромагнітної хвилі в керрівському середовищі // Вісник Харківського 
		національного університету імені В.Н. Каразіна. ``Радіофізика та 
		електроніка''. 2018. Вип. 29. С.11--16.
		
		\textit{Внесок здобувача: Розв'язання системи рівнянь Максвела з урахуванням
			нелінійних властивостей середовища для неоднорідності у вигляді плаского 
			диску з електричним струмом.}
		
		\item \textbf{Ахмедов Р.Д.}, Виокремлення корисної інформації з 
		надширокосмугової хвилі у ближній зоні випромінювання. ``Технология и 
		конструирование в электронной аппаратуре'', 2020, No 3-4, с. 3—10. DOI: 
		http://dx.doi.org/10.15222/TKEA2020.3-4.03
		
		\textit{Внесок здобувача: Розробка авторської методики виділення корисної 
			інформації з імпульсної надширокосмугової електромагнітної хвилі, проведення 
			числових симуляцій процесу випромінювання-поширювання-приймання імпульсів з
			урахуванням розробленої методики.}
		
		\setcounter{ItemsInMyWriting}{\value{enumi}}
	\end{enumerate}
	
%	\begin{center}
%		\textit{\textbf{Патенти:}}
%	\end{center}
%		
%	\begin{enumerate}
%		\setcounter{enumi}{\value{ItemsInMyWriting}}
%		
%		\item \textbf{Ахмедов Р. Д.}, Спосіб виділення корисної інформації з 
%		надширокосмугових (НШС) електромагнітних хвиль // Український інститут 
%		інтелектуальної власності. Київ. 2020.
%		
%		\textit{Внесок здобувача: Розроблено методику виділення корисної інформації 
%			з нестаціонарних імпульсних електромагнітних хвиль. Проведено порівняльну 
%			характеристику різних.}
%		
%		\setcounter{ItemsInMyWriting}{\value{enumi}}
%	\end{enumerate}
		
	\begin{center} 
		\textit{\textbf{Наукові праці у фахових виданнях, що входять до 
				міжнародних наукометричних баз:}}
	\end{center}
	
	\begin{enumerate}
		\setcounter{enumi}{\value{ItemsInMyWriting}}
		
		\item \textbf{Akhmedov R.}, Dumin O., Katrich V., Impulse radiation of antenna 
		with circular aperture // Telecommunications and Radio Engineering. Kharkiv. 
		2018. Vol. 77. P. 1767--1784.
		
		\textit{Внесок здобувача: Розв'язання задачі випромінювання імпульсу довільної
			геометричної форми лінзовою імпульсною антеною з круговою апертурою. Аналітична
			робота по отриманню перехідної функції для ближньої зони, як явної функції 
			від просторових координат та часу.}
		
		\setcounter{ItemsInMyWriting}{\value{enumi}}
	\end{enumerate}
		
	% \begin{center} 
	% \textit{\textbf{Наукові праці у фахових закордонних виданнях:}}
	% \end{center}
	
	\begin{center} 
		\textit{\textbf{Наукові праці апробаційного характеру (тези доповідей на 
				наукових конференціях) за темою дисертації:}}
	\end{center}
	
	\begin{enumerate}
		\setcounter{enumi}{\value{ItemsInMyWriting}}
		
		\item Dumin O.M., Katrich V.A., \textbf{Akhmedov R.D.}, Tretyakov O.A., 
		Dumina O.O., Evolutionary Approach for the Problems of Transient 
		Electromagnetic Field Propagation in Nonlinear Medium // 15th International 
		Conference on Mathematical Methods in Electromagnetic Theory (MMET).
		Dnipropetrovsk. 2014.
		
		\item Dumin O.M., Tretyakov O.A., \textbf{Akhmedov R.D.}, Stadnik Yu.B., 
		Katrich V.A., Dumina, O.O., Modal Basis Method for Propagation of 
		Transient Electromagnetic Fields in Nonlinear Medium // Proc. 7th 
		International Conference on Ultrawideband and Ultrashort Impulse Signals 
		(UWBUSIS). Kharkiv. 2014.
		
		\item Dumin O.M., Tretyakov O.A., \textbf{Akhmedov R.D.}, and Dumina O.O., 
		Transient Electromagnetic Field Propagation through Nonlinear Medium in 
		time domain // International Conference on Antenna Theory and Techniques, 
		21 -- 24 April, 2015. Kharkiv. 2015.
		
		\item Dumin O.M., \textbf{Akhmedov R.D.}, Dumina O.O., Propagation of 
		Transient Field Radiated from Plane Disk in Nonlinear Medium // 
		Ultrawideband and Ultrashort Impulse Signals, 5--11 September 2016. 
		Odessa. 2016.
		
		\item Dumin O., \textbf{Akhmedov R.}, Dumina O., Transient Field 
		Radiation of Plane Disk into Nonlinear Medium // Radio Electronics and 
		Info Communications, 11--16 September 2016. Kiev. 2016.
		
		\item Dumin O., \textbf{Akhmedov R.}, Katrich V., Dumina O., Transient 
		Radiation of Circle with Uniform Current Distribution // 2017 IEEE First 
		Ukraine Conference on Electrical and Computer Engineering (UKRCON), 
		May 29 -- June 2 2017. Kiev. 2017.
		
		\item \textbf{Akhmedov R.}, Dumin O., Ultrashort Impulse Radiation from 
		Plane Disk with Uniform Current Distribution // Ultrawideband and 
		Ultrashort Impulse Signals, 4--7 September 2018. Odessa. 2018.
		
		\item Dumin O., \textbf{Akhmedov R.}, Dumina O., Cherkasov D., Near Zone 
		of Plane Disk with Uniform Transient Current Distribution // 2017 IEEE 2nd 
		Ukraine Conference on Electrical and Computer Engineering (UKRCON), 
		June 2 -- June 9 2019. Lviv. 2019.
		
		\item Dumin O., \textbf{Akhmedov R.}, Katrich V., Cherkasov D., 
		Impulse Electromagnetic Wave Propagation in Kerr Medium // 2019 XXIVth 
		International Seminar/Workshop on Direct and Inverse Problems of 
		Electromagnetic and Acoustic Wave Theory (DIPED). Lviv. 2019.
		
		\item  \textbf{R. Akhmedov}, Neural Radio in DS-UWB IoT Applications // 2020 
		IEEE Ukrainian Microwave Week (UkrMW), Kharkiv, Ukraine, 2020, pp. 1073-1078, 
		doi: 10.1109/UkrMW49653.2020.9252611.
		
		\setcounter{ItemsInMyWriting}{\value{enumi}}
	\end{enumerate}
	
} % Локальние определения

\begin{document}

% Випромінювання нестаціонарних полів та їх розповсюдження в нелінійному просторі
\title(uk){Поле імпульсних антен в лінійному і неліннійному середовищі}
\title(en){NO TRANSLATION}

\author(uk){Ахмедов Ролан Джавадович}
\author(en){Akhmedov Rolan Dhavadovich}

\supervisor(uk){Думін Олександр Миколайович}
{кандидат фізико-математичних наук,доцент}{}

\speciality(uk)
[specialityname=Радіофізика і електроніка, degreefield=Радіофізика, 
specialityfile=specsci20150406n394.uk.csv]
{01.04.03}

\speciality(en)
[specialityname=Radiophisics and electronics, degreefield=Radiophisics,
specialityfile=specsci20150406n394.en.csv]
{01.04.03}

\udc{537.87}

\institution(uk)[altname=ХНУ імені В. Н. Каразіна]
{Харківський національний університет імені В. Н. Каразіна}
{Харків}
\institution(en)[altname=Karazin KNU]
{V. N. Karazin Kharkiv National University}
{Kharkiv}

\date{2020}

\maketitle

% \begin{abstract}[language=ukrainian, header=false]

\textbf{Ахмедов~Р.~Д. Поля iмпульсних антен в лiнiйному i нелiнiйному 
середовищi.} -- Квалiфiкацiйна наукова праця на правах рукопису.

Дисертацiя на здобуття наукового ступеня кандидата 
фiзико-математичних наук за спецiальнiстю 01.04.03 -- 
Радiофiзика i електронiка (фiзико-математичні науки). -- 
ХНУ~iменi~В.~Н.~Каразiна, Харкiв, 2020.

Дисертаційну роботу присвячено теоретичному дослідженню властивостей 
імпульсного електромагнітного  пікосекундного та наносекундного 
випромінювання в ближній та проміжній зонах. Ідеологію всієї роботи можна 
окреслити єдиним підходом до розв'язання задач випромінювання та приймання 
електромагнітних імпульсів з урахуванням ефектів ближньої зони, що 
полягає у відмові від спектральних перетворень та роботі в часовій області, 
що дозволяє уникнути появи ближньої зони, як особливого 
випадку розв'язання. Задачі, яким присвячена робота, здебільшого 
розглянуті з апертурними антенами імпульсного випромінювання в якості 
джерела поля. Увага до ближньої зони обумовлена декількома факторами:
ефектом концентрації енергії апертурними імпульсними 
антенам у ближній зоні, що спричиняє прояв слабконелінійних ефектів та
викривлення фронту імпульсу в зоні формування електромагнітної хвилі.

В огляді літератури проаналізовано спеціально наукові методи, актуальні 
для предмету дослідження. Огляд містить результати сучасних досліджень 
як закордонних, так і вітчизняних авторів. Детально проаналізовано
метод еволюційних рівнянь та особливості його застосування. Приведено 
найбільш поширені методи виокремлення інформації з часової послідовності,
що застосовуються в задачах комунікації та локації, а також представлено 
сучасні та перспективні для галузі методи науки про дані. Розглянуто 
можливість застосування теорії збурень для лінеаризації задачі поширення 
імпульсної наносекундної електромагнітної хвилі у середовищі з нелінійними 
індуктивними властивостями, які представлено у вигляді розкладу по малому 
параметру в матеріальних рівняннях середовища.

Автором вперше отримано аналітичний розв'язок задачі випромінювання 
поодинокого наносекундного електромагнітного імпульсу в нелінійне 
середовище з урахуванням процесу формування нестаціонарних 
електромагнітних хвиль у ближній зоні джерела. В якості джерела поля
розглянуто  однонапрямний рівномірний розподіл електричного 
струму у формі плаского диска. Представлений нелінійний розв'язок 
отримано для збуджувального імпульсу з часовою залежністю у вигляді 
ступеневої функції Хевісайда. Отриманий аналітичний вираз для 
напруженості електричного поля містить кратний інтеграл над 
швидкоосцилюючою функцією. Інтеграл було чисельно проаховано з 
застосуванням квадратурних формул Сімпсона-Рунге для багатовимірних
невласних інтегралів. Застосована методика дозволила отримати 
розв'язок з заданою точністю.

В роботі продемонстровано, що позитивною особливістю застосованого 
алгоритму є можливість узагальнення розв'язку з часовою залежністю 
струму збудження у вигляді функції Хевісайда до розв'язку нелінійної задачі 
випромінювання з довільною часовою залежністю струму.

Хоча кінцевий вираз для компонентів поля містить інтеграл, залежність 
від азимутального кута присутня в явному вигляді. Такий вираз з явною 
амплітудною та кутовою залежністю дозволив спостерігати деякі відомі 
ефекти взаємодії гармонійного поля з середовищем у випадку поширення 
наносекундних імпульсів. Також проведено аналіз отриманих графіків 
поля-поправки лінійного розв'язку. Проведено узагальнення розв'язку 
задачі випромінювання у Керрівське середовище  до задачі з 
поліноміальним нелінійним вектором поляризації. Для побудови нелінійного 
розв'язку знехтувано дисперсійними властивостями середовища, а також 
втратами провідності середовища.

Нелінійний розв'язок для Керрівського середовища отримано з 
лінійного наближення з використанням елементів теорії збурень та методу 
еволюційних рівнянь. Лінійну задачу випромінювання розв'язано для лінзової 
антени імпульсного випромінювання у наближенні плаского кругового 
синхронного однонапрямленого розподілу електричного струму. В роботі 
вперше представлено перехідну функцію для такої антени з явною 
залежністю від часу та просторових координат, яка справедлива в 
довільній точці спостереження.

Окрім розв'язку задач випромінювання явному виді проведено аналіз 
характеристик напрямленості лінзових антен імпульсного випромінювання при 
збуджені струмами з різними часовими формами як для ближньої так і дальньої 
зони випромінювання. Також проаналізовано влив точки спостереження на 
форму електромагнітного імпульсу з різними часовими залежностями 
збуджувального струму.

В роботі описано ефект інтерференції мінімумів та максимумів гармонійно 
промодульованого імпульсу вздовж напрямку поширення. При даному ефекті 
формується картина піків інтенсивності, споріднена до кілець Фур'є, що 
формуються внаслідок дифракції гармонійної хвилі на круглому отворі.

Отриманий аналітичний розв'язок задачі випромінювання у часовій області 
використано для моделювання процесу прийому-передачі електромагнітного 
імпульсу та виокремлення з нього корисної інформації. Аналіз отриманих результатів
та огляд літератури виявили ряд недоліків у сучасному способі виокремлення 
корисної інформації з імпульсної електромагнітної хвилі. В роботі 
представлено авторську методику виокремлення корисної інформації з 
надширокостугового імпульсу з урахуванням залежності його форми від 
точки спостереження. В основі запропонованої методики -- топологічно розділена 
на енодер та декорер фізична нейронна мережа з тривалою короткочасною 
пам'яттю у якості структурного елементу.

Моделювання процесу бездротової передачі інформації проведено з урахуванням 
зашумленості каналу. В якості передавальної антени використано лінзову антену
імпульсного випромінювання, а в якості приймальної детектор електричного поля.
Для моделювання багатокористувацького середовища використано імпульси різної 
форми і показана стійкість запропонованої системи до накладання імпульсів та до
розпізнавання сторонніх та власних імпульсів. Задля можливості відтворення 
отриманих результатів в дисертації представлено статистичні характеристики 
тренувальних даних, архітектуру нейронної мережі та візуалізацію процесу 
мінімалізації цільової функції. В роботі запропоновано методику впровадження 
отриманої моделі для розв'язання практичних задач зондування та 
телекомунікації шляхом застосування методики переносну нявчання.

Моделювання проведено з повнозв'язним та рекурентним енкодером. 
Проаналізовано обмеження та недоліки застосовання згорткових нейронних 
мереж в задачах аналізу часових послідовностей в реальному часі. За
результатами моделювання фізична нейронна мережа з застосуванням 
тривалої короткочасної пам'ятті дозволяє класифікувати надширокосмугові
наносекундні імпульси в реальному часі без оцифрування та з урахуванням 
взаємонакладання імпульсів. Отримані результати дозволяють вважати 
запропоновану методику розв'язанням задачі класифікації 
надширокосмугових нано- та пікосекундних імпульсів довільної 
форми при рівні амплітуди сигналу меншим за рівень шуму.
Моделювання показало, що особливо перспективно запропонована 
методика виглядає при аналізі імпульсів складної форми, наприклад 
великої кількості пелюсток чи фрактальної природи сигналу, коли досягнення 
критерію Найквіста при аналогово-цифровому перетворенні проблематичне 
через високу верхню частоту спектру імпульсу.

В роботі використано лише спеціально наукові та загальнонаукові методи
теоретичного аналізу, що знайшли експериментальне підтвердження.
Текст дисертації складають опубліковані та апробовані матеріали наукових 
досліджень. Матеріал, що викладено у роботі пов'язаний з декількома 
науково-дослідницькими проектами та програмами. Нові наукові результати,
що отримано в роботі не суперечать сучасним науковим знанням. На 
запропоновану автором методику виокремлення корисної інформації подано 
заяву на отримання винахідницького патенту.

Практично значним результатом кваліфікаційної роботи є методика обробки 
прийнятого надширокосмугового сигналу. Вирази для нелінійного поля, 
доповнені енергетичними діаграми, дозволяють теоретично оцінювати 
необхідність врахування нелінійних ефектів в практичних задачах, де 
застосовуються антени імпульсного випромінювання. Отримана модель 
поля лінзової антени імпульсного випромінювання у лінійному наближені 
може бути застосована для моделювання в реальному часі процесу 
поширення імпульсної надширокосмугової електромагнітної хвилі з 
урахуванням ефектів ближньої зони.

\keywords{часова область, електромагнітний імпульс, надширокосмугова 
електродинаміка, Керрівська нелінійність, слабка нелінійність, 
метод еволюційних рівнянь, машинне нявчання, рекурентні нейронні мережі,
тривала короткочасна пам'ять}

\end{abstract}

%%%%%%%%%%%%%%%%%%%%%%%%%%%%%%%%%%%%%%%%%%

 \begin{abstract}[language=english, header=false]
 	
\textbf{Akhmedov.~R. Field of Impuple Radiating Antennas in Linear and 
Nonlinear Medium} -- Qualifying scientific work on the rights of the manuscript.

Dissertation for the degree of a candidate of physical and 
mathematical sciences in specialty 01.04.03 -- radiophysics. 
V.~N.~Karazin Kharkiv National University, 
the Ministry of Education and Science of Ukraine, Kharkiv, 2020.

 In the thesis, we consider a problem of ...
 
 \keywords{time domain, electromagnetic pulse, ultrawidband, Kerr nonlinerity, 
 	weak nonlinerity, evalutionary approach, machine learning, 
 	recurent neural networks, long-short term memory}
 
\end{abstract}

% \nocite{Bar98fasp1,Bar98fasp2,PrB01umc}

% \begin{bibset}% [a]
%   {Список публікацій здобувача за~темою~дисертації}
%   % {Список публікацій здобувача}
%   \bibliographystyle{gost2008}
%   %
%   % Якщо не треба нумерація з крапкою, можна закоментувати наступні три рядки.
%   \makeatletter
%   \renewcommand\@biblabel[1]{#1.}
%   \makeatother
%   \bibliography{../my}
% \end{bibset}

\newpage

\begin{center} 
	\underline{\textbf{Список публікацій здобувача за темою дисертації}}
\end{center}

\vspace{1cm}

\mybibappendix


\tableofcontents

% \chapter*{Вступ}

%%%%%%%%%%%%%%%%%%%%%%%%%%%%%%%%%%%%%%%%%%%%%%%%%%%%%%%%%%%%%%%%%%%%%%%%%%%%%%%
\paragraph{Актуальність теми}

Час та частота - це абстракції, що описують одне явище природи - зміна
енергетичних взаємодій в системі. Для макроскопічної електродинаміки, зокрема,
існують підходи, що застосовують обидві абстракції. Вибір абстракції, тобто 
підходу, визначаються характером задачі, що вирішується. Історично склалось, що
більш широке розповсюдження отримала частотно-орієнтована методологія, яка в 
повній мірі виправдала себе. Однак новітні технології все частіше потребують 
розв'язків, які методи частотної області надати не зможуть. Наприклад, 
розв'язання задач розповсюдження та випромінювання в нестаціонарних 
неоднорідних середовищах, що характеризуються нелінійними та анізотропними 
ефектами. Це відновило інтерес до методів часової області.

%%%%%%%%%%%%%%%%%%%%%%%%%%%%%%%%%%%%%%%%%%%%%%%%%%%%%%%%%%%%%%%%%%%%%%%%%%%%%%%
\paragraph{Зв'язок роботи з науковими програмами}

Erasmus+

%%%%%%%%%%%%%%%%%%%%%%%%%%%%%%%%%%%%%%%%%%%%%%%%%%%%%%%%%%%%%%%%%%%%%%%%%%%%%%%
\paragraph{Мета та задача дослідження}

%%%%%%%%%%%%%%%%%%%%%%%%%%%%%%%%%%%%%%%%%%%%%%%%%%%%%%%%%%%%%%%%%%%%%%%%%%%%%%%
\paragraph{Методи дослідження}

\textcolor{red}{
Мотивуючись цим, вибираємо часовий метод для аналізу імпульсного випромінювання,
а саме метод еволюційних рівнянь. В якості основи для метода є вилучення 
поперечних компонент поля методом Рімана-Вольтера, також відомого у вітчизняній 
літературі як метод еволюційних рівнянь. Це дозволяє звести задачу розв'язання 
системи рівнянь Максвела до розв'язання диференціального рівняння другого 
порядку в часних похідних. У випадку вакуумного середовища це рівняння 
зводиться до рівняння Клейна-Гордона.}

В данній роботі використано метод теорії збуджень для побудови рекурентного
ітеративного методу врахування нелінійності.

Напрям пошуку солітоноподібних розвя'зків задачі випромінювання спрямовано 
згідно досліджень, що передбачають їх появу в нелінійний оптиці. Той факт,
що за основні ідеї та положення нелінійної оптики близькі до положень 
нелінійної радіофізики в теоретичномц плані дозволяє говорити про правильний 
напрям досліджень.

\textcolor{red}{Рекурентні нейронні мережі}

\textcolor{red}{Метод перехідної функції}

%%%%%%%%%%%%%%%%%%%%%%%%%%%%%%%%%%%%%%%%%%%%%%%%%%%%%%%%%%%%%%%%%%%%%%%%%%%%%%%
\paragraph{Наукова новизна отриманих результатів}

\textcolor{red}{Розв'язок декількох задач випромінювання в часовій області для 
лінійного та нелінійного простору}

\textcolor{red}{Адаптація методу еволюційних рівнянь для чисельного розв'язку
та побудова відповідного програмного комплексу, що розповсюджується під 
ліцензією GPL, як один з проектів GNU спів-товариства}

\textcolor{red}{Авторський метод синтезу протоколів передачі інформації}

%%%%%%%%%%%%%%%%%%%%%%%%%%%%%%%%%%%%%%%%%%%%%%%%%%%%%%%%%%%%%%%%%%%%%%%%%%%%%%%
\paragraph{Практичне значення отриманих результатів}

\begin{enumerate} 
	\item Шведський проект по аналізу властивостей надпровідникових матеріалів
	\item Передача інформації імпульсами на велику відстань
	\item Анаітичний розв'язок для лінзевих антен збуджених TEM рупором
	\item Оцінка нелінійних явищ що супроводжують випромінювання LIRA-и
	\item Новий підхід до прийому надширокосмугового сигналу без АЦП
\end{enumerate} 

%%%%%%%%%%%%%%%%%%%%%%%%%%%%%%%%%%%%%%%%%%%%%%%%%%%%%%%%%%%%%%%%%%%%%%%%%%%%%%%
\paragraph{Особистий вклад дисертанта}

Аналітичний розв'язок задачі плаского диску у всіх точках
Розв'язок задачі плаского диску слабкого нелінійного простору
Авторський метод синтезу протоколів передачі інформації

%%%%%%%%%%%%%%%%%%%%%%%%%%%%%%%%%%%%%%%%%%%%%%%%%%%%%%%%%%%%%%%%%%%%%%%%%%%%%%%
\paragraph{Апробація результатів дослідження}

\begin{enumerate} 
	\item Erasmus+
	\item Upsala Univ.
	\item Young scientist award UWBUSIS 2018
\end{enumerate} 

%%%%%%%%%%%%%%%%%%%%%%%%%%%%%%%%%%%%%%%%%%%%%%%%%%%%%%%%%%%%%%%%%%%%%%%%%%%%%%%
\paragraph{Публікації}

%%%%%%%%%%%%%%%%%%%%%%%%%%%%%%%%%%%%%%%%%%%%%%%%%%%%%%%%%%%%%%%%%%%%%%%%%%%%%%%
\paragraph{Зміст роботи}

%%%%%%%%%%%%%%%%%%%%%%%%%%%%%%%%%%%%%%%%%%%%%%%%%%%%%%%%%%%%%%%%%%%%%%%%%%%%%%%
\paragraph{Подяка}

\chapter*{Перелік умовних позначень}

% a b c d e f g h i j k l m n o p q r s t u v w x y z
% а б в г ґ д е є ж з и і ї й к л м н о п р с т у ф х ц ч ш щ ь ю я

BER -- (від англ. bit error rate/ratio) коефіцієнт бітових помилок;

GPU -- (від англ. graphics processing unit) графічний процесор;

GRU -- (від англ. gated recurrent unit) вентильний рекурентний вузол;

IPS -- (від англ. indor position system) система внутрішнього позиціювання;

IRA -- (від англ. impulse radiating antenna) антена імпульсного випромінювання;

LIRA -- (від англ. lens impulse radiating antenna) лінзова антена імпульсного
випромінювання;

LSTM -- (від англ. long short-term memory) тривала короткочасна пам'ять;

RIRA -- (від англ. reflector impulse radiating antenna) рефлекторна антена 
імпульсного випромінювання;

RNN -- (від англ. recurrent neural network) рекурентна нейронна мережа;

SoC -- (від англ. a system on a chip) однокристальна система;

TE -- (від англ. transvers electric) поперечна магнітна, тобто хвиля без 
$ E_z $ компоненти;

TM -- (від англ. transvers magnetic) поперечна електрична, тобто хвиля без 
$ H_z $ компоненти;

TEM -- (від англ. transvers electromagnetic) поперечна електромагнітна, тобто 
хвиля  без $ E_z $ та $ H_z $ компонент;

АШНМ -- апаратна штучна нейронна мережа;

СТВ -- спеціальна теорія відносності;

НШС -- надширока смуга (англ. UWB);

ШНМ -- штучна нейронна мережа.

\chapter{Методи роботи з імпульсними полями}
\label{ch:review}

%%%%%%%%%%%%%%%%%%%%%%%%%%%%%%%%%%%%%%%%%%%%%%%%%%%%%%%%%%%%%%%%%%%%%%%%%%%%%%
\section{Сучасна імпульсна та надширокосмугова радіоелектроніка}

% Розвиток науки і техніки
% Наслідком історичної ґенези радіотехнічного обладнання
Підхід частотної селекції інформації набув масового поширення в радіофізиці
перебігом історії XX-го століття \cite{imp:Nosich2001}. Проте, вимоги 
сучасності до радіотехнічних приладів значно підвищились: з'явилась потреба 
у створенні нових засобів комунікації та локації. Зокрема, одним із 
способів покращення технічних характеристик радіоапаратури стало 
використання надширокосмугових режимів роботи, а на початку XXI-го 
століття було закладено практичне застосування нелінійної природи 
надширокосмугових електромагнітних процесів \cite{imp:Chernogor2008}.

Електромагнітні коливання, що мають імпульсну та надширокосмугову природу, 
називаються нестаціонарними. Імпульсні та надширокосмугові явища, 
що мають електромагнітну природу, вивчаються таким розділом фізики як 
нестаціонарна електродинаміка. Прийнято розділяти поняття імпульсного та
надширокосмугового поля -- імпульсне поле не завжди називають 
надширокосмуговим. Згідно міжнародних стандартів, надширокосмуговим 
називається коливання електромагнітної енергії, відносна ширина спектра 
якого принаймні вища за $ 25\% $ \cite{imp:RadarStandard2007}, тобто:

\begin{equation} \label{eq:spectum_width}
\eta = \frac{f_{max} - f_{min}}{f_{max} + f_{min}} > 0.25,
\end{equation}
%
де $ f_{max} $ -- максимальна частота спектру сигналу, а $ f_{min} $ -- 
мінімальна частота спектру. Хоча границі спектру реального сигналу і 
визначаться досить умовно, існують декілька загальновизнаних способів їх 
визначити, серед яких найчастіше використовується рівень половинної енергії 
спектру ($ 0.707 $ за амплітудно-частотною характеристикою).

В природі спостерігати надширокосмугові електромагнітні хвилі можливо при 
ядерних реакціях \cite{imp:Baum2007}, при проходженні сонячного термінатора, 
при атмосферних явищах \cite{imp:Uman2006}, землетрусах 
\cite{imp:Hayakawa2008} тощо. Крім того, фоновий електромагнітний шум і 
власні завади радіотехнічного обладнання найчастіше мають 
надширокосмугову природу. Широка розповсюдженість таких явищ в 
навколишньому середовищі слугує аргументом на користь ефективності 
застосування надширокосмугових хвиль в науці і техніці.

Імпульсні радіохвилі штучно формують двома підходами: випромінюють антеною 
з низькою добротністю обмежений в часі гармонійний сигнал 
\cite{imp:Mesyas1963, imp:Mesyas1974} або стрибковий розряд електричного 
струму \cite{imp:BaumMN053}. Антено-фідерна система, для задовольняння 
умови низької добротності, виготовляється з максимально схожими 
характеристиками на широкому спектрі частот. Якщо спектр сформованого в 
такий спосіб електромагнітного коливання не задовольняє умові 
мінімальної ширини спектру \eqref{eq:spectum_width}, то його називають 
радіоімпульсом, який не розгладяться в даній роботі.

На практиці, однією з найважливіших якісних характеристик надширокосмугових
радіотехнічних пристроїв виявилась тривалість імпульсу. Імпульси меншої 
тривалості дозволяють, насамперед, концентрувати більшу кількість енергії,
що важливо, наприклад, для галузі медичної діагностики 
\cite{imp:Guardiola2010}. Також, менша тривалість імпульсу надає 
можливість передавати більшу кількість цифрових даних через радіоканал, що 
напряму витікає з формули Хартлі \cite{imp:Taub1986}. За тривалість 
випроміненого імпульсу відповідає цілий ряд параметрів серед яких і 
параметри антени, але визначальний вплив має електронний ключ, що подає 
струм живлення на антено-фідерну систему замикаючи, а інколи і розмикаючи 
електричне коло.

% нові фізичні принципи

Основним фактором, що підбурює інтерес до надширокосмугових систем є 
теоретична можливість поєднати особливості електромагнітного випромінювання 
на різних частотних діапазонах, що дозволить покращити характеристики 
існуючих радіоелектронних систем та розширить сферу їх застосування. 
Також, використання надширокосмугових імпульсних хвиль, дозволяє
розв'язати неможливі для вузькодіапазонних приладів технічні задачі. 
Деякі з таких технічних задач вже реалізували себе на ринку споживацької 
електроніки, наприклад:

\begin{enumerate}
\item автомобільний радар малої дальності дії - допоміжний засіб 
технічного зору для автопілотів п'ятого покоління, що здатен працювати 
з потрібною точністю у несприятливих погодних умовах, а тобто в 
дисперсному, поглинальному і провідному середовищі.
\cite{imp:Yarovoy2017};
\item пристрій підводного радіозв'язку - засіб бездротової 
телекомунікації для систем зв'язку, де один або більше учасників 
зв'язку, занурені у провідне середовище \cite{imp:Garcia2009, 
imp:Karagianni2015}.
% \item Засіб вимірювання товщини та якості фарби (толщік)
% \item Кліматичний радар
\end{enumerate}

Також, з початку 60-х років технічні засоби, робота яких базується на 
використанні надширокосмугових електромагнітних хвиль, активно 
використовуються військовими. Серед таких досліджень варто відмітити 
науково-дослідницький проект ATLAS-1 під керівництвом К. Баума 
\cite{imp:BaumSSN0267}, який було спрямовано на захист устаткування 
літальних апаратів від електромагнітного уражаючого фактору ядерного 
вибуху.

Завдяки відсутності носійної хвилі і роботі передавача в імпульсному режимі 
значно скорочується споживання електричної енергії передавачем, що породило 
значний інтерес до використання фізичних принципів випромінювання 
надширокосмугових сигналів в галузі техніки інтернету речей (IoT). 
Сучасні мережеві протоколи канального рівня стеку TCP/IP, такі як WiFi6 
(IEEE 802.11ax) \cite{imp:Khorov2019} використовують саме радіоімпульси,
як носій інформації, що негативно впливає на споживання електричної енергії.

Окремим напрямком використання надширокосмугових радіолокаційних систем 
стало їх застосування для систем внутрішнього позиціювання (IPS), що у 
поєднанні з методикою технічного зору SLAM (одночасна локалізація і 
картографування) дозволяє отримувати тривимірну мапу внутрішнього 
середовища \cite{imp:Segura2011}. Аналогічна система внутрішнього 
позиціювання, що базується використанні гармонійних сигналів 
характеризується меншою точністю \cite{imp:Zou2017}.

%%%%%%%%%%%%%%%%%%%%%%%%%%%%%%%%%%%%%%%%%%%%%%%%%%%%%%%%%%%%%%%%%%%%%%%%%%%%%%
\section{Дослідження нелінійних явищ при поширенні нестаціонарних хвиль}

\textcolor{red}{ Моделі врахування нелінійності // методи розв'язання 
нелініних задач:
%
\begin{enumerate}
\item розв'язання нелінійного рівняння (Черногор, Лазоренко)
\item теорія збурень (Шведи, ми)
\item ... (Яцик)
\item ...
\end{enumerate}
%
Моделі нелінійності повітряного середовища:
%
\begin{enumerate}
\item ...
\end{enumerate}
%
}

Що ще повинно бути окрім того, що представлено в розділі \ref{ch:nonlinear}.

%%%%%%%%%%%%%%%%%%%%%%%%%%%%%%%%%%%%%%%%%%%%%%%%%%%%%%%%%%%%%%%%%%%%%%%%%%%%%%
\section{Методи розв'язання зовнішніх задач випромінювання нестаціонарних хвиль}

Хоча, всі методи аналітичного та числового розв'язання задач 
випромінювання і базуються на одних і тих самих фізичних принципах, що 
закладаються рівняннями Максвела, важливим аспектом залишається вибір 
методу розв'язання системи рівнянь з урахування обмежень обраного методу. 

Постанова задач випромінювання в даній роботі накладає значні обмеження щодо 
методів розв'язання. Вибраний метод повинен надавати розв'язок задач 
випромінювання, як у ближній так і у дальній зоні, в усьому діапазоні 
частот електромагнітних хвиль. Такі обмеження визначаються 
необхідністю враховувати нелінійні ефекти поблизу до апертури випромінювача, 
тобто в просторовій зоні формування хвилі \cite{imp:BaumUWBSP1}.

Також, за умовами поставлених задач, необхідно отримати розв'язок у вигляді
плавної аналітичної функції, без особливих точок, з огляду на необхідність 
застосування теорії збурень до отриманого розв'язку. 

% Застосування ітеративної теорії збурень для врахування нелінійності
% найбільш ефективно при використанні на кожній ітерації такого методу 
% розв'язання системи рівнянь Максвела, який надає аналітичний зв'язк без 
% особливих точок. 

Розглянемо окремо обмеження області застосування деяких методів 
розв'язання системи рівнянь Максвела відносно напруженості 
електромагнітного поля та оцінено можливість їх застосування до задач, 
що розглядаються у даній роботі.

\paragraph{Метод скінченних різниць в часовій області}

Числовий метод, що дозволяє отримати значення інтенсивності електромагнітного 
поля з заданою точністю в кожен момент часу для дискретного набору точок, що
визначається наперед заданою сіткою значень \cite{imp:FDTD1966}. Метод 
представлено для одно-, двох- і тривимірних \cite{imp:FDTD1975} вільних 
середовищ та резонаторів. Адаптація алгоритму для GPU (графічних 
обчислювальних систем) \cite{imp:FDTD2011} роблять метод 
обчислювально-оптимальним і досить перспективним для розв'язання 
широкого кола задач випромінювання нестаціонарного струму.

Метод належить до сімейства динамічних алгоритмів, тобто визначає розв'язок 
в деякій точці за значеннями інтенсивності поля в точках навколо, що 
призводить до збільшення розрахункової складності для точок спостереження на 
великих відставаннях від джерела. Також метод можна застосувати лише до 
задач випромінювання з реальним джерелом, що унеможливлює розв'язання 
модельних задач, наприклад задачі, що розглядаються в даній роботі.

\paragraph{Методи частотної області}

Популярним підходом до спрощення розв'язання системи рівнянь Максвела відносно
напруженості електричного і магнітного поля є припущення, що часова залежність 
джерела електромагнітної енергії має гармонійний характер, наприклад 
$ e^{i \omega t} $, де $ \omega $ - фіксована частота. Сімейство таких 
методів \cite{imp:Shubarin1960} інколи застосовують для розв'язання 
нестаціонарних задач \cite{imp:Harmuth1981}, представляючи джерело у вигляді 
розкладу в ряд Фур'є. Серед таких методів можна виділити метод векторного 
потенціалу та метод функції Гріна.

Найбільш простим частотним методом розв'язку системи рівнянь Максвела є 
зменшення кількості шуканих змінних переходом до рівняння векторного 
потенціалу. У купі з каліброваним потенціалом Лоренца цей метод стає досить 
зручним для розв'язання нестаціонарних задач.

Сутність методу функції Гріна полягає в тому, що для розв'язання задачі 
випромінювання з довільним розподілом струму або заряду необхідно знати 
внесок елементарної ділянки з рівномірним розподілом струму. Таким чином, 
поділивши заданий розділ джерела електромагнітного поля на ділянки з 
рівномірним розподілом, можна обчислити значення поля від заданого джерела 
користуючись принципом суперпозиції. Такий елементарний випромінювач 
називається електричним диполем Герца.

\paragraph{Метод моментів до розсіювання електромагнітних хвиль}

Сімейство чисельних та аналітичних методів, які дозволяють розв'язувати
задачі дифракції надширокосмугових хвиль на провідних поверхнях.
Такі методи широко використовувались К. Баумом для розв'язування задач, які 
зустрінуться в даній роботі. Урахування ефектів в ближній зоні закладено в
математичний апарат методу, але обмежене складністю застосування цього 
математичного апарату. Наприклад, розв'язання задачі випромінювання 
рефлекторної антени у ближній зоні досить складне, через труднощі врахування 
ефектів взаємодії опромінювача та рефлектора у ближній зоні, тобто 
опромінювач розглядається в спрощеному виді, як джерело сферичної хвилі.

\paragraph{Метод еволюційних рівнянь}

Історично, метод еволюційних рівнянь -- адаптація методу модового базису 
\cite{imp:Tretyakov1986} до задач випромінювання поля обмеженим в часі і 
просторі розподілом струму у вільний простір \cite{imp:Tretyakov2004}.

Метод еволюційних рівнянь дозволяє звести систему рівнянь Максвела в 
диференціальній формі до одновимірної задачі зберігши явну залежність 
напруженості поля від часу. Таке перетворення здійснюється за рахунок 
проектування компонентів поля на базис функцій з аргументами у вигляді 
координат, поперечних до напрямку поширення хвилі \cite{imp:Dumin2010}. 
Таким чином шукана величина звільняється від залежності від поперечних 
координат та стають функціями лише поздовжньої координати та часу.

З іншого боку перехід від векторних шуканих функцій до скалярних здійснюється
за рахунок збільшення кількості рівнянь в системі. Отримана з системи Максвела
система рівнянь і називається еволюційною.

Отримати поперечний базис електромагнітного поля дозволяє зв'язок між 
поперечними та поздовжніми компонентами поля -- поздовжні компоненти 
породжуються поперечними і знову формують їх. 
Побудувати поперечний базис електромагнітного поля можна в довільній системі 
координат. Вибір необхідної системи координат залежить від геометрії задач 
випромінювання, що розглядаються. Ця робота, здебільшого, присвячена 
апертурним випромінювачам імпульсного струму, а отже краще підійдуть 
системи координат осьової симетрії, такі як циліндрична та сферична.

%%%%%%%%%%%%%%%%%%%%%%%%%%%%%%%%%%%%%%%%%%%%%%%%%%%%%%%%%%%%%%%%%%%%%%%%%%%%%%
\section{Розв'язання задач випромінювання методом еволюційних рівнянь
в часовій області}

Система рівнянь Максвела в диференціальній формі математично описує процес
випромінювання і формування електромагнітного поля. Історично, система рівнянь 
Максвела з'явилась в наслідок дослідження взаємозв'язку електричного і 
магнітного поля з довільною залежністю від часу \cite{imp:Maxwell1865}.

З моменту створення спеціальної теорії відносності \cite{imp:Einstein1905} 
актуальність та справедливість застосування системи рівнянь Максвела в задачах 
макроскопічної нестаціонарної електродинаміки не втрачено \cite{imp:Bray2006}.

Базовими поняттями електродинаміки є напруженість поля: електричного 
$ \vect{E} $ та магнітного $ \vect{H} $; а також індукція електричного 
$ \vect{D} $ та магнітного $ \vect{B} $ полів. Ці базові поняття нелінійно 
пов'язані матеріальними рівняннями середовища через електромагнітні 
властивості речовини в якій спостерігаються:

\begin{equation} \label{eq:MInduct}
\vect{D} = \epsilon_0 \vect{E} + \func{\vect{P}}{\vect{E},\vect{H}},
\end{equation}

\begin{equation} \label{eq:EInduct} 
\vect{B} = \mu_0 \vect{H} + \mu_0 \func{\vect{M}}{\vect{E},\vect{H}},
\end{equation}
%
де $ \vect{P} $ - поляризація речовини, а $ \vect{M} $ - її намагніченість.
$ \epsilon_0 $ та $ \mu_0 $ - фундаментальні константи з системи 
вимірювання CI, діелектрична та магнітна проникності абсолютного вакууму.

Рівності \eqref{eq:MInduct} та \eqref{eq:EInduct} є загальним видом 
взаємозв'язку індукції і поля та справедливі для всіх типів середовищ.
Метод модового базису, в свою чергу, дозволяє розглядати лише середовище 
певного виду - шарово-неоднорідні ізотропні середовища з лінійною поляризацією.
Таким чином діелектрична та магнітна проникності середовища, що розглядаються 
мають вид $ \epsilon(z,t) $ і $ \mu(z,t) $, де $ z $ - поздовжня координата 
поширення хвилі, а $ t $ - змінна часу.

Систему рівнянь Максвела можна поділити на два типи рівнянь. Перші, роторні,
пов'язують зміну напруженості магнітного поля зі зміною електричної індукції, 
а зміну напруженості електричного поля зі зміною магнітної індукції. Ці закони 
носять імена своїх першовідкривачів: закон Ампера та закон індукції Фарадея 
відповідно і в диференціальній формі мають наступний вид:

\begin{equation} \label{eq:AmpereLow}
\crossprod{\nabla}{\vect{H}} = 
\frac{\partial \vect{D}}{\partial t} + \vect{J^\sigma} + \vect{J^e},
\end{equation}

\begin{equation} \label{eq:FaradayInduction}
- \crossprod{\nabla}{\vect{E}} =
\frac{\partial \vect{B}}{\partial t} + \vect{J^h},
\end{equation}
%
де $ J^\sigma $ - струм провідності середовища, $ J^e $ - сторонній 
електричний струм, а $ J^h $ - сторонній струм умовних магнітних зарядів, що
інколи використовується, як зручна інтерпретація електричного струму.

Другий тип рівнянь в системі Максвела, дивергентні - пов'язують появу 
електричної та магнітної індукції з наявністю вільних зарядів у просторі 
та називається теоремами Гауса для електричних та магнітних зарядів:

\begin{equation} \label{eq:GaussTheorem}
\dotprod{\nabla}{\vect{D}} = \rho^\sigma + \rho^e,
\end{equation}

\begin{equation} \label{eq:GaussMagnetic}
\dotprod{\nabla}{\vect{B}} = \rho^h,
\end{equation}
%
де $ \rho^e $ - розподіл густини електричних зарядів, $ \rho^h $ - розподіл
густини магнітних зарядів, $ \rho^\sigma $ - розподіл густини заряду, що
утворюється внаслідок наявності провідних властивостей середовища 
поширення хвилі.

Задля зручності застосування рівнянь Максвела, об'єднують різні складові 
неоднорідності в системі в одну, вводячи узагальнене джерело поля, що 
характеризується густиною розподілу електричних і магнітних струмів та 
зарядів. Такий підхід цікавий тим, що можна врахувати і нелінійні складові 
вектору поляризації в якості джерел стороннього струму, що буде показано далі. 
Введемо узагальнений електричний $ \vect{J} $ та $ \vect{I} $ магнітній 
струми і надалі працюватимемо з ним вважаючи, що всі можливі джерела струму 
враховано цими доданками:

\begin{equation} \label{eq:e_current}
\vect{J} = \vect{J^\sigma} + \vect{J^e},
\end{equation}

\begin{equation} \label{eq:m_current}
\vect{I} = \vect{J^h}.
\end{equation}

Припущення лінійності поляризації \eqref{eq:EInduct} та намагніченості 
\eqref{eq:MInduct} в матеріальних рівняннях, а також використання нотації
\eqref{eq:e_current} і \eqref{eq:m_current} дозволяють записати роторні 
рівняння \eqref{eq:AmpereLow} і \eqref{eq:FaradayInduction} наступним чином:

\textcolor{blue}{ \begin{equation*} \begin{aligned}
\crossprod{\nabla}{\vect{H}} = \epsilon_0 \partder{}{t} \left[ 
\vect{E} + \left( \epsilon - 1 \right) \vect{E} \right] + 
\partder{\vect{P^\prime}}{t} + \vect{J^\sigma} + \vect{J^e}= \\
= \epsilon_0 \partder{}{t} \left( \epsilon \vect{E} \right) +
\partder{\vect{P^\prime}}{t} + \vect{J^\sigma} + \vect{J^e} = 
\epsilon_0 \left( \partder{\epsilon}{t} 
\vect{E} + \epsilon \partder{\vect{E}}{t} \right) + 
\partder{\vect{P^\prime}}{t} + \vect{J^\sigma} + \vect{J^e}
\end{aligned} \end{equation*} }
%
\begin{equation} \label{eq:rotHfromE}
\crossprod{\nabla}{\vect{H}} = 
\epsilon_0 \partder{}{t} \left( \epsilon \vect{E} \right) + \vect{J}
\end{equation}

\begin{equation} \label{eq:rotEfromH} 
- \crossprod{\nabla}{\vect{E}} = 
\mu_0 \partder{}{t} \left( \mu \vect{H} \right) + \vect{I}
\end{equation}

Схожа ситуація і для джерел що представлені розподілом заряду. Нехай,
всі розподіли електричного заряду включаються до $ \varrho $, а всі розподіли 
магнітного заряду до $ g $. Тоді, підставляючи матеріальні рівняння до теореми 
Гауса отримаємо:

\textcolor{blue}{ \begin{equation*} \begin{aligned}
\dotprod{\nabla}{ \left( \epsilon_0 \epsilon \vect{E} + 
\vect{P^\prime} \right) } = \rho^\sigma + \rho^e \\
\dotprod{\nabla}{ \epsilon_0 \epsilon \vect{E} } = \rho^\sigma + \rho^e -
\dotprod{\nabla}{ \vect{P^\prime} }
\end{aligned} \end{equation*} }
%
\begin{equation} \label{eq:divE} 
\epsilon_0 \dotprod{\nabla}{ \epsilon(z,t) \vect{E} } = \varrho,
\end{equation}
%
\begin{equation} \label{eq:divH}
\mu_0 \dotprod{\nabla}{ \mu(z,t) \vect{H} } = g.
\end{equation}

Диференціальні рівняння першого порядку \eqref{eq:divE}, \eqref{eq:divH} та 
векторні другого \eqref{eq:rotHfromE}, \eqref{eq:rotEfromH} формують систему 
рівнянь Максвела відносно невідомих векторних величин $ \vect{E} $ і 
$ \vect{H} $. Для спрощення цієї системи пропонується використати метод 
розділення змінних Фур'є. Аналогічно до методу функції Гріна з класичної 
електродинаміки, спрощення відбувається шляхом зменшення кількості невідомих 
\textcolor{red}{[Джерело]}, вилучаючи їх з рівняння.

Першим етапом застосування методу еволюційних рівнянь задля розв'язання системи
рівнянь Максвела є відокремлення поздовжніх компонент поля $ E_z $ та $ H_z $.
Ця процедура є зменшенням розмірності рівнянь, а отже повинна збільшити їх 
кількість. Відокремлення поздовжніх компонент поля дозволить знайти розв'язок 
початкової задачі у вигляді суперпозиції двох розв'язків: відносно поперечних 
електричних (ТЕ) і поперечних магнітних (ТМ) полів. Вочевидь, вибір методу 
еволюційних рівнянь в якості способу розвязаня задачі випромінювання є 
доцільним коли заздалегідь відомо, що шукане поле має вигляд ТЕ, ТМ чи ТЕМ. 
Такого висновку можна дійти з огляду на те, що половина отриманих рівнянь 
матиме лише тривіальний розв'язок.

% Метод Функції Гріна як і будь-який метод частотної області, вибирає саме час, 
% як змінну для виключення, обмежуючи себе розгляданням квазі-стаціонарних 
% процесів. Метод еволюційних рівнянь, в свою чергу, пропонує виключення 
% просторов змінної.

% Виключення саме цієї просторової залежності зумовлено тісним зв'язком 
% координати поширення з координатою часу через принцип причинності. Його 
% сутність в термінології спеціальної теорії відносності полягає в тому, що дві 
% події можуть бути причинно зв'язані одна з одної тоді, і тільки коли, інтервал 
% між ними часоподібний, що напряму слідує з того, що ніяка взаємодія не може 
% поширюватись швидше за світло. \cite[ст. 22]{imp:LandauII}. В 
% електродинамічному сенсі це означає, що поле не може поширитись далі у 
% вільному просторі, ніж може пройти світло за той самий час та по тій самій осі 
% випромінювання. Математично це можна записати, як $ ct - z > 0 $, де $ z $
% поздовжна просторова координата поширення, а $ c = 2,998 \cdot 10^8 $ м/с 
% -- фундаментальна константа, швидкість світла в вакуумі.

З рівнянь Максвела відокремимо векторну компоненту $ \vect{z_0} $ для всіх 
величин. Так як складовими рівнянь Максвела є не просто векторні величини, а
векторно-диференціальними операторами з векторними аргументами, то виділяти 
поздовжню компоненту необхідно саме з них. Як відомо, будь-який вектор можна 
розписати, як суму добутків ортів та відповідних проекцій, але робити це 
необхідно з урахуванням коефіцієнтів Ламе \cite{imp:Korn1974} для заданої 
системи координат. Як згадувалось раніше, ця дисертація розглядає апертурні 
випромінювачі круглої форми, а отже розглянемо векторні оператори в контексті 
циліндричної системи координат. Так, оператор $ \nabla $ можна записати як 
$ \nabla_\perp + \vect{z_0} \partder{}{z} $, а довільний вектор
$ \vect{A} $, як $ \vect{A_\perp} + \vect{z_0} A_z $. Користуючись визначенням 
векторного добутку лінійної комбінації векторів, з \eqref{eq:rotHfromE} 
отримаємо два незалежні рівняння. Тоді роторні рівняння Максвела матимуть розділятися на чотири і матимуть вид:

\textcolor{blue}{ \begin{equation*} \begin{aligned}
\rot{\vect{A}} = \crossprod{\nabla}{\vect{A}} = \crossprod
{\left( \nabla_\perp + \vect{z_0} \partder{}{z} \right)}
{\left( \vect{A_\perp} + \vect{z_0} A_z \right)} = \\
= \crossprod{\nabla_\perp}{\vect{A_\perp}} + 
\crossprod{\nabla_\perp}{\vect{z_0} A_z} +
\crossprod{\vect{z_0} \partder{}{z}}{\vect{A_\perp}} +
\crossprod{ \vect{z_0} \partder{}{z} }{ \vect{z_0} A_z } = \\
= \crossprod{\nabla_\perp}{\vect{A_\perp}} + 
\crossprod{\nabla_\perp}{\vect{z_0}} A_z +
\partder{}{z} \crossprod{\vect{z_0}}{\vect{A_\perp}}
\end{aligned} \end{equation*} }
%
\textcolor{blue}{ \begin{equation*} \begin{aligned}
\crossprod{\nabla}{\vect{H}} = 
\crossprod{\nabla_\perp}{\vect{H_\perp}} + 
\crossprod{\nabla_\perp}{\vect{z_0}} H_z +
\partder{}{z} \crossprod{\vect{z_0}}{\vect{H_\perp}} = \\
= \epsilon_0 \partder{}{t} \left( \epsilon  \vect{E_\perp} + 
\epsilon \vect{z_0} E_z \right) + \vect{J_\perp} + \vect{z_0} J_z
\end{aligned} \end{equation*} }
%
\begin{equation} \label{eq:rotHt} 
\crossprod{\nabla_\perp}{\vect{z_0}} H_z +
\partder{}{z} \crossprod{\vect{z_0}}{\vect{H_\perp}} =
\epsilon_0 \partder{}{t} \left( \epsilon  \vect{E_\perp} \right) + 
\vect{J_\perp},
\end{equation}
%
\textcolor{blue}{ \begin{equation*} \begin{aligned}
\dotprod{\vect{z_0}}{\crossprod{\nabla_\perp}{\vect{H_\perp}}} =
\triple{\vect{z_0}}{\nabla_\perp}{\vect{H_\perp}}
\end{aligned} \end{equation*} }
%
\begin{equation} \label{eq:rotHz}
\triple{\vect{z_0}}{\nabla_\perp}{\vect{H_\perp}} = 
\epsilon_0 \partder{}{t} \left( \epsilon  E_z \right) + J_z,
\end{equation}
%
\textcolor{blue}{ \begin{equation*} \begin{aligned}
- \crossprod{\nabla}{\vect{E}} = 
- \crossprod{\nabla_\perp}{\vect{E_\perp}} - 
\crossprod{\nabla_\perp}{\vect{z_0}} E_z -
\partder{}{z} \crossprod{\vect{z_0}}{\vect{E_\perp}} = \\
= \mu_0 \partder{}{t} \left( \mu  \vect{H_\perp} + 
\mu \vect{z_0} H_z \right) + \vect{I_\perp} + \vect{z_0} I_z
\end{aligned} \end{equation*} }
%
\begin{equation} \label{eq:rotEt} 
- \crossprod{\nabla_\perp}{\vect{z_0}} E_z -
\partder{}{z} \crossprod{\vect{z_0}}{\vect{E_\perp}} = 
\mu_0 \partder{}{t} \left( \mu  \vect{H_\perp} \right) + \vect{I_\perp},
\end{equation}
%
\textcolor{blue}{ \begin{equation*} \begin{aligned}
- \dotprod{\vect{z_0}}{\crossprod{\nabla_\perp}{\vect{E_\perp}}} = 
- \triple{\vect{z_0}}{\nabla_\perp}{\vect{E_\perp}}
\end{aligned} \end{equation*} }
%
\begin{equation} \label{eq:rotEz}
- \triple{\vect{z_0}}{\nabla_\perp}{\vect{E_\perp}} =
\mu_0 \partder{}{t} \left(\mu H_z \right) + I_z.
\end{equation}

З теорем Гауса \eqref{eq:divE} та \eqref{eq:divH} також виключимо 
поздовжню компоненту $ \vect{z_0} $, користуючись комутативними та 
асоціативними властивостями скалярного добутку векторів.
%
\textcolor{blue}{ \begin{equation*} \begin{aligned}
\dotprod{\nabla}{\vect{A}} = \dotprod
{\left( \nabla_\perp + \vect{z_0} \partder{}{z} \right)}
{\left( \vect{A_\perp} + \vect{z_0} A_z \right)} = \\
= \dotprod{\nabla_\perp}{\vect{A_\perp}} + 
\dotprod{\nabla_\perp}{\vect{z_0} A_z}  +
\dotprod{\vect{z_0} \partder{}{z}}{\vect{A_\perp}} +
\dotprod{\vect{z_0} \partder{}{z}}{\vect{z_0} A_z} = \\
= \dotprod{\nabla_\perp}{\vect{A_\perp}} +
\dotprod{\nabla_\perp}{\vect{z_0}} A_z +
\dotprod{\vect{z_0}}{\partder{\vect{A_\perp}}{z}} +
\dotprod{\vect{z_0}}{\vect{z_0}} \partder{A_z}{z} = \\
= \dotprod{\nabla_\perp}{\vect{A_\perp}} + \partder{A_z}{z}
\end{aligned} \end{equation*} }
%
\begin{equation} \label{eq:divEt} 
\epsilon_0 \partder{}{z} \left( \epsilon E_z \right) = 
\varrho - \epsilon_0 \epsilon \dotprod{\nabla_\perp}{\vect{E_\perp}}
\end{equation}
%
\begin{equation} \label{eq:divHt}
\mu_0 \partder{}{z} \left( \mu H_z \right) = 
g - \mu_0 \mu \dotprod{\nabla_\perp}{\vect{H_\perp}}
\end{equation}

Як зазначалось раніше, в данні роботі розглядається пошарово неоднорідне 
середовище, а отже $ \epsilon = \epsilon(z,t) $ и $ \mu = \mu(z,t) $. Тому
маємо змогу винести показники проникності середовища з під оператора у 
від'ємнику.

Випишемо в окрему систему тільки ті рівняння, що явно містять $ H_z $ та 
виразимо з них саме цю компоненту. Тепер, діючи на рівняння \eqref{eq:rotHt} 
операторами $ \mu_0 \partder{}{t} \mu $ і $ \mu_0 \partder{}{z} \mu $ 
виключимо поздовжну магнітну компоненту з рівнянь Максвелла, підставивши, 
відповідно, \eqref{eq:divHt} та \eqref{eq:rotEz}. В результаті маємо систему 
не з трьох рівнянь, а вже з двох відносно поперечних компонент поля. 
Результат зашипимо не в вигляді системи, а як чотиривимірне векторне рівняння:

\textcolor{blue}{ \begin{equation*} \begin{aligned}
\begin{cases} 
\crossprod{\nabla_\perp}{\vect{z_0}} H_z =
\epsilon_0 \partder{}{t} \left( \epsilon \vect{E_\perp} \right) -
\partder{}{z} \crossprod{\vect{z_0}}{\vect{H_\perp}} + \vect{J_\perp} \\
- \triple{\vect{z_0}}{\nabla_\perp}{\vect{E_\perp}} =
\mu_0 \partder{}{t} \left(\mu H_z \right) + I_z \\ 
\mu_0 \partder{}{z} \left( \mu H_z \right) = 
g - \mu_0 \mu \dotprod{\nabla_\perp}{\vect{H_\perp}}
\end{cases}
\end{aligned} \end{equation*} }
%
\textcolor{blue}{ \begin{equation*} \begin{aligned}
\begin{cases} 
\left. \crossprod{\nabla_\perp}{\vect{z_0}} H_z = \vect{F_H} 
\right| \cdot \mu_0 \partder{}{z} \mu \\
\left. \crossprod{\nabla_\perp}{\vect{z_0}} H_z = \vect{F_H} 
\right| \cdot \mu_0 \partder{}{t} \mu \\
\mu_0 \partder{}{z} \left( \mu H_z \right) = 
g - \mu_0 \mu \dotprod{\nabla_\perp}{\vect{H_\perp}} \\
\mu_0 \partder{}{t} \left(\mu H_z \right) =
\triple{\nabla_\perp}{\vect{z_0}}{\vect{E_\perp}} - I_z
\end{cases}
\end{aligned} \end{equation*} }
%
\textcolor{blue}{ \begin{equation*} \begin{aligned}
\begin{cases} 
\crossprod{\nabla_\perp}{\vect{z_0}} \left(
g - \mu_0 \mu \dotprod{\nabla_\perp}{\vect{H_\perp}} \right) =
\mu_0 \partder{}{z} \left( \mu \vect{F_H} \right) \\
\crossprod{\nabla_\perp}{\vect{z_0}} \left(
\triple{\nabla_\perp}{\vect{z_0}}{\vect{E_\perp}} - I_z \right) = 
\mu_0 \partder{}{t} \left( \mu \vect{F_H} \right)
\end{cases}
\end{aligned} \end{equation*} }
%
\textcolor{blue}{ \begin{equation*} \begin{aligned}
\begin{cases} 
- \mu_0 \mu \dotprod{\crossprod{\nabla_\perp}{\vect{z_0}} \nabla_\perp}
{\vect{H_\perp}} = \mu_0 \partder{}{z} \left( \mu \vect{F_H} \right) -
\crossprod{\nabla_\perp}{\vect{z_0}} g \\
\left. \crossprod{\nabla_\perp 
\triple{\nabla_\perp}{\vect{z_0}}{\vect{E_\perp}}
}{\vect{z_0}} = \mu_0 \partder{}{t} \left( \mu \vect{F_H} \right) +
\crossprod{\nabla_\perp}{\vect{z_0}} I_z \right| \times \vect{z_0}
\end{cases}
\end{aligned} \end{equation*} }
%
\textcolor{blue}{ \begin{equation*} \begin{aligned}
\crossprod{ \crossprod
{\nabla_\perp \triple{\nabla_\perp}{\vect{z_0}}{\vect{E_\perp}}}
{\vect{z_0}} }{ \vect{z_0} } = \crossprod{ \crossprod{\nabla_\perp \phi}
{\vect{z_0}} }{\vect{z_0}} = \\ = - \crossprod{ \vect{z_0} }{ 
\crossprod{\nabla_\perp \phi}{\vect{z_0}} } = - \dotprod{\nabla_\perp \phi}
{ \dotprod{\vect{z_0}}{\vect{z_0}} } + \dotprod{\vect{z_0}}
{ \dotprod{\vect{z_0}}{\nabla_\perp \phi} } = \\ = - \nabla_\perp \phi = 
- \nabla_\perp 
\dotprod{\crossprod{\nabla_\perp}{\vect{z_0}}}{\vect{E_\perp}} = 
\dotprod{\nabla_\perp \crossprod{\vect{z_0}}{\nabla_\perp}}{\vect{E_\perp}}
\end{aligned} \end{equation*} }
%
\textcolor{blue}{ \begin{equation*} \begin{aligned}
\crossprod {\crossprod{\nabla_\perp}{\vect{z_0}} I_z}{\vect{z_0}} = 
- \crossprod {\vect{z_0}}{\crossprod{\nabla_\perp}{\vect{z_0}} I_z} = \\
= - \dotprod{{\nabla_\perp}}{\dotprod{\vect{z_0}}{\vect{z_0}}} I_z + 
\dotprod{\vect{z_0}}{\dotprod{\vect{z_0}}{{\nabla_\perp}}} I_z = 
- \nabla_\perp I_z 
\end{aligned} \end{equation*} }
%
\textcolor{blue}{ \begin{equation*} \begin{aligned}
\begin{cases} 
\dotprod{\crossprod{\vect{z_0}}{\nabla_\perp} \nabla_\perp} {\vect{H_\perp}} = 
\mu^{-1} \partder{}{z} \left( \mu \vect{F_H} \right) +
\left( \mu_0 \mu \right)^{-1} \crossprod{\vect{z_0}}{\nabla_\perp} g \\
\dotprod{\nabla_\perp \crossprod{\vect{z_0}}{\nabla_\perp}}{\vect{E_\perp}}
= - \mu_0 \partder{}{t} \left( \mu \crossprod{\vect{z_0}}{\vect{F_H}} \right) -
\nabla_\perp I_z 
\end{cases}
\end{aligned} \end{equation*} }
%
\begin{equation} \label{eq:wboHinit}
\left( \begin{array}{c} 
\dotprod{\crossprod{\vect{z_0}}{\nabla_\perp} \nabla_\perp} {\vect{H_\perp}} \\
\dotprod{\nabla_\perp \crossprod{\vect{z_0}}{\nabla_\perp}}{\vect{E_\perp}} \\
\end{array} \right) = \left( \begin{array}{c} 
\frac{1}{\mu} \partder{}{z} \left( \mu \vect{F_H} \right) +
\frac{1}{\mu_0 \mu} \crossprod{\vect{z_0}}{\nabla_\perp} g \\
- \mu_0 \partder{}{t} \left( \mu \crossprod{\vect{z_0}}{\vect{F_H}} \right) -
\nabla_\perp I_z 
\end{array} \right),
\end{equation}
%
де
%
\begin{equation} \label{eq:F_H}
\vect{F_H} = \epsilon_0 \partder{}{t} \left( \epsilon \vect{E_\perp} \right) - 
\partder{}{z} \crossprod{\vect{z_0}}{\vect{H_\perp}} + \vect{J_\perp}.
\end{equation}

Переозначення $ \vect{F_H} $ не несе фізичного змісту, а введено лише для 
спрощення виду формул. Аналогічним чином виключимо компоненту $ E_z $. 
Отримане векторне рівняння матиме наступний вид.
%
\begin{equation} \label{eq:wboEinit}
\left( \begin{array}{c} 
\dotprod{\nabla_\perp \crossprod{\vect{z_0}}{\nabla_\perp}} {\vect{H_\perp}} \\
\dotprod{\crossprod{\vect{z_0}}{\nabla_\perp} \nabla_\perp}{\vect{E_\perp}} \\
\end{array} \right) = \left( \begin{array}{c} 
- \epsilon_0 \partder{}{t} \left( \epsilon \crossprod{\vect{F_E}}{\vect{z_0}} 
\right) - \nabla_\perp J_z \\
\frac{1}{\epsilon} \partder{}{z} \left( \epsilon \vect{F_E} \right) +
\frac{1}{\epsilon_0 \epsilon} \crossprod{\nabla_\perp \varrho}{\vect{z_0}}
\end{array} \right),
\end{equation}
%
де
%
\begin{equation} \label{eq:F_E}
\vect{F_E} = \mu_0 \partder{}{t} \left( \mu  \vect{H_\perp} \right) +
\partder{}{z} \crossprod{\vect{z_0}}{\vect{E_\perp}} + \vect{I_\perp}.
\end{equation}

Перейшовши до чотиривимірних векторних величин ми знову збільшуємо розмірність 
рівнянь, а отже зменшуємо їх кількість. Тепер система рівнянь Максвела
представлена системою з двох векторних рівнянь четверного порядку відносно 
всіх компонент електромагнітного поля. 

Помічаємо, що поперечні компоненти можуть бути виділені в окрему векторну 
змінну: ліва частина рівнянь \eqref{eq:wboEinit} та \eqref{eq:wboHinit} може 
бути представлена у вигляді лінійного диференціального оператора, що діє на
чотиривимірний вектор 
$ \vect{X} = \left( \vect{E_\perp} \vect{H_\perp} \right)^\intercal $. Тоді,
ліва частина \eqref{eq:wboHinit} матиме вигляд:

\begin{equation} \label{eq:W_H}
\widehat{W}_H \vect{X} = \left( \begin{array}{cc} \widehat{0} & 
\crossprod{\vect{z_0}}{\nabla_\perp} \nabla_\perp \\
\nabla_\perp \crossprod{\vect{z_0}}{\nabla_\perp} &
\widehat{0} \\ \end{array} \right) \left( \begin{array}{c} 
\vect{E_\perp} \\ \vect{H_\perp} \\ \end{array} \right),
\end{equation}
%
де $ \widehat{0} $ - квадратна нульова матриця розмірності 2, а 
$ \widehat{W}_H $ називають крайовим оператором магнітного поля. Аналогічно,
випишемо вираз для крайового оператора електричного поля $ \widehat{W}_E $:

\begin{equation} \label{eq:W_E}
\widehat{W}_E \vect{X} = \left( \begin{array}{cc} \widehat{0} & 
\nabla_\perp \crossprod{\vect{z_0}}{\nabla_\perp} \\
\crossprod{\vect{z_0}}{\nabla_\perp} \nabla_\perp &
\widehat{0} \\ \end{array} \right) \left( \begin{array}{c} 
\vect{E_\perp} \\ \vect{H_\perp} \\ \end{array} \right).
\end{equation}

Вочевидь, оператори $ \widehat{W}_E $ і $ \widehat{W}_H $ не можуть бути 
отримані один з другого лінійними операціями, що вказує на різну фізику 
процесів, що вони описують. Можна довести 
\cite{imp:Dumin1996, imp:Tretyakov1993}, що крайові оператори поля є 
самоспряженими, тобто:

\begin{equation}
\dotprod{\widehat{W}_H \vect{Y_i}}{\vect{Y_j}} -
\dotprod{\vect{Y_i}}{\widehat{W}_H \vect{Y_j}} = 0,
\end{equation}

\begin{equation}
\dotprod{\widehat{W}_E \vect{Z_i}}{\vect{Z_j}} -
\dotprod{\vect{Z_i}}{\widehat{W}_E \vect{Z_j}} = 0,
\end{equation}
%
де $ \vect{Y} $ і $ \vect{Z} $ - деякі чотиривимірні вектори, за розмірністю 
рівні напруженості поперечних компонентів поля. Спираючись на ермітовість 
операторів $ \widehat{W}_H $ і $ \widehat{W}_E $, можна стверджувати, що 
чотиривимірні вектори $ \vect{X}, \vect{Y}, \vect{Z} $, а отже і всі поперечні 
компоненти електромагнітного поля можна представити у вигляді розкладу по 
деякому ортогональному базису. Як буде показано далі
$ \left\{ \vect{Y_i} \right\}_{i=-\infty}^\infty $ та 
$ \left\{ \vect{Z_i} \right\}_{i=-\infty}^\infty $ є ортами базису 
поперечного електромагнітного поля:

\begin{equation} \label{eq:vectY}
\func{\vect{Y_{\pm m}}}{z,t;\nu} =
\left( \begin{array}{c} 
\sqrt[-2]{\epsilon_0} 
\crossprod{\nabla_\perp \func{\Psi_m}{z,t;\nu}}{\vect{z_0}} \\ 
\sqrt[-2]{\mu_0} \nabla_\perp \func{\Psi_m}{z,t;\nu} \\ 
\end{array} \right),
\end{equation}

\begin{equation} \label{eq:vectZ}
\func{\vect{Z_{\pm n}}}{z,t;\chi} =
\left( \begin{array}{c} 
\sqrt[-2]{\epsilon_0} \nabla_\perp \func{\Phi_n}{z,t;\chi} \\ 
\sqrt[-2]{\mu_0} \crossprod{\nabla_\perp \func{\Phi_n}{z,t;\chi}}{\vect{z_0}} \\ 
\end{array} \right).
\end{equation}

По аналогії до вектору $ \vect{X} $, що складається з ортогональних 
$ \vect{E_\perp} $ та $ \vect{H_\perp} $, вектори \eqref{eq:vectY} і 
\eqref{eq:vectZ} складаються з ортогональних двовимірних компонентів 
збудованих на основі функцій $ \Psi_m $ і $ \Phi_m $, що є розв'язками
наступних рівнянь:

\begin{equation} \label{eq:equation_psi}
\left( \Delta_\perp + \sqrt{\epsilon_0 \mu_0} p_m \right) 
\func{\Psi_m}{z,t;\nu} = 0,
\end{equation}

\begin{equation} \label{eq:equation_phi}
\left( \Delta_\perp + \sqrt{\epsilon_0 \mu_0} q_n \right)
\func{\Phi_n}{z,t;\chi} = 0.
\end{equation}

В рівнянні \eqref{eq:equation_psi} $ \Psi_m $ - власна функція, а $ p_m $ 
власне число оператора $ \widehat{W}_H $ на полі векторів 
$ \left\{ \vect{Y_i} \right\}_{i=-\infty}^\infty $. Аналогічно і для 
\eqref{eq:equation_phi}, де $ \Phi_n $ - власна функція, а $ q_n $ - власне
число оператора $ \widehat{W}_E $ на вектором полі
$ \left\{ \vect{Z_i} \right\}_{i=-\infty}^\infty $. Користуючись 
властивостями функцій $ \Psi_m $ та $ \Phi_n $, а також операторів 
\eqref{eq:W_E} -- \eqref{eq:W_H}, можна довести ортогональність всіх 
$ \vect{Y_i} $ і $ \vect{Z_i} $ \cite{imp:Tretyakov1994}.

Представимо довільне поперечне електромагнітне поле $ \vect{X} $ у вигляді 
розкладу по ортогональним базисним функціям $ \vect{Y_i} $ і $ \vect{Z_i} $:

\begin{equation} \begin{aligned} \label{eq:modalYZ}
\func{\vect{X}}{z,t} = 
\sum_{m=-\infty}^\infty \int_0^\infty d \chi 
\func{A_m}{z,t;\chi} \func{\vect{Y_m}}{z,t;\chi} + \\
+ \sum_{n=-\infty}^\infty \int_0^\infty d \nu 
\func{B_n}{z,t;\nu} \func{\vect{Z_n}}{z,t;\nu},
\end{aligned} \end{equation}
%
де $ \vect{X} $ вектор, що складається з поперечних компонентів 
електромагнітного поля, $ \func{A_m}{z,t;\chi} $ і $ \func{B_n}{z,t;\nu} $ 
невідомі скарні коефіцієнти трьох змінних - простору, часу і деякого 
неперервного спектрального параметру. Помічаємо, що в загальному вигляді 
кількість невідомих коефіцієнтів - нескінченна. Користуючись властивостями 
операторів $ \widehat{W}_H $ і $ \widehat{W}_E $ можна довести, що для 
реальних задач \cite{imp:Legenkiy2010}, а тобто для обмежених в часі і в 
просторі джерел електромагнітної енергії, кількість невідомих коефіцієнтів 
обмежена дійсним числом \cite{imp:Tretyakov2004, imp:Tretyakov2010}. Таким
чином використання методу еволюційних рівнянь накладає наступну енергетичну 
умову до електромагнітних полів, що розглядаються:

\begin{equation} \label{eq:restrictXX}
\dotprod{\vect{X_1}}{\vect{X_2}} = \frac{1}{S} \int_{S} dS
\left( \epsilon_0 \vect{E_1} \vect{E_2} + \mu_0 \vect{H_1} \vect{H_2} \right),
\end{equation}
%
де $ \vect{E_i}, \vect{H_i} $ -- компоненти векторів $ \vect{X_1} $ та 
$ \vect{X_2} $ . Записавши вираз \eqref{eq:restrictXX} в змінних 
функціонального простору $ L^3 $ над дійсним полем чисел $ \R $ в 
циліндричних координатах, для якого було записано систему рівнянь Максвела 
\eqref{eq:AmpereLow} -- \eqref{eq:GaussMagnetic}, маємо наступну умову:

\begin{equation} \label{eq:energy_restrict}
\int_0^\infty dt \int_0^\infty dz 
\int_0^{2\pi} d\varphi \int_0^\infty \rho d \rho
\left(  \vect{E}^2 + \vect{H}^2 \right) < \infty.
\end{equation}

Рівність \eqref{eq:modalYZ} зводить розв'язання системи неоднорідних 
диференціальний рівнянь другого порядку в часткових похідних  
відносно пов'язаних між собою векторних функцій до пошуку кінцевої кількості 
незалежних скалярних вагових функцій. Перед тим, як знаходити еволюційні 
коефіцієнти перепишемо розклад \eqref{eq:modalYZ} так, щоб ліва 
частина рівності містила окремі компоненти електромагнітного поля. Для цього 
скористаємось відомим розв'язком коливальних рівнянь \eqref{eq:equation_psi} 
і \eqref{eq:equation_phi} в явному вигляді \textcolor{red}{[Посилання]}:

\begin{equation}
\func{\Psi_m}{\rho, \varphi; \nu} = 
\frac{J_m (\nu \rho)}{\sqrt{\nu}} e^{im\varphi},
\end{equation}

\begin{equation}
\func{\Phi_n}{\rho, \varphi; \chi} = 
\frac{J_n (\chi \rho)}{\sqrt{\chi}} e^{in\varphi},
\end{equation}
%
де $ \func{J_i}{z} $ -- циліндрична функція Бесселя. \textcolor{red}{TODO: 
описати фізичний зміст функції $ \func{\Psi_m}{\rho, \varphi; \nu} $; чому у 
розв'язку рівнянь нема власного числа $ p_m $?}.

Тепер, користуючись наступним перевизначенням коефіцієнтів розкладу 
поперечного електромагнітного поля:

\begin{equation}
A_m + A_{-m} = V_m^h,
\end{equation}

\begin{equation}
A_m - A_{-m} = I_m^h,
\end{equation}

\begin{equation}
B_n + B_{-n} = V_n^e,
\end{equation}

\begin{equation}
B_n - B_{-n} = I_n^e,
\end{equation}
%
підставимо вирази в явному виді для $ \Psi_m $ і $ \Phi_n $ в базисні функції 
модового розкладу \eqref{eq:modalYZ} і отримаємо наступний розклад поперечних 
компонент векторів інтенсивності електромагнітного поля:

\begin{equation}
\vect{H_\perp} = \frac{1}{\sqrt{\mu_0}} \left( 
\sum \limits_{m=-\infty}^{\infty} \int \limits_{0}^{\infty} d \nu
I_m^h \nabla_\perp \Psi_m + \sum \limits_{n=-\infty}^{\infty}
\int \limits_{0}^{\infty} d \chi I_n^e 
\crossprod{\vect{z_0}}{\nabla_\perp \Phi_n} \right),
\end{equation}

\begin{equation} 
\vect{E_\perp} = \frac{1}{\sqrt{\epsilon_0}} \left( 
\sum \limits_{m=-\infty}^{\infty} \int \limits_{0}^{\infty} 
d \nu V_m^h \crossprod{ \nabla_\perp \Psi_m }{ \vect{z_0} } +
\sum \limits_{n=-\infty}^{\infty} \int \limits_{0}^{\infty}
d \chi V_n^e \nabla_\perp \Phi_n \right).
\end{equation}

Розкладання поздовжніх компонентів поля по отриманому базису можна здійснити
користуючись \eqref{eq:divEt} та \eqref{eq:divHt}:

\begin{equation} 
H_z (r,t) = \frac{1}{\sqrt{\mu_0}} \sum \limits_{m=-\infty}^\infty
\int \limits_0^\infty \nu^2 d \nu h_m \Psi_m,
\end{equation}

\begin{equation} 
E_z (r,t) = \frac{1}{\sqrt{\epsilon_0}} \sum \limits_{n=-\infty}^\infty
\int \limits_0^\infty \chi^2 d \chi e_n \Phi_n.
\end{equation}

Скалярні функції $ e_n, h_m, V_m^h, V_n^e, I_m^h, I_n^e  $ називають 
еволюційними коефіцієнтами, а систему рівнянь що їх визначає еволюційною:

\begin{equation} \label{eq:evo1}
\partial_z (\mu h_m) = \mu I_m^h + \frac{\sqrt[-2]{\mu_0}}{2 \pi}
\int_0^{2\pi} d \varphi \int_0^{\infty} \rho d \rho
\Psi_m^* (\nu) g;
\end{equation}

\begin{equation} \label{eq:evo2}
\partial_{ct} (\mu h_m) = - V_m^h - \frac{\sqrt{\epsilon_0}}{2 \pi}
\int_0^{2\pi} d \varphi \int_0^{\infty} \rho d \rho
\Psi_m^* (\nu) I_z;
\end{equation}

\begin{equation} \label{eq:evo3}
- \partial_{ct} (\epsilon V_m^h) - \partial_z I_m^h + \nu^2 h_m = 
\frac{\sqrt{\mu_0}}{2 \pi} \int_0^{2\pi} d \varphi 
\int_0^{\infty} \rho d \rho \crossprod{\vect{z_0}}{\vect{J_\perp}}
\nabla_\perp \Psi_m^* (\nu);
\end{equation}

\begin{equation} \label{eq:evo4}
\partial_{ct} (\epsilon e_n) = - I_n^e - 
\frac{\sqrt{\mu_0}}{2 \pi} \int_0^{2\pi} d \varphi 
\int_0^{\infty} \rho d \rho \Phi_n^* (\chi) J_z;
\end{equation}

\begin{equation} \label{eq:evo5}
\partial_{z} (\epsilon e_n) = \epsilon V_n^e + 
\frac{\sqrt[-2]{\epsilon_0}}{2 \pi} \int_0^{2\pi} d \varphi 
\int_0^{\infty} \rho d \rho \Phi_n^* (\chi) \varrho;
\end{equation}

\begin{equation} \label{eq:evo6}
- \partial_{ct}(\mu I_n^e) - \partial_z V_n^e + \chi^2 e_n = 
\frac{\sqrt{\epsilon_0}}{2 \pi} \int_0^{2\pi} d \varphi 
\int_0^{\infty} \rho d \rho \crossprod{\vect{I_\perp}}{\vect{z_0}}
\nabla_\perp \Phi_n^* (\chi).
\end{equation}

Рівняння \eqref{eq:evo1} -- \eqref{eq:evo6} формують так звану систему 
еволюційних рівнянь, але також інколи використають перевантажену систему 
рівнянь, доповнену наступними співвідношеннями еволюційних коефіцієнтів:

\begin{equation}
\partial_{ct} (\epsilon V_n^h) + \partial_z I_n^h = 
- \frac{\sqrt{\mu_0}}{2 \pi} \int_0^{2\pi} d \varphi 
\int_0^{\infty} \rho d \rho 
\dotprod {\vect{J_\perp}} {\nabla_\perp \Phi_n^* (\chi)};
\end{equation}

\begin{equation}
\partial_{ct}(\mu I_m^e) + \partial_z V_m^e = - 
\frac{\sqrt{\epsilon_0}}{2 \pi} \int_0^{2\pi} d \varphi 
\int_0^{\infty} \rho d \rho 
% \dotprod {\vect{I_\perp}} {\nabla_\perp \Phi_m^* (\nu)}
\dotprod {\crossprod {\vect{I_\perp}} {\vect{z_0}} } 
{ \nabla_\perp \Phi_m^* (\nu) }.
\end{equation}

%%%%%%%%%%%%%%%%%%%%%%%%%%%%%%%%%%%%%%%%%%%%%%%%%%%%%%%%%%%%%%%%%%%%%%%%%%%%%%
\section{Ефект електромагнітного снаряду}

При дослідженні імпульсного поля пласких розподілів струму було 
помічено сповільнене згасання енергії поля з відстанню \cite{imp:Wu1989}. 
Дослідження цього явища показало, що концентрація енергії вздовж осі 
випромінювання спостерігається для пласких розподілів струму довільної форми 
\cite{imp:Wu1985}.

Ефект «електромагнітного снаряду» був відкритий у задачі випромінювання 
плаского диску з рівномірним розподілом нестаціонарного електричного струму 
\cite{imp:Wu1985}. Він спостерігається в прожекторній зоні випромінювача і 
полягає у тому, що енергія поля згасає повільніше ніж зворотно пропорційно 
квадрату відстані \cite{imp:Sodin1992-10}, за рахунок її концентрації в 
одному напрямку. Для відтворення цього явища в лабораторних умовах 
використовували діелектричні лінзи опромінені TEM хвилеводом \cite{imp:Wu1991}.

Імпульсний режим роботи радіотехнічних приладів дозволяє локалізувати енергію 
електромагнітного поля в часі та просторі у вигляді електромагнітного 
імпульсу, що може викликати необхідність врахування нелінійної природи 
електромагнітних хвиль. Через ефект електромагнітного снаряду, що збільшує 
концентрацію енергії імпульсного поля, врахування нелінійних ефектів стає 
ще актуальнішим.

Математичний опис ефекту електромагнітного снаряду був незалежно отриманий 
в різних дослідженнях, як консервативними методами частотної області
\cite{imp:Wu1987, imp:Samsonov1986}, так і методом еволюційних рівнянь у 
часовій області \cite{imp:Dumin1996}. Також ця задача розв'язувалась 
чисельно \textcolor{red}{[Посилання Бутрим]}. В наведених роботах присутні
аналітичні вирази для інтенсивності та енергії в окремих інтервалах 
простору-часу, як наприклад на осі випромінювання або під деяким кутом до 
неї, але аналітичний та зручний для аналізу розв'язок, що справедливий для 
довільної точки спостереження знайдено не було.

%%%%%%%%%%%%%%%%%%%%%%%%%%%%%%%%%%%%%%%%%%%%%%%%%%%%%%%%%%%%%%%%%%%%%%%%%%%%%%
\section{Методи виділення інформації з надширокосмугових та імпульсних сигналів}

Радіо-телекомунікаційна система -- засіб бездротової передачі інформації, що 
має три обов'язкові складові: передавач, приймач і середовище поширення 
хвилі, що справедливо як для резонансних, так і для широкосмугових систем. 
Передавач формує електромагнітні хвилі в заданому напрямку,  що несуть корисну 
інформацію. Середовище виступає в якості носія інформації. Найпростішою моделлю
такого середовища найчастіше використовують канал адитивного білого гаусового 
шуму (AWGN) \cite{imp:Lazorenko2009}. Приймач декодує отриману з заданого 
напрямку хвилю в інформацію. За сучасними потребами людства, найчастіше, 
декодована інформація передасться у вигляді маловольтажного цифрового коду. 
Приймально-передавальна техніка в задачах телекомунікації, за окремим 
виключенням, поєднується в один технічний пристрій, як, наприклад, 
короткохвильові та імпульсні радіостанції. Приймальні пристрої для 
радіолокаційних задач принципово не вирізняються від телекомунікаційних 
систем засобами виділення корисної інформації із хвилі.

% TODO: ілюстрація передавач - середовище - приймач

Першим етапом виділення корисної інформації з електромагнітної хвилі є 
перетворення хвилі у вільному просторі в електричний струм у дроті. В техніці, 
що базується на фізичних принципах резонансу для цього використовують 
детекторний діод, однак для більшості задач надширокосмугових пристроїв 
такий підхід недопустимий і використовують масштабно-часове перетворення
\cite{imp:Lazorenko2009}. Таке перетворення зберігає форму імпульсу, але 
допускає його масштабування, тобто функція $ f(t) $ перетворюється на 
$ r f(qt) $, де $ r $ -- деякий нормувальний коефіцієнт, який, здебільшого, 
визначається внутрішнім опором приймальної антени, а $ q $ -- коефіцієнт 
масштабування. При $ q < 1 $ тривалість сигналу менша за тривалість імпульсу, 
а при $ q > 1 $, що зустрічається частіше, -- більша. Отриманий струм 
$ r f(qt) $ називають розгорткою сигналу.

Наступним етапом є декодування отриманої розгортки, тобто виділення корисної 
інформації зі струму. Розглянемо доступні методи та моделі розв'язання цієї 
задачі.

\begin{figure}[htbp] \begin{center}
\includegraphics[scale=0.6]{method_model}
\caption{Класифікація методів та моделей аналізу часових послідовностей} 
\label{fig:method_model}
\end{center} \end{figure}

На рис.~\ref{fig:method_model} зображено класифікацію методів та моделей 
\cite{imp:Chuchueva2012}, де модель -- функціональне представлення, що з 
достатньою точністю описує досліджуваний процес, а метод являє собою 
послідовність дій які необхідно виконати для побудови моделі. Евристичні 
(інтуїтивні) методи аналізу базуються на судженнях і оцінках експертів, 
та використовуються в таких дисциплінах, як маркетинг, філософія, економіка 
і навряд підходять для виділення інформації з аналогового 
сигналу. Формалізовані методи більш придатні для розв'язання цієї задачі,
через те, що дозволяють побудувати математичну залежіть (модель), що 
пов'язує аналоговий сигнал та інформацію, що він несе і робить це з 
заданою точністю. Доступні моделі розділяють на статистичні та структурні 
\cite{imp:Chuchueva2012}. 

Моделі часових послідовностей базуються на методах, що шукають залежності
в середині самого нестаціонарного процесу і на цій основі виконують 
необхідний аналіз. Моделі часових послідовностей поділяються на статистичні
(регресія, EM-алгоритми, експоненційне стискання, еволюційні алгоритми, тощо)
та структурні (машинне навчання, ланцюги Маркова, класифікаційні дерева, 
узагальнене перетворення Хафа, каскади Хара, watershed, тощо). Статистичні 
моделі не придатні для застосування в розв'язанні задач виділення корисної 
інформації через їх алгоритмічну обчислювальну складність, натомість, 
константа складність деяких структурних алгоритмів робіть їх досить 
привабливими. Моделлю предметної області (радіофізики), у випадку виділення 
корисної інформації з сигналу, є послідовне застосування лінійного фільтра, 
низькошумного підсилювача, аналогово-цифрового перетворювача (АЦП) і модуля 
цифрової обробки FPGА, що переводить розв'язання в область структурної моделі
аналізу часової послідовності. В радіофізиці, такі методи застосовують як 
для задач локації \cite{imp:Dumin2017}, так і комунікації \cite{imp:Taok2009}.

Структурні моделі аналізу часових послідовностей, можна розділити за типом
інформації, що виділяється зі вхідних даних:

\begin{enumerate}

	\item Класифікація часової послідовності (Sequence classificaion). 
	Визначення приналежності сигналу, як цілого, до певного виду (класу) 
	з наперед відомими характеристиками. Активно застосовується в задачах 
	багатопроменевої і багатокористувацької надширокосмугової комунікації. 
	Прикладом найпростішого бінарного класифікатору є визначення наявності 
	сигналу за граничним значенням (threshold value), тобто на проміжках 
	спостереження, де значення досліджуваної функції більше за наперед 
	визначене значення, бінарний класифікатор вказує True, а в інших 
	випадках False.

	\item Маркування послідовності (Sequence labeling). Визначення 
	приналежності сигналу в кожен момент часу до певного виду з наперед 
	відомими характеристиками. Активно застосовується в задачах 
	багатопроменевої і багатокористувацької надширокосмугової комунікації.

	\item Пошук аномалей (Anomaly detection). Пошук проміжку часової 
	послідовності, який вибивається з загального вигляду даних та має 
	невизначену природу. Корисний для задач, де метод максимальної 
	правдоподібності застосувати неможливо через невизначеність 
	характеристик аномалії, що шукається, наприклад задачі автоматичного 
	визначення наявності випадкових завад невідомої природи.

	\item Передбачення послідовності (Sequence forcasting). Передбачення 
	значень часової послідовності засновуватись на минулих значеннях цієї 
	послідовності. В радіофізиці активно використовуються, як моделі з 
	учителем так і без.

	\item Генерація послідовності (Sequence generation). Генерація нової 
	часової послідовності на основі наявної з урахуванням зовнішніх факторів. 
	Генерація нової послідовності може проходити, як в порядку продовження 
	або доповнення існуючої, так і в порядку створення окремої послідовності.

	\item Фільтрування послідовності (Sequence filtering). Фільтрація 
	часової послідовності від завад або сторонніх сигналів, що вона 
	містить. Спосіб визначення завад та їх ліквідація залежить від 
	імплементації конкретної моделі. Найрозповсюдженіше сімейство моделей
	фільтрації часової послідовності в радіофізиці -- лінійна фільтрація, 
	наприклад: частотні фільтри, фільтр Калмана та узгоджений фільтр.

\end{enumerate}

З іншого боку, як в здачах локації, інколи застосовують методи обробки 
зображень, формуючи карти інтенсивності з отриманого сигналу. Тобто,
задача аналізу часових послідовностей розглядається в контексті 
моделей комп'ютерного зору. Через обчислювальну складність, такі методи 
застосовують лише в задачах де є необхідність накопичувати дані і паралельно 
оброблювати вже готовий пакет даних. Таким чином, до структурних моделей 
аналізу часових послідовностей можна віднести, також, і деякі моделі 
аналізу зображень:

\begin{enumerate}

	\item Класифікація зображень (Image classificaion). Визначення 
	приналежності зображення, як цілого, до певного виду (класу) з наперед 
	відомими характеристиками.

	\item Детектування об'єктів (Object detetion). Визначення положень 
	обмежувальних прямокутників для об'єктів з наперед відомими 
	характеристиками, що містяться на зображені, яке досліджується.

	\item Сегментація об'єктів (Instance segmentation). Процес розділення 
	цифрового зображення на декілька сегментів довільної форми. Під сегментом 
	мається на увазі множина пікселів, які часто називають суперпікселями. 
	Така модель сегментації не розділяє об'єкти одного типу між собою.

	\item Семантична сегментація (Semantic segmentation). Процес розділення 
	цифрового зображення на декілька сегментів довільної форми, при тому, що
	форми об'єктів одного типу семантично розділені. Найрозповсюдженішим 
	підходом до семантичного розділення об'єктів є бінарні піксельні маски. 
	Типова модель для розв'язку таких задач -- штучна згорткова нейронна 
	мережа-автоенкодер UNet.

	\item Фільтрація зображення (Image filtering). Фільтрація зображення 
	від сторонніх шумів. Зручно застосовувати для отримання обробки 
	спектральних характеристик реальних сигналів 
	\textcolor{red}{[Коленов Дима]}.

\end{enumerate}

Для вибору класу моделі виділення інформації з сигналу, а також для 
дослідження даних і імплементації моделі прийнято застосовувати  
методологію аналізу даних CRISP-DM \cite{imp:CRISPDM2000}.

\section{Структурні моделі аналізу процесів}

Фізичний процес -- змінна в часі величина (інколи векторна). На практиці
опис процесу найчастіше являє собою безперервну часову послідовність, 
що задано математичною функцією, або дискретну часову послідовність, задану 
масивом значень змінної в часі величини. Структурною моделлю аналізу процесу
називатимемо формалізовану модель, що виділяє з нього корисну інформацію 
засновуючись на закономірностях в значеннях досліджуваної величини у 
плині часу.

Враховуючи шум в моделі інформаційного радіо-каналу \cite{imp:Shihovcev2011}, 
досліджуваний сигнал стає випадковим процесом, що ускладнює виділення 
корисної інформації для реальних задач. Джерелом шумів можуть бути як і 
зовнішні фактори (невідомі сторонні джерела електромагнітного поля) так і
саме обладнання прийму-передачі. Через це ускладнення бенчмарк-модель 
виділення інформації, що базується на виявлені наявності сигналу за 
пороговим значенням напруженості струму стає малопридатною. Крім того,
додатковим фактором, що впливає на точність виділення інформації є розбіжності
протікання реального фізичного процесу передачі-поширення-прийму 
електромагнітного поля та моделі, що описує цей процес. Прикладом такої 
розбіжності може стати застосування спрощень, що не враховують нелінійну 
природу електромагнітного поля або ефекти ближньої зони випромінювання.
В таких випадках інколи застосовують методи з учителем, або калібрувальними 
параметрами.

Таким чином, основними показниками вибору моделі стають обчислювальна 
складність її алгоритмів та точність самого методу. Для порівняння методів,
а також для визначення їх абсолютної та відносної точності користаються 
широким класом метрик, але частіше доводиться визначати власні. 
Найпростішими метриками, що можуть застосовуватись в широкому класі задач є 
середня абсолютна помилка, а також середня квадратичні помилки 
\cite{imp:Willmott2005}, але через їх недоліки часто приходять до 
використання більш складних методів оцінки точності. Недоліки таких метрик 
часто допомагає вирішити логарифмізація значення середньої помилки і деякі 
регуляризації отриманої функції. Прикладом метрики в задачах класифікації 
можна навести F-метрики \cite{imp:Tharwat2018}, коли для задач сегментації та 
індикації використовують Коэффициент Жаккара (Intercection Over Union або 
IoU) \cite{imp:Jaccard1901}.

% декілька методів одразу

Набір моделей, які можна застосувати до часових послідовностей, значно 
розширюється, якщо застосовувати методику ковзного вікна, тобто при аналізі 
безперервної часової послідовності розглядати одномоментно деякий проміжок 
дискретних значень, що рухається в часі. Наприклад, на вхід методу лінійної 
регресії \cite{imp:Xin2009} можна подати вектор, що складається з дискретних 
послівних значень взятих з деяким інтервалом з аналогового сигналу. Якщо 
інтервал і тривалість вікна будуть вибрані правильно, то модель лінійної
регресії можливо застосувати, щоб виявити наявність імпульсу. 

З приходом ери нейронних мереж \cite{imp:Rosenblatt1957}, їх застосування в
задачах дискримінантного аналізу витісняє інші методики. Так, для більш 
складних задач, таких як класифікація імпульсу, методика ковзного вікна 
дозволяє застосовувати повнозв'язані і згорткові нейронні мережі 
\cite{imp:Plakhtii2019}.

\begin{figure}[htbp] \begin{center}
\includegraphics[scale=0.6]{perceptron}
\caption{Математична модель біологічного нейрону}
\label{fig:perceptron}
\end{center} \end{figure}

На рис.~\ref{fig:perceptron} зображено одну з математичних моделей (модель 
Розенблатта) біологічного нейрону. Вона виділяє дві функції нейрону: суматору 
і активаційну. Біологічній нейрон розглядають, як нелінійний пристрій з 
декількома вхідними інтерфейсами (дендритами) та єдиним вихідним (аксон).
Також сам нейрон характеризується своїм внутрішнім параметрам -- зміщення 
$ b $, який відповідає за порогове значення реагування нейрону. Суматора 
функція $ S $ в моделі нейрона описує механізм акумулювання довільної 
кількості ($ n $) вхідних сигналів, поєднаних в вектор $ \vect{x} $ 
розмірністю $ n + 1 $, де останній елемент вектору завжди $ 1 $:

\begin{equation}
\func{S}{\vect{x}} = \vect{w}^\intercal \vect{x},
\end{equation}
%
де $ \vect{w}^\intercal = \left( w_1, w_2, ..., w_n, b \right) $, а
$ \vect{x} = \left( x_1, x_2, ..., x_n, 1 \right)^\intercal $
Активаційна функція $ \func{A}{\func{S}{\vect{x}}} $ описує механізм 
реагування на акумульовані вхідні сигнали. Для класичного перцептрону 
активаційною є функція Хевісайда. Також часто застосовують сигмоїдальні 
активаційні функції та функції ReLu \cite{imp:Kussul2004}.

Разом з розповсюдженням аналогової радіоелектроніки почали заявлятись 
структурні моделі пристосовані саме для аналізу аналогових сигналів 
та часових послідовностей \cite{imp:Markov1906}. Проривом стала можливість
зберігати минулі значення послідовності у якості стану системи 
(ланцюга Маркова) і на основі цього стану проводити аналіз поточного 
значення випадкового процесу. Такий підхід дозволив скоротити кількість 
параметрів моделі у порівнянні з лінійною регресією. У порівнянні з 
багатошаровим перцептроном модель, ланцюгів Маркова показувала низьку 
запам'ятовувальну здатність, а тому в складних задачах розрізнення великої 
кількості класів показувала гірші результати.

Лише поява рекурентних нейронних мереж у 1988 році \cite{imp:Rumelhart1988} 
дозволила застосовувати на практиці моделі, що зберігають інформацію про 
випадковий процес в якості свого внутрішнього стану і не потребують 
застосування методу ковзного вікна. Цей метод та подальші його модифікації
стали перспективним напрямком розвитку структурних моделей аналізу процесів. 

\section{Рекурентні нейронні мережі короткотривалої та довготривалої пам'яті}

Рекурентна нейронна мережа може бути представлена, як ланцюг, що складається
з ланок (штучних нейронів), які відрізняються лише своїм внутрішнім станом 
(рис.~\ref{fig:rnn_unrolled}). Ця модель споріднена типовому радіофізичному 
терміну -- зворотній зв'язківок.

\begin{figure}[htbp] \begin{center}
\includegraphics[scale=0.4]{rnn_unrolled}
\caption{Проходження сигналу крізь рекурентну нейронну мережу}
\label{fig:rnn_unrolled}
\end{center} \end{figure}

На рис.~\ref{fig:rnn_unrolled} літерою А позначено ланку ланцюга, $ x_i $ --
дискретне значення вхідної часової послідовності (вхід), а $ h_t $ -- якісна 
або кількісна характеристика виділена нейронною мережею з сигналу (вихід). 
На відміну від класичного перцептрону Розенблатта, ланки рекурентного ланцюга 
мають додатковий вихід, що передає свій внутрішній стан на наступну ланку.

Входом для такої математичної моделі є дискредитована часова послідовність 
довільної тривалості, що властиво для радіофізичних задач виділення інформації
з сигналу. Виходом також є часова послідовність, але не завжди з таким же 
періодом дискретизації. 

Випадок, коли частота дискретизації входу та виходу співпадає, тобто кожному 
елементу вхідної послідовності $ x_i $ підставляється окремий елемент 
вихідної $ h_i $ є найчастіше вживаним в радіофізиці. Тут кожен дискретний 
вихід моделі $ h_i $ базується на проміжку даних 
$ \left[ x_{i-j} , x_i \right] $, де $ j $ -- довільна 
кількість врахованих елементів входу, яка обмежена максимальною 
запам'ятовувальною знатністю моделі. Зазвичай, максимальна 
запам'ятовувальна знатність є гіперпараметром тренувального алгоритму 
рекурентрої мережі. Моделі такого типу відомі в закордонній літературі, як 
many-to-many.

Також зустрічаються моделі для яких частота дискретизації входу та виходу 
не співпадає. В моделях де одному вхідному дискретному значенню 
співставляєтья деякий проміжок вихідної послідовності називають one-to-many.
Прикладом one-to-many задачі є підвищення якості звуку. Задачі, в яких 
навпаки, деякому проміжку вхідних значень $ \left[ x_{i-N} , x_i \right] $, 
де $ N $ -- стала константа, співставляється одне вихідне значення $ h_i $
називають many-to-one. На відміну від many-to-many задач, тут 
дискретне вихідне значення базується на проміжку вхідних даних сталої 
тривалості, що накладає деякі обмеження на застосування моделі. Такий 
підхід еквівалентний застосуванню методики ковзного вікна.

Підхід із застосуванням медики ковзного вікна має суттєвий недолік: 
тривалість вікна залишається сталою, коли тривалість сигналу залежить від 
багатьох факторів, а отже вибирати тривалість вікна треба з огляду на 
максимально можливу тривалість імпульсу. Це призводить до того, що для 
більш коротких імпульсів модель може втратити точність. В задачах локації, 
де корисна інформація знаходиться в післяімпульсних коливаннях, тривалість 
яких взагалі не обмежена, застосування методики ковзного вікна принципово 
неможливе без обмеження на глибину зондування.

Сьогодні, прижились дві реалізації ланок рекурентних мереж, які застосовують,
як для many-to-one так і для many-to-many -- це LSTM та GRU, які дуже схожі 
за своїми можливостями. Класичні RNN без механізму забування вважаються менш 
ефективними через низьку максимальну запам'ятовувальну знатність.

\begin{figure}[htbp] \begin{center}
\includegraphics[scale=0.5]{lstm_inside}
\caption{Внутрішня структура ланки ланцюга LSTM}
\label{fig:lstm_inside}
\end{center} \end{figure}

На рис.~\ref{fig:lstm_inside} з \cite{imp:Varsamopoulos2018} зображено 
внутрішню будову ланки рекурентної мережі довготривалої та короткотривалої 
пам'яті. Символом множення на рисунку позначена операція множення чисел, а 
плюсом -- додавання. Також, жовтими прямокутниками позначено функції, що діють 
на число: $ \tanh $ -- гіперболічний тангенс та $ \sigma $ -- сигмоїда.

Передача внутрішнього стану $ C_t $ на наступну ланку здійснюється по 
верхній горизонтальній стрілці, де цей внутрішній стан змінюється на основі
поточного значення вхідної часової послідовності $ x_t $ та передбачення 
моделі на минулій ланці $ h_{t-1} $.

Першим етапом обробки часової послідовності є так званий поріг забування
(або forget gate). При проходженні крізь нього ми визначаємо яку 
кількість інформації про часову послідовність слід забути:

\begin{equation}
f_t = \func{\sigma}{\dotprod{W_f}{\crossprod{h_{t-1}}{x_t}} + b_f},
\end{equation}
%
де $ W_f $ і $ b_f $ -- матричні тренувальні параметри.

Наступний етап визначає на скільки потенційно важливим може стати поточне 
значення нейронної мережі та чи треба зберігати його у внутрішньому стані. 
Цей етап називають прохідний поріг (або input gate):

\begin{equation}
i_t = \func{\sigma}{\dotprod{W_i}{\crossprod{h_{t-1}}{x_t}} + b_i},
\end{equation}

\begin{equation}
\tilde{C_t} = \func{\tanh}{\dotprod{W_C}{\crossprod{h_{t-1}}{x_t}} + b_C},
\end{equation}
%
де $ W_C, b_C $ -- матричні тренувальні параметри.

Користуючись отриманими значеннями прохідного порогу і порогу забування
можна змінити внутрішній стан моделі:

\begin{equation}
C_t = f_t C_{t-1} + i_t \tilde{C_t}.
\end{equation}

Отримавши новий внутрішній стан моделі можна отримати поточне вихідне 
значення користуючись поточним вхідним значенням послідовності, як порогом 
забування:

\begin{equation}
o_t = \func{\sigma}{\dotprod{W_o}{\crossprod{h_{t-1}}{x_t}} + b_o},
\end{equation}

\begin{equation}
h_t = o_t \func{\tanh}{C_t},
\end{equation}
%
де $ W_o, b_o $ -- матричні тренувальні параметри.

Алгоритмом зворотнього поширення помилки знаходять тренувальні 
параметри $ W_f, b_f, W_C, b_C, W_o, b_o $ так само як і для повнозв'язних 
мереж з єдиною особливістю -- при визначенні похідних враховують також
зміну внутрішнього стану $ \partder{C_t}{x_t} $.

\chapter{Перехідна функція антени з круговою апертурою}
\label{ch:linear}

%%%%%%%%%%%%%%%%%%%%%%%%%%%%%%%%%%%%%%%%%%%%%%%%%%%%%%%%%%%%%%%%%%%%%%%%%%%%%%%
\section{Наближення кругової апертури}

На початку 70х років інтерес до імпульсної радіофізики був збуджений
мілітарним застосуванням переваг нанширокосмугових радарних та 
телекомунікаційних систем, як в Україні \cite{imp:Dumin1996} так 
і за кордоном \cite{imp:BaumIN0105}. Соьгодні, наробітки минулого 
століття знайшли своє застосування у системах інтернету речей, як 
наприклад Apple SoC U1 % \cite{}.

Широким класом технічних рішень для напрямленого випромінювання 
надширокосмугового електричного струму є антени імпульсного віпромінювання.
Серех них можна виділити чотири класи за способом вирівнювання фронту хвиль:
%
\begin{enumerate}
	\item Без вирівнювання сферичного фронту
	\item Лінзеві сповільнювачі
	\item Рефлекторні антени
	\item Комбіновані архітектури \cite{imp:BaumSSN0379}
\end{enumerate}

Живлення такого класу антен зазвичай виконується ТЕМ рупором, що підєднується 
до коаксіального кабеля через балун. Кожен з перелічиних типів антен має свої
переваги, недоліки та сферу застосування. Даний розділ присвячуєтся 
дослідженню саме лінзевих антен імпульсного випрмінювання (LIRA). 

Антени типу LIRA мають численні переваги над рефлекторними. Перш за все це
більш високий ковіціент підсилення антенни \cite{imp:BaumUWBSP1}. Подруге,
лінзеві антени не мають області тіні від опромінювача та краще узгоджуються
на практиці \cite{imp:BaumSSN0377}. Також експерементальне порівняння LIRA 
та RIRA показують меншої тривалості. З недоліків варто відзначити важкість 
виготовлення лінз точної форми та вагу антени.

Розглянемо задачу збудження такої антени нестаціонарним імпульсним струмом,
деякої часової залежності $ f(t) $, з умовою існування першої та другої 
похідної. Ефективна тривалісь перехідного процессу $ f(t) $ за метрикою FWHM 
розглядається в межах від десятків піко-секунд до декількох нано-секунд.

Сферична хвиля проходе через систему діелектричних лінз розташованих у 
розкриві формуючи квазі-одномоментне збудження плаского фронту у розкриві. 
Формою розкриву зазвичай вибирають кругову апертуру. Таким чином, у першому 
наближенні, у розкриві формується рівномірно розподілений сторонній плаский 
електричний струм напрямлений від одного плеча рупора до іншого. Вперше така
апроксимація була запропонована 1985р. \cite{imp:Wu1985} та 
емпірично перепірена 1991р. \cite{imp:Wu1991}.

Серед лінзевого класу антен, що формують розподіл сторонього струму у вигяди 
плаского диску варто відзачити антену Рис.~\ref{fig:lira_baum}, що спершу
представив Ву \cite{imp:Wu1987}, а згодом і Баум \cite{imp:BaumSSN0377}. 
Лінза антени виконана у формі витягнутого сфероїда, а розкрив ТЕМ рупора 
цілком заповнено діелектриком. Таким чином мінімалізується відбиття. 
Зкруглений рупор починається в одному фокусі еліпсоїда, а закінціється в 
другому, таким чином радіус розкриву є факальним параметром еліпсоїда.
В якості матеріалу для лінзи пропонується використовувати поліетилен 
високої густини ($\epsilon = 2.3 $).

\begin{figure}[htbp] \begin{center}
\includegraphics[scale=0.5]{Baum_LIRA}
\caption{Геометрія лінзевої антени Баума та Ву} \label{fig:lira_baum}
\end{center} \end{figure}

В 1991р. Ву представив антену, що краще підходить під наближення  
плаского диску електричного струму \cite{imp:Wu1991}. Гіперболічна лінза 
забеспечує розташування площини рівних фаз в самому розкриві, що і утворює
плаский диск \ref{fig:lira_wu}. Цікавою варіацією цієї антени є заміна 
діелектричного наповнення $ \epsilon_1 $ та лінзи $ \epsilon_2 $ на матрвал,
діелектрична характеристика якого є функцією координат $ \epsilon(\rho, z) $.
Таким чином відбиття від внутрішньої поверхні лінзи зникне і характиристики 
антени покращаться. Важливим мінусом такої архітектуре стає важкість 
виготовлення лінзи.

\begin{figure}[htbp] \begin{center}
\includegraphics[scale=0.5]{Wu_LIRA}
\caption{Геометрія лінзевої антени Ву} \label{fig:lira_wu}
\end{center} \end{figure}

Прямим призначенням таких антен є телекомунікація, радіолокація і лабораторні 
вимірювання. Проте цікавість таких антен пояснюється ще і аномально повільним
згасанням енергії імпульсного поля з відстанню, що було теоретично 
передбачено \cite{imp:Wu1987}. Цей ефект відомий за назвою електромагнітний 
сняряд.

Фізична модель плаского диску описує поле LIRA лише у першому наближенні.
Така модель не враховує вихровий магнітний сторонній струм, що існує на ряду 
з пласким електричним, а також не враховує струми що течуть назад в генератор 
відбившись від краю рупора - поле такого струму залишає довгий "хвіст" після 
основного імпульу. Проте емпіричне дослідженнях 
\cite{imp:BaumSSN0396,imp:BaumSSN0401} показує, що при належному 
узгодженні, паразитний вплив відбиття фактично відсутній, а перехідна 
функція отримана експерементальним шляхом відповідає розвязку, що дає 
плаский диск.

Розв'язок задачі плаского диску шукали разними методами. Першими були 
отримані наближені розв'язки в частотній області 
\cite{imp:Wu1985,imp:Sodin1992-10}. Також, розвязок для цієї задачі знайдено 
у часовій області \cite{imp:Dumin1996}. Недоліком наявних розвязків є те,
що вони не надають часову залежніть напруженності поля в довільних точках 
спостереження в явному виді. При самодії поля крізь середовище на значення 
напруженності поля в кожній точці спостереження та в будь-який момент впливать
всі пов'язані зі спостереженням події. Таким чином наявність розв'язання в 
будьякій точці спостереження - необхідна умова.
%
%%%%%%%%%%%%%%%%%%%%%%%%%%%%%%%%%%%%%%%%%%%%%%%%%%%%%%%%%%%%%%%%%%%%%%%%%%%%%%%
\section{Розв'язання методом еволюційних рівнянь}
%
Розглянемо сторонній електричний нестаціонарний струм $ \vect{j_0} (r,t) $ 
в якості единого джерела електромагнітнго поля. Нехай, струм однонапрямлений, 
рівномірнорозподілений та має форму плаского диску нульової товщини. Для 
розв'язання прямої задачі електродинаміки для довільної часової $ f(t) $ 
залежності нестаціонарного струму $ \vect{j_0} (r,t) $, достатньо отримати
розв'язок для перехідної функції, тобто $ f(t) = H(t) $, де $ H(t) $ - 
функція Хевісайда. Тоді, математично його можна описати в циліндричних
координатах $ \rho, \varphi, z $, як
%
\begin{equation}
\vect{j_0} \left( r, t \right) = \vect{J} = \vect{x_0} A_0 H(t) \delta(z) 
\left(  H(\rho) - H(\rho - R) \right),
\end{equation}
%
де $ A_0 $ - максимальна амплітуда струму, $ R $ - радіус диску,
$ \delta(z) $ - символ Кронекера, а 
$ \vect{x_0} = \vect{\rho_0} \cos \varphi - \vect{\varphi_0} \sin \varphi $ 
- декартовий орт OX. 
%
\begin{figure}[htbp] \begin{center}
\includegraphics[scale=0.55]{PlaneDisk}
\caption{Геометрія випромінювача} \label{fig:pdisk}
\end{center} \end{figure}
%
\textcolor{lightgray} { \begin{equation*} \begin{aligned}
\begin{cases}
\vect{\rho_0} = \vect{x_0} \cos \varphi + \vect{y_0} \sin \varphi \\
\vect{\varphi_0} = - \vect{x_0} \sin \varphi + \vect{y_0} \cos \varphi
\end{cases} \Rightarrow \mathbf{A} = \left( \begin{array}{cc}
\cos \varphi & \sin \varphi \\
- \sin \varphi & \cos \varphi
\end{array} \right)
\end{aligned} \end{equation*} }
%
\textcolor{lightgray} { \begin{equation*} \begin{aligned}
\vect{j_0} \left( \vect{\rho_0}, \vect{\varphi_0} \right) = 
\mathbf{A} \vect{j_0} \left( \vect{x_0}, \vect{y_0} \right) = \\
= H(t) \delta(z) (  H(\rho) - H(\rho - R) ) 
( \vect{\rho_0} \cos \varphi - \vect{\varphi_0} \sin \varphi )
\end{aligned} \end{equation*} }
%
Для застосування методу еволюційних рівнянь знайдемо модовий розклад 
струму
%
\begin{equation} \label{eq:jm_base}
j_m \left( r, t; \nu \right) = \frac{\sqrt{\mu_0}}{2\pi} 
\int \limits_{0}^{2\pi} d \varphi \int \limits_{0}^{\infty} \rho d \rho 
\vect{j_0} \crossprod{ \nabla_\perp \Psi_m^* }{ \vect{z_0} },
\end{equation}
%
де $ \Psi_m^* $ - комплексно спряжена бізисна функція \cite{imp:Dumin2010}.
%
\textcolor{lightgray} { \begin{equation*} \begin{aligned}
\crossprod{ \nabla_\perp \Psi_m^* }{ \vect{z_0} } = 
- \sqrt{\nu} e^{-im\varphi} \left( 
\vect{\varphi_0} \frac{J_{m-1} (\nu \rho) - J_{m+1} (\nu \rho)}{2} + 
\right. \\ + \left. i m \vect{\rho_0} \frac{J_m (\nu \rho)}
{\rho \nu} \right) = - \sqrt{\nu} e^{-im\varphi} \left( 
\vect{\varphi_0} \frac{J_{m-1} (\nu \rho) - J_{m+1} (\nu \rho)}{2} + 
\right. \\ + \left. i \vect{\rho_0} \frac{J_{m-1} (\nu \rho) + 
J_{m+1} (\nu \rho)}{2} \right)
\end{aligned} \end{equation*} }
%
\textcolor{lightgray} { \begin{equation*} \begin{aligned}
\vect{j_0} \crossprod{ \nabla_\perp \Psi_m^* }{ \vect{z_0} } = 
- \sqrt{\nu} ( \cos m \varphi - i \sin m \varphi ) 
H(t) \delta(z) ( H(\rho) - H(\rho - R) ) \cdot \\ \cdot \left( 
i \frac{J_{m-1} (\nu \rho) + J_{m+1} (\nu \rho)}{2} \cos \varphi
- \frac{J_{m-1} (\nu \rho) - J_{m+1} (\nu \rho)}{2} \sin \varphi
\right)
\end{aligned} \end{equation*} }
%
\textcolor{lightgray} { \begin{equation*} \begin{aligned}
j_m = \frac{\sqrt{\mu_0}}{2\pi} \sqrt{\nu} \delta(z) H(t) \cdot \\
\cdot \Big( \int \limits_{0}^{2\pi} d \varphi \sin \varphi 
( \cos m \varphi - i \sin m \varphi) \int \limits_{0}^{R} 
\frac{J_{m-1} (\nu \rho) - J_{m+1} (\nu \rho)}{2} \rho d \rho - \\
- i \int \limits_{0}^{2\pi} d \varphi \cos \varphi 
( \cos m \varphi - i \sin m \varphi) \int \limits_{0}^{R} 
\frac{J_{m-1} (\nu \rho) + J_{m+1} (\nu \rho)}{2} \rho d \rho \Big)
\end{aligned} \end{equation*} }
%
\textcolor{lightgray} { \begin{equation*} \begin{aligned}
j_m = \frac{\sqrt{\mu_0}}{2\pi} \sqrt{\nu} \delta(z) H(t) 
i\pi ( \delta_{m,-1} - \delta_{m,1} ) \int \limits_{0}^{R} 
\frac{J_{m-1} (\nu \rho) - J_{m+1} (\nu \rho)}{2} \rho d \rho - \\
- \frac{\sqrt{\mu_0}}{2\pi} \sqrt{\nu} \delta(z) H(t) 
i\pi ( \delta_{m,-1} + \delta_{m,1} ) \int \limits_{0}^{R} 
\frac{J_{m-1} (\nu \rho) + J_{m+1} (\nu \rho)}{2} \rho d \rho =
\end{aligned} \end{equation*} }
%
\textcolor{lightgray} { \begin{equation*} \begin{aligned}
= i \frac{\sqrt{\mu_0 \nu}}{4} \delta(z) H(t)
\delta_{m,-1} \int \limits_{0}^{R} \left( J_{-2} (\nu \rho) - 
J_0 (\nu \rho) \right) \rho d \rho - \\
- i \frac{\sqrt{\mu_0 \nu}}{4} \delta(z) H(t)
\delta_{m,1} \int \limits_{0}^{R} \left( J_{0} (\nu \rho) - 
J_2 (\nu \rho) \right) \rho d \rho - \\
- i \frac{\sqrt{\mu_0 \nu}}{4} \delta(z) H(t)
\delta_{m,-1} \int \limits_{0}^{R} \left( J_{-2} (\nu \rho) +  
J_0 (\nu \rho) \right) \rho d \rho - \\
- i \frac{\sqrt{\mu_0 \nu}}{4} \delta(z) H(t)
\delta_{m,1} \int \limits_{0}^{R} \left( J_{0} (\nu \rho) +
J_2 (\nu \rho) \right) \rho d \rho =
\end{aligned} \end{equation*} }
%
\textcolor{lightgray} { \begin{equation*} \begin{aligned}
= - i \frac{\sqrt{\mu_0 \nu}}{2} \delta(z) H(t) 
(\delta_{m,1} + \delta_{m,-1}) 
\int \limits_{0}^{R} \left( J_{0} (\nu \rho) + 
J_2 (\nu \rho) \right) \rho d \rho - \\
- i \frac{\sqrt{\mu_0 \nu}}{2} \delta(z) H(t) 
(\delta_{m,1} + \delta_{m,-1}) 
\int \limits_{0}^{R} \left( J_{0} (\nu \rho) -
J_2 (\nu \rho) \right) \rho d \rho = \\
= - i \frac{\sqrt{\mu_0 \nu}}{2} \delta(z) H(t) 
(\delta_{m,1} + \delta_{m,-1}) 
\int \limits_{0}^{R} J_{0} (\nu \rho) \rho d \rho
\end{aligned} \end{equation*} }
%
\textcolor{lightgray} { \begin{equation*} \begin{aligned}
\int \limits_{0}^{R} J_{0} (\nu \rho) \rho d \rho = 
\frac{1}{\nu^2} \int \limits_{0}^{R} J_{0} (\nu \rho) \nu \rho d \nu \rho =
\left. \frac{\rho J_1 (\nu \rho) }{\nu} \right|_{0}^{R} = 
\frac{R J_1 (\nu R)}{\nu}
\end{aligned} \end{equation*} }
%
Після інтегрування $ \eqref{eq:jm_base} $ отримаємо тільки дві не нульові 
рівнімоди, визначені через симполи кронекера $ \delta_{m,\pm1} $.
%
\begin{equation} 
j_m (z, t; \nu) = - i R A_0 \frac{\sqrt{\mu_0}}{2} \delta(z) H(t) 
\frac{\delta_{m,1} + \delta_{m,-1}}{\sqrt{\nu}} J_1 (\nu R)
\end{equation}
%
Тепер знайдемо продольні модові коефіцієнти $ h_1 $ та $ h_{-1} $ розв'язанням 
рівняннь Клейна-Гордона
%
\textcolor{lightgray} { \begin{equation*} \begin{aligned}
- \epsilon \partial_{ct} (V_m^h) - \partial_z I_m^h + \nu^2 h_m = 
\frac{\sqrt{\mu_0}}{2 \pi} \int_0^{2\pi} d \varphi 
\int_0^{\infty} \rho d \rho \crossprod{\vect{z_0}}{\vect{J_\perp}}
\nabla_\perp \Psi_m^* (\nu) 
\end{aligned} \end{equation*} }
%
\textcolor{lightgray} { \begin{equation*} \begin{aligned}
\crossprod{\vect{z_0}}{\vect{J_\perp}} \nabla_\perp \Psi_m^* (\nu) =
\vect{J_\perp} \crossprod{\nabla_\perp \Psi_m^* (\nu)}{\vect{z_0}}
\end{aligned} \end{equation*} }
%
\textcolor{lightgray} { \begin{equation*} \begin{aligned}
\epsilon \partial_{ct} \left( \mu \partial_{ct} h_m \right) -
\mu^{-1} \partial_z \left( \mu  \partial_z h_m \right) + 
\nu^2 h_m = j_m (z,t,\nu)
\end{aligned} \end{equation*} }
%
\begin{equation} \begin{aligned} \label{eq:klein_gordon}
\frac{\epsilon \mu}{c^2} \frac{\partial^2 h_m}{\partial t^2} - 
\frac{\partial^2 h_m}{\partial z^2} + \nu^2 h_m = j_m (z,t,\nu).
\end{aligned} \end{equation}
%
Рівняння \eqref{eq:klein_gordon} було отримано з припущенням, що середовище 
в якому розповсюджується поле однорідне, стаціонарне та характерезуеться 
видносною діелектричною $ \epsilon $ та магнітною $ \mu $ проникністями.
Буде зречно позначити швидкість світла в цьому середовищі за 
$ \mathit{v} = \frac{c}{\sqrt{\epsilon \mu}} $. Рівняння Клейна-Гордона
має відомий розв'язок через функцію Рімана:
%
\begin{equation} \label{eq:klein_gordon_sol}
h_m (z, t; \nu) = \iint_S j_m (t',z') G(t,t',z,z') dt' dz',
\end{equation}
%
де $ G(t,t',z,z') $ функція Рімана 
%
\begin{equation*}
G = \frac{\mathit{v}}{2} H \left( \mathit{v} (t-t') - (z-z') \right)
J_0 \left( \nu \sqrt{\mathit{v}^2 (t-t')^2 - (z-z')^2} \right).
\end{equation*}
%
З вигляду розв'язоку \eqref{eq:klein_gordon_sol} можна зробити висновок, що
функція Рімана $ G(t,t',z,z') $ - це аналог фунгції Гріна в часовому просторі,
а тобто розв'язок рівняння Клейна-Гордона є еквівалентом принціпу суперпозиції
сферичних нестаціонарних хвиль, що випромінюються кожною з точок джерела у
деякій точці спостереження у визначений час.
%
\textcolor{lightgray} { \begin{equation*} \begin{aligned}
h_m (z, t; \nu) = - i \mathit{V} R \frac{\sqrt{\mu_0}}{4} 
\frac{\delta_{m,1} + \delta_{m,-1}}{\sqrt{\nu}} J_1 (\nu R)
\int \limits_{0}^{\infty} \delta(z) \cdot \\ \cdot
\int \limits_{t - \frac{z}{\mathit{V}}}^{0} 
J_0 \left( \nu \sqrt{\mathit{V}^2 (t-t')^2 - (z-z')^2} \right) dt' dz' = 
i \mathit{V} R \frac{\sqrt{\mu_0}}{4} 
\frac{\delta_{m,1} + \delta_{m,-1}}{\sqrt{\nu}} J_1 (\nu R)
\cdot \\ \cdot \int \limits_{0}^{\infty} \delta(z)
\int \limits_{0}^{t - \frac{z}{\mathit{V}}} 
J_0 \left( \nu \sqrt{\mathit{V}^2 (t-t')^2 - (z-z')^2} \right) dt' dz
\end{aligned} \end{equation*} }
%
\textcolor{lightgray} { \begin{equation*} \begin{aligned}
h_m (z, t; \nu) = i \mathit{V} R \frac{\sqrt{\mu_0}}{4} 
\frac{\delta_{m,1} + \delta_{m,-1}}{\sqrt{\nu}} J_1 (\nu R)
\int \limits_{0}^{t - \frac{z}{\mathit{V}}} 
J_0 \left( \nu \sqrt{\mathit{V}^2 (t-t')^2 - z^2} \right) dt'
\end{aligned} \end{equation*} }
%
Користуючись властивостями дельта-функції Дірака та функції Хевісайда 
запишимо поздовжні модові коефіцієнти $ h_1 $ та $ h_{-1} $ в наступному 
виді:
%
\begin{equation} \label{eq:hm_int}
h_m = \frac{i R A_0}{4} \frac{\delta_{m,1} + \delta_{m,-1}}
{\sqrt{\nu} \sqrt{\epsilon_0 \epsilon \mu}} J_1 (\nu R) 
\int \limits_{0}^{t - \frac{z}{v}} 
J_0 \left( \nu \sqrt{v^2 (t-t')^2 - z^2} \right) dt'
\end{equation} }
%
Зараз зручно перейти до отримання поперечних модових коефіціентів.
Для отримання виразу для $ V_m^h = - \frac{\mu}{c} \partder{h_m}{t} $
необовязково брати інтеграл в $ h_m $ спробуємо спростити вираз 
скориставшись залежністю через похідну по часу, тобто застосуємо 
правило інтерування Лейбніца \cite{imp:Flanders1973}, помітивши, що
%
\begin{equation*} \begin{aligned}
\partder{}{t'} J_0 \left( \nu \sqrt{v^2 (t-t')^2 - z} \right) =
- \partder{}{t} J_0 \left( \nu \sqrt{v^2 (t-t')^2 - z} \right) 
\end{aligned} \end{equation*}
%
отримаємо
%
\textcolor{lightgray} { \begin{equation*} \begin{aligned}
\partder{}{\theta} \int_{a(\theta)}^{b(\theta)} f(x,\theta) dx = 
\int_{a(\theta)}^{b(\theta)} \partder{f}{\theta} dx + 
f\big( b(\theta), \theta \big) \partder{b}{\theta} -
f\big( a(\theta), \theta \big) \partder{a}{\theta}
\end{aligned} \end{equation*} }
%
\textcolor{lightgray} { \begin{equation*} \begin{aligned}
\partder{}{t} J_0 \left( \nu \sqrt{v^2 (t-t')^2 - z} \right) = 
- \nu J_1 \left( \nu \sqrt{v^2 (t-t')^2 - z} \right) 
\partder{}{t} \sqrt{v^2 (t-t')^2 - z} = \\
-  J_1 \left( \nu \sqrt{v^2 (t-t')^2 - z} \right)
\frac{2 \nu v^2 (t-t')}{2 \sqrt{v^2 (t-t')^2 - z}} = - \nu v^2 (t-t') 
\frac{J_1 \left( \nu \sqrt{v^2 (t-t')^2 - z} \right)}
     {\sqrt{v^2 (t-t')^2 - z}}
\end{aligned} \end{equation*} }
%
\textcolor{lightgray} { \begin{equation*} \begin{aligned}
\partder{}{t} \int \limits_{0}^{t - \frac{z}{v}} 
J_0 \left( \nu \sqrt{v^2 (t-t')^2 - z^2} \right) dt' = \\
= \int \limits_{0}^{t - \frac{z}{v}} 
\partder{}{t} J_0 \left( \nu \sqrt{v^2 (t-t')^2 - z^2} \right) dt' +
J_0 (0) - 0 \cdot \left( \nu \sqrt{v^2 (t-t')^2 - z^2} \right) = \\
= - \int \limits_{0}^{t - \frac{z}{v}} 
\partder{}{t'} J_0 \left( \nu \sqrt{v^2 (t-t')^2 - z^2} \right) dt' + 1 =
- \Big. J_0 \left( \nu \sqrt{v^2 (t-t')^2 - z^2} \right) \Big|_{0}^{t - \frac{z}{v}} + 1 = \\
- J_0 \left( \nu \sqrt{z^2 - z^2} \right) + J_0 \left( \nu \sqrt{v^2 t^2 - z^2} \right) + 1 = 
J_0 \left( \nu \sqrt{v^2 t^2 - z^2} \right)
\end{aligned} \end{equation*} }
%
\begin{equation*} \begin{aligned}
\partder{}{t} \int \limits_{0}^{t - \frac{z}{v}} 
J_0 \left( \nu \sqrt{v^2 (t-t')^2 - z^2} \right) dt' =
J_0 \left( \nu \sqrt{v^2 t^2 - z^2} \right),
\end{aligned} \end{equation*}
%
\textcolor{lightgray} { \begin{equation*} \begin{aligned}
V_m^h = - \frac{\mu}{c} \partder{h_m}{t} = 
\sqrt{\mu_0} \sqrt{\frac{\mu}{\epsilon}} \frac{iR A_0}{4} 
\frac{\delta_{m,1} + \delta_{m,-1}}{\sqrt{\nu}} J_1 (\nu R)
J_0 \left( \nu \sqrt{\mathit{v}^2 t^2 - z^2} \right)
\end{aligned} \end{equation*} }
%
тоді можемо записати формулу для коефійієнтів $ V_m^h $
%
\begin{equation} \label{eq:vmh}
V_m^h (z, t; \nu) = - \frac{iR A_0}{4} \sqrt{\frac{\mu_0 \mu}{\epsilon}} 
\frac{\delta_{m,1} + \delta_{m,-1}}{\sqrt{\nu}} J_1 (\nu R)
J_0 \left( \nu \sqrt{\mathit{v}^2 t^2 - z^2} \right).
\end{equation}
%
Далі отримаемо модовий коефіціент $ I_m^h $, що знадобиься для визначення
магнітних компонентів поля. Для цього запишемо поздовжний магнітний модовий
коефіцієнт \cite{eq:hm_int} через спеціальну функцію Ломмеля для двох 
змінних (дійсної та уявної) \cite{imp:Boersma1961}:
%
\textcolor{lightgray} { \begin{equation*} \begin{aligned}
\int \limits_{0}^{t - \frac{z}{\mathit{v}}} 
J_0 \left( \nu \sqrt{\mathit{v}^2 (t-t')^2 - z^2} 
\right) dt' = \left[ \begin{array}{cc} 
\nu \mathit{v} (t-t') = s & t' = t - \frac{ds}{\nu \mathit{v}} \\
dt' = -\frac{ds}{\nu \mathit{v}} & \\
s(0) = \nu \mathit{v} t & s \left( t - \frac{z}{\mathit{v}} \right) = \nu z
\end{array} \right] = \\ = - \frac{1}{\nu \mathit{v}} 
\int_{\nu \mathit{v} t}^{\nu z} ds 
J_0 (\sqrt{s^2 - \nu^2 z^2}) = \frac{1}{\nu \mathit{v}} 
\int_{\nu z}^{\nu \mathit{v} t} ds
J_0 (\sqrt{s^2 - \nu^2 z^2})
\end{aligned} \end{equation*} }
%
\textcolor{lightgray} { \begin{equation*} \begin{aligned}
\int_{\nu z}^{\nu \mathit{v} t} ds e^{-i0s} J_0 (\sqrt{s^2 - \nu^2 z^2}) = \\ 
= \frac{1}{i} (U_1[W_+,Z] + i U_2[W_+,Z] - U_1[W_-,Z] - i U_2[W_-,Z]) = \\
= \frac{1}{i} (-U_1[W_-,Z] + i U_2[W_+,Z] - U_1[W_-,Z] - i U_2[W_+,Z]) = \\
= \left[ \begin{array}{c} W_\pm = \pm i (\nu \mathit{v} t - \nu z) \\
Z = \sqrt{\nu^2 \mathit{v}^2 t^2 - \nu^2 z^2} \end{array} \right] = 
2i U_1 \left[ -i \nu (\mathit{v}t-z), \nu \sqrt{\mathit{v}^2 t^2-z^2} \right]
\end{aligned} \end{equation*} }
%
\textcolor{lightgray} { \begin{equation*} \begin{aligned}
\int \limits_{0}^{t - \frac{z}{\mathit{v}}} 
J_0 \left( \nu \sqrt{\mathit{v}^2 (t-t')^2 - z^2} 
\right) dt' = \frac{2i}{\nu \mathit{v}} U_1 
\left[ -i \nu (\mathit{v}t-z), \nu \sqrt{\mathit{v}^2t^2-z^2} \right]
\end{aligned} \end{equation*} }
%
\textcolor{lightgray} { \begin{equation*} \begin{aligned}
h_m (z, t; \nu) = \mathit{v} \sqrt{\mu_0} \frac{iR A_0}{4} 
\frac{\delta_{m,1} + \delta_{m,-1}} {\sqrt{\nu}} J_1 (\nu R) 
\frac{2i}{\nu \mathit{v}} U_1 \left[ W_-, Z \right]
\end{aligned} \end{equation*} }
%
\begin{equation} \label{eq:hm_lommel}
h_m (z, t; \nu) = - \sqrt{\mu_0} \frac{R A_0}{2} 
\frac{\delta_{m,1} + \delta_{m,-1}}
{\nu^{3/2}} J_1 (\nu R) U_1 \left[ W_-, Z \right]
\end{equation}
%
Тепер підставивши \eqref{eq:hm_lommel} в вираз для коефіціенту
$ I_{m}^{h} = \partder{h_m}{z} $ отримаємо:
%
\textcolor{lightgray} { \begin{equation*} \begin{aligned}
I_{m}^{h} = \partder{h_m}{z} = 
- \sqrt{\mu_0} \frac{R A_0}{2} 
\frac{\delta_{m,1} + \delta_{m,-1}}
{\nu^{3/2}} J_1 (\nu R) \partder{}{z} U_1 [ W_-, Z ]
\end{aligned} \end{equation*} }
%
\textcolor{lightgray} { \begin{equation*} \begin{aligned}
\begin{array}{lcr}
\derivat{W_-}{z} = i \nu & &
\derivat{Z}{z} = \frac{\nu}{2 \sqrt{\mathit{V}^2 t^2 - z^2}} (-2z) = 
- \frac{\nu z}{\sqrt{\mathit{V}^2 t^2 - z^2}} \\
\end{array}
\end{aligned} \end{equation*} }
%
\textcolor{lightgray} { \begin{equation*} \begin{aligned}
\left( \frac{Z}{W} \right)^2 = 
\left( - \frac{ \sqrt{\mathit{V}^2 t^2-z^2}}{i(\mathit{V} t-z)} \right)^2 =
\left( \frac{ i \sqrt{\mathit{V}^2 t^2-z^2}}{\mathit{V}t-z} \right)^2 =
- \frac{\mathit{V}^2 t^2-z^2}{(\mathit{V} t-z)^2} = 
- \frac{\mathit{V}t+z}{\mathit{V}t-z}
\end{aligned} \end{equation*} }
%
\textcolor{lightgray} { \begin{equation*} 
\partder{}{Z} U_n (W,Z) = - \frac{Z}{W} U_{n+1} (W,Z)
\end{equation*} }
%
\textcolor{lightgray} { \begin{equation*}
2 \partder{}{W} U_n (W,Z) = U_{n-1} (W,Z) + 
\left( \frac{Z}{W} \right)^2 U_{n+1} (W,Z)
\end{equation*} }
%
\textcolor{lightgray} { \begin{equation*} \begin{aligned}
\partder{}{z} U_1 \left[ -i \nu (ct-z), \nu \sqrt{c^2t^2-z^2} \right] =
\partder{}{z} U_1[W,Z] = \partder{U_1}{W} \derivat{W}{z} + 
\partder{U_1}{Z} \derivat{Z}{z} = \\
= \frac{i \nu}{2} \left( U_0 - \frac{ct+z}{ct-z} U_2 \right) -
\frac{\nu z}{\sqrt{c^2t^2 - z^2}} 
\left( - \frac{i \sqrt{c^2t^2-z^2}}{ct-z} \right) U_2 = \\
= \frac{i \nu}{2} U_0 - \frac{i \nu}{2} \frac{ct+z}{ct-z} U_2 +
\frac{i \nu z}{ct-z} U_2 = \\ = \frac{i \nu}{2} U_0 - \frac{i \nu}{2} U_2
\left( \frac{ct}{ct-z} + \frac{z}{ct-z} - \frac{2z}{ct-z} \right) = 
\frac{i \nu}{2} (U_0[W_-,Z] - U_2[W_-,Z])
\end{aligned} \end{equation*} }
%
\begin{equation} \label{eq:imh}
I_{m}^{h} = - \sqrt{\mu_0} \frac{iR A_0}{4} 
\frac{\delta_{m,1} + \delta_{m,-1}}{\sqrt{\nu}} 
J_1 (\nu R) \left( U_0 [ W_-, Z ] - U_2 [ W_-, Z ] \right)
\end{equation}
%
Електричні модові коефіцієнити $ e_n $, $ I_n^e $, $ V_n^e $ для всіх $ n $
рівні нулю. Математино, це виходить з того, що розв'язок однорідного рівняння 
Клейна-Гордона відносно $ e_n $ має тільки тривіальний розв'язок. Такі модові
розклади характерні для TEM хвиль.
%
Таким чином отримано аналітично всі еволюційні коефіцієнти. Підставиво
\eqref{eq:vmh} в розклад вектору напруженності електричного поля по
базисним функціям \cite{imp:Dumin2010}. Таким чином отримаємо електричне 
поле в циліндричних компонентах $ \vect{\rho_0}, \vect{\varphi_0}, \vect{z_0} $,
як функцію циліндричних координат $ \rho, \varphi, z $ та часу $ t $.
%
\textcolor{lightgray} { \begin{equation*} \begin{aligned}
\vect{E_\perp} = \frac{1}{\sqrt{\epsilon_0}} \left( 
\sum \limits_{m=-\infty}^{\infty} \int \limits_{0}^{\infty} 
d \nu V_m^h \crossprod{ \nabla_\perp \Psi_m }{ \vect{z_0} } +
\sum \limits_{n=-\infty}^{\infty} \int \limits_{0}^{\infty}
d \chi V_n^e \nabla_\perp \Phi_n \right)
\end{aligned} \end{equation*} }
%
\textcolor{lightgray} { \begin{equation*} \begin{aligned}
\crossprod{ \nabla_\perp \Psi_m }{ \vect{z_0} } = 
- e^{im\varphi} \left( \vect{\varphi_0} \sqrt{\nu} 
\frac{J_{m-1} (\nu \rho) - J_{m+1} (\nu \rho)}{2} - 
i m \vect{\rho_0} \frac{J_m (\nu \rho)}{ \rho \sqrt{\nu}} \right)
\end{aligned} \end{equation*} }
%
\textcolor{lightgray} { \begin{equation*} \begin{aligned}
\vect{E_\perp} = \frac{1}{\sqrt{\epsilon_0}} \int_{0}^{\infty} 
V_{-1}^h \crossprod{ \nabla_\perp \Psi_{-1}  }{ \vect{z_0} } +
\frac{1}{\sqrt{\epsilon_0}} \int \limits_{0}^{\infty} 
V_{1}^h \crossprod{ \nabla_\perp \Psi_{1} }{ \vect{z_0} } = \\
= \frac{i R A_0}{4} \sqrt{\frac{\mu_0 \mu}{\epsilon_0 \epsilon}} 
e^{- i \varphi} \int_{0}^{\infty} \frac{J_1 (\nu R)}{\sqrt{\nu}} 
J_0 \left( \nu \sqrt{c^2 t^2 - z^2} \right) \cdot \\
\cdot \left( \vect{\varphi_0} \sqrt{\nu} 
\frac{J_2 (\nu \rho) - J_0 (\nu \rho)}{2} +
i \vect{\rho_0} \frac{J_1 (\nu \rho)}{ \rho \sqrt{\nu}} \right) - \\
+ \frac{i R A_0}{4} \sqrt{\frac{\mu_0 \mu}{\epsilon_0 \epsilon}}
e^{i \varphi} \int \limits_{0}^{\infty} \frac{J_1 (\nu R)}{ \sqrt{\nu}}
J_0 \left( \nu \sqrt{c^2 t^2 - z^2} \right) \cdot \\
\cdot \left( \vect{\varphi_0} \sqrt{\nu}
\frac{J_0 (\nu \rho) - J_2 (\nu \rho)}{2} - 
i \vect{\rho_0} \frac{J_1 (\nu \rho)}{ \rho \sqrt{\nu}} \right)
\end{aligned} \end{equation*} }
%
\textcolor{lightgray} { \begin{equation*} \begin{aligned}
E_\varphi = \frac{i R A_0}{8} \sqrt{\frac{\mu_0 \mu}{\epsilon_0 \epsilon}} 
e^{-i \varphi} \int \limits_{0}^{\infty} J_1 (\nu R)
J_0 \left( \nu \sqrt{c^2 t^2 - z^2} \right)
\left( J_2 (\nu \rho) - J_0 (\nu \rho) \right) + \\
+ \frac{i R A_0}{8} \sqrt{\frac{\mu_0 \mu}{\epsilon_0 \epsilon}} 
e^{i \varphi} \int \limits_{0}^{\infty} J_1 (\nu R)
J_0 \left( \nu \sqrt{c^2 t^2 - z^2} \right)
\left( J_0 (\nu \rho) - J_2 (\nu \rho) \right) = \\
= \frac{i R A_0}{4} \sqrt{\frac{\mu_0 \mu}{\epsilon_0 \epsilon}}
\frac{e^{i \varphi} - e^{-i \varphi} }{2} \int \limits_{0}^{\infty} 
J_1 (\nu R) J_0 \left( \nu \sqrt{c^2 t^2 - z^2} \right) 
\left( J_0 (\nu \rho) - J_2 (\nu \rho) \right) =
\end{aligned} \end{equation*} }
%
\textcolor{lightgray} { \begin{equation*} \begin{aligned}
= \frac{R A_0}{4} \sqrt{\frac{\mu_0 \mu}{\epsilon_0 \epsilon}} 
\frac{e^{i \varphi} - e^{-i \varphi} }{2i} \int \limits_{0}^{\infty} 
J_1 (\nu R) J_0 \left( \nu \sqrt{c^2 t^2 - z^2} \right) 
\left( J_2 (\nu \rho) - J_0 (\nu \rho) \right) = \\
= \frac{R A_0}{4} \sqrt{\frac{\mu_0 \mu}{\epsilon_0 \epsilon}} \sin \varphi 
\int \limits_{0}^{\infty} J_1 (\nu R) 
J_0 \left( \nu \sqrt{c^2 t^2 - z^2} \right) 
\left( J_2 (\nu \rho) - J_0 (\nu \rho) \right)
\end{aligned} \end{equation*} }
%
\textcolor{lightgray} { \begin{equation*} \begin{aligned}
J_2 (\nu \rho) - J_0 (\nu \rho) = \frac{2}{\nu \rho} J_1 (\nu \rho) - 
2 J_0 (\nu \rho)
\end{aligned} \end{equation*} }
%
\textcolor{lightgray} { \begin{equation*} \begin{aligned}
E_\varphi = \frac{R A_0}{2} \sqrt{\frac{\mu_0 \mu}{\epsilon_0 \epsilon}}
\sin \varphi \int \limits_{0}^{\infty} J_1 (\nu R) 
J_0 \left( \nu \sqrt{c^2 t^2 - z^2} \right) 
\left( \frac{J_1 (\nu \rho)}{\nu \rho} - J_0 (\nu \rho) \right)
\end{aligned} \end{equation*} }
%
\textcolor{lightgray} { \begin{equation*} \begin{aligned}
E_\rho = \frac{i R A_0}{4} \sqrt{\frac{\mu_0 \mu}{\epsilon_0 \epsilon}}  
e^{- i \varphi} \int \limits_{0}^{\infty} \frac{J_1 (\nu R)}{\sqrt{\nu}} 
J_0 \left( \nu \sqrt{c^2 t^2 - z^2} \right) 
\left( - i \frac{J_1 (\nu \rho)}{\rho \sqrt{\nu}} \right) + \\
+ \mu \frac{i R A_0}{4} \sqrt{\frac{\mu_0}{\epsilon_0}}  e^{i \varphi}
\int \limits_{0}^{\infty} \frac{J_1 (\nu R)}{\sqrt{\nu}}
J_0 \left( \nu \sqrt{c^2 t^2 - z^2} \right) 
\left( - i \frac{J_1 (\nu \rho)}{ \rho \sqrt{\nu}} \right) = \\
= \mu \frac{R A_0}{2} \sqrt{\frac{\mu_0 \mu}{\epsilon_0 \epsilon}} 
\frac{e^{i \varphi} + e^{-i \varphi}}{2}
\int \limits_{0}^{\infty} \frac{J_1 (\nu R)}{\sqrt{\nu}}
J_0 \left( \nu \sqrt{c^2 t^2 - z^2} \right) 
\frac{J_1 (\nu \rho)}{ \rho \sqrt{\nu}} = \\
= \mu \frac{R A_0}{2} \sqrt{\frac{\mu_0 \mu}{\epsilon_0 \epsilon}} 
\cos \varphi \int \limits_{0}^{\infty} \frac{d \rho}{\nu \rho} 
J_1 (\nu \rho) J_1 (\nu R) J_0 \left( \nu \sqrt{c^2 t^2 - z^2} \right)
\end{aligned} \end{equation*} }
%
\begin{equation} \label{eq:linear_e_cyl}
\vect{E} \left( r, t \right) = \frac{A_0}{2} 
\sqrt{\frac{\mu_0 \mu}{\epsilon_0 \epsilon}}
\Big( \vect{\rho_0} I_1 \cos \varphi - 
\vect{ \varphi_0 } \left( I_2 - I_1 \right) \sin \varphi \Big),
\end{equation}
%
де
%
\begin{equation*}
I_1 = R \int \limits_{0}^{\infty} \frac{d \nu}{\nu \rho} J_1 (\nu \rho) 
J_1 (\nu R) J_0 \left( \nu \sqrt{\frac{c^2 t^2}{\epsilon \mu} - z^2} \right)
\end{equation*}
з аналітичним розвязком, що представсено в додатку \ref{sec:i1anal}, а
%
\begin{equation*}
I_2 = R \int_{0}^{\infty} d \nu J_1 (\nu R) J_0 (\nu \rho) 
J_0 \left( \nu \sqrt{\frac{c^2 t^2}{\epsilon \mu} - z^2} \right),
\end{equation*}
%
з аналітичним розвязком, що представсено в додатку \ref{sec:i2anal}.

Розглянемо вектор наапруенноті електричного поля в базису Декартової системи
координат, тоді: 
%
\textcolor{lightgray} { \begin{equation*} \begin{aligned}
\mathbf{A} = \left( \begin{array}{cc}
\cos \varphi & \sin \varphi \\
- \sin \varphi & \cos \varphi
\end{array} \right) \begin{array}{ccc}
	& \det A = 1 		&	\\
	& A^{-1} = A^{T}	&
\end{array} 
\mathbf{A^{-1}} = \left( \begin{array}{cc}
\cos \varphi & - \sin \varphi \\
\sin \varphi & \cos \varphi
\end{array} \right) 
\end{aligned} \end{equation*} }
%
\textcolor{lightgray} { \begin{equation*} \begin{aligned}
\vect{E} = 
\mathbf{A^{-1}} \vect{E} \left( \vect{\rho_0}, \vect{\varphi_0} \right) = 
\frac{A_0}{2} \sqrt{\frac{\mu_0 \mu}{\epsilon_0 \epsilon}}
\left( \begin{array}{cc} \cos \varphi & - \sin \varphi \\
\sin \varphi & \cos \varphi \end{array} \right)
\left( \begin{array}{c} I_1 \cos \varphi \\
- (I_2 - I_1) \sin \varphi \end{array} \right) = \\
= \frac{A_0}{2} \sqrt{\frac{\mu_0 \mu}{\epsilon_0 \epsilon}}
\left( \begin{array}{c} I_1 \cos^2 \varphi + (I_2 - I_1) \sin^2 \varphi \\
I_1 \sin \varphi \cos \varphi - (I_2 - I_1) 
\sin \varphi \cos \varphi \end{array} \right)
\end{aligned} \end{equation*} }
%
\begin{equation} \begin{aligned} \label{eq:Exyz}
\left( \begin{array}{c} E_x \\ E_y \\ E_z \end{array} \right) = 
\frac{A_0}{2}  \sqrt{\frac{\mu_0 \mu}{\epsilon_0 \epsilon}} 
\left( \begin{array}{c} 
I_1 \cos^2 \varphi + (I_2 - I_1) \sin^2 \varphi \\
- I_2 \sin \varphi \cos \varphi \\
0
\end{array} \right)
\end{aligned} \end{equation}

З аналітичних розвязків для інтегралів $ I_1 $ та $ I_2 $ бачимо, що 
компоненти поля - шматочно визначені функції з областю визначення 
$ S = S_1 \cup S_2 \cup S_3 $, де
%
\begin{equation} \begin{aligned} \label{eq:s1zone}
S_1 \subset (\rho - R)^2 \ge \frac{c^2t^2}{\epsilon \mu} - z^2 > 0,
\end{aligned} \end{equation}
%
\begin{equation} \begin{aligned} \label{eq:s2zone}
S_2 \subset (\rho - R)^2 < \frac{c^2t^2}{\epsilon \mu} - z^2 < (\rho + R)^2,
\end{aligned} \end{equation}
%
\begin{equation} \begin{aligned} \label{eq:s3zone}
S_3 \subset \frac{c^2t^2}{\epsilon \mu} - z^2 \ge (\rho + R)^2.
\end{aligned} \end{equation}

Звертаючись до схематичного зображення причинного звязку спостергігача та 
джерела (Рис.~\ref{fig:part_rad}) помічаємо, що область $ S_1 $ стосується 
просторово-чаасових подій, які причинно не пов'язані з жодним з крайніх точок 
джерела. Саме тут спостерігається ефект електромагнітного снаряду: 
спостерігач  в цій області простору-часу завжди причинно пов'язаний з 
частиною джерела, яка має круглу форму.

\begin{figure}[h] \begin{center}
\includegraphics[scale=0.6]{PartialRadiation}
\caption{Фізичний зміст областей випромінювання} \label{fig:part_rad}
\end{center} \end{figure}

Область $ S_2 $ відповідає за події, коли частина кайніх точок джерела вже 
причинно повзязана зі спостерігачем. Тобто частина джерела про яку вже відомо
спостерігачу (часово-поєднані), вже не має круглої форми та ще не охоплює 
всього джерела. В цій області спостерігається деякий перехідний процес в
якому значення прехідної функції поступово згасає на нуль.

Спостервгачі в області $ S_3 $ вже отримали всю інформацію про джерело. Для
них перехідний процес скінчено і зміни напруженності поля не спостерігається,
а отже напруженность електричного поля тут відсутня.

Для зони $ S_3 $ тільки $ E_x $ компонента електричного поля не дорівнює
нулю. Окрім того амплітуда $ E_x $ постійна для всіх подій з облсті $ S_3 $.

Перейдемо до магнітних складових пеехідної функції плаского диску. Для 
отримання магних компонент поля скористаємось еволюційним коефіцієнтом 
\eqref{eq:imh}.

\textcolor{lightgray} { \begin{equation*} \begin{aligned}
\vect{H_\perp} = \frac{1}{\sqrt{\mu_0}} \left( 
\sum \limits_{m=-\infty}^{\infty} \int \limits_{0}^{\infty} d \nu
I_m^h \nabla_\perp \Psi_m + \sum \limits_{n=-\infty}^{\infty}
\int \limits_{0}^{\infty} d \chi I_n^e 
\crossprod{\vect{z_0}}{\nabla_\perp \Phi_n} \right)
\end{aligned} \end{equation*} }
%
\textcolor{lightgray} { \begin{equation*} \begin{aligned}
\nabla_\perp \Psi_m = e^{i m \varphi} \left( \vect{\rho_0} 
\sqrt{\nu} \frac{ J_{m-1}(\nu \rho) - J_{m+1}(\nu \rho) }{2} +
i m \vect{\varphi_0} \frac{J_m(\nu \rho)}{\sqrt{\nu} \rho} \right)
\end{aligned} \end{equation*} }
%
\textcolor{lightgray} { \begin{equation*} \begin{aligned}
\vect{H_\perp} = \frac{1}{\sqrt{\mu_0}} \left( 
\int \limits_{0}^{\infty} d \nu I_{-1}^h \nabla_\perp \Psi_{-1} +
\int \limits_{0}^{\infty} d \nu I_1^h \nabla_\perp \Psi_1 \right) = \\
= - \frac{A_0}{\sqrt{\mu_0}} \int \limits_{0}^{\infty} d \nu
\sqrt{\mu_0} \frac{iR}{4} J_1 (\nu R)
\frac{ U_0 [ W_-, Z ] - U_2 [ W_-, Z ] }{\sqrt{\nu}}  
e^{- i \varphi} \cdot \\ \cdot \left( \vect{\rho_0} 
\sqrt{\nu} \frac{ J_{2}(\nu \rho) - J_{0}(\nu \rho) }{2} +
i \vect{\varphi_0} \frac{J_1(\nu \rho)}{\sqrt{\nu} \rho} \right) -
\frac{A_0}{\sqrt{\mu_0}} \int \limits_{0}^{\infty} d \nu 
\sqrt{\mu_0} \frac{iR}{4} J_1 (\nu R) \cdot \\
\cdot \frac{ U_0 [ W_-, Z ] - U_2 [ W_-, Z ] }{\sqrt{\nu}} 
e^{i \varphi} \left( \vect{\rho_0} 
\sqrt{\nu} \frac{ J_{0}(\nu \rho) - J_{2}(\nu \rho) }{2} +
i \vect{\varphi_0} \frac{J_1(\nu \rho)}{\sqrt{\nu} \rho} \right)
\end{aligned} \end{equation*} }
%
\textcolor{lightgray} { \begin{equation*} \begin{aligned}
H_\varphi = \frac{R A_0}{4} 
\frac{e^{i \varphi} + e^{- i \varphi}}{\rho} \int \limits_{0}^{\infty} 
\frac{d\nu}{\nu} (U_0[ W_-, Z ] - U_2[ W_-, Z ]) J_1(\nu R) J_1(\nu \rho) = \\
= \frac{R}{2} \cos \varphi \int \limits_{0}^{\infty}
\frac{d\nu}{\nu \rho} (U_0[ W_-, Z ] - U_2[ W_-, Z ]) 
J_1(\nu R) J_1(\nu \rho)
\end{aligned} \end{equation*} }
%
\textcolor{lightgray} { \begin{equation*} \begin{aligned}
H_\rho = \frac{R A_0}{4} \frac{e^{i \varphi} - e^{- i \varphi}}{2i}
\int \limits_{0}^{\infty} d \nu (J_{0}(\nu \rho) - J_{2}(\nu \rho))
J_1(\nu R) (U_0[ W_-, Z ] - U_2[ W_-, Z ]) = \\
= \frac{R}{2} \sin \varphi \int \limits_{0}^{\infty} d \nu 
(J_0(\nu \rho) - \frac{J_1(\nu \rho)}{\nu \rho})
J_1(\nu R) (U_0[ W_-, Z ] - U_2[ W_-, Z ]) = \\
\end{aligned} \end{equation*} }
%
\textcolor{lightgray} { \begin{equation*} \begin{aligned}
\vect{H_\perp} \left( r, t \right) = \frac{A_0}{2} \left( 
\vect{\rho_0} \left( I_4 - I_3 \right) \sin \varphi +
\vect{\varphi_0} I_3 \cos \varphi  \right)
\end{aligned} \end{equation*} }
%
\textcolor{lightgray} { \begin{equation*} \begin{aligned}
H_z (r,t) = \frac{1}{\sqrt{\mu_0}} \sum \limits_{m=-\infty}^{\infty}
\int \limits_0^\infty \nu^2 d \nu h_m \Psi_m
\end{aligned} \end{equation*} }
%
\textcolor{lightgray} { \begin{equation*} \begin{aligned}
\Psi_m (\nu) = \frac{J_m(\nu \rho)}{\sqrt{\nu}} e^{im \varphi} 
\end{aligned} \end{equation*} }
%
\textcolor{lightgray} { \begin{equation*} \begin{aligned}
H_z (r,t) = 
\frac{1}{\sqrt{\mu_0}} \int \limits_0^\infty \nu^2 d \nu h_{1} \Psi_{1} +
\frac{1}{\sqrt{\mu_0}} \int \limits_0^\infty \nu^2 d \nu h_{-1} \Psi_{-1}
\end{aligned} \end{equation*} }
%
\textcolor{lightgray} { \begin{equation*} \begin{aligned}
H_z (r,t) = R A_0 \frac{e^{im \varphi}-e^{-im \varphi}}{2} \int_0^\infty 
d \nu J_1(\nu \rho) J_1 (\nu R)
U_1 \left[ -i \nu (ct-z), \nu \sqrt{c^2t^2-z^2} \right]
\end{aligned} \end{equation*} }
%
\textcolor{lightgray} { \begin{equation*} \begin{aligned}
H_z (r,t) = - R A_0 \sin \varphi \int_0^\infty 
d \nu J_1(\nu \rho) J_1 (\nu R) U_1 [ W_-, Z ] = \\
= - i R A_0 \sin \varphi \int_{0}^{\infty} J_1 \left( \nu R \right)
J_1 \left( \nu \rho \right) U_1 [ W_-, Z ]
\end{aligned} \end{equation*} }
%
\textcolor{lightgray} { \begin{equation*} \begin{aligned}
H_z \left( r, t \right) = - A_0 I_5 \sin \varphi
\end{aligned} \end{equation*} }
%
\begin{equation} \label{eq:linear_h_cyl}
\vect{H} (r, t) = \frac{A_0}{2} \Big( 
\vect{\rho_0} \left( I_4 - I_3 \right) \sin \varphi +
\vect{\varphi_0} I_3 \cos \varphi -
2 \vect{z_0} I_5 \sin \varphi \Big),
\end{equation}
%
де 
%
\begin{equation*}
I_3 = R \int \limits_{0}^{\infty}
\frac{d\nu}{\nu \rho} J_1(\nu R) J_1(\nu \rho)
\Big( U_0[ W, Z ] - U_2[ W, Z ] \Big),
\end{equation*}
%
\begin{equation*}
I_4 = R \int \limits_{0}^{\infty} d\nu J_1(\nu R) J_0(\nu \rho)
\Big( U_0[ W, Z ] - U_2[ W, Z ] \Big),
\end{equation*}
%
\begin{equation*}
I_5 = i R \int \limits_0^\infty 
d \nu J_1(\nu \rho) J_1 (\nu R)
U_1 \left[ -i \nu \left( \frac{ct}{\sqrt{\epsilon \mu}} - z \right), 
\nu \sqrt{\frac{c^2t^2}{\epsilon \mu}-z^2} \right].
\end{equation*}

В Декартовому базисі вектор напруженності магнітного поля матиме вигляяд

\begin{equation} \begin{aligned} \label{eq:Hxyz}
\left( \begin{array}{c} H_x \\ H_y \\ H_z \end{array} \right) = 
\frac{A_0}{2} \sqrt{\frac{\mu_0 \mu}{\epsilon_0 \epsilon}} 
\left( \begin{array}{c}
- I_4 \sin \varphi \cos \varphi \\
I_3 \cos^2 \varphi + (I_4 - I_3) \sin^2 \varphi \\
- 2 I_5 sin \varphi
\end{array} \right).
\end{aligned} \end{equation}

Для інтегралів $ I_3, I_4, I_5 $ аналітичні розв'язки, які представлено в 
додатку \ref{ch:lommel}, вдалось знайти лише на осі випромінювання 
($ \rho = 0 $). Відзначимо, що всі інтеграли мають дійсні значення, що витікає
з властивостей функції Ломмеля. Застсовуючи визначення функції Ломеля для  
виразу \eqref{eq:linear_h_cyl} на великій відстані від джерела, де плаский 
диск можна розгядати, як матеріальну точку, тобто $ z >> R $ помічаємо, що 
%
\begin{equation*}
U_0[ W, Z ] - U_2[ W, Z ] = 
J_0 \left( \frac{c^2t^2}{\epsilon \mu}  - z^2 \right),
\end{equation*}
%
тоді ортогональні поперечні компоненти електромагнітного поля попарно 
пропорційні через значення імпедансу вільного простору, в якому 
розповлюджується хвиля
%
\begin{equation} \label{eq:e2h}
\frac{E_x}{H_y} = \frac{E_y}{H_x} = 
\sqrt{\frac{\mu_0 \mu}{\epsilon_0 \epsilon}},
\end{equation}
%
що відповідає властивостям пласної хвилі, а також підтверждує можливість 
апроксимації антен імпульснго випромінювання фізичною моделю плаского 
сторонього електричного струму. Такий висновку можна дійти з того, що
саме пласку хвилю очікується побачити після випрямлення сфкричного фонту
від ТЕМ рупора в плаский гіперболічною, або витягноную сферичною лінзою.

Звертаючись до аналітики з додатку \ref{ch:lommel} також помічаємо, що 
рівність \eqref{eq:e2h} також строго виконується і вблизу апертури, для 
всієї тривалості перехідного процесу. Рівність не виконується лише для 
області $ S_3 $, де $ \vect{E} = 0 $, а $ \vect{H}_\perp = const $.

%%%%%%%%%%%%%%%%%%%%%%%%%%%%%%%%%%%%%%%%%%%%%%%%%%%%%%%%%%%%%%%%%%%%%%%%%%%%%%%
\section{Властивості перехідної функції плаского диску}

\begin{figure}[h] \begin{center}
\includegraphics[scale=1.6]{MissileEffect}
\caption{Ефект електромагнітного снаряду ($ z = 2 $ м)} \label{fig:emp_rho}
\end{center} \end{figure}
%
\begin{figure}[h] \begin{center}
\includegraphics[scale=0.6]{TransientEffect}
\caption{Ефект електромагнітного снаряду ($ \rho = 0.2 $ м)} \label{fig:emp_z}
\end{center} \end{figure}
%
\begin{figure}[h] \begin{center}
\includegraphics[scale=0.7]{LinearPulsShape}
\caption{Кутова залежнысть формы імпульсу ($ \rho = R/2 .. 2R $ м)} 
\label{fig:emp_shape}
\end{center} \end{figure}
%
\begin{figure}[h] \begin{center}
\includegraphics[scale=0.6]{StaticOnAxis}
\caption{Магнітно-статичне поле ($ \rho = 0 $ м)} \label{fig:emp_h_rho}
\end{center} \end{figure}
%
\begin{figure}[h] \begin{center}
\includegraphics[scale=0.7]{LinearMagnetic}
\caption{Магнітно-статичне поле ($ z = 2 $ м)} \label{fig:emp_h_rho}
\end{center} \end{figure}
%
\begin{figure}[h] \begin{center}
\includegraphics[scale=0.9]{Ex_vs_Hz}
\caption{$E_x$ і $H_z$ в ($\rho = R/2$, $\varphi = -\pi/2$, $z = R$)} 
\label{fig:ex_vs_hz}
\end{center} \end{figure}

Область електромагнітного знаряду $ S_1 $ характерезується сталою амплітудою 
електричного і мігнітного поля в кожен момент часу. Також в цій області 
існують лише дві кмпоненти поля $ E_x $ та $ H_y $, що видно з аналітики в 
додатках \ref{ch:bessel} та \ref{ch:lommel}, а отже в області 
електромгнітного знаряду спостерігається вироджена ТЕМ хвиля. Спостерігач на 
осі випромінювання знаходиться в області елктромагнітного знаряду вподовж 
всього перехідного процесу, бо стає \textcolor{red}{часовоподібним} відносно 
всіх крайніх точок джерела одномоментно. Таким чином для $ \rho = 0 $ область 
визначення векотрів \eqref{eq:Exyz} та \eqref{eq:Hxyz} завжди $ S_1 $, отже
значення сталих амплітуд поля в області електромагнітного знаряду знайдемо
підставивши $ \rho = 0 $ в \eqref{eq:Exyz} та \eqref{eq:Hxyz}:

\begin{equation*} \begin{aligned}
\vect{E} \{ S_1 \} = 
\vect{x_0} \frac{A_0}{4} 
\sqrt{\frac{\mu_0 \mu}{\epsilon_0 \epsilon}};
\end{aligned} \end{equation*}
%
\begin{equation*} \begin{aligned}
\vect{H} \{ S_1 \} = \vect{y_0} \frac{A_0}{4}.
\end{aligned} \end{equation*}
%

Співвідносячи форму імпульсу ра Рис.~\ref{fig:emp_rho} з тим, що 
$ H_z \{ S_1 \} = 0 $ бачимо, що поперечне електромагнітне поле в бласті 
$ S_1 $ "випереджує" поздовжне та ніби породжує його в області $ S_2 $. В 
роботах Фарадея зустрічається твердження, що випромінює не антена а простір 
довкола неї. Саме це і спостерігається в області електромагнітного знаряду - 
поздовжна компонента поля, без якої неможливе розповсюдження хвилі у 
вільному просторі \textcolor{red}{[Хармут]}, породжується простором у 
якому з’являється поперечні компоненти. Таким чином, можимо узагальнити 
твеждженн Хармута для нестаціонарного процесу: розповсюдження хвилі, а тобто 
($ E_z \neq 0 $ чи $ H_z \neq 0 $) почнется тільки тоді, коли спостерігач 
дізнається про те, що розподіл заряду стороннього джерела змінюєтья у часі 
або у просторі, а до цього, хвиля не розповсюджується, не дивлячись на те, 
що спостерігач і джерело - вже причино пов'язані. 

Можемо зробити висновок, що для стаціонарного процесу поняття поля и хвилі 
тотожні, коли у часовій області інколи ні: так наприклад область визначеня 
електрично поля, що випромінюеться пласким диском $ S_1 S_2 S_3 $, а для 
область визначення хвилі лише $ S_2 S_3 $, тобто електромагнітний сназяд 
неможна назвити хвилею, а лише полем.

%
Такий ефект носить назву електромагнітного знаряду. Хоча поле в явному
виді не залежить від точки спостереження, залежність присутня в визначенні 
зон випромінювання поля плаского диску. Користуючить ними можна вирахувати 
тривалість ефекту.
%
\begin{equation*} \begin{aligned}
\frac{c \tau}{\sqrt{\epsilon \mu}} = \sqrt{R^2+z^2} - z.
\end{aligned} \end{equation*}
%
При $ R = 1 $ можна ввести апроксимацію цієї формули, як  
%
\begin{equation*} \begin{aligned}
\frac{c \tau}{\sqrt{\epsilon \mu}} \left( z \gg 2R \right) \approx 
\begin{cases}
\frac{2R}{z} , R \geq 1 \\
\frac{R^2}{2z} , R \leq 1
\end{cases}
\end{aligned} \end{equation*}


За визначенням дальньої зони - це область простору, де $ E_\varphi = H_\rho $ та
$ H_\varphi = E_\rho $. Дня лінійного поля плаского диску ці умови виконується, 
коли $ U_0(W_-,Z) - U_2(W_-,Z) = J_0(Z) $. Остання тотожність математично вірна
тоді і тільки коли 
%
\begin{equation} \label{eq:FraunhoferDistance}
\left. \lim_{z \to \infty} \left( \frac{ct-z}{ct+z} \right)^m 
\right|_{m > 1} = 0
\end{equation}
%
Остання тотожність є умовою дальньої зони для антени що породжує 
нестаціонарне поле. Варто зазначити, що при рості $ z $ росте і $ t $, 
відповідно до принципу причинності, тобто умова $ ct - z > 0 $ виконується.
%
\begin{figure}[h] \begin{center}
\includegraphics[scale=0.5]{SingulatiyFactorization}
\caption{Розклад сингулярності джерела} \label{fig:singulatiy_factorization}
\end{center} \end{figure}
%
Побудуємо діаграму спрямованності випроміювача корістуючись різницею 
максимального та мінімального значення електричного поля, збудженого 
випромінюванням перехідної функції, як функцію азимутального кута в площінах. 

\textcolor{red} { Плаский диск можна розглядати, як генератор пласної хвилі,
тоді антенни Баума та Ву можна розглядати, як генератори квазі-пласкої хвилі }

%%%%%%%%%%%%%%%%%%%%%%%%%%%%%%%%%%%%%%%%%%%%%%%%%%%%%%%%%%%%%%%%%%%%%%%%%%%%%%
\section{Збуджуючий імпульс довільної форми}

Узагальнюючи принцип суперпозиції...

Користуючитсь інтегралом Дюамеля \cite[ст. 40]{imp:Kharkevich1950} можно 
отримати випромінювання...
%
\begin{equation}
\vect{E} = \int_{0}^{t} f(\tau) \vect{E_0} (t - \tau) d \tau,
\end{equation}
%
де $ \vect{E_0} $ - імпльсна характеристика антенни, а $ f(\tau) $ - 
плавна функція часу довільного виду. Користуючись переставною властивістью 
інтегралу цю формулу
%
Чим більше кінець відгуку схожий на початок тим ближче точка спостереження
до осі випромінювання. Це справедливо для будьякої антенни та довільної форми
імпульсу.

%%%%%%%%%%%%%%%%%%%%%%%%%%%%%%%%%%%%%%%%%%%%%%%%%%%%%%%%%%%%%%%%%%%%%%%%%%%%%%
\section{Енергія імпульсного випромінювання}

Побудова класичної енергетичної діаграми спрямованості мало інформативна 
для імпульсного випромінювання. Будуватимемо поперечні зрізи значень енергій
для різних довжин імпульсів та для різних відстаней від джерела.
Для збереженя кутового розміру зрізів візьмемо йщго $ z + 2 \cdot R $.
%
\begin{equation}
W = \int_{0}^{\infty} \vect{E}^2 (t) dt
\end{equation}
%
\textcolor{lightgray} { \begin{equation}
W = \int_{0}^{\infty} \left( E_\rho^2 + E_\varphi^2 + E_z^2 \right) dt
\end{equation} }
%
Якщо відомо час приходу імпульсу в точку спостереження та його тривалість, 
можна обмежити область інтегрування.
%
\begin{equation}
W = \int_{\tau_1}^{\tau_2} \left( E_\rho^2 + E_\varphi^2 + E_z^2 \right) dt
\end{equation}
%
\textcolor{lightgray} { \begin{equation*}
(\rho-R)^2 > v^2t^2 - z^2
\end{equation*} }
%
\textcolor{lightgray} { \begin{equation*}
v^2t^2 = (\rho-R)^2 + z^2
\end{equation*} }
%
\textcolor{lightgray} { \begin{equation*}
vt = \sqrt{(\rho-R)^2 + z^2}
\end{equation*} }
%
\begin{equation*}
\tau_1 = \sqrt{(\rho-R)^2 + z^2}
\end{equation*}
%
\begin{equation*}
\tau_2 = \tau + \sqrt{(\rho-R)^2 + z^2}
\end{equation*}
%
Для зручності застосування формули домножимо ліву і праву частину на
швидкість світла в середовищі $ v $ та згадаємо, що $ E_z = 0 $.
%
\begin{equation}
v W(r) = \int_{\tau_1}^{\tau_2} 
\Big( E_\rho^2 (vt,r) + E_\varphi^2 (vt,r) \Big) dvt
\end{equation}

\chapter{Розповсюдження випромінювання плаского диску в нелінійному середовищі}
\label{ch:nonlinear}

%%%%%%%%%%%%%%%%%%%%%%%%%%%%%%%%%%%%%%%%%%%%%%%%%%%%%%%%%%%%%%%%%%%%%%%%%%%%%%%%
\section{Матеріальні рівняння, як модель нелінійного середовища}

Взаємодію поля крізь середовище, завдяки теорії суперпозиції, можна
представити у вигляді додаткового стороннього джерела поля, що буде 
просторово розподілене в усій області розповсюдження породжувальної 
сильної хвилі. Для сильних хвиль, що мають імпульсну природу 
крім просторового розподілу доводиться розглядати, ще і причинний 
зв'язок. Назвемо таке джерело вторинним.

Розглянемо модель, де характер взаємодії електромагнітного поля і середовища 
задається в матеріальних рівняннях, де поляризація та намагніченість 
розглядаються, як \textcolor{red}{джерела} електромагнітної індукції.

\begin{equation*}
\vect{D} = \epsilon_0 \vect{E} + \vect{P} \left( \vect{E}, \vect{H} \right) =
\epsilon_0 \vect{E} + \epsilon_0 \chi_e \left( \vect{E}, \vect{H} \right)
\end{equation*}

\begin{equation*}
\vect{B} = \mu_0 \vect{H} + \mu_0 \vect{M} \left( \vect{E}, \vect{H} \right) =
\mu_0 \vect{H} + \mu_0 \chi_m \left( \vect{E}, \vect{H} \right)
\end{equation*}

Виключимо з розглядання гіротропні і не-хіральні середовища, тоді взаємний 
вплив магнітної та електричної індукції зникне і вектор поляризації стане 
функцією лише електричної напруженості, а намагніченість - лише магнітної
напруженості. Також, клас середовищ, що розглядається, обмежимо сталими, 
однорідними та ізотропними властивостями. Розглянемо середовище з лінійною 
магнітною індукцією

\begin{equation}
\vect{B} = \mu_0 \vect{H} + \mu_0 \chi_m \left( \vect{H} \right) =
\mu_0 \left( \chi_m + 1 \right) \vect{H} = \mu_0 \mu \vect{H}
\end{equation}
%
та нелінійною електричною індукцією

\begin{equation} \label{eq:d_voltera}
\vect{D} = \epsilon_0 \vect{E} + \epsilon_0 \chi_e \left( \vect{E} \right) = 
\epsilon_0 \vect{E} + \epsilon_0 \sum_{k=1}^{\infty} \int_0^t
\chi_e^{(k)} (\tau) \vect{E}^k (\tau) d \tau,
\end{equation}
%
де $\chi_e^{(k)} (t) $ коефіцієнти розкладу Вольтера нелінійної функції 
$ \chi_e $ по параметру $ \vect{E} $.

Розклад Вольтера ілюструє затримку у відгуках середовища на помірно сильні 
збудження. Розклад \eqref{eq:d_voltera} може застосовуватись у випадках,
коли породжуюче поле розповсюджуватись у середовищі не змінює його квантовий
стан. Такий тип нелінійності називається параметричним та випадку слабкої 
нелінійності. Тепер припустимо, що нелінійні ефекти в середовищі, що 
розглядається, не є інерційними за часом та породжують індукційний відгук 
миттєво. Тоді, від розкладу в ряд Вольтера перейдемо до Тейлорівського 
ряду, як моделі нелінійності:

\begin{equation} \label{eq:d_teilor}
\vect{D} = \epsilon_0 \vect{E} + 
\epsilon_0 \sum_{k=1}^{\infty} \chi_e^{(k)} \vect{E}^k (t).
\end{equation}

Розглянемо другий доданок в виразі \eqref{eq:d_teilor}: з точки зору 
матеріальних рівнянь він описує поляризаційні властивості середовища,
а згадуючи властивість симетрії вектору поляризації 
$ \vect{P} \left( \vect{E} \right) = - \vect{P} \left( - \vect{E} \right) $
лише непарні доданки ряду Тейлора можуть бути не нульовими. Також,
відокремивши лінійну поляризацію отримаємо вираз

\begin{equation} \label{eq:d_teilor_odd}
\vect{D} = \epsilon_0 \epsilon \vect{E} + 
\epsilon_0 \sum_{k=1}^{\infty} \chi_e^{(2k+1)} \vect{E}^{2k+1} (t).
\end{equation}

Для широкого класу прикладних задач \textcolor{red}{[UPSALA]} розглядається
лише перший нелінійний доданок розкладу. Таким чином отримаємо кубічну 
нелінійну складову вектору поляризації, що зустрічається в оптиці під 
назвою нелінійна поляризація Керра

\begin{equation} \label{eq:d_kerr}
\vect{D} = 
\epsilon_0 \epsilon \vect{E} + \epsilon_0 \chi_e^{(3)} \vect{E}^{3} = 
\epsilon_0 \epsilon \vect{E} + \vect{P}^\prime.
\end{equation}

Хоча розклад в ряд тейлора \eqref{eq:d_teilor_odd} не гарантує зменшення 
впливу кожного наступного доданку, тобто $ \chi_e^{(i)} < \chi_e^{(i+1)} $
на практиці врахування лише першого нелінійного доданку найчастіше дає
гарну точність до наближення слабкої нелінійності і врахуванням доданків
вищих порядків нехтують.

\textcolor{red}{ Які нелінійні ефекти спостерігаються в 
керрівському середовищі?}

Перша складова вектору електричної індукції \eqref{eq:d_kerr} відповідає 
полю при лінійному наближенні $ \vect{E} $, що породжене деяким 
струмом $ \vect{J} $, а другий доданок відповідає нелініному Керрівському 
відгуку середовища, який згідно з принципом суперпозиції, можна розглянути, 
як деяке поле $ \vect{E}^\prime $, породжене додатковим розподілом струму 
зміщення $ \vect{J}^\prime $ (далі вторинне джерело). Тоді, відповідно до
аналогії зі струмом зміщення, індукований нелінійний вторинний струм

\begin{equation} \label{eq:j_kerr}
\vect{J^\prime} = \partder{\vect{P}^\prime}{t} = 
\epsilon_0 \chi_e^{(3)} \partder{\vect{E}^3}{t}.
\end{equation}

Згідно описаної моделі, деяке стороннє джерело $ \vect{J} $ породжує 
імпульсне електромагнітне поле $ \vect{E} $ та $ \vect{H} $ розраховане
з припущенням лінійності електромагнітної індукції. Це лінійне поле, 
розповсюджуючись зі втратами крізь середовище, формує струм зміщення 
$ \vect{J}^\prime $, який в свою чергу є джерелом поля $ \vect{E}^\prime $ 
та $ \vect{H}^\prime $. Тоді, нелінійна самодія хвилі крізь середовище, 
згідно принципу суперпозиції $ \vect{E} + \vect{E}^\prime $ та 
$ \vect{H} + \vect{H}^\prime $.

Вторинне джерело $ \vect{J}^\prime $ забирає енергію породжуючої хвилі та 
формується частиною енергії втрат $ \vect{E} $, що характеризують середовище.

\textcolor{red}{ Розглянута модель, не пояснює нелінійні ефекти в вакуумі. 
Можливо, що природа нелінійних ефектів значно глибша і їх ефект впливає на 
характер енергетичної взаємодії та змінює простір. Відповідно довжина вектора 
в декартовому сенсі втрачає зміст. }

Вторинне джерело поля не є реальним джерелом, в прямому розумінні. 
Джерело $ \vect{J^\prime} $ наближено моделює нелінійну природу фізичних 
явищ розповсюдження сильних електромагнітних хвиль. Згадуючи всі обмеження,
які були велені при побудові моделі зазначимо властивості середовищ для 
яких цю модель можна застосовувати... \textcolor{red}{TODO...}

Розглянемо в якості породжувальної хвилі $ \vect{E} $ поле породжене 
пласким диском електричного струму з часовою залежністю у вигляді 
функції Хевісайда - моментальний стрибок амплітуди струму від нуля до 
значення $ A_0 $.

\textcolor{blue}{ \begin{equation*} 
\vect{J^\prime} = 
\vect{\rho_0}    \partder{}{t} P_\rho^\prime    \left( \vect{E} \right) + 
\vect{\varphi_0} \partder{}{t} P_\varphi^\prime \left( \vect{E} \right) + 
\vect{z_0}       \partder{}{t} P_z^\prime       \left( \vect{E} \right) 
\end{equation*} }
%
\textcolor{blue}{ \begin{equation*}
\vect{P^\prime} \left( \vect{E} \right) = \epsilon_0 \chi_e^{(3)} 
\dotprod{ \vect{E} }{ \vect{E} } \cdot \vect{E} 
\end{equation*} }
%
\textcolor{blue}{ \begin{equation*} \begin{aligned}
\vect{P^\prime} \left( \vect{E} \right) = 
\frac{ {A_0}^3 \epsilon_0 \chi_e^{(3)} }{ 8 } \left( \frac{\mu_0 \mu}
{\epsilon_0 \epsilon} \right)^{3/2} \left( {I_1}^2 \cos^2 \varphi + 
\left( I_2 - I_1 \right)^2 \sin^2 \varphi \right) \cdot \\ 
\cdot \Big( \vect{\rho_0} I_1 \cos \varphi - 
\vect{ \varphi_0 } \left( I_2 - I_1 \right) \sin \varphi \Big)
\end{aligned} \end{equation*} }
%
\textcolor{blue}{ \begin{equation*} \begin{aligned}
\vect{P^\prime} \left( \vect{E} \right) = 
\frac{ {A_0}^3 \epsilon_0 \chi_e^{(3)} }{ 8 } \left( \frac{\mu_0 \mu}
{\epsilon_0 \epsilon} \right)^{3/2} \left( {I_1}^2 \cos^2 \varphi + 
\left( I_2 - I_1 \right)^2 \sin^2 \varphi \right) \cdot \\ 
\cdot \Big( \vect{\rho_0} I_1 \cos \varphi - 
\vect{ \varphi_0 } \left( I_2 - I_1 \right) \sin \varphi \Big)
\end{aligned} \end{equation*} }
%
\textcolor{blue}{ \begin{equation*}
\vect{E} = \frac{A_0}{2} \sqrt{\frac{\mu_0 \mu}{\epsilon_0 \epsilon}}
\Big( \vect{\rho_0} I_1 \cos \varphi - 
\vect{ \varphi_0 } \left( I_2 - I_1 \right) \sin \varphi \Big)
\end{equation*} }
%
\textcolor{blue}{ \begin{equation*} \begin{aligned}
\vect{E}^2 = \frac{A_0^2}{4} \frac{\mu_0 \mu}{\epsilon_0 \epsilon}
\Big( I_1^2 \cos^2 \varphi + \left( I_2 - I_1 \right)^2 \sin^2 \varphi \Big)
\end{aligned} \end{equation*} }
%
\textcolor{blue}{ \begin{equation*} \begin{aligned}
\partder{ \vect{E}^2 }{t} = \frac{A_0^2}{4} 
\frac{\mu_0 \mu}{\epsilon_0 \epsilon}
\left( 2 I_1 \partder{I_1}{t} \cos^2 \varphi + 
2 ( I_2 - I_1 ) \left( \partder{I_2}{t} - \partder{I_1}{t} \right) 
\sin^2 \varphi \right)
\end{aligned} \end{equation*} }

Користуючись виразом для вторинного струму Керра \eqref{eq:j_kerr} та 
напруженістю електричного поля \eqref{eq:linear_e_cyl}, запишемо компоненти 
струму.

\textcolor{blue} { \begin{equation*} \begin{aligned}
\partder{P_\rho^\prime}{t}   = \frac{ {A_0}^3 \epsilon_0 \chi_e^{(3)} }{ 8 } 
\left( \frac{\mu_0 \mu} {\epsilon_0 \epsilon} \right)^{3/2} \left(
\left( {I_1}^2 \cos^2 \varphi + ( I_2 - I_1 )^2 \sin^2 \varphi \right)
\partder{I_1}{t} \cos \varphi + \right. \\
\left. + I_1 \cos \varphi \left( 2 I_1 \partder{I_1}{t} \cos^2 \varphi + 
2 ( I_2 - I_1 ) \left( \partder{I_2}{t} - \partder{I_1}{t} \right) 
\sin^2 \varphi \right) \right) = \\ 
= \frac{ {A_0}^3 \epsilon_0 \chi_e^{(3)} }{ 8 } 
\left( \frac{\mu_0 \mu} {\epsilon_0 \epsilon} \right)^{3/2} \left(
\partder{I_1}{t} {I_1}^2 \cos^3 \varphi + \partder{I_1}{t} ( I_2 - I_1 )^2 
\cos \varphi \sin^2 \varphi + \right. \\
\left. + 2 {I_1}^2 \partder{I_1}{t} \cos^3 \varphi + 
2 I_1 ( I_2 - I_1 ) \left( \partder{I_2}{t} - \partder{I_1}{t} \right) 
\cos \varphi \sin^2 \varphi \right)
\end{aligned} \end{equation*} }
%
\begin{equation*} \begin{aligned}
\partder{P_\rho^\prime}{t} = \frac{ {A_0}^3 \epsilon_0 \chi_e^{(3)} }{ 8 } 
\left( \frac{\mu_0 \mu} {\epsilon_0 \epsilon} \right)^{3/2} \left(
3 {I_1}^2 \partder{I_1}{t} \cos^3 \varphi + \right. \\
+ \left. ( I_2 - I_1 ) \cos \varphi \sin^2 \varphi \left( 
\partder{I_1}{t} ( I_2 - I_1 ) + 2 I_1 \left( \partder{I_2}{t} - 
\partder{I_1}{t} \right) \right) \right)
\end{aligned} \end{equation*}
%
\textcolor{blue} { \begin{equation*} \begin{aligned}
\partder{P_\varphi^\prime}{t}   = 
- \frac{ {A_0}^3 \epsilon_0 \chi_e^{(3)} }{ 8 } 
\left( \frac{\mu_0 \mu} {\epsilon_0 \epsilon} \right)^{3/2} \left(
\left( {I_1}^2 \cos^2 \varphi + ( I_2 - I_1 )^2 \sin^2 \varphi \right)
\left( \partder{I_2}{t} - \partder{I_1}{t} \right) \sin \varphi + \right. \\
\left. + (I_2 - I_1) \sin \varphi \left( 2 I_1 \partder{I_1}{t} \cos^2 \varphi + 
2 ( I_2 - I_1 ) \left( \partder{I_2}{t} - \partder{I_1}{t} \right) 
\sin^2 \varphi \right) \right) = \\ 
= - \frac{ {A_0}^3 \epsilon_0 \chi_e^{(3)} }{ 8 } 
\left( \frac{\mu_0 \mu} {\epsilon_0 \epsilon} \right)^{3/2} \left(
{I_1}^2 \left( \partder{I_2}{t} - \partder{I_1}{t} \right) 
\sin \varphi \cos^2 \varphi + \right. \\ \left. 
+ ( I_2 - I_1 )^2 \left( \partder{I_2}{t} - \partder{I_1}{t} \right) 
\sin^3 \varphi + 2 I_1 \partder{I_1}{t} (I_2 - I_1) 
\sin \varphi \cos^2 \varphi + \right. \\ 
+ \left. 2 ( I_2 - I_1 )^2 \left( \partder{I_2}{t} - \partder{I_1}{t} \right) 
\sin^3 \varphi \right)
\end{aligned} \end{equation*} }
%
\begin{equation*} \begin{aligned}
\partder{P_\varphi^\prime}{t} = 
- \frac{ {A_0}^3 \epsilon_0 \chi_e^{(3)} }{ 8 } 
\left( \frac{\mu_0 \mu} {\epsilon_0 \epsilon} \right)^{3/2} \left(
3 ( I_2 - I_1 )^2 \left( \partder{I_2}{t} - \partder{I_1}{t} \right)
\sin^3 \varphi \right. + \\
+ \left. I_1 \sin \varphi \cos^2 \varphi \left( 
I_1 \left( \partder{I_2}{t} - \partder{I_1}{t} \right) + 
2 \partder{I_1}{t} (I_2 - I_1) \right) \right)
\end{aligned} \end{equation*}

Так як поздовжня напруженість електричного поля відсутня 

\begin{equation*} \begin{aligned}
\partder{P_z^\prime}{t} = 0.
\end{aligned} \end{equation*}

Вирази для поперечних компонентів поля стають шматочно-визначеними,
через свої залежності від $ I_1 $ та $ I_2 $, а також від їх похідних в 
кожному з доданків, згрупованих по залежностях від азимутального кута.
Область визначення інтегралів $ I_1 $ та $ I_2 $ відома та має вигляд 
$ S_1 \cup S_2 \cup S_3 $, де кожна з під-областей 
\eqref{eq:s1zone}-\eqref{eq:s3zone} залежить від часу.
Тоді, строго виписана похідна міститиме дельта-функції в точках дотику
часово-просторових областей випромінювання $ S_1 $, $ S_2 $, $ S_3 $.
Користуючись неоднозначністю векторного потенціалу 
\cite[ст. 77]{imp:LandauII} звільнимося від дельта-функцій в виразі для 
похідних від $ I_1 $ та $ I_2 $, тоді

\begin{equation*} \begin{aligned}
\frac{1}{v} \partder{I_\alpha}{t} = 
\frac{1}{v} \partder{ I_\alpha \{ S_{2} \} }{t} 
\Big( H \left( vt^2 - z^2 - (\rho - R)^2 \right)  - 
H \left( vt^2 - z^2 - (\rho + R)^2 \right) \Big),
\end{aligned} \end{equation*}
%
де $ v = c/\sqrt{\epsilon \mu} $ - швидкість світла в середовищі при 
лінійному наближенні вектору поляризації. Таким чином область часу-простору,
де розподілений вторинний струм обмежена лише $ S_2 $, а у всіх точках 
спостереження, що відповідають співвідношенню

\begin{equation*} \begin{aligned}
S^\prime \in S_2 \subset (\rho-R)^2 < vt^2 - z^2 < (\rho+R)^2.
\end{aligned} \end{equation*}

Тоді явні вирази для похідних будуть:

\textcolor{blue}{ \begin{equation*} \begin{aligned}
I_1 \left\{ S_2 \right\} = \frac{\rho^2 + R^2}{4 \pi \rho^2} \arccos 
\frac{c^2 t^2 - z^2 - \rho^2 - R^2}{2 \rho R}  -
\frac{\sqrt{4 \rho^2 R^2 - (\rho^2 + R^2 - c^2t^2 + z^2)^2}}{4 \pi \rho^2} - \\
- \frac{ |\rho^2 - R^2| }{2 \pi \rho^2} 
\arctan \sqrt{ \frac{(\rho - R)^2}{(\rho + R)^2} \cdot
\frac{\left( \rho + R \right)^2 - \left( c^2t^2 - z^2 \right)} 
{\left( c^2t^2 - z^2 \right) - \left( \rho - R \right)^2} }
\end{aligned} \end{equation*} }
%
\textcolor{blue}{ \begin{equation*} \begin{aligned}
\partder{I_1 \left\{ S_2 \right\}}{t} = \frac{\rho^2 + R^2}{4 \pi \rho^2}
\partder{}{t} \arccos \frac{c^2 t^2 - z^2 - \rho^2 - R^2}{2 \rho R} - \\
- \partder{}{t} \frac{\sqrt{4 \rho^2 R^2 - (\rho^2 + R^2 - c^2t^2 + z^2)^2}}
{4 \pi \rho^2} - \\ - \frac{ |\rho^2 - R^2| }{2 \pi \rho^2} \partder{}{t} 
\arctan \sqrt{ \frac{(\rho - R)^2}{(\rho + R)^2} \cdot
\frac{\left( \rho + R \right)^2 - \left( c^2t^2 - z^2 \right)} 
{\left( c^2t^2 - z^2 \right) - \left( \rho - R \right)^2} }
\end{aligned} \end{equation*} }
%
\textcolor{blue}{ \begin{equation*} \begin{aligned}
\partder{}{t} \arccos \frac{c^2 t^2 - z^2 - \rho^2 - R^2}{2 \rho R} = 
- \frac{2 c^2 t}
{ \sqrt{4 \rho^2 R^2 - \left(c^2 t^2 - z^2 - \rho^2 - R^2 \right)^2} }
\end{aligned} \end{equation*} }
%
\textcolor{blue}{ \begin{equation*} \begin{aligned}
- \partder{}{t} \left( \rho^2 + R^2 - c^2t^2 + z^2 \right)^2 = 
- 2 (\rho^2 + R^2 -c^2t^2 + z^2) (-2 c^2 t)
\end{aligned} \end{equation*} }
%
\textcolor{blue}{ \begin{equation*} \begin{aligned}
\partder{}{t} \frac{\sqrt{4 \rho^2 R^2 - (\rho^2 + R^2 - c^2t^2 + z^2)^2}}
{4 \pi \rho^2} = \frac{1}{8 \pi \rho^2} 
\frac{ 4 c^2 t (\rho^2 + R^2 - c^2 t^2 + z^2) }
{ \sqrt{4 \rho^2 R^2 - (\rho^2 + R^2 - c^2t^2 + z^2)^2} } = \\
= \frac{c^2 t}{2 \pi \rho^2} \frac{\rho^2 + R^2 - c^2 t^2 + z^2}
{ \sqrt{4 \rho^2 R^2 - (\rho^2 + R^2 - c^2t^2 + z^2)^2} }
\end{aligned} \end{equation*} }
%
\textcolor{blue}{ \begin{equation*} \begin{aligned}
\partder{}{t} \arctan \sqrt{ \frac{x}{y} } = 
\frac{1}{1 + \frac{x}{y}} \frac{1}{2} 
\sqrt \frac{y}{x} \partder{}{t} \frac{x}{y}
\end{aligned} \end{equation*} }
%
\textcolor{blue}{ \begin{equation*} \begin{aligned}
- \frac{1}{(\rho + R)^2} + \frac{1}{(\rho - R)^2} = 
\frac{- (\rho-R)^2 + (\rho+R)^2 }{ (\rho^2 - R^2)^2 }
\end{aligned} \end{equation*} }
%
\textcolor{blue}{ \begin{equation*} \begin{aligned}
\partder{}{t} \arctan \sqrt{ \frac{(\rho - R)^2}{(\rho + R)^2}
\frac{\left( \rho + R \right)^2 - \left( c^2t^2 - z^2 \right)} 
{\left( c^2t^2 - z^2 \right) - \left( \rho - R \right)^2} } = 
\partder{}{t} \arctan \sqrt{ \frac
{1 - \frac{c^2t^2 - z^2}{\left( \rho + R \right)^2} } 
{ \frac{c^2t^2 - z^2}{ \left( \rho - R \right)^2 } - 1} } = \\
= \frac{1}{1 + \frac{1 - \frac{c^2t^2 - z^2}{\left( \rho + R \right)^2} } 
{ \frac{c^2t^2 - z^2}{ \left( \rho - R \right)^2 } - 1} } \frac{1}{2}
\sqrt{ \frac{ \frac{c^2t^2 - z^2}{ \left( \rho - R \right)^2 } - 1 }
{1 - \frac{c^2t^2 - z^2}{\left( \rho + R \right)^2} } } 
\frac{ - \frac{2 c^2 t}{\left( \rho + R \right)^2} 
\left( \frac{c^2t^2 - z^2}{ \left( \rho - R \right)^2} - 1 \right) - 
\frac{ 2 c^2 t }{ \left( \rho - R \right)^2 } 
\left( 1 - \frac{c^2t^2 - z^2}{\left( \rho + R \right)^2} \right) }
{\left( \frac{c^2t^2 - z^2}{ \left( \rho - R \right)^2 } - 1 \right)^2} = \\
= - c^2 t \frac{ \frac{c^2t^2-z^2}{(\rho-R)^2} - 1 }
{ \frac{c^2t^2-z^2}{(\rho-R)^2} - 1 + 1 - 
\frac{c^2t^2 - z^2}{\left( \rho + R \right)^2} }
\sqrt{ \frac{ \frac{c^2t^2 - z^2}{ \left( \rho - R \right)^2 } - 1}
{1 - \frac{c^2t^2 - z^2}{\left( \rho + R \right)^2} } } \frac
{ \frac{c^2t^2 - z^2}{ \left( \rho^2 - R^2 \right)^2 } - \frac{1}{(\rho+R)^2} + 
\frac{1}{(\rho-R)^2} - \frac{ c^2t^2 - z^2 }{ \left( \rho^2 - R^2 \right)^2 } }
{ \left( \frac{c^2t^2 - z^2}{ \left( \rho - R \right)^2} - 1 \right)^2 } = \\
= - \frac{4 \rho R c^2 t}{ \left( \rho^2 - R^2 \right)^2 } 
\frac{ 1 }{ \frac{c^2t^2-z^2}{(\rho-R)^2} - 
\frac{c^2t^2 - z^2}{\left( \rho + R \right)^2} }
\sqrt{ \frac{ \frac{c^2t^2 - z^2}{ \left( \rho - R \right)^2 } - 1}
{1 - \frac{c^2t^2 - z^2}{\left( \rho + R \right)^2} } } \frac
{ 1 }{ \frac{c^2t^2 - z^2}{ \left( \rho - R \right)^2} - 1 } = \\
= - \frac{c^2 t}{ c^2 t^2 - z^2 } \frac{1} { 
\sqrt{ 1 - \frac{c^2t^2 - z^2}{(\rho + R)^2 } } 
\sqrt{ \frac{c^2t^2 - z^2}{ (\rho - R)^2 } - 1} }
\end{aligned} \end{equation*} }
%
\textcolor{blue}{ \begin{equation*} \begin{aligned}
\partder{ I_1 \{ S_2 \} }{t} = - \frac{c^2 t}{2 \pi \rho^2}
\frac{\rho^2 + R^2}
{ \sqrt{4 \rho^2 R^2 - \left(c^2 t^2 - z^2 - \rho^2 - R^2 \right)^2} } - \\
- \frac{c^2 t}{2 \pi \rho^2} \frac{\rho^2 + R^2 - c^2 t^2 + z^2}
{ \sqrt{4 \rho^2 R^2 - (\rho^2 + R^2 - c^2t^2 + z^2)^2} } + \\ 
+ \frac{ c^2 t }{2 \pi \rho^2} \frac{|\rho^2 - R^2|}{ c^2 t^2 - z^2 } \frac{1} 
{ \sqrt{ 1 - \frac{c^2t^2 - z^2}{(\rho + R)^2 } } 
\sqrt{ \frac{c^2t^2 - z^2}{ (\rho - R)^2 } - 1} }
\end{aligned} \end{equation*} }
%
\begin{equation} \begin{aligned} \label{eq:i1_partder}
\frac{1}{v} \partder{ I_1 \{ S_2 \} }{t} = \frac{ vt }{2 \pi \rho^2} 
\frac{ (\rho^2 - R^2)^2  (v^2 t^2 - z^2)^{-1} } 
{ \sqrt{ (\rho + R)^2 - v^2t^2 + z^2 } 
\sqrt{ v^2t^2 - z^2 - (\rho - R)^2 } } - \\
- \frac{vt}{2 \pi \rho^2} \frac{2 (\rho^2 + R^2) - (v^2 t^2 - z^2)}
{ \sqrt{4 \rho^2 R^2 - (v^2t^2 - z^2 - \rho^2 - R^2)^2} };
\end{aligned} \end{equation}
%
\textcolor{blue}{ \begin{equation*} \begin{aligned}
\partder{ I_2 \{ S_2 \} }{t} = \frac{1}{\pi} \partder{}{t} \arccos 
\frac{c^2t^2 - z^2 + \rho^2 - R^2}{2 \rho \sqrt{c^2t^2 - z^2}} = \\
= - \frac{1}{\pi} \frac{1} { \sqrt{ 1 - \frac{ (c^2t^2 - z^2 + \rho^2 - R^2)^2 }
{4 \rho^2 (c^2t^2 - z^2)^2} } } \frac{1}{2 \rho} \partder{}{t} 
\frac{c^2t^2 - z^2 + \rho^2 - R^2} {\sqrt{c^2t^2 - z^2}} = \\
= - \frac{1}{2 \rho \pi} \frac{1} 
{ \sqrt{ 1 - \frac{ (c^2t^2 - z^2 + \rho^2 - R^2)^2 }
{4 \rho^2 (c^2t^2 - z^2)} } } \frac{2c^2t \sqrt{c^2t^2 - z^2} - 
\frac{c^2t}{\sqrt{c^2t^2 - z^2}} (c^2t^2 - z^2 + \rho^2 - R^2)
}{c^2t^2 - z^2} = \\ = - \frac{c^2 t}{2 \pi \rho} \frac{1} 
{ \sqrt{ 1 - \frac{ (c^2t^2 - z^2 + \rho^2 - R^2)^2 }
{4 \rho^2 (c^2t^2 - z^2)} } } \frac{2 \sqrt{c^2t^2 - z^2} - 
\frac{c^2t^2 - z^2 + \rho^2 - R^2}{\sqrt{c^2t^2 - z^2}}}{c^2t^2 - z^2} = \\
= - \frac{c^2 t}{2 \pi \rho (c^2t^2 - z^2)} \frac{ 2 \sqrt{c^2t^2 - z^2} - 
\frac{c^2t^2 - z^2 + \rho^2 - R^2}{\sqrt{c^2t^2 - z^2}} } 
{ \sqrt{ 1 - \frac{ (c^2t^2 - z^2 + \rho^2 - R^2)^2 }
{4 \rho^2 (c^2t^2 - z^2)} } } = \\
= - \frac{c^2 t}{\pi (c^2t^2 - z^2) } 
\frac{ 2 (c^2t^2 - z^2) - (c^2t^2 - z^2 + \rho^2 - R^2) } 
{ \sqrt{ 4 \rho^2 (c^2t^2 - z^2) - (c^2t^2 - z^2 + \rho^2 - R^2)^2 } } = \\
= - \frac{c^2 t}{\pi (c^2t^2 - z^2) } \frac{ c^2t^2 - z^2 -  \rho^2 + R^2 } 
{ \sqrt{ 4 \rho^2 (c^2t^2 - z^2) - (c^2t^2 - z^2 + \rho^2 - R^2)^2 } }
\end{aligned} \end{equation*} }
%
\begin{equation} \begin{aligned} \label{eq:i2_partder}
\frac{1}{v} \partder{ I_2 \{ S_2 \} }{t} = 
- \frac{vt}{\pi (v^2t^2 - z^2) } \frac{ v^2t^2 - z^2 - \rho^2 + R^2 } 
{ \sqrt{ 4 \rho^2 (v^2t^2 - z^2) - (v^2t^2 - z^2 + \rho^2 - R^2)^2 } }.
\end{aligned} \end{equation}

Графічно можемо визначити, що проекція вторинного струм обернена за знаком 
відносно тієї ж проекції компоненти напруженості електричного поля.

\begin{figure}[h] \begin{center}
\includegraphics[scale=0.5]{Jperp_A1}
\caption{Нелінійність амплітуди вторинного джерела}
\label{fig:jx_secondary}
\end{center} \end{figure}

З виразів \eqref{eq:i1_partder}, \eqref{eq:i2_partder} очевидна нелінійна 
залежність струму від $ A_0 $.

%%%%%%%%%%%%%%%%%%%%%%%%%%%%%%%%%%%%%%%%%%%%%%%%%%%%%%%%%%%%%%%%%%%%%%%%%%%%%%%%
\section{Енергетичний розподіл поля плаского диску в лінійному наближенні}

Вторинний електричний струм розподілений в усьому напівпросторі $ z > 0 $,
але за рахунок згасання енергії з відстанню в межах $ [1/R, 1/R^2] $ (в 
залежності від напрямку спостереження) можемо обмежити область де треба 
враховувати нелінійні ефекти за рахунок високої концентрації енергії. Для 
визначення параметричних меж застосування введеної моделі нелінійності та 
оцінки границі зони, де нелінійні ефекти треба враховувати, розглянемо 
енергетичні характеристики поля в ближній зоні

Тепер розглянемо енергетичний розподіл від поля плаского диску при різних
часових залежностей $ f(t) $ стороннього струму. Побудова класичної 
енергетичної діаграми спрямованості мало інформативне дослідження нелінійних
ефектів - в даній роботі важливим є енергетичний розподіл в ближній зоні.

Розглянемо густину енергії електромагнітного поля $ \vect{E} $, збудженого 
пласким диском електричного струму з довільною часовою залежністю $ f(t) $ за 
визначенням \cite{imp:Schantz2018}, нехтуючи при цьому енергетичним внеском 
$ \vect{H} $ спираючись на відсутність нелінійного внеску останньої.

\begin{equation} \label{eq:energy}
W = \frac{\epsilon_0}{2} \int_0^\infty \vect{E}^2 dt
\end{equation}

Користуючись властивостями перехідної функції можемо обмежити
область інтегрування за часом, а також спростити підінтегральний вираз

\begin{equation} \label{eq:energy}
W = \frac{\epsilon_0}{2} \frac{\mu_0 \mu}{\epsilon_0 \epsilon}
\int_{ct_1}^{c\tau_0+ct_3} \left( E_\rho^2 + E_\varphi^2 \right) dt
\end{equation}

Також оцінено енергію перехідної функції, а тобто \ref{eq:energy} при 
часовій залежності сигналу у вигляді функції Гевісайда $ f(t) = H(t) $:

\textcolor{blue}{ \begin{equation*} \begin{aligned}
\vect{E}^2 = \frac{A_0^2}{4} \frac{\mu_0 \mu}{\epsilon_0 \epsilon}
\Big( I_1^2 \cos^2 \varphi + \left( I_2 - I_1 \right)^2 \sin^2 \varphi \Big)
\end{aligned} \end{equation*} }
%
\begin{equation} \label{eq:energy_tr}
W_{tr} = \frac{\epsilon_0 A_0^2}{8} \frac{\mu_0 \mu}{\epsilon_0 \epsilon}
\int_{ct_1}^{ct_3}  \Big( I_1^2 \cos^2 \varphi + 
\left( I_2 - I_1 \right)^2 \sin^2 \varphi \Big) dt.
\end{equation}

Розглядаючи особливий випадок \ref{eq:energy_tr} для $ \rho = 0 $, помічаємо,
що енергія випромінювання плаского диску, завжди лежить у наступних межах для 
сигналів з часовою залежністю $ f(t) $ з областю значень 
$ \left[ -1, 1 \right] $ та тривалістю $ \tau_0 $:

\textcolor{blue}{ \begin{equation*}
\left. W \right|^{\rho=0} = \frac{\epsilon_0 A_0^2}{32} 
\frac{\mu_0 \mu}{\epsilon_0 \epsilon} \Big( \sqrt{R^2+z^2} - z \Big)
\end{equation*} }
%
\textcolor{blue}{ \begin{equation*}
\int_{ct_1}^{c\tau_0+ct_3} 
\left( \int_0^t f(\tau) d \tau < \tau_0 \right) dt < \tau_0 R
\end{equation*} }
%
\begin{equation}
0 \leq W_{max} \left( \tau_0, f(t), \vect{r} \right) < 
\frac{\epsilon_0 \tau_0 R A_0^2}{32} \frac{\mu_0 \mu}{\epsilon_0 \epsilon},
\end{equation}
%
де $ W_{max} $ - густина енергії, $ \tau_0 $ - ефективна тривалість імпульсу 
за визначеною метрикою, $ A_0 $ - максимальна амплітуда сигналу, $ R $ - 
радіус апертури, а вираз під коренем - імпеданс у середовищі розповсюдження 
хвилі.

Будуватимемо поперечні зрізи значень енергій
для різних довжин імпульсів та для різних відстаней від джерела.
Для збереження кутового розміру зрізів візьмемо його $ z + 2R $.

\textcolor{red}{ TODO: побудувати поздовжні та поперечні енергетичні зрізи
для декількох $ f(t) $ }

\textcolor{red}{ Як видно з рисунків, форма збудження має значний вплив на 
розподіл енергії у ближній зоні, де вклад нелінійної поправки найбільший, а 
отже обмежувати область врахування вторинного струму на основі енергетичних 
характеристик породжувальної хвилі треба з урахуванням часової залежності 
струму }

\textcolor{red}{ TODO: можна строго порівняти діаграму спрямованості 
отриману в часовій області з діаграмою Баума, а також оцінити її відхилення 
в ближній зоні }

%%%%%%%%%%%%%%%%%%%%%%%%%%%%%%%%%%%%%%%%%%%%%%%%%%%%%%%%%%%%%%%%%%%%%%%%%%%%%%%%
\section{Модовий розподіл Керрівської поправки}

Застосуємо метод еволюційних рівнянь для розв'язання задачі випромінювання 
вторинним струмом \eqref{eq:j_kerr}. Для цього, спершу, запишемо модовий
розподіл джерела, який є правою частиною рівняння Клейна-Гордона. 

\textcolor{blue} { \begin{equation*}
j_m \left( r, t; \nu \right) = \frac{\sqrt{\mu_0}}{2\pi} 
\int \limits_{0}^{2\pi} d \varphi \int \limits_0^\infty \rho d \rho 
\vect{j_0} \crossprod{ \nabla_\perp \Psi_m^* }{ \vect{z_0} }
\end{equation*} }
%
\textcolor{blue} { \begin{equation*} 
\vect{J^\prime} = 
\vect{\rho_0}    \partder{}{t} P_\rho^\prime    \left( \vect{E} \right) + 
\vect{\varphi_0} \partder{}{t} P_\varphi^\prime \left( \vect{E} \right) + 
\vect{z_0}       \partder{}{t} P_z^\prime       \left( \vect{E} \right) 
\end{equation*} }
%
\textcolor{blue} { \begin{equation*} \begin{aligned}
\crossprod{ \nabla_\perp \Psi_m^* }{ \vect{z_0} } =
- \vect{\rho_0} i m e^{-im\varphi} \frac{J_m (\nu \rho)}{\rho \sqrt{\nu}}
- \vect{\varphi_0} \sqrt{\nu} e^{-im\varphi} 
\frac{J_{m-1} (\nu \rho) - J_{m+1} (\nu \rho)}{2}
\end{aligned} \end{equation*} }
%
\textcolor{blue} { \begin{equation*} \begin{aligned}
\vect{J^\prime} \crossprod{ \nabla_\perp \Psi_m^* }{ \vect{z_0} } = 
- i e^{-im\varphi} m \frac{J_m (\nu \rho)}{\rho \sqrt{\nu}}
\partder{}{t} P_\rho^\prime \left( \vect{E} \right) - \\
- \sqrt{\nu} e^{-im\varphi} \frac{J_{m-1} (\nu \rho) - J_{m+1} (\nu \rho)}{2}
\partder{}{t} P_\varphi^\prime \right)
\end{aligned} \end{equation*} }

\begin{equation*} \begin{aligned}
j_m = - \frac{\sqrt{\mu_0}}{2\pi} 
\int_{0}^{2\pi} d \varphi \int \limits_{0}^{\infty} \rho d \rho
e^{-im\varphi} \left( i  m \frac{J_m (\nu \rho)}{\rho \sqrt{\nu}}
\partder{j_\rho^\prime}{t} + \sqrt{\nu}
\frac{J_{m-1} (\nu \rho) - J_{m+1} (\nu \rho)}{2}
\partder{j_\varphi^\prime}{t} \right)
\end{aligned} \end{equation*}

Спершу, перейдемо від комплексної області визначення модового розподілу 
до уявної. Згрупувавши доданки за тригонометричними функціями,
знайдемо інтеграли за азимутальним кутом $ \varphi $, користуючись 
аналітичними інтегралами \eqref{eq:int_exp3}, \eqref{eq:int_exp4}, 
\eqref{eq:int_exp5}, \eqref{eq:int_exp6}. Отримаємо вирази для інтегралів
від компонентів вектору вторинного нелінійного струму з ядром інтегралу у 
вигляді комплексної експоненти.

\textcolor{blue} { \begin{equation*} \begin{aligned}
\int_0^{2\pi} e^{-i m \varphi} \cos^3 \varphi d \varphi = 
\frac{\pi}{4} \delta_{m,-3} + \frac{\pi}{4} \delta_{m,3} + 
\frac{3 \pi}{4} \delta_{m,-1} + \frac{3 \pi}{4} \delta_{m,1}
\end{aligned} \end{equation*} }
%
\textcolor{blue} { \begin{equation*} \begin{aligned}
\int_0^{2\pi} e^{-i m \varphi} \cos \varphi \sin^2 \varphi d \varphi = 
\frac{\pi \delta_{m,1} }{4} + \frac{\pi \delta_{m,-1} }{4} - 
\frac{\pi \delta_{m,-3} }{4} - \frac{\pi \delta_{m,3} }{4}
\end{aligned} \end{equation*} }
%
\textcolor{blue} { \begin{equation*} \begin{aligned}
\int_{0}^{2 \pi} d \varphi e^{-im \varphi} \partder{P_\rho^\prime}{t} = 
\frac{ {A_0}^3 \epsilon_0 \chi_e^{(3)} }{ 8 } \int_{0}^{2\pi} d \varphi
e^{-im\varphi} \left( \frac{\mu_0 \mu} {\epsilon_0 \epsilon} \right)^{3/2} 
\left( 3 {I_1}^2 \partder{I_1}{t} \cos^3 \varphi + \right. \\
+ \left. ( I_2 - I_1 ) \cos \varphi \sin^2 \varphi \left( 
\partder{I_1}{t} ( I_2 - I_1 ) + 2 I_1 \left( \partder{I_2}{t} - 
\partder{I_1}{t} \right) \right) \right) = \\
= \frac{ {A_0}^3 \epsilon_0 \chi_e^{(3)} }{ 8 } 
\left( \frac{\mu_0 \mu} {\epsilon_0 \epsilon} \right)^{3/2}
\left( \frac{3 \pi}{4} {I_1}^2 \partder{I_1}{t} \left( \delta_{m,-3} + 
\delta_{m,3} + 3 \delta_{m,-1} + 3 \delta_{m,1} \right) + \right. \\
+ \frac{\pi }{4} \left. ( I_2 - I_1 ) \left( \delta_{m,1} + 
\delta_{m,-1} - \delta_{m,-3} - \delta_{m,3} \right) \left( 
\partder{I_1}{t} ( I_2 - I_1 ) + 2 I_1 \left( \partder{I_2}{t} - 
\partder{I_1}{t} \right) \right) \right)
\end{aligned} \end{equation*} }
%
\begin{equation*} \begin{aligned}
\frac{\epsilon_0 \chi_e^{(3)}}{2 \pi} \int_{0}^{2\pi} d \varphi 
e^{-im \varphi} \partder{P_\rho^\prime}{t} = 
\frac{ {A_0}^3 \epsilon_0 \chi_e^{(3)}}{ 64 } 
\left( \frac{\mu_0 \mu} {\epsilon_0 \epsilon} \right)^{3/2} \cdot \\ 
\cdot \left( 3 {I_1}^2 \partder{I_1}{t} \left( \delta_{m,-3} + 
\delta_{m,3} + 3 \delta_{m,-1} + 3 \delta_{m,1} \right) + \right. \\
+ \left. ( I_2 - I_1 ) \left( \delta_{m,1} + \delta_{m,-1} - 
\delta_{m,-3} - \delta_{m,3} \right) \left( 
\partder{I_1}{t} ( I_2 - I_1 ) + 2 I_1 \left( \partder{I_2}{t} - 
\partder{I_1}{t} \right) \right) \right)
\end{aligned} \end{equation*}
%
\textcolor{blue} { \begin{equation*} \begin{aligned}
\int_{0}^{2\pi} e^{-i m \varphi} \sin^3 \varphi d \varphi = 
\frac{3 \pi i}{4} \delta_{m,-1} - \frac{3 \pi i}{4} \delta_{m,1} - 
\frac{\pi i}{4} \delta_{m,-3} + \frac{\pi i}{4} \delta_{m,3}
\end{aligned} \end{equation*} }
%
\textcolor{blue} { \begin{equation*} \begin{aligned}
\int_0^{2\pi} e^{-i m \varphi} \sin \varphi \cos^2 \varphi d \varphi = 
\frac{\pi i }{4} \delta_{m,-1} - \frac{\pi i }{4} \delta_{m,1} -
\frac{\pi i }{4} \delta_{m,3} + \frac{\pi i }{4} \delta_{m,-3}
\end{aligned} \end{equation*} }
%
\textcolor{blue} { \begin{equation*} \begin{aligned}
\int_{0}^{2 \pi} d \varphi e^{-im \varphi} \partder{P_\varphi^\prime}{t} = \\
= - \frac{ {A_0}^3 \epsilon_0 \chi_e^{(3)} }{ 8 } 
\int_{0}^{2 \pi} d \varphi e^{-im \varphi}
\left( \frac{\mu_0 \mu} {\epsilon_0 \epsilon} \right)^{3/2} \left(
3 ( I_2 - I_1 )^2 \left( \partder{I_2}{t} - \partder{I_1}{t} \right)
\sin^3 \varphi \right. + \\
+ \left. I_1 \sin \varphi \cos^2 \varphi \left( 
I_1 \left( \partder{I_2}{t} - \partder{I_1}{t} \right) + 
2 \partder{I_1}{t} (I_2 - I_1) \right) \right) = 
- \frac{ {A_0}^3 \epsilon_0 \chi_e^{(3)} }{ 8 } \cdot \\ 
\cdot \left( \frac{\mu_0 \mu} {\epsilon_0 \epsilon} \right)^{3/2} \left(
\frac{3 \pi i}{4} ( I_2 - I_1 )^2 \left( \partder{I_2}{t} - 
\partder{I_1}{t} \right) \left( 3 \delta_{m,-1} - 3 \delta_{m,1} - 
\delta_{m,-3} + \delta_{m,3} \right) \right. + \\
+ \left. \frac{\pi i}{4} I_1 \left( \delta_{m,-1} - \delta_{m,1} - 
\delta_{m,3} + \delta_{m,-3} \right) \left( 
I_1 \left( \partder{I_2}{t} - \partder{I_1}{t} \right) + 
2 \partder{I_1}{t} (I_2 - I_1) \right) \right)
\end{aligned} \end{equation*} }
%
\begin{equation*} \begin{aligned}
\frac{\epsilon_0 \chi_e^{(3)}}{2 \pi} \int_{0}^{2 \pi} d \varphi 
e^{-im \varphi} \partder{P_\varphi^\prime}{t} = 
- \frac{ {A_0}^3 \epsilon_0 \chi_e^{(3)}  i}{ 64 }
\left( \frac{\mu_0 \mu} {\epsilon_0 \epsilon} \right)^{3/2} \cdot \\ 
\cdot \left( 3 ( I_2 - I_1 )^2 \left( \partder{I_2}{t} - 
\partder{I_1}{t} \right) \left( 3 \delta_{m,-1} - 3 \delta_{m,1} - 
\delta_{m,-3} + \delta_{m,3} \right) \right. + \\
+ \left. I_1 \left( \delta_{m,-1} - \delta_{m,1} - 
\delta_{m,3} + \delta_{m,-3} \right) \left( 
I_1 \left( \partder{I_2}{t} - \partder{I_1}{t} \right) + 
2 \partder{I_1}{t} (I_2 - I_1) \right) \right)
\end{aligned} \end{equation*}

Як видно з останніх виразів, інтригування за кутом $ \varphi $ дає дискретний 
модовий розподіл вторинного струму.  Також помічаємо, що у розподілах присутні 
лише моді з номерами $ \pm 1 $ та $ \pm 3 $ - вклад вторинного струму в інші
моди відсутній. Випишемо окрему кожну з ненульових мод розподілу вторинного 
струму. Для цього введемо нові змінні:

\begin{equation} \begin{aligned} \label{eq:alpha}
\alpha = 3 {I_1}^2 \partder{I_1}{t}
\end{aligned} \end{equation}

\begin{equation} \begin{aligned} \label{eq:beta}
\beta = ( I_2 - I_1 ) \left( \partder{I_1}{t} ( I_2 - I_1 ) + 
2 I_1 \left( \partder{I_2}{t} - \partder{I_1}{t} \right) \right)
\end{aligned} \end{equation}

\begin{equation} \begin{aligned} \label{eq:gamma}
\gamma = 3 ( I_2 - I_1 )^2 \left( \partder{I_2}{t} - \partder{I_1}{t} \right)
\end{aligned} \end{equation}

\begin{equation} \begin{aligned} \label{eq:lambda}
\lambda = I_1^2 \left( \partder{I_2}{t} - 
\partder{I_1}{t} \right) + 2 I_1 \partder{I_1}{t} (I_2 - I_1)
\end{aligned} \end{equation}

Запишемо модовий розподіл струму через нові позначення, спростивши вираз 
можна отримати:

\textcolor{blue} { \begin{equation*} \begin{aligned}
j_m = - \frac{\sqrt{\mu_0}}{2\pi} 
\int_0^{2\pi} d \varphi \int \limits_0^\infty \rho d \rho
e^{-im\varphi} \left( i m \frac{J_m (\nu \rho)}{\rho \sqrt{\nu}}
j_\rho^\prime + \sqrt{\nu}
\frac{J_{m-1} (\nu \rho) - J_{m+1} (\nu \rho)}{2}
j_\varphi^\prime \right)
\end{aligned} \end{equation*} }
%
\textcolor{blue} { \begin{equation*} \begin{aligned}
j_m = - \frac{\sqrt{\mu_0}}{2\pi} 
\int_0^{2\pi} d \varphi \int \limits_0^\infty \rho d \rho 
e^{-im\varphi} \cdot \\ \cdot 
\left( i \sqrt{\nu} \frac{J_{m-1} (\nu \rho) + J_{m+1} (\nu \rho)}{2}
j_\rho^\prime + \sqrt{\nu} 
\frac{J_{m-1} (\nu \rho) - J_{m+1} (\nu \rho)}{2}
j_\varphi^\prime \right)
\end{aligned} \end{equation*} }
%
\textcolor{blue} { \begin{equation*} \begin{aligned}
j_m = - \frac{\sqrt{\mu_0} \sqrt{\nu}}{4\pi} 
\int_0^{2\pi} d \varphi \int \limits_0^\infty \rho d \rho 
e^{-im\varphi} \Big(
J_{m-1} (\nu \rho) ( i j_\rho^\prime + j_\varphi^\prime ) +
J_{m+1} (\nu \rho) ( i j_\rho^\prime - j_\varphi^\prime ) \Big)
\end{aligned} \end{equation*} }
%
\textcolor{red} { \begin{equation} \begin{aligned}
j_1 = \frac{i A_0^3 \sqrt{\mu_0} \epsilon_0 \chi_e^{(3)} \sqrt{\nu}}{128}
\left( \frac{\mu_0 \mu}{\epsilon_0 \epsilon} \right)^{3/2}
\int_0^\infty \rho d \rho \cdot \\ \cdot
\Big( J_0 (\nu \rho) ( 3 \alpha + \beta + 3 \gamma + \lambda) - J_2 (\nu \rho)
( 3 \alpha + \beta - 3 \gamma - \lambda ) \Big)
\end{aligned} \end{equation} }
%
\textcolor{blue} { \begin{equation*} \begin{aligned}
j_{-1} = \frac{i A_0^3 \sqrt{\mu_0} \epsilon_0 \chi_e^{(3)}}{64}
\left( \frac{\mu_0 \mu}{\epsilon_0 \epsilon} \right)^{3/2}
\int_0^\infty \rho d \rho \cdot \\ \cdot
\Big( \frac{J_1 (\nu \rho)}{\rho \sqrt{\nu}}
( 3 \alpha + \beta ) - \sqrt{\nu}
\frac{J_0 (\nu \rho) - J_2 (\nu \rho)}{2}
( 3 \gamma + \lambda ) \Big)
\end{aligned} \end{equation*} }
%
\textcolor{blue} { \begin{equation*} \begin{aligned}
j_{-1} = \frac{i A_0^3 \sqrt{\mu_0} \epsilon_0 \chi_e^{(3)} \sqrt{\nu}}{128}
\left( \frac{\mu_0 \mu}{\epsilon_0 \epsilon} \right)^{3/2}
\int_0^\infty \rho d \rho \cdot \\ \cdot
\Big( J_0 (\nu \rho) ( 3 \alpha + \beta - 3 \gamma - \lambda ) + 
J_2 (\nu \rho) ( 3 \alpha + \beta + 3 \gamma + \lambda ) \Big)
\end{aligned} \end{equation*} }
%
\textcolor{red} { \begin{equation} \begin{aligned}
j_{-1} = j_{1}
\end{aligned} \end{equation} }
%
\textcolor{red} { \begin{equation} \begin{aligned}
j_3 = \frac{i A_0^3 \sqrt{\mu_0} \epsilon_0 \chi_e^{(3)}}{64}
\left( \frac{\mu_0 \mu}{\epsilon_0 \epsilon} \right)^{3/2}
\int_0^\infty \rho d \rho \cdot \\ \cdot
\left( 3 m \frac{J_3 (\nu \rho)}{\rho \sqrt{\nu}}
( \alpha - \beta ) - \sqrt{\nu}
\frac{J_2 (\nu \rho) - J_4 (\nu \rho)}{2}
( \gamma - \lambda ) \right)
\end{aligned} \end{equation} }
%
\textcolor{blue} { \begin{equation*} \begin{aligned}
j_{-3} = - \frac{i A_0^3 \sqrt{\mu_0} \epsilon_0 \chi_e^{(3)}}{64}
\left( \frac{\mu_0 \mu}{\epsilon_0 \epsilon} \right)^{3/2}
\int_0^\infty \rho d \rho \cdot \\ \cdot
\left( 3 m \frac{J_3 (\nu \rho)}{\rho \sqrt{\nu}}
( \alpha - \beta ) - \sqrt{\nu}
\frac{J_2 (\nu \rho) - J_4 (\nu \rho)}{2}
( \gamma - \lambda ) \right)
\end{aligned} \end{equation*} }
%
\textcolor{blue} { \begin{equation*} \begin{aligned}
j_{-3} = - \frac{i A_0^3 \sqrt{\mu_0} \epsilon_0 \chi_e^{(3)} \sqrt{\nu}}{128}
\left( \frac{\mu_0 \mu}{\epsilon_0 \epsilon} \right)^{3/2}
\int_0^\infty \rho d \rho \cdot \\ \cdot
\Big( J_2 (\nu \rho) ( \alpha - \beta - \gamma + \lambda) + 
J_4 (\nu \rho) ( \alpha - \beta + \gamma - \lambda) \Big)
\end{aligned} \end{equation*} }
%
\textcolor{red} { \begin{equation} \begin{aligned}
j_{-3} = - j_{3}
\end{aligned} \end{equation} }

\textcolor{red}{ TODO: побудувати залежність $ j_m / vt $ від 
$ v^2t^2 - z^2 $ та $ \nu $ для всіх $ m $ у вигляді heatmaps. }

Для чисельного розрахунку невласного інтегралу по $ \rho $, що містяться в 
модових розподілах струму $ j_m $ зручно звузити межі інтегрування 
користуючись областю визначення під-інтегральної функцій $ S_2 $:

\textcolor{blue} { \begin{equation*} \begin{aligned}
(\rho - R)^2 \leq v^2t^2 - z^2 \leq (\rho + R)^2
\end{aligned} \end{equation*} }
%
\textcolor{blue} { \begin{equation*} \begin{aligned}
| \rho - R | - \sqrt{v^2t^2 - z^2} \leq 0 \leq \rho + R - \sqrt{v^2t^2 - z^2}
\end{aligned} \end{equation*} }
%
\textcolor{blue} { \begin{equation*} \begin{aligned}
- R - \sqrt{v^2t^2 - z^2} \leq - \rho \leq R - \sqrt{v^2t^2 - z^2}
\end{aligned} \end{equation*} }
%
\textcolor{blue} { \begin{equation*} \begin{aligned}
R + \sqrt{v^2t^2 - z^2} \geq \rho \geq - R + \sqrt{v^2t^2 - z^2}
\end{aligned} \end{equation*} }
%
\begin{equation} \begin{aligned}
\left| \sqrt{v^2t^2 - z^2} - R \right| \leq \rho \leq \sqrt{v^2t^2 - z^2} + R.
\end{aligned} \end{equation}

Задля відокремлення розмірних коефіцієнтів перевизначимо модовий розподіл 
струму:

\begin{equation} \begin{aligned}
j_m = - \frac{i A_0^3 \sqrt{\mu_0} \epsilon_0 \chi_e^{(3)} \sqrt{\nu}}{128}
\left( \frac{\mu_0 \mu}{\epsilon_0 \epsilon} \right)^{3/2} \hat{j_m}.
\end{aligned} \end{equation}

Так як поздовжня компонента вторинного струму $ J_z $ відсутня, рівняння 
Клейна-Гордона відносно поздовжнього електричного еволюційного коефіцієнту 
є однорідним, що згідно методом функції Рімана дає нульовий розв'язок для
цього коефіцієнту. Отже, як і у лінійному наближенні, електромагнітне поле 
з урахуванням ефектів слабкої нелінійності залишається ТЕ типу:

\textcolor{blue}{ \begin{equation*}
- \partial_{ct}(\mu I_n^e) - \partial_z V_n^e + \chi^2 e_n = 0
\end{equation*} }
%
\textcolor{blue}{ \begin{equation*}
\frac{\epsilon \mu}{ \sqrt{\epsilon_0 \mu_0}} 
\frac{\partial^2 e_n}{\partial t^2} - 
\frac{\partial^2 e_n}{\partial z^2} + \chi^2 e_n = 
- \frac{\sqrt{\mu_0}}{2 \pi c} 
\int_0^{2\pi} d \varphi 
\int_0^\infty \rho d \rho \Phi_n^* (\chi) \partder{J_z}{t} = 0
\end{equation*} }
%
\textcolor{blue}{ \begin{equation*}
e_n (z, t; \chi) = \iint_S j_n (t',z', \chi) G(t,t',z,z') dt' dz' = 0
\end{equation*} }
%
\textcolor{blue}{ \begin{equation*}
I_n^e = - \partial_{ct} (\epsilon e_n) - 
\frac{\sqrt{\mu_0}}{2 \pi} \int_0^{2\pi} d \varphi 
\int_0^{\infty} \rho d \rho \Phi_n^* (\chi) J_z
\end{equation*} }
%
\textcolor{blue}{ \begin{equation*}
\partial_{z} e_n = V_n^e
\end{equation*} }
%
\begin{equation} \label{eq:e_evolution}
e_n (z, t; \nu) = V_n^e (z, t; \nu) = I_n^e (z, t; \nu) = 0
\end{equation}

Пошук виразів для еволюційних коефіцієнтів починаємо з поздовжнього 
магнітного коефіцієнту $ h_m $. Для розглянутої фізичної моделі ізотропного
та стаціонарного середовища без втрат коефіцієнт $ h_m $ є розв'язком 
рівняння Клейна-Гордона. В лінійному випадку це рівняння містить електричну 
і магнітну сприйнятливості; для нелінійної нотації скористаємось поняттям 
ефективної сприйнятливості та методикою запропонованою R. Ziolkowski
\cite{imp:Ziolkowski1993}.

\begin{equation} \label{eq:klein_gordon_nl}
\frac{(\epsilon + \chi_e^{(3)}) \mu}{c^2} 
\frac{\partial^2 h_m}{\partial t^2} - 
\frac{\partial^2 h_m}{\partial z^2} + 
\nu^2 h_m = j_m (t',z'; \nu),
\end{equation}
%
де $ j_m (t',z'; \nu) $ - мода $ m $ дискретного розподілу стороннього 
джерела, $ \chi_e^{(3)} $ - відносна нелінійна електрична сприйнятливість 
середовища третього порядку, а величина $ \epsilon + \chi_e^{(3)} $ в 
літературі зустрічається, як ефективна нелінійна Керрівська сприйнятливість 
\cite{imp:Ziolkowski1993}. Тоді, розв'язком \eqref{eq:klein_gordon_nl} за 
методом функції Рімана буде \eqref{eq:klein_gordon_sol} з ядром у вигляді 
функції Рімана

\begin{equation*}
G(t,t',z,z') = \frac{v}{2} H \left( v (t-t') - (z-z') \right)
J_0 \left( \nu \sqrt{v^2 (t-t')^2 - (z-z')^2} \right),
\end{equation*}
%
де $ v $ - швидкість світла в середовищі з урахуванням нелінійного 
Керрівського сповільнення

\begin{equation}
v = \frac{c}{\sqrt{ \left(\epsilon + \chi_e^{(3)}\right) \mu}} = 
\left( \epsilon_0 
\left( \epsilon + \chi_e^{(3)} \right) \mu_0 \mu \right)^{-1/2}.
\end{equation}

Для отримання нелінійних поправок до напруженості електричного поля достатньо 
поперечного модового коефіцієнту $ V_m^h $, який лінійно залежить від $ h_m $

\textcolor{blue}{ \begin{equation*} 
h_m (z, t; \nu) = \iint_S j_m (t',z') G(t,t',z,z') dt' dz',
\end{equation*} }
%
\textcolor{blue}{ \begin{equation*}
V_m^h = - \mu \partial_{ct} (h_m)
\end{equation*} }
%
\textcolor{blue} { \begin{equation*} \begin{aligned} 
V_m^h = - \mu \partial_{ct} \int_0^\infty \int_0^\infty j_m (t',z') G dz' dt'
\end{aligned} \end{equation*} }
%
\textcolor{blue} { \begin{equation*} \begin{aligned} 
V_m^h = - \frac{v \mu}{2c}
\partder_{t} \int_0^\infty dz' \int_0^\infty 
dt' H \left( v (t-t') - (z-z') \right) \cdot \\
\cdot J_0 \left( \nu \sqrt{v^2 (t-t')^2 - (z-z')^2} \right) j_m (t',z')
\end{aligned} \end{equation*} }
%
\textcolor{blue} { \begin{equation*} \begin{aligned} 
V_m^h = - \frac{1}{2} \sqrt{\frac{\mu}{\epsilon + \chi_e^{(3)}}}
\partder_{t} \int_0^\infty dz' \int_0^\infty 
dt' H \left( v (t-t') - (z-z') \right) \cdot \\
\cdot J_0 \left( \nu \sqrt{v^2 (t-t')^2 - (z-z')^2} \right) j_m (t',z')
\end{aligned} \end{equation*} }
%
\textcolor{blue} { \begin{equation*} \begin{aligned} 
H \left( v (t-t') - z + z' \right) = 
H \left( - vt' - ( z - vt - z' ) \right) = \\
= H \left( vt' + ( z - vt - z' ) \right) = 
H \left( vt' - ( vt - z + z' ) \right)
\end{aligned} \end{equation*} }
%
\begin{equation} \begin{aligned} 
V_m^h = \frac{i A_0^3 \epsilon_0 \chi_e^{(3)}}{2^8}
\sqrt{\frac{\mu_0 \mu}{\epsilon + \chi_e^{(3)}}} 
\left( \frac{\mu_0 \mu}{\epsilon_0 \epsilon} \right)^{3/2} \sqrt{\nu} 
\cdot \\ \cdot \partder{}{t} \int_0^\infty dz' \int_0^{vt - z + z'} dt'
J_0 \left( \nu \sqrt{v(t-t')^2 - (z-z')^2} \right) \hat{j_m} (vt',z')
\end{aligned} \end{equation}

Тепер, користуючись правилом інтегрування Лейбніца, спростимо отриманий 
вираз взявши аналітично похідну за часом:
%
\textcolor{blue} { \begin{equation*} \begin{aligned}
\partder{}{\tau} 
J_0 \left( \nu \sqrt{\Delta \tau^2 - \Delta z^2} \right) = 
- \nu \Delta \tau 
\frac{J_1 \left( \nu \sqrt{\Delta \tau^2 - \Delta z^2} \right)}
{\sqrt{\Delta \tau^2 - \Delta z^2}}
\end{aligned} \end{equation*} }
%
\textcolor{blue} { \begin{equation*} \begin{aligned}
\partder{}{\theta} \int_{a(\theta)}^{b(\theta)} f(x,\theta) dx = 
\int_{a(\theta)}^{b(\theta)} \partder{f}{\theta} dx + 
f\big( b(\theta), \theta \big) \partder{b}{\theta} -
f\big( a(\theta), \theta \big) \partder{a}{\theta}
\end{aligned} \end{equation*} }
%
\textcolor{blue} { \begin{equation*} \begin{aligned}
\partder_{t} \int_0^{vt - z + z'} dt'
J_0 \left( \nu \sqrt{v(t-t')^2 - (z-z')^2} \right) \hat{j_m} (vt',z') = \\
= \int_0^{vt - z + z'} dvt' \partder{J_0}{vt} \hat{j_m} (vt',z') + \\
+ \left. 
J_0 \left( \nu \sqrt{v(t-t')^2 - (z-z')^2} \right) \hat{j_m} (vt',z')
\right|^{vt' = vt - z + z'}
\end{aligned} \end{equation*} }
%
\textcolor{blue} { \begin{equation*} \begin{aligned}
\left. v(t-t')^2 - (z-z')^2 \right|^{vt' = vt - z + z'} = 
vt^2 - (vt - z + z')^2 - (z-z')^2 = \\
= vt^2 - (vt - z)^2 + 2 z' (vt - z) + z'^2 - (z-z')^2 = \\
= 2 vt z - z^2 + 2 z' (vt - z) + z'^2 - z^2 + 2 z z' - z'^2 = \\
= 2 vt z - 2 z^2 + 2 z' (vt - z) + 2 z z' = 
2 \big( z (vt - z) + z' (vt - z) + z z' \big) = \\
2 \big( z (vt - z) + z' (vt - z + z) \big) = 
2 \big( z (vt - z) + vt z' \big) = \\
= 2 \big( vt z - z^2 + vt z' \big) = 2 vt (z + z') - 2 z^2
\end{aligned} \end{equation*} }
%
\textcolor{blue} { \begin{equation*} \begin{aligned}
\partder_{t} \int_0^{vt - z + z'} dt'
J_0 \left( \nu \sqrt{v(t-t')^2 - (z-z')^2} \right) \hat{j_m} (vt',z') = \\
= - \nu v (t-t') 
\frac{J_1 \left( \nu \sqrt{v(t-t')^2 - (z-z')^2} \right)}
{\sqrt{v(t-t')^2 - (z-z')^2}} + \\
+ J_0 \left( \nu \sqrt{2 vt (z + z') - 2 z^2} \right) 
\hat{j_m} (vt - z + z',z')
\end{aligned} \end{equation*} }
%
\begin{equation} \begin{aligned} \label{eq:vmh_nl}
V_m^h = \frac{i A_0^3 \epsilon_0 \chi_e^{(3)}}{2^8}
\sqrt{\frac{\mu_0 \mu}{\epsilon + \chi_e^{(3)}}} 
\left( \frac{\mu_0 \mu}{\epsilon_0 \epsilon} \right)^{3/2} \sqrt{\nu}
\int_0^\infty dz' \cdot \\ \cdot 
\left\{ J_0 \left( \nu \sqrt{2 vt (z + z') - 2 z^2} \right) 
\hat{j_m} (vt - z + z',z') - \right. \\ 
\left. - \nu \int_0^\infty v (t-t') 
\frac{J_1 \left( \nu \sqrt{v(t-t')^2 - (z-z')^2} \right)}
{\sqrt{v(t-t')^2 - (z-z')^2}} \hat{j_m} (vt',z') dt' \right\}
\end{aligned} \end{equation}

\begin{equation} \begin{aligned} \label{eq:vmh_norm}
V_m^h = \frac{i A_0^3 \epsilon_0 \chi_e^{(3)}}{2^8}
\sqrt{\frac{\mu_0 \mu}{\epsilon + \chi_e^{(3)}}} 
\left( \frac{\mu_0 \mu}{\epsilon_0 \epsilon} \right)^{3/2} 
\sqrt{\nu} \hat{V_m^h}
\end{aligned} \end{equation}

Як видно з \eqref{eq:vmh_nl}, властивості симетрії відносно номеру моди 
$ m $ у еволюційних коефіцієнтів зберігаються відносно модового розподілу 
струму, тому:

\begin{equation} \begin{aligned} \label{eq:vp1_vm1}
V_1^h = V_{-1}^h
\end{aligned} \end{equation}

\begin{equation} \begin{aligned} \label{eq:vp3_vm3}
V_3^h = - V_{-3}^h
\end{aligned} \end{equation}

\textcolor{red} { \begin{equation*} \begin{aligned}
\frac{J_1 \left( \nu \sqrt{\Delta \tau^2 - \Delta z^2} \right)}
{\nu \sqrt{\Delta \tau^2 - \Delta z^2}} =
\frac{J_0 \left( \nu \sqrt{\Delta \tau^2 - \Delta z^2} \right) +
J_2 \left( \nu \sqrt{\Delta \tau^2 - \Delta z^2} \right)}{2}
\end{aligned} \end{equation*} }


%%%%%%%%%%%%%%%%%%%%%%%%%%%%%%%%%%%%%%%%%%%%%%%%%%%%%%%%%%%%%%%%%%%%%%%%%%%%%%%%
\section{Числовий розрахунок нелінійної поправки}

Для розв'язання задачі випромінювання у вільний простір, згідно методу 
модового базису, необхідно підставити в розклад компонентів поля знайдені 
методом еволюційних рівнянь коефіцієнти. Як було доведено в 
\eqref{eq:e_evolution}, всі електричні еволюційні коефіцієнти нульові,
а отже за визначенням \textcolor{red}{[ТРЕТЯКОВ]} поздовжня електрична
компонента відсутня, як і у лінійному випадку:

\begin{equation} \label{eq:ez_kerr}
E'_z = \frac{1}{\sqrt{\epsilon_0}} \sum_{n=-\infty}^{\infty}
\int_0^\infty \chi^2 d \chi e_n (\nu | vt, z) \Phi_n (\nu | \rho, \phi) = 0,
\end{equation}
%
де $ \Phi_n $ - базисна функція розкладу \textcolor{red}{[ТРЕТЯКОВ]}.

З \eqref{eq:ez_kerr} пласка TE хвиля залишається TE при розповсюдженні 
крізь нелінійне середовище при слабких ефектах самодії, навіть при 
нестаціонарному збудженні.

Поперечні електричні компоненти поля, в свою чергу, визначені наступним 
розкладом \textcolor{red}{[ТРЕТЯКОВ]}:

\begin{equation} \begin{aligned}
\vect{E_\perp} = \frac{1}{\sqrt{\epsilon_0}} \left( 
\sum \limits_{m=-\infty}^{\infty} \int \limits_{0}^{\infty} 
d \nu V_m^h \crossprod{ \nabla_\perp \Psi_m }{ \vect{z_0} } +
\sum \limits_{n=-\infty}^{\infty} \int \limits_{0}^{\infty}
d \chi V_n^e \nabla_\perp \Phi_n \right),
\end{aligned} \end{equation}
%
де $ \Psi_m $ та $ \Phi_n  $ є базисні функції розкладу поля, а $ V_m^h $
та $ V_n^e $ - еволюційні коефіцієнти, відомі з виразів \eqref{eq:e_evolution}
та \eqref{eq:vmh_nl}. Користуючись властивостями симетрії мод 
\eqref{eq:vp1_vm1} та \eqref{eq:vp3_vm3}, не важко помітити, що

%
\textcolor{blue} { \begin{equation*} \begin{aligned}
\crossprod{ \nabla_\perp \Psi_m }{ \vect{z_0} } = 
- e^{im\varphi} \left( \vect{\varphi_0} \sqrt{\nu} 
\frac{J_{m-1} (\nu \rho) - J_{m+1} (\nu \rho)}{2} - 
i m \vect{\rho_0} \frac{J_m (\nu \rho)}{ \rho \sqrt{\nu}} \right)
\end{aligned} \end{equation*} }
%
\textcolor{blue} { \begin{equation*} \begin{aligned}
E'_\rho = \frac{1}{\sqrt{\epsilon_0}} \sum_{m=-\infty}^{\infty} 
i m e^{im\varphi} \int_{0}^{\infty} \frac{d \nu}{\sqrt{\nu}} 
V_m^h \frac{J_m(\nu \rho)}{\rho}
\end{aligned} \end{equation*} }

\textcolor{blue} { \begin{equation*} \begin{aligned}
\sum_{m=-\infty}^\infty m e^{im \varphi} V_m^h J_m(\nu \rho) = \\ =
  e^{  i \varphi} V_{ 1}^h J_{ 1}(\nu \rho) - 
  e^{- i \varphi} V_{-1}^h J_{-1}(\nu \rho) + \\ +
3 e^{ 3i \varphi} V_{ 3}^h J_{ 3}(\nu \rho) - 
3 e^{-3i \varphi} V_{-3}^h J_{-3}(\nu \rho) = \\ =
\left( e^{ i\varphi} + e^{- i\varphi} \right) V_1^h J_1(\nu \rho) + 
3 \left( e^{3i\varphi} + e^{-3i\varphi} \right) V_3^h J_3(\nu \rho) = \\
= 2 \cos \varphi V_1^h J_1(\nu \rho) + 
6 \cos 3 \varphi V_3^h J_3(\nu \rho)
\end{aligned} \end{equation*} }

\begin{equation} \begin{aligned}
E'_\rho = \frac{2 i \cos \varphi}{\sqrt{\epsilon_0}}
\int_0^\infty \sqrt{\nu} d \nu V_1^h \frac{J_1(\nu \rho)}{\nu \rho} +
\frac{6 i \cos 3 \varphi}{\sqrt{\epsilon_0}}
\int_0^\infty \sqrt{\nu} d \nu V_3^h \frac{J_3(\nu \rho)}{\nu \rho}.
\end{aligned} \end{equation}

Тепер, користуючись раніше введеним нормуванням еволюційного коефіцієнта 
\eqref{eq:vmh_norm} запишемо вираз для нелінійної поправки до напруженості 
електричного поля $ E'_\rho $ в зручному для порівняння з лінійним наближенням
\eqref{eq:linear_e_cyl} вигляді:

\textcolor{blue} { \begin{equation*} \begin{aligned}
E'_\rho = - \frac{A_0^3 \epsilon_0 \chi_e^{(3)}}{2^7}
\sqrt{\frac{\mu_0 \mu}{\epsilon_0 \left( \epsilon + \chi_e^{(3)} \right)}} 
\left( \frac{\mu_0 \mu}{\epsilon_0 \epsilon} \right)^{3/2} \cdot \\
\cdot \int_0^\infty \nu d \nu \left(
\cos \varphi \hat{V_1^h} \frac{J_1(\nu \rho)}{\nu \rho} +
3 \cos 3\varphi \hat{V_3^h} \frac{J_3(\nu \rho)}{\nu \rho} 
\right)
\end{aligned} \end{equation*} }

\textcolor{blue} { \begin{equation*} \begin{aligned}
\epsilon_0 \chi_e^{(3)}
\sqrt{\frac{\mu_0 \mu}{\epsilon_0 \left( \epsilon + \chi_e^{(3)} \right)}} 
\left( \frac{\mu_0 \mu}{\epsilon_0 \epsilon} \right)^{3/2} 
\frac{\sqrt{\epsilon}}{\sqrt{\epsilon}} =
\frac{\epsilon_0 \sqrt{\epsilon} \chi_e^{(3)}}
{\sqrt{\epsilon + \chi_e^{(3)}}} 
\left( \frac{\mu_0 \mu}{\epsilon_0 \epsilon} \right)^2
\end{aligned} \end{equation*} }
%
\textcolor{blue} { \begin{equation*} \begin{aligned}
\frac{\sqrt{\epsilon}}
{\sqrt{\epsilon + \chi_e^{(3)}}} =
\frac{\sqrt{\epsilon \mu}}{c} 
\frac{c}{\sqrt{\mu \left( \epsilon + \chi_e^{(3)} \right)}} = 
\frac{v_{NL}}{v_{LN}}
\end{aligned} \end{equation*} }

\begin{equation} \begin{aligned} \label{eq:erho_kerr}
E'_\rho = - \frac{\epsilon_0 \chi_e^{(3)} A_0^3}{2^7}
\frac{v_{NL}}{v_{LN}}
\left( \frac{\mu_0 \mu}{\epsilon_0 \epsilon} \right)^2
\left(\hat{E}_\rho^{(1)} \cos \varphi +
\hat{E}_\rho^{(3)} \cos 3 \varphi \right),
\end{aligned} \end{equation}
%
де відношення $ v_{NL} / v_{LN} < 1 $ є коефіцієнтом нелінійного 
сповільнення породженої вторинної хвилі $ \vect{E'} $, а 
$ \hat{E}_\rho^{(m)} $ - функція, подібна за змістом та значенням до 
інтегральних виразів $ I_1 $ та $ I_2 $ з розв'язку у наближенні
лінійного розповсюджування \eqref{eq:linear_e_cyl}:

\begin{equation} \begin{aligned} \label{eq:erho_norm}
\hat{E}_\rho^{(m)} = \hat{E}_\rho^{(m)} (vt,\rho,z) = 
\int_0^\infty \nu d \nu \frac{m J_m(\nu \rho)}{\nu \rho} 
\hat{V_m^h} (\nu | vt,\rho,z).
\end{aligned} \end{equation}

Також серед множників помічаємо імпеданс вільного простору 
$ (\mu_0 \mu) / (\epsilon_0 \epsilon) $, куб максимальної амплітуди 
нестаціонарного струму плаского диску $ A_0^3 $ та абсолютну нелінійну 
сприйнятливість $ \epsilon_0 \chi_e^{(3)} $.

\textcolor{red}{РИСУНОК, ЩО ІЛЮСТРУЄ АМПЛІТУДНИЙ ЕФЕКТ САМОДІЮ ПРИ ВАРІЦІЇ 
ЛІНІЙНОГО КОЕФІЦІЄНТУ ВІДБИТТЯ}

З виразу \eqref{eq:erho_kerr} добре видно один з проявів амплітудної
самодії електромагнітного випромінювання. Як видно з лінійного розв'язку
збільшення коефіцієнту заломлення середовища "розтягує" нестаціонарний 
імпульс. Самодія цього ефекту у нелінійному середовищі зменшує 
максимальну амплітуду поля-поправки в 
$ \sqrt{\epsilon} / \sqrt{\epsilon} \chi_e^{(3)} $  разів, що відповідає 
швидкості світла з урахуванням нелінійної сприйнятливості до швидкості 
світла в середовищі у лінійному наближення. Варто зазначити, що цей ефект 
мало-значущій та складає лише $ 0.01\% $ від поля поправки, а тому ним 
можна знехтувати в цьому випадку. Варто перевірити його внесок при сильній 
нелінійній самодії. Даний ефект можна сприймати, як поправку до імпедансу 
вільного простору при врахуванні нелінійної сприйнятливості третього порядку:

\begin{equation*} \begin{aligned}
\frac{v_{NL}}{v_{LN}}
\left( \frac{\mu_0 \mu}{\epsilon_0 \epsilon} \right)^2 = 
\sqrt{\frac{\mu_0 \mu}{\epsilon_0 \left( \epsilon + \chi_e^{(3)} \right)}}
\sqrt[3]{\frac{\mu_0 \mu}{\epsilon_0 \epsilon}} 
\end{aligned} \end{equation*}

Розв'язок відносно нелінійної поправки \eqref{eq:erho_kerr} містить кутову 
залежність та константні коефіцієнти в явному вигляді. Інші змінні, тобто
$ vt, \rho, z $ представлені в $ \hat{E}_\rho^{(m)} $, що в свою чергу
є невласним кратним інтегралом дійсної області значень:

\textcolor{blue} { \begin{equation*} \begin{aligned}
\hat{V_m^h} = \int_0^z dz' 
\left\{ J_0 \left( \nu \sqrt{2 vt (z + z') - 2 z^2} \right) 
\hat{j_m} (vt - z + z',z') - \right. \\ 
\left. - \nu \int_0^{vt - z + z'} dvt' v (t-t') 
\frac{J_1 \left( \nu \sqrt{v(t-t')^2 - (z-z')^2} \right)}
{\sqrt{v(t-t')^2 - (z-z')^2}} \hat{j_m} (vt',z')  \right\}
\end{aligned} \end{equation*} }

\textcolor{red} { \begin{equation*} \begin{aligned}
j_1 = \frac{i A_0^3 \sqrt{\mu_0} \epsilon_0 \chi_e^{(3)} \sqrt{\nu}}{128}
\left( \frac{\mu_0 \mu}{\epsilon_0 \epsilon} \right)^{3/2}
\int_0^\infty \rho d \rho \cdot \\ \cdot
\Big( J_0 (\nu \rho) ( 3 \alpha + \beta + 3 \gamma + \lambda) - J_2 (\nu \rho)
( 3 \alpha + \beta - 3 \gamma - \lambda ) \Big)
\end{aligned} \end{equation*} }

\begin{equation} \begin{aligned}
\hat{V_m^h} = \int_{0}^{\infty} dz'
\left\{ J_0 \left( \nu \sqrt{2 vt (z + z') - 2 z^2} \right) 
\int_{0}^{\infty} \rho' d \rho'
f (\rho',vt - z + z',z') - \right. \\ 
\left. - \nu \int_{0}^{\infty} dvt' v (t-t') 
\frac{J_1 \left( \nu \sqrt{v(t-t')^2 - (z-z')^2} \right)}
{\sqrt{v(t-t')^2 - (z-z')^2}} 
\int_{0}^{\infty} \rho' d\rho'
f (\rho',vt',z')  \right\},
\end{aligned} \end{equation}
%
де $ f ( \nu | vt', \rho', z') $ - лінійна комбінація функцій виду
$ I_\alpha I_\beta \partder{I_\alpha}{vt} $ введених раніше 
\eqref{eq:alpha} - \eqref{eq:lambda}, як $ \alpha, \beta, \gamma, \lambda $:

\textcolor{red} { \begin{equation} \begin{aligned}
f ( \nu | vt', \rho', z') = 
J_0 (\nu \rho') (3 \alpha + \beta + 3 \gamma + \lambda) - \\
- J_2 (\nu \rho') (3 \alpha + \beta - 3 \gamma - \lambda).
\end{aligned} \end{equation} }

Інтегрування за штрихованими змінними є виконнанням принципу суперпозиції 
відносно точкових джерел у вигляді яких можна представити плаский диск, та
визначені відповідно у межах $ 0 \leq \rho' \leq \rho $,
$ 0 \leq z' \leq z $ та $ 0 \leq t' \leq t $. Користуючись цим, а також 
областю визначення під-інтегральних функцій та принципом причинності функції 
Рімана $ v(t-t')-(z-z') > 0 $, можна обмежити область інтегрування в 
останньому виразі:

\begin{equation} \begin{aligned}
0 \leq vt' \leq vt - z + z';
\end{aligned} \end{equation}

\begin{equation} \begin{aligned}
0 \leq z' \leq \min(z,2R),
\end{aligned} \end{equation}
%
де верхня межа значень $ z' $ також обмежена за рахунок попереднього аналізу 
енергетичних властивостей випромінювання в ближній зоні. Також, для окремих 
інтегралів за змінними $ \rho' $ область інтегрування буде різною. Для 
інтегралу в першому доданку

\begin{equation} \begin{aligned}
\left| \sqrt{v^2t'^2 - z'^2} - R \right| \leq \rho' \leq 
\sqrt{v^2t'^2 - z'^2} + R,
\end{aligned} \end{equation}
%
а в другому при $ vt' = vt - z + z' $, відповідно:

\textcolor{blue} { \begin{equation*} \begin{aligned}
\left. v^2 t'^2 - z'^2 \right|^{vt' = vt - z + z'} = 
(vt - z + z')^2 - z'^2 = (vt - z)^2 + 2 z' (vt - z) = \\
(vt - z) (vt - z + 2 z') = (vt - z) (vt + z') + (vt - z) (z + z')
\end{aligned} \end{equation*} }

\begin{equation} \begin{aligned}
\left| \sqrt{(vt - z) (vt - z + 2 z')} - R \right| \leq \rho' \leq 
\sqrt{(vt - z) (vt - z + 2 z')} + R.
\end{aligned} \end{equation}

Аналітичний розрахунок нормованого еволюційного коефіцієнту $ \hat{V_m^h} $
є недоцільним та може виявитись взагалі неможливим, тому залишаються лише
чисельні квадратурні методи розв'язку, які дадуть гарну точність в випадку,
кратних визначених інтегралів. Зовнішнім інтегралом в \eqref{eq:erho_kerr} є 
інтеграл за неперервним спектральним параметром $ \nu $. На жаль, фізичного
обґрунтування для обмеження цієї області інтегрування немає, а отже,
доцільно розділити чисельне розв'язання на два етапи:

\begin{enumerate}
	\item Чисельний розрахунок еволюційного коефіцієнту $ \hat{V_m^h} $ 
	для деякої області значень по параметру $ \nu $;
	\item Аналіз частотних характеристик під-інтегральної функції за $ \nu $ 
	та обмеження області інтегрування на основі аналізу, що дозволить 
	розрахувати абсолютну похибку чисельного методу.
\end{enumerate}

Аналогічно можна знайти поправку до $ \varphi $ проекції вектору 
напруженості електричного поля:

\textcolor{blue} { \begin{equation*} \begin{aligned}
E_\varphi = - \frac{1}{2 \sqrt{\epsilon_0}} \sum_{m=-\infty}^{\infty} 
e^{im\varphi} \int_{0}^{\infty} \sqrt{\nu} d \nu 
V_m^h \left( J_{m-1} (\nu \rho) - J_{m+1} (\nu \rho) \right)
\end{aligned} \end{equation*} }

\textcolor{blue} { \begin{equation*} \begin{aligned}
\sum_{m=-\infty}^\infty e^{im \varphi}
V_m^h \left( J_{m-1} (\nu \rho) - J_{m+1} (\nu \rho) \right) = \\
e^{  i \varphi} V_{ 1}^h \left( J_0 (\nu \rho) - J_2 (\nu \rho) \right) +
e^{- i \varphi} V_{-1}^h \left( J_2 (\nu \rho) - J_0 (\nu \rho) \right) + \\
e^{ 3i \varphi} V_{ 3}^h \left( J_2 (\nu \rho) - J_4 (\nu \rho) \right) +
e^{-3i \varphi} V_{-3}^h \left( J_4 (\nu \rho) - J_2 (\nu \rho) \right) = \\
= \left( e^{  i \varphi} - e^{- i \varphi} \right)
\left( J_0 (\nu \rho) - J_2 (\nu \rho) \right) +
\left( e^{ 3i \varphi} - e^{-3i \varphi} \right) 
\left( J_2 (\nu \rho) - J_4 (\nu \rho) \right) = \\
= 2i \sin \varphi \left( J_0 (\nu \rho) - J_2 (\nu \rho) \right) +
2i \sin 3 \varphi \left( J_2 (\nu \rho) - J_4 (\nu \rho) \right)
\end{aligned} \end{equation*} }

\textcolor{blue} { \begin{equation*} \begin{aligned}
E_\varphi =
- \frac{i \sin \varphi}{\sqrt{\epsilon_0}} \int_0^\infty d \nu
\sqrt{\nu} V_1^h \left( J_0 (\nu \rho) - J_2 (\nu \rho) \right) - \\
- \frac{i \sin 3 \varphi}{\sqrt{\epsilon_0}} \int_0^\infty d \nu
\sqrt{\nu} V_3^h \left( J_2 (\nu \rho) - J_4 (\nu \rho) \right)
\end{aligned} \end{equation*} }

\begin{equation} \begin{aligned} \label{eq:ephi_kerr}
E'_\varphi = \frac{\epsilon_0 \chi_e^{(3)} A_0^3}{2^7}
\frac{v_{NL}}{v_{LN}}
\left( \frac{\mu_0 \mu}{\epsilon_0 \epsilon} \right)^2
\left(\hat{E}_\varphi^{(1)} \sin \varphi +
\hat{E}_\varphi^{(3)} \sin 3 \varphi \right)
\end{aligned} \end{equation}
%
де 

\begin{equation} \begin{aligned} \label{eq:ephi_norm}
\hat{E}_\varphi^{(m)} (vt, \rho, z) = 
\int_0^\infty \nu d \nu V_m^h (\nu | vt, \rho, z)
\left( J_{m-1} (\nu \rho) - J_{m+1} (\nu \rho) \right)
\end{aligned} \end{equation}

В виразах \eqref{eq:erho_kerr} та \eqref{eq:ephi_kerr} спостерігається
утворення поля з тригонометричною кутовою залежністю кратних порядків.
Схожий ефект спостерігається, також, в задачах розповсюджування пласкої 
хвилі в частотних характеристиках при гармонійних часових залежностях.


Для аналізу та порівняння з лінійним розв'язком, зручно перейти до 
декартових проекцій вектору напруженості, тоді

\textcolor{blue} { \begin{equation*} \begin{aligned}
E'_x = E'_\rho \cos \varphi - E'_\varphi \sin \varphi
\end{aligned} \end{equation*} }
%
\textcolor{blue} { \begin{equation*} \begin{aligned}
E'_x = - \frac{\epsilon_0 \chi_e^{(3)} A_0^3}{2^7} \frac{v_{NL}}{v_{LN}}
\left( \frac{\mu_0 \mu}{\epsilon_0 \epsilon} \right)^2 \cdot \\ \cdot
\left(\hat{E}_\rho^{(1)} \cos^2 \varphi +
\hat{E}_\rho^{(3)} \cos \varphi \cos 3 \varphi - 
\hat{E}_\varphi^{(1)} \sin^2 \varphi -
\hat{E}_\varphi^{(3)} \sin \varphi \sin 3 \varphi \right)
\end{aligned} \end{equation*} }
%
\textcolor{blue} { \begin{equation*} \begin{aligned}
E'_x = - \frac{\epsilon_0 \chi_e^{(3)} A_0^3}{2^7} \frac{v_{NL}}{v_{LN}}
\left( \frac{\mu_0 \mu}{\epsilon_0 \epsilon} \right)^2 \cdot \\ \cdot
\int_0^\infty \nu d \nu V_1^h \left( 
\left( J_0 (\nu \rho) + J_2 (\nu \rho) \right) \cos^2 \varphi -
\left( J_0 (\nu \rho) - J_2 (\nu \rho) \right) \sin^2 \varphi \right) - \\
- \frac{\epsilon_0 \chi_e^{(3)} A_0^3}{2^7} \frac{v_{NL}}{v_{LN}}
\left( \frac{\mu_0 \mu}{\epsilon_0 \epsilon} \right)^2 \cdot \\ \cdot
\int_0^\infty \nu d \nu V_3^h \left( 
\left( J_2 (\nu \rho) + J_4 (\nu \rho) \right) \cos \varphi \cos 3 \varphi -
\left( J_2 (\nu \rho) - J_4 (\nu \rho) \right) \sin \varphi \sin 3 \varphi 
\right)
\end{aligned} \end{equation*} }

\textcolor{red} {TODO: ефект відставання та сповільнення хвилі}

\textcolor{red} {TODO: ефект самофокусування за рахунок залежності від кута}

%%%%%%%%%%%%%%%%%%%%%%%%%%%%%%%%%%%%%%%%%%%%%%%%%%%%%%%%%%%%%%%%%%%%%%%%%%%%%%%%
\section{Розповсюдження прямокутного імпульсу в нелінійному середовищі}

\textcolor{red} {TODO: Порушення закону збереження та принципу суперпозиції}

\textcolor{red} {TODO: Може вдається якось виділити залежність від 
тривалості імпульсу аналітично???}

%%%%%%%%%%%%%%%%%%%%%%%%%%%%%%%%%%%%%%%%%%%%%%%%%%%%%%%%%%%%%%%%%%%%%%%%%%%%%%%%
\section{Узагальнення для слабкої нелінійності}

Опираючись на геометрію джерела та на властивості модового базису можна 
довести, що 

\textcolor{red} { \begin{equation} \begin{aligned} \label{eq:erho_norm}
\hat{E}_\rho^{(1)} = \int_0^\infty \nu d \nu 
\frac{J_1(\nu \rho)}{\nu \rho} \hat{V_1^h} \approx
\int_0^\infty \nu d \nu \frac{J_1(\nu \rho)}{\nu \rho} 
\frac{J_1(\nu R) J_0(\nu \sqrt{v^2t^2-z^2})}{\nu} = I_1,
\end{aligned} \end{equation} }
%
отже компонент з лінійною залежністю від кута, повторює за формою 
імпульс отриманий у лінійному наближенні та менший за амплітудою на 
декілька порядків, а отже його внеском можна знехтувати.

\begin{equation*} \begin{aligned}
\lim_{n \to \infty} 
\sqrt{ \frac{\epsilon}{ \epsilon + \chi_e^{(2n+1)}} } = 1
\end{aligned} \end{equation*}

\textcolor{blue} { \begin{equation*} \begin{aligned}
E'_\rho = - \frac{\epsilon_0 \chi_e^{(3)} A_0^3}{2^7}
\frac{v_{NL}}{v_{LN}}
\left( \frac{\mu_0 \mu}{\epsilon_0 \epsilon} \right)^2
\left(\hat{E}_\rho^{(1)} \cos \varphi +
\hat{E}_\rho^{(3)} \cos 3 \varphi \right)
\end{aligned} \end{equation*} }

\textcolor{blue} { \begin{equation*} \begin{aligned}
E'_\rho = - \frac{\epsilon_0 \chi_e^{(3)} A_0^3}{2^7}
\frac{v_{NL}}{v_{LN}}
\left( \frac{\mu_0 \mu}{\epsilon_0 \epsilon} \right)^2 
\hat{E}_\rho^{(3)} \cos 3 \varphi
\end{aligned} \end{equation*} }

\begin{equation*} \begin{aligned}
E'_\rho = - \frac{1}{2} \sum_{n=1}^{\infty} 
\frac{\epsilon_0 \chi_e^{(2n+1)} A_0^{2n+1} }{ 4^{2n+1} }
\left( \frac{\mu_0 \mu}{\epsilon_0 \epsilon} \right)^{n+1}
\hat{E}_\rho^{(2n+1)} \cos (2n + 1) \varphi
\end{aligned} \end{equation*}

% \begin{tabular}{ | l | l | }
% \hline 
% Призначення змінної                          & Область визначенням       \\ 
% \hline
% Відстань, що проходить сигнал за час $t$     & $ 0 \le vt' \le vt $      \\ 
% \hline
% Відстань, від точки спостереження до джерела & $ 0 \le z' \le z $        \\  
% \hline
% Принцип причинності для проміжних подій      & $ 0 < vt - vt' - z + z' $ \\ 
% \hline
% Наслідок з 3 та 1                            & $ 0 < vt' < vt - z + z' $ \\ 
% \hline
% \end{tabular}

\chapter{Передача інформації у часовому просторі короткоімпульсними 
електромагнітними полями}
\label{ch:neuron}

%%%%%%%%%%%%%%%%%%%%%%%%%%%%%%%%%%%%%%%%%%%%%%%%%%%%%%%%%%%%%%%%%%%%%%%%%%%%%%%
\section{Недоліки класичного імпульсного радіо}

Велика інформаційна ємність імпульсного надширокосмугового випромінювання, 
а також низьке споживання енергії для випромінювання за рахунок 
короткотривалих збуджень робить його привабленим для технічного використання.
Новою сферою застосування послідовної надширокосмугової радіоелектроніки 
(DS-UWB) сьогодні стає інтернет речей. Основними чинниками для цього стали 
високий рівень інформаційної безпеки, порівняно низький рівень споживання 
електроенергії та стійкість до вузькосмугових завад. 
В задачах інтернету речей, робота таких пристроїв на маленькій відстані є, 
скоріш, перевагою, а ніж недоліком. Цей чинник зменшує радіо-забруднення 
приміщення за рахунок низьких потужностей та напрямленності випромінювання.

В оглядовій роботі \cite{imp:ChannelImplementation} приведені основні схеми
роботи імпульсного радіо послідовної передачі (без застосування 
стробоскопічного принципу), що використовуються в різноманітних сферах, 
починаючи з телекомунікації і закінчуючи радарними задачами, які можна 
узагальнити наступною схемою (Рис.~\ref{fig:emp_radio}).

\begin{figure}[htbp] \begin{center}
\includegraphics[scale=0.45]{classical_radio}
\caption{Класична схема імпульсного радіо} \label{fig:emp_radio}
\end{center} \end{figure}

Існуючі принципи аналогової обробки прийнятого імпульсного радіосигналу 
успадковані від схемотехніки, що використовувались для гармонійних сигналів 
\cite{imp:ComunicationsOverview}: для обробки отриманого з антени 
електричного струму використовується послідовна фільтрація та підсилення з 
подальшим оцифровуванням за допомогою АЦП і цифрової обробки в модулях FPGA 
(Рис.~\ref{fig:emp_radio}).

Кожен з послідовних етапів аналогової обробки, направлений на покращення 
окремої характеристики сигналу, неминуче впливає і на інші його характеристики,
що накопичує похибку та губить частину інформації про сигнал:

\begin{enumerate}
	\item фільтрація разом з інформацією про шум частково зачіпає спектр, 
	а відповідно і форму корисного сигналу і не прибирає шум повністю;
	\item кавзі-лінійне підсилювання незначним чином впливає на 
	форму імпульсу за рахунок нелінійності амплітудно-частотної 
	характеристики, а також підсилює не відфільтрований шум;
	\item аналогово-цифрове перетворення сигналу губить частину інформації 
	сигналу та шуму за рахунок дискретизації, що також ускладнює числову 
	обробку.
\end{enumerate}

Таким чином, на числову обробку потрапляє дещо видозміненій від початкового
сигнал. Також сам алгоритм числової обробки в модулі FPGA, зазвичай,
вибирається простим, через умову обробки сигналу в квазі-реальному часі.

\textcolor{red}{
Серед методів обробки сигналу в модулях FPGA, з лінійною складністю по часу 
виконання, можна виділити сімейство методів корелятивного порівняння, що 
базується на фільтрі Калмана. Ці методи, фактично, порівнюють отриманий сигнал 
з еталонним і надають коефіцієнт відповідності, чого достатньо для бінарної
класифікації наявності сигналу. }

В ближній зоні антени форма сигналу сильно залежить від напрямку 
спостереження \cite{imp:Wu1985, imp:Sodin1992-10, my:Telecom2018}. Це 
призводить до падіння точності роботи описаного підходу порівняння з 
еталоном. Також, погіршання якості роботи кореляційних алгоритмів впаде 
при прийманні сильних електромагнітних імпульсних хвиль, через неілнійну 
природу розвовсюдження в просторі та в компонентах антенно-фідерної стстеми.

Імпульсні надширокосмугові радіотехнічні пристрої мають теоретичні переваги 
над вузькосмуговими в плані інформаційної ємності, але на практиці, не 
вдається використовувати ці переваги повною мірою через складність обробки 
надширокосмугових сигналів \cite{imp:ChannelLimitations}. 

Для перетворення електромагнітної хвилі у вільному просторі в надширокосмуговий 
електричний струм у дротах прийнято використовувати масштабно-часове 
перетворення, яке може бути технічно виконано різними способами 
\cite{imp:Astanin1989}, а процес виділення корисної інформації з сигналу без 
її втрати може бути покращено, як буде показано далі.

%%%%%%%%%%%%%%%%%%%%%%%%%%%%%%%%%%%%%%%%%%%%%%%%%%%%%%%%%%%%%%%%%%%%%%%%%%%%%%%
\section{Імпульсний радіоприймач на базі апаратної нейронної мережі}

Всі задачі випромінювання, поширення чи дифракції хвилі формалізуються
у вигляді деяких параметричних рівнянь відносно компонентів струму чи 
компонентів поля. Коли аналітичне розв'язання такого рівняння знайти не 
вдається, а числовий метод розв'язання має значну обчислювальну складність, 
доцільно використати штучні нейронні мережі (ШНМ) прямого поширення, графові 
моделі машинного навчання або імпульсні нейронні мережі у купі з методами їх 
навчання для пошуку розв'язків.

Рівень оптимізації сучасних програмних інструментів машинного навчання, таких 
як CUDA та Tensorflow, а також рівень розвитку апаратних інструментів GPU/ASIC
дозволяють аналізувати цифрові часові послідовності за час порядку 
десятків мілісекунд, що дозволяє використовувати такі інструменти при роботі 
з сигналом у квазі-реальному часі, опрацьовуючи сигнал після АЦП. Недоліком 
такого методу може стати висока ціна кінцевих виробів, а також високий 
рівень споживання енергії. З іншого боку, галузь що швидко розвивається - 
аналогові штучні нейронні мережі виконані за технологією CMOS. На такі 
пристрої можна подати сигнал напряму з антенно-фідерної системи. Такі 
пристрої вже широко застосовуються радіотехніками в галузі когнітивного 
радіо \cite{imp:Husseini2010}, а також адаптивних вузькосмугових антенних 
систем \cite{imp:Zbynek2002}. В цих задачах, апаратні ШНМ мережі 
використовується для оптимізації деяких параметрів прийому-передачі сигналу 
в режимі реального часу.

Останнім часом, технічний розвиток в галузі апаратних штучних нейронних мереж 
дозволив втілювати їх різноманітні топологічні особливості в електронних 
аналогових пристроях. Проаналізуємо можливість застосування цих технологій для 
задач класифікації отриманого сигналу (sequence-to-label) та визначення його 
присутності в кожен момент часу (sequence-to-sequence). Результат розв'язання 
таких задач аналізу даних дозволить оцінити практичну цінність 
такого методу обробки радіосигналу в різних задачах 
прикладної електродинаміки: радіолокації, телекомунікації, вимірювання, 
зондування тощо. 

Важливим напрямком розвитку CMOS технології для покращення технічних 
характеристик імпульсного радіо стає зменшення техпроцесу. Наразі існують
готові прилади CMOS LSTM з техпроцесом 180нм. Використаннм техпроцесу 2нм,
що активно розвивається, можна збільшити швидкість обробки сигналів у 
порівнянні з класичними схемами імпульсного радіо з аналогічною цифровою 
обробкою у $ 10^9 $ разів.

\begin{figure}[htbp] \begin{center}
\includegraphics[scale=0.45]{neuron_radio}
\caption{Схема імпульсного радіо на нейронній схемотехніці} 
\label{fig:neural_radio}
\end{center} \end{figure}

Штучна нейронна мережа тут є електричним колом, внутрішня передавальна 
функція якого визначається лінійною комбінацією деякого набору матричних 
характеристик, які визначаються різноманітними методами оптимізації. Таким 
чином, задача обробки прийнятого радіосигналу зводиться до пошуку необхідних 
матричних параметрів. Для цього гарно підходять градієнтні методи навчання 
з учителем, де конкретна імплементація процесу тренування та набір 
тренувальних даних залежатиме від типу задачі, що розв'язується.

В якості нейронної архітектури для кіл обробки радіосигналу розглянемо
схему encoder-decoder \cite{imp:Ying2017}. 
В цій архітектурі нейронна мережа топологічно розбивається на 
дві частини. Перша частина трансформує вхідну часову послідовність в деякий 
набір параметрів, які однозначно характеризують вхідний сигнал. Тобто, 
encoder проектує вхідний сигнал на деякий ознаковий
простір. Друга частина мережі, decoder, перетворює набір ознак в ту якісну 
або кількісну характеристику, яку передбачає постанова задачі. Наприклад,
для задачі телекомунікації - це інформаційне повідомлення, або для радарної 
задачі - це положення та тип цілі. Такий підхід забезпечить повторне 
використання попередньо-навчених encoder-ів для різноманітних задач 
радіофізики. Технічна можливість додати декілька різних декодерів в 
електричні кола радіоприймальної схеми зробить цю концепцію промислово 
придатною.

Вихідний сигнал, що продукує апаратна нейронна мережа, визначається 
активаційною функцією вихідного нейрону. Цифровий вихідний сигнал можна 
отримати ступеневою активаційною функцією (персептрон Розенблата) та зручно 
використовувати описаний пристрій, як мереживний інтерфейс комп'ютера чи 
джерело керуючого сигналу для робототехніки.

Прикладом застосування аналогових нейронних мереж у радіо є покращення 
характеристик роботи CDMA телефонії, шляхом послідовної лінійної фільтрації
з подальшою обробкою повнозв'язною аналоговою нейронною мережею прямого 
поширення \textcolor{red}{[ПОСИЛАННЯ]}. Принципова відмінність нейронного
радіо від прототипів полягає у відмові від обробки кожної характеристики 
сигналу окремо, і делегації його аналізу, як цілого, пристрою, системотехніка 
якого реалізує деякий математичний граф. В досліджені 
\cite{imp:Zhang2009} задачу розрізнення збуджень різного типу 
виріщують за допомогою лінійного фільтру і, як буде показано далі, цю ж 
задачу краще розв'язує енкодер штучної нейронної мережі за рахунок 
врахування складної природи поширення імпульсних електромагнітних 
хвиль.

Як і класична схема імпульсного радіо, запропонована схема передбачає 
можливість застосовувати одну і ту ж антену для передачі та прийму сигналу. 
Також використання нейропроцесору не виключає застосування традиційних 
методів аналогової обробки: підсилення, фільтрація.

Важливим аспектом є те, що використання ПЗУ дозволить програмувати нейронний 
процесор шляхом встановлення нових параметрів аналогових нейронів керуючими 
напругами, що дає змогу змінювати технічні характеристики пристроїв без 
втручання в його схемотехніку. Але практика не завжди потребує такої 
гнучкості: частіше, для всього періоду використання вбудованої електроніки, 
її призначення та порядок роботи залишаються постійним, а отже керуюча 
напруга може задаватись резисторами, що значно здешевить пристрій.

Порівняння рівня споживання електроенергії аналогового нейропроцесора та 
аналогічного GPU/ASIC \cite{imp:AnalogLSTM} показало кращу енергоефективність 
для першого. Відтак, використовуючи аналогового нейропроцесор замість АЦП та
FPGA можна скоротити споживання енергії надшироосмуговим радіо.

%%%%%%%%%%%%%%%%%%%%%%%%%%%%%%%%%%%%%%%%%%%%%%%%%%%%%%%%%%%%%%%%%%%%%%%%%%%%%%%
\section{Формування тренувальних даних}

Розглянемо задачу односторонньої передачі інформації через нейронне радіо. 
В якості передавальної антени розглядається антена типу LIRA. Для спрощення, 
в якості приймальної антени розглянемо ідеальний надширокосуговий вимірювач 
напруженності електричного поля, який не змінює форму отриманого сигналу,
тобто гарантує відсутність впливу приймальної антени на отриманий сигнал. 
Сигнал з приймальної антено-фідерної системи подається на деяку апаратну 
нейронну мережу. Напруженість електричного поля, породженого випромінювальною 
антеною $ \vect{E}_{tx} $, можна визначити в довільній точці спостереження при 
довільному збуджені з урахуваннями ефектів ближньої зони, користуючись 
згорткою \eqref{eq:duhamel} по перехідній функції 
$ \vect{E}_0 $. Тоді, отриманий з антенно-фідерної системи сигнал, буде 
пропорційний до компонентів напруженості електричного поля випромінювальної 
антени. Здійснюючи гіпотетичне вимірювання в такій площині, що спостерігається 
$ OX $ компонента напруженості поля, а приймальна лінія ідеально узгоджена і 
немає втрат, отриманий сигнал матиме вигляд:

\begin{equation}
f_{rx} \left( \vect{r}, t \right) = 
\int_0^t \derivat{f_{tx}}{\tau} \vect{E_0} (\vect{r}, t - \tau) d \tau,
\end{equation}
%
де $ \vect{E_0} $ - перехідна функція передавальної антени типу LIRA, а 
$ f_{tx} (t) $ - форма сигналу, що збуджує передавальну антену.

Метою даного моделювання є пошук оптимальної нейронної архітектури, а також 
її вагових коефіцієнтів, що дозволить співвіднести прийнятий сигнал з деяким 
типом збудження на передавачі в умовах завад, та з урахуванням деяких ефектів 
ближньої зони.

Тренувальний набір даних для цієї задачі складатиметься з пар часових 
послідовностей - струм збудження передавальної антени $ f_{tx} (t) $ та струм, 
що буде отримано приймачем $ f_{rx} (t) $ при різних розташуваннях 
приймача $ \vect{r} $ відносно системи координат передавача, введеної як на 
Рис.~\ref{fig:pdisk}. Для максимально правильного функціонування мережі в 
заданих умовах, набір тренувальних даних повинен містити вичерпну інформацію про 
поведінку поля у всій області функціонування антенної системи, а тобто містити 
вимірювання в ближній і дальній зонах. Користуючись визначенням дальньої зони,
отримуємо максимальне віддалення від джерела, де необхідно проводити 
вимірювання - $ 0 \leq z \leq 8R $. Користуючись направленістю антен типу 
LIRA, обмежимо радіус поперечного зрізу циліндричної області, де проводяться 
вимірювання $ 0 \leq \rho \leq R $, а користуючись симетрією джерела розглянемо 
не весь зріз, а лише його першу чверть $ 0 \leq \varphi \leq \pi / 2 $.

Таким чином, просторовий об'єм, в якому необхідно провести вимірювання 
складає $ \pi R^2 / 2 $. З огляду на низьку сходимість деяких алгоритмів 
навчання заповнимо отримана область великою кількістю (10000) випадково 
розміщених і рівномірно розподілених уявних вимірювань для тренувального 
набору даних, а також в чотири рази меншим тестувальним набором - 
2500 зразків. Для наближення моделі до реальних умов, до кожного з семплів 
додамо деяку випадкову заваду. Такий підхід відомий в літературі, як 
задача аналізу AWGN каналу, де енергія шуму визначається як квадрат 
середнього відхилення, що можна записати математично:

\begin{equation} \label{eq:snr}
SNR = 10 * \log_{10} \left( \frac{W}{\sigma^2} \right),
\end{equation}
%
де $ W $ - енергія електромагнітної хвилі, а $ \sigma $ - другий статистичний 
момент моделі білого гаусівського шуму. У виразі не враховано вплив постійної 
фонової напруженості поля, яка фігурує в моделі білого шуму для каналу зв'язку 
через відсутність його впливу на часову залежність прийнятого сигналу.

\begin{figure}[htbp] \begin{center}
\includegraphics[scale=0.4]{P4AWGN}
\caption{Енергія моделі білого шуму в залежності від параметру} \label{fig:P4AWGN}
\end{center} \end{figure}

Через напрямлені властивості антени, навіть при постійному рівні завад, 
отриманий набір даних складатиметься зі зразків з різним значенням SNR: більшим 
на осі та меншим на периферії (Рис.~\ref{fig:SNR}). Для якісного навчання 
необхідно, щоб розподіл тренувальних даних за значенням SNR відповідав 
імовірнісному розподілу прийнятих сигналів за значенням SNR в умовах 
реального використання. Припустимо, що користувачі пристроїв будуть 
намагатись вести прийомо-передачу максимізуючи SNR, тоді реальний імовірнісний 
розподіл прийнятих сигналів матиме вигляд гаусівського, через статистичні 
відхилення від ідеальних параметрів та в умовах завад.

Для якісного процесу навчання необхідно, щоб тренувальний набір даних не 
лише містив всю палітру значень SNR при кожній епосі навчання, а ще і 
послідовно зменшував його середнє значення кожного разу, досягаючи 
локального мінімуму цільової функції.

\begin{figure}[htbp] \begin{center}
\includegraphics[scale=0.9]{Dataset_SNR}
\caption{Зашумленість набору тренувальних даних} \label{fig:SNR}
\end{center} \end{figure}

На Рис.~\ref{fig:SNR} зображено гістограму, де висота стовбчика ілюструє 
кількість зразків в датасеті з відповідним значенням SNR. Для генерації 
тренувальних даних застосовано рандомайзер mt19937 в реалізації стандартної
бібліотеки шаблонів С++ для 64-х розрядних систем за стандартом ISO C++17.

Для перевірки можливостей нейронного радіо вирізняти різні види сигналів 
розглянемо відразу ряд збуджувальних імпульсів:

\begin{equation} \label{eq:type_void}
f_0 = void = 0;
\end{equation}
%
\begin{equation} \label{eq:type_sinc}
f_1 = sinc = \sinc(t-\tau/2);
\end{equation}
%
\begin{equation} \label{eq:type_gauss}
f_2 = gauss = \exp(- (t-\tau/2)^2 );
\end{equation}
%
\begin{equation} \label{eq:type_gauss_perp}
f_3 = gauss\_perp = \partder{}{t} \exp(- (t-\tau/2)^2 );
\end{equation}
%
тоді отриманий набір даних матиме 4 класи, де окремим класом є білий шум. Для
дотримання збалансованості даних в процесі навчання, до рандомайзера додамо 
випадковий рівномірно розподілений дискретний параметр, що відповідатиме 
за тип збудження: \textit{void}, \textit{sinc}, \textit{gauss}, 
\textit{gauss\_perp}. Для максимізації якості навчання розглянемо 
лише датасет, де вибірки за типом збудження будуть кількісно збалансовані. 

Процес навчання штучної нейронної мережі може проходити як на комп'ютері так і 
на апаратному модулі за рахунок створення спеціальних програмних драйверів
до апаратної частини. В межах даного дослідження 
достатньо навчання на комп'ютері, як на ресурсах CPU так і на ресурсах GPU.
Для цього набір тренувальних даних необхідно дискретизувати. Частоту 
дискретизації вибираємо, користуючись критерієм Найквіста.

Тобто кожен зразок тренувального набору складатиметься з часової
послідовності, що відповідає прийнятому сигналу та метаданих, що описують 
переданий сигнал: значення енергетичного SNR, тип збудження, 
точка спостереження, ефективна тривалість збудження та інше. Для цього
зручно використати формати зберігання даних JSON та TFRecord.

\textcolor{red}{TODO: порівняти методи навчання NLP з DeepUWB}

%%%%%%%%%%%%%%%%%%%%%%%%%%%%%%%%%%%%%%%%%%%%%%%%%%%%%%%%%%%%%%%%%%%%%%%%%%%%%%%
\section{Моделювання детекції сигналу для повнозв'язних моделей}

Розглянемо декілька моделей нейронних мереж та визначимо недоліки та переваги
кожної з них. Сімейство моделей, що розглядається звузимо лише до тих, які 
просто виготовляти. Перші моделі штучних нейронних мереж прямого 
поширення, що були запропоновані - повнозв'язні з одним прихованим шаром.

\textcolor{red}{TODO: плавні активаційні функції через аналогову імплементацію}

\begin{figure}[htbp] \begin{center}
\includegraphics[scale=0.65]{simple_radio}
\caption{Імпульсне радіо на основі багатошарового перцептрону} 
\label{fig:mp_radio}
\end{center} \end{figure}

Такі мережі є популярним інструментом розв'язання задач різного типу 
\cite{imp:Kussul2004}, в основному, через досить простий алгоритм навчання.
Для подачі часової послідовності (сигналу) на її вхід обов'язковим стає 
застосування методики ковзного вікна шляхом його затримки, що зменшує частоту
дискретизації вихідного сигналу з антени та втрачає частину 
даних. Такий підхід забезпечує подачу до нейронної мережі ковзного вікна даних 
та не враховує результат обробки минулого вікна при переході до наступного.

Можна підрахувати, що розмірність вхідного шару визначається частотою 
дискретизації. Для імпульсів переданих антеною типу LIRA з одиничним 
електричним розміром вхідний шар складатиметься не менше ніж з 300 штучних 
нейронів. Так як сигнал може знаходитись в будь-якій частині вікна прихований 
шар повинен містити не меншу кількість нейронів для збереження якісного 
результату в кожен момент часу. З іншого боку, для більшості задач 
електродинаміки важливо визначити, що сигнал взагалі був в якомусь з вікон,
що накладаються, тому кількість нейронів в другому шарі можна дещо зменшити.
Останній вихідний шар, що виконує функцію декодеру матиме розмірність 
якісної або кількісної характеристики, яку передбачає постанова задачі.
У випадку, що розглядається - це тип збудження, який може приймати чотири 
дискретні значення. Таким чином, кількість параметрів, що підлягають 
тренуванню - $ 76400 $. Для розв'язання електромагнітних задач на кшталт
зондування, де необхідно враховувати перевідбиття, розмір вікна спостереження
почне залежати від постанови задачі, що призведе до зростання кількості 
тренувальних параметрів до величини порядку $ 10^7 $.

\begin{figure}[htbp] \begin{center}
\includegraphics[scale=0.9]{FC_S2L_loss}
\caption{Зміна значення цільової функції повнозв'язної моделі  
в процесі тренування} \label{fig:fcnn_loss}
\end{center} \end{figure}

Не дивлячись на величезну кількість тренувальних параметрів, навчання 
проходить досить швидко за рахунок малої глибини моделі. 
На Рис.~\ref{fig:fcnn_loss} зображено зміну значень цільової функції в 
процесі тренування. Її пораховано на тренувальних та на тестувальних даних 
та зображено окремими кривими. Перетин тренувальної та тестувальної 
цільової функції після третьої епохи є індикатором перенавчання моделі. 
Для тренування використовувалась техніка Dropeout (виключення окремих 
нейронв з деяких етапів навчання), що в купі з перенавчанням є фактором, 
що вказує на недостатню інформаційну ємність повнозв'язної моделі для 
розв'язання поставленої задачі.

Задачею радіоприймача є перетворення в реальному часі сигналу з антени в 
деяку послідовність корисних даних. Згідно класифікації задач аналізу даних, 
цю задачу можна розглядати як many-to-many так і many-to-one. Повнозв'язна 
нейронна мережа працює саме за схемою many-to-one, коли деякому вікну, що 
спостерігається, призначається одна якісна характеристика -- вектор 
імовірностей присутності сигналів кожного з видів.

Точність роботи, визначена за метрикою середньої відносної помилки,
справедлива лише до третьої епохи, через проблему перенавчання, що наступає 
коли криві перетинаються. Таким чином, мінімальне значення середньої 
відносної помилки для штучної нейронної мережі пов'язаного типу складе 
$ 89 \% $.

Серед недоліків застосування описаної архітектури в якості нейронного радіо
можна відмітити велику кількість тренувальних параметрів, що ускладнить 
електронну схему пристрою. Цю проблему можна вирішити використанням 
рекурентних штучних нейронних мереж, які одномоментно приймають на вхід 
одне значення часової послідовності і накопичують інформацію про сигнал 
з плином часу за рахунок зміни свого внутрішнього стану.

\textcolor{red}{TODO: Оцінка стійкості апаратних нейронних мереж до 
внутрішніх шумів}

%%%%%%%%%%%%%%%%%%%%%%%%%%%%%%%%%%%%%%%%%%%%%%%%%%%%%%%%%%%%%%%%%%%%%%%%%%%%%%%
\section{Моделювання детекції сигналу для рекурентних моделей}

На відміну від повнозв'язної моделі, рекурентні можуть працювати як за схемою
many-to-one так і за схемою many-to-many, коли в кожен момент часу деякому 
вхідному сигналу співвідноситься якісна або кількісна вихідна характеристика.

У випадку many-to-one вірний прогноз нейронної мережі стосується ковзного 
вікна в цілому, а отже, точність визначення меж сигналу у часі буде визначена 
розміром вікна спостереження, яке в декілька разів ширше за сигнал. Моделі, 
що працюють за схемою many-to-many позбавлені цього недоліку.

Так як режим роботи нейронної мережі, а тобто many-to-one або many-to-many,
визначається алгоритмом тренування, апаратна реалізація пристрою залишається 
однаковою для many-to-one і many-to-many схем, що зручно для прикладного 
застосування.

Проста архітектура та мала кількість тренувальних параметрів є 
перевагами класичної топології рекурентної штучної нейронної мережі.
Її основним недоліком є нестабільність процесу навчання -- проблеми 
exploding gradient і vanishing gradient. Саме для вирішення цих проблем 
були створені архітектури GRU та LSTM \cite{imp:Hochreiter1997}.

\begin{figure}[htbp] \begin{center}
\includegraphics[scale=0.7]{lstm_radio}
\caption{Імпульсне радіо на основі нейронної мережі 
довго-короткотривалої пам'яті} \label{fig:lstm_radio}
\end{center} \end{figure}

На Рис.~\ref{fig:lstm_radio} зображено нейронне радіо з використанням 
рекурентних штучних нейронних мереж. Вихідний шар (decoder) залишився без 
змін - його розмірність визначається практичними потребами. В поточному 
досліді нейрони останнього шару мають сигмоїдальні активаційні функції 
та відповідають імовірностям спостерігати сигнал певного типу. Вхідний 
шар ШНМ є рекурентним, тобто закладається з ланцюжку однакових нейронів.
Описана штучна нейронна мережа має лише 38-116 змінних параметрів в 
залежності від типу рекурентного шару.

Розглянемо в якості вхідного шару LSTM ланцюжок. Його перевага над 
GRU в контексті нейронного радіо -- універсальність: він працює як за схемою
many-to-one так і за схемою many-to-many. Єдиним недоліком LSTM у порівнянні 
з GRU топологією стане більш тривале поширення сигналу крізь такий шар,
що можна нівелювати, використовуючи менший техпроцес.

Спершу розглянемо режим роботи many-to-one, щоб порівняти результат з 
повнозв'язаною нейронною мережею.

\begin{figure}[htbp] \begin{center}
\includegraphics[scale=0.35]{LSTM_S2L_loss}
\caption{Зміна значення цільової функції моделі LSTM 
в процесі тренування} \label{fig:lstm_loss}
\end{center} \end{figure}

На Рис.~\ref{fig:lstm_loss} зображено зміну значень цільової функції в процесі
навчання на тренувальних даних. Помічаємо, що швидкість навчання 
не рівномірна -- спостерігаються ``плато'' зі сталими значеннями цільової 
функції, не приймаючи до уваги випадкових викидів. Подальший аналіз показав, 
що кожен з таких відрізків відповідає за навчання розпізнаванню кожного з 
типу сигналів, що вивчаються. Як можна помітити на Рис.~\ref{fig:lstm_loss},
кожен наступний тип сигналу вивчається довше минулого. Порядок вивчення 
імпульсів теж виявився не випадковим: чим більше осциляцій відносно нуля має 
імпульс, тим довше і пізніше він вивчається.

Бачимо, що застосування нейронних мереж замість лінійної фільтрації 
дозволяє покращити розпізнавальну здатність у ближній зоні антени, де форма 
сигналу досить мінлива \cite{my:UWBUSIS2018}.

Використовуючи рекурентні нейронні мережі можна досягти точності в 
$ 99.7\% $, що значно перевищує результати повнозв'язної нейронної мережі. 
Однак, з рис.~\ref{fig:lstm_loss} видно, що використання рекурентних 
нейронних мереж помітно сповільнює процес навчання: кількість епох 
тренування зросла на два порядки.

Розглянемо тренування за моделлю many-to-many. Топологія мережі залишається 
як на рис.~\ref{fig:lstm_radio}, а дані для тренування доповнимо анотацією 
для кожного моменту часу замість анотування деякого вікна спостереження.

\begin{figure}[htbp] \begin{center}
\includegraphics[scale=0.9]{lstm-seq2seq-loss}
\caption{Зміна значення цільової функції моделі LSTM
в процессі тренування} \label{fig:lstm_seq2seq_loss}
\end{center} \end{figure}

На Рис.~\ref{fig:lstm_seq2seq_loss} зображено зміну значень цільової функції
для задачі маркування послідовності (many-to-many). Тут зображено процес 
навчання штучної нейронної мережі з рис.~\ref{fig:lstm_radio}. Цільова
функція тренувального процесу спрямована на максимізацію здібності визначати
імовірності присутності сигналу певного виду в певний момент спостереження.
При переході від many-to-one до many-to-many тренування сповільнилось.

\begin{figure}[htbp] \begin{center}
\includegraphics[scale=0.9]{seq2seq_example}
\caption{Приклад правильного аналізу} \label{fig:seq2seq_example}
\end{center} \end{figure}

На Рис.~\ref{fig:seq2seq_example} зображено приклад роботи рекурентної 
штучної нейронної мережі для класифікації прийнятого надширокосмугового сигналу 
в кожен момент часу. В представленому зразку даних спостерігається сигнал,
породжений збудженням антени типу LIRA, що має часову залежність у вигляді 
похідної від Гаусіна \eqref{eq:type_gauss_perp} та спостерігається з 
таким відхиленням від осі $ OZ $, що збуджене поле виглядає як поле породжене 
іншим збудженням \eqref{eq:type_gauss}. Також на рисунку зображені 
імовірності приналежності сигналу в кожен момент часу до певного збудження 
чи шуму (відсутності збудження).

Для проміжку часу, що передує видимому імпульсу імовірності приналежності 
сигналу до одного з типів залишаються 
приблизно рівними та складають близько $ 25 \% $. Тобто при спостеріганні 
білого шуму значення виходу нейронної мережі, фактично, не правильне. Не 
дивлячись на це, наявність сигналу можна визначити, аналізуючи всі вихідні 
значення ШНМ: якщо імовірності наявності сигналів кожного з типів (в тому 
числі і його відсутності) рівні, то спостерігається лише шум і цю системн 
похибку можна врахувати. Таку похибку пояснити тим, що нейронна мережа 
намагається виділити в шумі сигнал кожної з вивчених форм, а не знайшовши 
сигналу повертає мінімальний рівноімовірний результат. Така систематична 
похибка легко виявляється та нейтралізується евристичним аналізом або 
додатковим модулем, що виконує операцію XOR (наприклад нейропроцесор з 
трьох нейронів).

На рис.~\ref{fig:lstm_seq2seq_loss} проілюстровано, що навіть у ближній зоні, 
де форма імпульсу може змінюватись настільки, що стає більше схожою на інший 
сигнал, нейронне радіо гарантує стійкий режим роботи. З моменту, коли сигнал 
візуально спостерігається (індекси часової послідовності 
$ \left[ 45, 55 \right]$ ), деякий час імовірність приналежності сигналу до 
деяких типів зростає одночасно. Це можна пояснити через схожість 
градієнта часової послідовності на градієнти сигналів різних типів. Далі, 
мережа визначається з вибором і тримає його весь час тривалості сигналу. 
Стійка детекція сигналу за пороговим значенням значенням $ 0.707 $ займає 
близько $ 70\% $ тривалості сигналу.

Точність роботи мережі на валідаційному датасеті впала до $ 98.9\% $,
що є закономірним при підвищенні точності визначення тривалості сигналу.

\begin{figure}[htbp] \begin{center}
\includegraphics[scale=0.9]{lstm_seq2seq_bad}
\caption{Приклад неправильного аналізу} \label{fig:lstm_seq2seq_bad}
\end{center} \end{figure}

На рис.~\ref{fig:lstm_seq2seq_bad} зображено приклад неправильного
розпізнавання сигналу. Зразок містить сигнал, породжений збудженням антени 
типу LIRA, що має часову залежність у вигляді \eqref{eq:type_sinc}. 
Рекурентна штучна нейронна мережа надала нестійку
детекцію трьох сигналів в хронологічній послідовності: \textit{sinc}, 
\textit{gauss}, \textit{sinc}. Детекція першого сигналу викликана 
одномоментною схожістю сигналу на \textit{sinc}, що викликало ланцюжкову 
реакцію для подальших помилкових детекції сигналів: мережа гадає, що 
помилково детектований сигнал накладається на реальний сигнал і робить 
невірне передбачення класу його приналежності.

%%%%%%%%%%%%%%%%%%%%%%%%%%%%%%%%%%%%%%%%%%%%%%%%%%%%%%%%%%%%%%%%%%%%%%%%%%%%%%%
\section{Моделювання детекції сигналу для графічних моделей}

\textcolor{red}{TODO: HMM для захищеного радіо-вимикача}

\textcolor{red}{TODO: HNN для захищеного радіо-вимикача}

%%%%%%%%%%%%%%%%%%%%%%%%%%%%%%%%%%%%%%%%%%%%%%%%%%%%%%%%%%%%%%%%%%%%%%%%%%%%%%%
\section{Універсальна топологія нейронного радіо та її застосування}

Як показано раніше, повнозв'язна штучна нейронна мережа прямого поширення
дозволяє провести перетворення сигналу на інформацію, що покращить якісні 
характеристики детектора у порівнянні з класичним імпульсним радіо. 
Застосування енкодеру у вигляді рекурентної нейронної мережі робить нейронне 
радіо можливим для практичної реалізації через суттєве зменшення кількості 
штучних нейронів, а також покращить якість роботи, за рахунок топологічного 
врахування принципу причинності і принципу суперпозиції. Також архітектуру 
можна покращити, замінивши повнозв'язний шар графічним.

Така рекурентно-графічна мережа дозволить якісно краще розв'язувати задачі 
імпульсної телефонії для багатокористувацького середовища, де врахування форми 
імпульсу, його тривалості і ефектів ближньої зони особливо критичне. Отримана 
архітектура штучної нейронної мережі також зустрічається і використовується в 
задачах аналізу розпізнавання усної мови на кшталт розпізнавання голосових 
команд \textcolor{red}{[ПОСИЛАННЯ]} або в задачах аналізу даних з сенсорів
присутності людини в приміщенні \textcolor{red}{[ПОСИЛАННЯ]}. 

\textcolor{red}{TODO: BER (bit error rate) calculation in DS-UWB system in AWGN}

Для первинного навчання нейронного радіо застосовуються дані, отримані 
теоретичним моделюванням, замість експериментальних вимірювань задля спрощення 
практичного застосування пристроїв та їх виготовлення. При такому підході,
різниця реальних даних і тренувальних викликає падіння точності детектування.
Шляхом вирішення цієї проблеми є застосування методів переносу навчання, які 
широко використовуються в задачах аналізу часових послідовностей. 

Сутність методів переносу навчання полягає в дотренуванні окремих елементів 
мережі, користуючись експериментально отриманими даними, для адаптації її 
параметрів для реальних умов. Таким чином, первинне тренування на даних
теоретичних моделювань дозволяє суттєво зменшити об'єм емпіричних вимірювань,
необхідних для проведення дотренування у порівнянні з тренуванням без 
первинного наближення \textcolor{red}{[ПОСИЛАННЯ]}.

Запропонована рекурентно-графічна архітектура також дозволяє отримати приріст 
в швидкості передачі даних за рахунок побудови протоколів комунікації, де для
кодування використовуються імпульси різної форми (high radix networking).

\begin{equation}
C = \frac{1}{N_{smp}} \frac{\log_2 \left( 1 + SNR \right)}{1/B + \tau_{RMS}} 
\end{equation}

\begin{figure}[htbp] \begin{center}
\includegraphics[scale=0.7]{channel_capacity}
\caption{Інформаційна ємність імпульсного випромінювання} \label{fig:info_cap}
\end{center} \end{figure}

Для налагодження якісного імпульсного зв'язку використовують послідовності 
імпульсів для кодування одного символу, що дозволяє розв'язати задачу 
імпульсної комунікації в умовах перевідбиттів (multi-path problem).

При використанні запропонованої рекурентно-графічної моделі, графічний 
decoder не вирішує проблему перевідбиттів (multipath): недостатня 
запам'ятовувальна здатність графічної моделі призводить до лінійного 
погіршення якості детектування імпульсу при кількісному збільшенні імпульсів, 
що кодують сигнал. Очевидно, що топологічне ускладнення декодеру, наприклад 
застосування LSTM, дозволить вирішити цю проблему. З іншого боку, замість 
ускладнення декодеру, простіше застосувати FPGA для аналізу цифрового сигналу, 
що повертається нейронним радіо.

Зазвичай надширокосмугова імпульсна радіолокація виконується 
через вимірювання часу надходження відбитого випромінювання. Використання 
нейронної мережі дозволить підвищити точність такого вимірювання через 
визначення не тільки часу, а і азимутального кута прийому. Для цього навчимо 
нейронну мережу розпізнавати напрямок до цілі за формою імпульсу, яка сильно 
змінюється у ближній зоні.

При розв'язанні задач зондування, де інформація про об'єкт зондування 
розташована у після-імпульсних коливаннях, тривалість яких фактично 
нескінченна, а тривалість ROI залижіть від глибини розташовання цілі. 
Очевидним недоліком запропонованої рекурентно-графічної моделі стає лінійне 
погіршення якості роботи енкодеру, який запам'ятовуватиме все триваліші
відгуки середовища на імпульс. Просте подовження рекурентного ланцюжку 
енкодену не вирішить цієї проблеми і спостерігатиметься лінійне погіршення 
якості його роботи.

Для задач розпізнавання геометрично складних цілей в середовищі зі сторонніми
об'єктами енкодер потребує архітектурних ускладнень. Серед методів, які 
дозволяють збільшити запам'ятовувальну здатність енкодеру: Bi-LSTM, 
LSTM-to-LSTM при many-to-many зв'язками та Attention based LSTM.

\textcolor{red}{TODO: denoising autoencoder (NDA)}

\textcolor{red}{TODO: генерація сигналу для передачі декодером мережі для 
розв'язання задач адаптивних антен, або для випромінювання сигналу-відповіді 
на вхідний запит по радіоаналу}

%%%%%%%%%%%%%%%%%%%%%%%%%%%%%%%%%%%%%%%%%%%%%%%%%%%%%%%%%%%%%%%%%%%%%%%%%%%%%%
\section*{Висновки до розділу \ref{ch:neuron}}

Імпульсні надширокосмугові радіотехнічні пристрої мають теоретичні переваги 
над вузькосмуговими в плані інформаційної ємності, але на практиці, не 
вдається використовувати ці переваги повною мірою через складність обробки 
надширокосмугових сигналів \cite{imp:ChannelLimitations}, про що також 
свідчать результати проведених симуляцій. Отже нейронне радіо може стати 
перспективним напрямком розвитку телекомунікаційних систем, після 
впровадження 5G технології.

Одним з напрямків дослідження щодо розвитку нейронного радіо може стати
застосування в якості аналогово модуля рекурентної нейронної мережі з 
комплексними тренувальними параметрами \cite{imp:NIPS2018}. За результатами 
досліджень, така модель краще підходять для аналізу імпульсних часових 
послідовностей, але станом на сьогодні не існує відповідних пристоїв, що 
працюють з аналоговим струмом.

Перспективним напрямком дослідження в області нейронного радіо є 
застосування імпульсних штучних нейронних мереж замість штучної нейронної 
мережі прямого поширення. Тренування таких мереж здійснюється шляхом 
самоорганізації системи під зовнішнім впливом з позитивним підкріпленням. 
Такі мережі простіше виконати у виді аналогової мікросхеми, ніж мережі
прямого поширення. З огляду малого обсягу інструментального 
апарату для навчання таких моделей цей підхід в даному досліджені не 
розглядається, але швидкий розвиток подібних технологій залишає їх 
дослідження перспективним в майбутньому.

Результати цього розділу впроваджують деякі ідеї опубліковані в роботах 
автора \cite{my:Telecom2018, my:UKRCON2019} та відображені в роботах автора
\textcolor{red}{[ПОСИЛАННЯ]}. Представлені матеріали знайшли своє застосування
в проектах з відкритим кодом та відомі як DeepUWB.


\chapter*{Висновки}

\begin{enumerate}
%
\item Побудовано аналітичне розв'язання у вигляді кусково визначеної функції для 
задачі випромінювання круглої апертури при нестаціонарному збуджені у вигляді 
прямокутної функції. Розв'язок отримано без наближення дальної зони та визначено 
для всіх точок спостереження в кожен момент часу. Використання моделі круглої 
апертури, як моделі антен типу LIRA перевірено на експерементальних даних в 
окремих точках та на даних отриманих методом FDTD з комерційного електромагнітного 
симулятора CST Studio.
%
\item Отримане розв'язання задачі випромінювання плаского диску при збуджені у 
вигляді функції Хевісайда в лінійному наближенні має чітку просторово-часову 
зональність та ілюструє твердження Фарадея, що випромінює не антена, а простір 
довколі неї. Отримані області випромінювання наступають послідовно для довільної 
точки спостереженя. Остання за часом настання область $ S_3 $ відповідає 
стаціонарному (усталеному) процесу випромінювання, коли всі точки апертури 
поєднані зі спостерегічем за принципом причинності. Настанню усталеного процесу 
передує область деякого транзитивного процесу $ S_2 $, поки поле від всієї 
апертури не досягне спостерігача. Найпершою для спостерігача просторово-часовою 
областю випромінювання в прожекторній зоні круглої апертури настає область 
електромагнітного снаряду $ S_1 $, де з хвилі у ТЕМ рупора формується ТЕ хвиля 
у вільному просторі.
%
\item 
%
\end{enumerate}
%


%\begin{bibset}{Список використаних джерел}
\bibliographystyle{acm}
% Для сортування літератури за алфавітом використовуйте
%\bibliographystyle{gost71s}
\bibliography{../my,../import}
%\end{bibset}
%GATHER{xampl-mybib.bib}

%\begin{bibset}[a]{Список публікацій автора}
%\bibliographystyle{acm}
%\bibliography{mybib}
%\end{bibset}


\appendix
\chapter{Деякі властивості тригонометричних функції}
\label{ch:trigonometric}

В цьому розділі представлено деякі маловідомі властивості тригонометричних 
функцій для зручності споживання викладеного в кваліфікаційній роботі 
матеріалу.

\textcolor{blue}{
\begin{equation*}
\derivat{}{\varphi} \arccos \varphi = - \frac{1}{ \sqrt{1 - \varphi^2} }
\end{equation*}
%
\begin{equation*}
\derivat{}{\varphi} \arctan \varphi = \frac{1}{1 + \varphi^2}
\end{equation*}
%
\begin{equation*}
\cos \alpha \cos \beta = \frac{1}{2} 
\left(  \cos (\alpha + \beta) + \cos (\alpha - \beta) \right)
\end{equation*}
%
\begin{equation*}
\sin \alpha \cos \beta = \frac{1}{2} 
\left( \sin (\alpha + \beta) + \sin (\alpha - \beta) \right)
\end{equation*}
%
\begin{equation*}
\sin \alpha \sin \beta = \frac{1}{2} 
\left( \cos (\alpha - \beta) - \cos (\alpha + \beta) \right)
\end{equation*}
%
\begin{equation*}
e^{im \varphi} = \cos m \varphi + i \sin m \varphi
\end{equation*}
%
\begin{equation*}
e^{-im \varphi} = \cos m \varphi - i \sin m \varphi
\end{equation*}
%
\begin{equation*}
\sin \varphi = \frac{e^{i \varphi} - e^{- i \varphi}}{2i}
\end{equation*}
%
\begin{equation*}
\cos \varphi = \frac{e^{i \varphi} + e^{- i \varphi}}{2}
\end{equation*}
%
\begin{equation*}
\arctan \frac{1}{x} = \frac{\pi}{2} - \arctan x
\end{equation*}
%
\begin{equation*}
\pi - \arccos x = \arccos (-x)
\end{equation*}
} % textcolor blue
%
\begin{equation}
\arccos x - \arccos y = \mp \arccos \left( 
xy + \sqrt{(1-x^2)(1-y^2)} \right),
\left\{ \begin{array}{c} x \ge y \\ x < y  \end{array} \right\}
\end{equation}
%
\begin{equation}
\arctan x - \arctan y = 
\arctan \frac{x-y}{1+xy}, xy > -1 
\end{equation}
%
\begin{equation}
\arctan x - \arctan y = \pm \pi + \arctan \frac{x-y}{1+xy}, 
\left\{ \begin{array}{c} x > 0 \\ x < 0  \end{array} \right\}, xy < -1 
\end{equation}

\section{Визначення символу Кронакера, через інтеграли}

В роботі часто застосовується визначення символу Кронакера через 
інтеграл на комплексній площині над експонентою з уявним показником.
Тут зібрані деякі не табличні інтеграли, що застосовані в роботі. 

\begin{equation} \begin{aligned} \label{eq:int_exp0}
\int_{0}^{2\pi} e^{\pm i (m-n) \varphi} d \varphi = 2 \pi \delta_{m,n} 
\end{aligned} \end{equation}
%
\begin{equation} \begin{aligned} \label{eq:int_exp1}
\int \limits_{0}^{2\pi} d \varphi \sin \varphi 
\left( \cos m \varphi - i \sin m \varphi \right) = 
i \pi \left( \delta_{m,-1} - \delta_{m,1} \right)
\end{aligned} \end{equation}
%
\textcolor{blue}{ \begin{equation*} \begin{aligned}
\int_{0}^{2\pi} d \varphi \sin \varphi 
\left( \cos m \varphi - i \sin m \varphi \right) = \int_{0}^{2\pi} d \varphi
\left( \sin \varphi \cos m \varphi - i \sin \varphi \sin m \varphi \right) = \\
= \frac{1}{2} \int_{0}^{2\pi} d \varphi \left( \sin (\varphi + m \varphi) + 
\sin (\varphi - m \varphi) - i \cos (\varphi - m \varphi) + 
i \cos (\varphi + m \varphi) \right) = \\
= \frac{i}{2} \int_{0}^{2\pi} d \varphi \left( -i \sin (\varphi + m \varphi) -
i \sin (\varphi - m \varphi) - \cos (\varphi - m \varphi) + 
\cos (\varphi + m \varphi) \right) = \\
= \frac{i}{2} \int_{0}^{2\pi} d \varphi \left( e^{-i (\varphi + m \varphi)} - 
e^{i (\varphi - m \varphi)} \right) = 
i \pi \left( \delta_{m,-1} - \delta_{m,1} \right)
\end{aligned} \end{equation*} }
%
\begin{equation} \begin{aligned} \label{eq:int_exp2}
\int \limits_{0}^{2\pi} d \varphi \cos \varphi 
( \cos m \varphi - i \sin m \varphi) = \pi ( \delta_{m,-1} + \delta_{m,1} )
\end{aligned} \end{equation}
%
\textcolor{blue}{ \begin{equation*} \begin{aligned}
\int_{0}^{2\pi} d \varphi \cos \varphi 
\left( \cos m \varphi - i \sin m \varphi \right) = \int_{0}^{2\pi} d \varphi
\left( \cos \varphi \cos m \varphi - i \cos \varphi \sin m \varphi \right) = \\
= \frac{1}{2} \int_{0}^{2\pi} d \varphi \left( 
\cos (\varphi + m \varphi) + \cos (\varphi - m \varphi) - 
i \sin (m \varphi + \varphi) - i \sin (m \varphi - \varphi) \right) = \\
= \frac{1}{2} \int_{0}^{2\pi} d \varphi \left( 
\cos (\varphi + m \varphi) + \cos (\varphi - m \varphi) - 
i \sin (m \varphi + \varphi) + i \sin (\varphi - m \varphi) \right) = \\
= \frac{1}{2} \int_{0}^{2\pi} d \varphi 
\left( e^{-i (1 + m) \varphi} - e^{i (1 - m) \varphi} \right) = 
\pi \left( \delta_{m,-1} + \delta_{m,1} \right)
\end{aligned} \end{equation*} }
%
\begin{equation} \begin{aligned} \label{eq:int_exp3}
\int_0^{2\pi} e^{-i m \varphi} \cos \varphi \sin^2 \varphi d \varphi = 
\frac{\pi \delta_{m,1} }{4} + \frac{\pi \delta_{m,-1} }{4} - 
\frac{\pi \delta_{m,-3} }{4} - \frac{\pi \delta_{m,3} }{4}
\end{aligned} \end{equation}
%
\textcolor{blue}{ \begin{equation*} \begin{aligned}
e^{-i m \varphi} \cos \varphi \sin^2 \varphi = e^{-i m \varphi} 
\frac{e^{i\varphi} + e^{-i\varphi}}{2} \frac{1 - \cos 2\varphi}{2} = \\
\frac{2e^{-i(m-1)\varphi} + 2e^{-i(m+1)\varphi}}{8} - 
\frac{e^{2i\varphi} + e^{-2i\varphi}}{2} 
\frac{2e^{-i(m-1)\varphi} + 2e^{-i(m+1)\varphi}}{8} = \\
\frac{2e^{-i(m-1)\varphi} + 2e^{-i(m+1)\varphi}}{8} - 
\frac{e^{-i(m-3)\varphi} + e^{-i(m+1)\varphi} + 
e^{-i(m-1)\varphi} + e^{-i(m+3)\varphi}}{8} = \\
= \frac{e^{-i(m-1)\varphi}}{8} + \frac{e^{-i(m+1)\varphi}}{8} -
\frac{e^{-i(m-3)\varphi}}{8} - \frac{e^{-i(m+3)\varphi}}{8}
\end{aligned} \end{equation*} }
%
\begin{equation} \begin{aligned} \label{eq:int_exp4}
\int_{0}^{2\pi} e^{-i m \varphi} \sin^3 \varphi d \varphi = 
\frac{3 \pi i}{4} \delta_{m,-1} - \frac{3 \pi i}{4} \delta_{m,1} - 
\frac{\pi i}{4} \delta_{m,-3} + \frac{\pi i}{4} \delta_{m,3}
\end{aligned} \end{equation}
%
\textcolor{blue}{ \begin{equation*} \begin{aligned}
e^{-i m \varphi} \sin^3 \varphi = e^{-i m \varphi} 
\frac{1 - \cos 2\varphi}{2} \frac{e^{i\varphi} - e^{-i\varphi}}{2i} = \\
= \frac{e^{-i(m-1)\varphi} - e^{-i(m+1)\varphi}}{4i} - 
\frac{e^{-i(m-3)\varphi} + e^{-i(m+1)\varphi} -
e^{-i(m-1)\varphi} - e^{-i(m+3)\varphi}}{8i} = \\
= \frac{3 e^{-i(m-1)\varphi}}{8i} - \frac{3 e^{-i(m+1)\varphi}}{8i} - 
\frac{e^{-i(m-3)\varphi}}{8i} + \frac{e^{-i(m+3)\varphi}}{8i} = \\
= \frac{3i e^{-i(m+1)\varphi}}{8} - \frac{3i e^{-i(m-1)\varphi}}{8} + 
\frac{i e^{-i(m-3)\varphi}}{8} - \frac{i e^{-i(m+3)\varphi}}{8} 
\end{aligned} \end{equation*} }
%
\textcolor{blue}{ \begin{equation*} \begin{aligned}
\int_{0}^{2\pi} e^{-i m \varphi} \sin^3 \varphi d \varphi = 
\frac{i\pi}{4} \left( 3 \delta_{m,-1} - 3 \delta_{m,1} + 
\delta_{m,3} - \delta_{m,-3} \right)
\end{aligned} \end{equation*} }
%
\begin{equation} \begin{aligned} \label{eq:int_exp5}
\int_0^{2\pi} e^{-i m \varphi} \sin \varphi \cos^2 \varphi d \varphi = 
\frac{\pi i }{4} \delta_{m,-1} - \frac{\pi i }{4} \delta_{m,1} -
\frac{\pi i }{4} \delta_{m,3} + \frac{\pi i }{4} \delta_{m,-3}
\end{aligned} \end{equation}
%
\textcolor{blue}{ \begin{equation*} \begin{aligned}
e^{-i m \varphi} \sin \varphi \cos^2 \varphi = 
\cos^2 \varphi e^{-i m \varphi} \frac{e^{i\varphi} - e^{-i\varphi}}{2i} = \\
= \frac{1 + \cos 2\varphi}{2} 
\frac{e^{i(1-m)\varphi} - e^{-i(1+m)\varphi}}{2i} = \\
= \frac{e^{i(1-m)\varphi} - e^{-i(1+m)\varphi}}{4i} + 
\frac{e^{2i\varphi} + e^{-2i\varphi}}{2} 
\frac{e^{i(1-m)\varphi} - e^{-i(1+m)\varphi}}{4i} = \\
\frac{ 2 e^{i(1-m)\varphi} - 2 e^{-i(1+m)\varphi}}{8i} +
\frac{e^{i(3-m)\varphi} + e^{-i(1+m)\varphi} - 
e^{-i(m-1)\varphi} - e^{-i(3+m)\varphi}}{8i} = \\
= -\frac{ i e^{-i (m-1) \varphi} }{8} + \frac{ i e^{-i (m+1) \varphi} }{8} -
\frac{ i e^{-i (m-3) \varphi} }{8} + \frac{ i e^{-i (m+3) \varphi} }{8}
\end{aligned} \end{equation*} }
%
\begin{equation} \begin{aligned} \label{eq:int_exp6}
\int_{0}^{2\pi} e^{-i m \varphi} \cos^3 \varphi d \varphi = 
\frac{\pi}{4} \delta_{m,-3} + \frac{\pi}{4} \delta_{m,3} + 
\frac{3 \pi}{4} \delta_{m,-1} + \frac{3 \pi}{4} \delta_{m,1}
\end{aligned} \end{equation}
%
\textcolor{blue}{ \begin{equation*} \begin{aligned}
e^{-i m \varphi} \cos^3 \varphi = 
\cos^2 \varphi e^{-i m \varphi} \frac{e^{i \varphi} + e^{-i \varphi}}{2} =
\frac{\cos^2 \varphi}{2} 
\left( e^{-i (1+m) \varphi} + e^{i (1-m) \varphi} \right) = \\
= \frac{ 1 + \cos 2 \varphi } { 4 } 
\left( e^{-i (1+m) \varphi} + e^{i (1-m) \varphi} \right) = \\
= \frac{e^{-i(1+m) \varphi} + e^{i(1-m) \varphi}}{4} + 
\frac{e^{-i(1+m) \varphi} + e^{i(1-m) \varphi}}{4}
\frac{e^{2i\varphi} + e^{-2i\varphi}}{2} = \\
= \frac{e^{-i(1+m) \varphi} + e^{i(1-m) \varphi}}{4} +
\frac{ e^{i(1-m) \varphi} + e^{-i(3+m) \varphi} + 
e^{i(3-m) \varphi} + e^{-i(1+m) \varphi} }{8} = \\
= \frac{3 e^{i(1-m) \varphi}}{8} + \frac{e^{-i(3+m) \varphi}}{8} +
\frac{e^{i(3-m) \varphi}}{8} + \frac{ 3 e^{-i(1+m) \varphi} }{8}
\end{aligned} \end{equation*} }
%
\textcolor{blue}{ \begin{equation*} \begin{aligned}
\int_{0}^{2\pi} d \varphi \left( \frac{3 e^{i(1-m) \varphi}}{8} + 
\frac{e^{-i(3+m) \varphi}}{8} + \frac{e^{i(3-m) \varphi}}{8} + 
\frac{ 3 e^{-i(1+m) \varphi} }{8} \right) = \\
= \frac{\pi}{4} \delta_{m,-3} + \frac{\pi}{4} \delta_{m,3} + 
\frac{3 \pi}{4} \delta_{m,-1} + \frac{3 \pi}{4} \delta_{m,1}
\end{aligned} \end{equation*} }

\include{vector}
\chapter{Інтеграли від циліндричної функції Бесселя першого роду}
\label{ch:bessel}

Циліндрична функція Бесселя першого роду -- базисна функція методу еволюційних 
рівнянь. Її властивості широко застосовуються в багатьох дослідженнях присвячених
нестаціонарним сигналам та процесам. Тут зібрано основні властивості 
функції Бесселя, використані в роботі, а також продемонстровано спосіб отримання 
аналітичного розв'язку для деяких не табличних інтегралів з
ядром у вигляді добутку декількох функцій Бесселя.

%%%%%%%%%%%%%%%%%%%%%%%%%%%%%%%%%%%%%%%%%%%%%%%%%%%%%%%%%%%%%%%%%%%%%%%%%%%%%%%
% Визначення на лінійні властивості
\begin{equation}
J_{-n} \left( z \right) = \left( -1 \right)^n J_n \left( z \right)
\end{equation}
%
\begin{equation} \label{eq:bessel_order_change}
J_{n+1} \left( z \right) + J_{n-1} \left( z \right) = 
\frac{2n}{z} J_n \left( z \right)
\end{equation}
% Асимптотичні властивості
\begin{equation} \label{eq:limJ1toZ}
\lim_{z \to 0} \left. \frac{J_1 \left( z \right)}{z} \right. = \frac{1}{2}
\end{equation}
% Інтегродиференціальні властивості
%\textcolor{blue}{
%\begin{equation*}
%2 \derivat{}{z} J_n \left( z \right) = 
%J_{n-1} \left( z \right) - J_{n+1} \left( z \right) 
%\end{equation*}
%%
%\begin{equation*}
%\derivat{}{z} J_n \left( z \right) = 
%J_{n-1} \left( z \right) - \frac{n}{z} J_{n} \left( z \right) 
%\end{equation*}
%%
%\begin{equation*}
%\derivat{}{z} J_n \left( z \right) = 
%\frac{n}{z} J_{n} \left( z \right) - J_{n+1} \left( z \right) 
%\end{equation*}
%%
%\begin{equation*}
%\derivat{}{z} \frac{ J_n \left( z \right) }{ z^n }  = 
%- \frac{ J_{n+1} \left( z \right) }{ z^n }
%\end{equation*}
%%
%\begin{equation*}
%\derivat{}{z} \left( z^n J_n \left( z \right) \right)  = 
%z^n J_{n-1} \left( z \right)
%\end{equation*}
%}

%%%%%%%%%%%%%%%%%%%%%%%%%%%%%%%%%%%%%%%%%%%%%%%%%%%%%%%%%%%%%%%%%%%%%%%%%%%%%%%
\section{Отримання інтегралу $ I_1 $ в явному виді} \label{sec:i1anal}
%

Отриманий інтеграл є азимутально-симетричним амплітудним 
коефіцієнтом для компонентів векторів напруженості поля, 
породженого антенами імпульсного випромінювання. Очевидно,
що значення інтегралу -- функція часу, лінійних просторових 
координат та радіусу апертури.

\begin{equation} \label{eq:int1start}
I_1 = R \int\limits_{0}^{\infty} \frac{d\nu}{\rho \nu} 
J_1 \left( \nu R \right) J_1 \left( \nu \rho \right) 
J_0 \left( \nu \sqrt{c^2 t^2 - z^2} \right)
\end{equation}

Спробуємо знайти аналітичне значення виразу \eqref{eq:int1start}, 
притупивши, що він сходиться. Інтеграли такого виду 
зустрічаються в \cite[ст. 398]{imp:Watson1922}.

\begin{equation} \begin{aligned} \label{eq:intJJJtable}
\int\limits_{0}^{\infty} \frac{d t}{t^{\lambda + \nu}} 
J_\mu \left( at \right) J_\nu \left( bt \right) J_\nu \left( ct \right) =
\frac{ \left( bc/2 \right) ^\nu }
{ \Gamma \left( \nu + 1/2 \right) \Gamma \left( 1/2 \right) } \cdot \\
\cdot \int\limits_{0}^{\infty} \int\limits_{0}^{\pi}
\frac{J_\mu \left( at \right) J_\nu \left( \omega t \right)}
{\omega^\nu t^\lambda} \sin^{2\nu}{\phi} d\phi dt, \\
\omega = \sqrt{b^2 + c^2 - 2bc \cos \phi} \\
\Re \left( \nu \right) > - \frac{1}{2};
\Re \left( \mu + \nu + 2 \right) > \Re \left( \lambda + 1 \right) > 0
\end{aligned} \end{equation}

%\textcolor{blue}{ \begin{equation*} \begin{aligned}
%a = \sqrt{c^2 t^2 - z^2}; b = R; c = \rho; \lambda = 0 \\
%\nu = 1; \mu = 0; \omega = \sqrt{R^2 + \rho^2 - 2 \rho R \cos \phi} \\
%\int\limits_{0}^{\infty} \frac{d\nu}{\nu} 
%J_1 \left( \nu R \right) J_1 \left( \nu \rho \right) 
%J_0 \left( \nu \sqrt{c^2 t^2 - z^2} \right) = 
%\frac{R^2}{ 2 \Gamma \left( 3/2 \right) \Gamma \left( 1/2 \right) } \cdot \\
%\int\limits_{0}^{\pi} 
%\frac{\sin^2{\phi}}{\sqrt{R^2 + \rho^2 - 2 \rho R \cos \phi}}
%\int\limits_{0}^{\infty} d \nu J_1 \left( \nu \omega \right) 
%J_0 \left( \nu \sqrt{c^2 t^2 - z^2} \right) d \phi
%\end{aligned} \end{equation*} }
%
%\textcolor{blue}{ \begin{equation*} \begin{aligned}
%\Gamma \left( 3/2 \right) \Gamma \left( 1/2 \right) = 
%\frac{\sqrt{\pi}}{2} \cdot \sqrt{\pi} = \frac{\pi}{2} 
%\end{aligned} \end{equation*} }
%
%\textcolor{blue}{ \begin{equation*} \begin{aligned}
%I_1 = \frac{R^2}{\pi} \int\limits_{0}^{\pi} 
%\frac{\sin^2{\phi}}{\sqrt{R^2 + \rho^2 - 2 \rho R \cos \phi}}
%\int\limits_{0}^{\infty} d \nu J_1 \left( \nu \omega \right) 
%J_0 \left( \nu \sqrt{c^2 t^2 - z^2} \right) d \phi
%\end{aligned} \end{equation*} }

Використання формули \eqref{eq:intJJJtable} дозволяє спростити $ I_1 $ до 
інтегралу по двом функціям Бесселя в ядрі замість трьох. Використаємо наступну 
формулу з \cite{imp:Golubovic2013} для пошуку рішення нового інтегралу. 

\begin{equation} \begin{aligned} \label{eq:intJJtable}
\int\limits_{0}^{\infty} d \nu
J_n \left( a \nu \right) J_{n-1} \left( b \nu \right) = \begin{cases} 
b^{n-1} / a^n , 0 < b < a \\
1 / 2 b , 0 < a = b \\
0 , 0 < a < b
\end{cases} 
\end{aligned} \end{equation}

%\textcolor{blue}{ \begin{equation*} \begin{aligned}
%\int\limits_{0}^{\infty} d \nu J_1 \left( \nu \omega \right) 
%J_0 \left( \nu \sqrt{c^2 t^2 - z^2} \right) = \begin{cases}
%\left( R^2 + \rho^2 - 2 \rho R \cos \phi \right)^{-1/2}, 0 < b < a \\
%\frac{1}{2} \left( c^2 t^2 - z^2 \right)^{-1/2}, 0 < a = b \\
%0 , 0 < a < b
%\end{cases} 
%\end{aligned} \end{equation*} }
%
%\textcolor{blue}{ \begin{equation*} \begin{aligned}
%\sqrt{R^2 + \rho^2 - 2 \rho R \cos \phi} > \sqrt{c^2 t^2 - z^2} \\
%R^2 + \rho^2 - 2 \rho R \cos \phi > c^2 t^2 - z^2 \\
%\cos \phi < \frac{R^2 + \rho^2}{2 \rho R} - \frac{c^2 t^2 - z^2}{2 \rho R} \\
%\phi > \arccos \left( \frac{\rho^2 + R^2}{2 \rho R} - 
%\frac{c^2 t^2 - z^2}{2 \rho R} \right), 0 \leq \phi \leq \pi
%\end{aligned} \end{equation*} }
%
%\textcolor{blue}{ \begin{equation*}
%\phi > \arccos \left( \frac{\rho^2 + R^2}{2 \rho R} - 
%\frac{c^2 t^2 - z^2}{2 \rho R} \right)
%\end{equation*} }
%
%\textcolor{blue}{ \begin{equation*} \begin{aligned}
%\begin{cases}
%\frac{\rho^2 + R^2}{2 \rho R} - \frac{c^2 t^2 - z^2}{2 \rho R} \leq 1 \\
%\frac{\rho^2 + R^2}{2 \rho R} - \frac{c^2 t^2 - z^2}{2 \rho R} \geq - 1
%\end{cases}
%\begin{cases}
%\rho^2 + R^2 - c^2 t^2 + z^2 \leq 2 \rho R \\
%\rho^2 + R^2 - c^2 t^2 + z^2 \geq - 2 \rho R
%\end{cases}
%\end{aligned} \end{equation*} }
%
%\textcolor{blue}{ \begin{equation*} \begin{aligned}
%\begin{cases}
%0 \leq \left( R - \rho \right)^2 \leq c^2 t^2 - z^2 \\ 
%\left( \rho + R \right)^2 \geq c^2 t^2 - z^2 \geq 0
%\end{cases}
%\begin{cases}
%0 \leq R \leq \rho + \sqrt{c^2 t^2 - z^2} \\
%R \geq \left| \rho - \sqrt{c^2 t^2 - z^2} \right| \geq 0
%\end{cases}
%\begin{cases}
%R \leq f_+(r,t) \\
%R \geq \left| f_-(r,t) \right|
%\end{cases} 
%\end{aligned} \end{equation*} }

\begin{equation*} \begin{aligned}
I_1 \in \begin{cases}
S_1: \{ 0 \leq \phi \leq \psi \}, 0 < R < 
\left| \rho - \sqrt{c^2 t^2 - z^2} \right| \\
S_2: \{ \psi \leq \phi \leq \pi \}, \left| \rho - \sqrt{c^2 t^2 - z^2} \right| \leq 
R \leq \rho + \sqrt{c^2 t^2 - z^2} \\
S_3: \{ 0 \leq \phi \leq \pi \}, R > \rho + \sqrt{c^2 t^2 - z^2}
\end{cases} 
\end{aligned} \end{equation*}

\begin{equation*} \begin{aligned}
I_1 \{ S_1 \} = 0
\end{aligned} \end{equation*}

%\textcolor{blue}{ \begin{equation*} \begin{aligned}
%I_1 = \frac{R^2}{\pi} \int\limits_{0}^{\pi} 
%\frac{\sin^2{\phi}}{\sqrt{R^2 + \rho^2 - 2 \rho R \cos \phi}}
%\int\limits_{0}^{\infty} d \nu J_1 \left( \nu \omega \right) 
%J_0 \left( \nu \sqrt{c^2 t^2 - z^2} \right) d \phi = \\
%= \frac{R^2}{\pi} \int_{\psi}^{\pi}
%\frac{\sin^2{\phi}}{\sqrt{R^2 + \rho^2 - 2 \rho R \cos \phi}}
%\frac{1}{\sqrt{R^2 + \rho^2 - 2 \rho R \cos \phi}} d \phi = \\
%= \frac{R^2}{\pi} \int_{\psi}^{\pi}
%\frac{\sin^2{\phi}}{R^2 + \rho^2 - 2 \rho R \cos \phi} d \phi = 
%\frac{1}{\pi} \int_{\psi}^{\pi}
%\frac{\sin^2{\phi}}{1 + \frac{\rho^2}{R^2} - \frac{2 \rho}{R} \cos \phi} d \phi
%\end{aligned} \end{equation*} }

Згідно властивістю адитивності при розбиттях для інтегралів Рімана, 
значення інтегралу в одній точці не впливає на значення інтегралу у 
визначених межах, а отже:

\begin{equation*} \begin{aligned}
I_1 = \frac{1}{\pi} \int_{\psi}^{\pi}
\frac{\sin^2{\phi}}{1 + \frac{\rho^2}{R^2} - 
\frac{2 \rho}{R} \cos \phi} d \phi \\
\psi = \arccos \left( \frac{\rho^2 + R^2}{2 \rho R} - 
\frac{c^2 t^2 - z^2}{2 \rho R} \right)
\end{aligned} \end{equation*}

%\textcolor{blue}{ \begin{equation*} \begin{aligned}
%\int \frac{\sin^2{\phi}}{a + b \cos \phi} d \phi = 
%\int \frac{1 - \cos^2{\phi}}{a + b \cos \phi} d \phi = 
%\int\frac{d \phi}{a + b \cos \phi}  -
%\int \frac{\cos^2{\phi}}{a + b \cos \phi} d \phi = \\
%= \int \frac{d \phi}{a + b \cos \phi}  - 
%\int \frac{\cos^2{\phi}}{a + b \cos \phi} d \phi -
%\frac{a}{b} \int \frac{\cos \phi}{a + b \cos \phi} d \phi + \\
%+ \frac{a}{b} \int \frac{\cos \phi}{a + b \cos \phi} d \phi = 
%\int \frac{d \phi}{a + b \cos \phi} +
%\frac{a}{b} \int \frac{\cos \phi}{a + b \cos \phi} d \phi - \\
%- \int \frac{\cos^2{\phi} + \frac{a}{b} \cos \phi} {a + b \cos \phi} d \phi =
%\int \frac{d \phi}{a + b \cos \phi} + 
%\frac{a}{b} \int\limits_{0}^{\psi} \frac{\cos \phi}{a + b \cos \phi} d \phi -
%\end{aligned} \end{equation*} }
%
%\textcolor{blue}{ \begin{equation*} \begin{aligned}
%- \frac{1}{b} \int \frac{\cos \phi + a/b} {a/b +  \cos \phi} \cos \phi d \phi = 
%\int \frac{d \phi}{a + b \cos \phi} + 
%\frac{a}{b} \int \frac{\cos \phi}{a + b \cos \phi} d \phi - \\
%- \frac{1}{b} \int \cos \phi d \phi = \int \frac{d \phi}{a + b \cos \phi} - 
%\frac{1}{b} \int \cos \phi d \phi + \frac{a}{b^2} \int
%\frac{a - a + b \cos \phi}{a + b \cos \phi} d \phi = \\ 
%= \int\frac{d \phi}{a + b \cos \phi} - \frac{1}{b} \int \cos \phi d \phi +
%\frac{a}{b^2} \int \frac{a + b \cos \phi}{a + b \cos \phi} d \phi - \\ 
%- \frac{a^2}{b^2} \int \frac{d \phi}{a + b \cos \phi} = 
%\left( 1 - \frac{a^2}{b^2} \right) \int\frac{d \phi}{a + b \cos \phi} - 
%\frac{1}{b} \int \cos \phi d \phi + \frac{a}{b^2} \int d \phi
%\end{aligned} \end{equation*} }
%
%\textcolor{blue}{ \begin{equation*} \begin{aligned}
%\int_{\psi}^{\pi} \frac{\sin^2{\phi}}{a + b \cos \phi} d \phi =  
%\left( 1 - \frac{a^2}{b^2} \right)
%\int_{\psi}^{\pi} \frac{d \phi}{a + b \cos \phi} -
%\frac{\sin \pi - \sin \psi}{b} + \frac{a}{b^2} (\pi - \psi) = \\
%= \left( 1 - \frac{a^2}{b^2} \right)
%\int_{\psi}^{\pi} \frac{d \phi}{a + b \cos \phi} +
%\frac{\sin \psi}{b} + \frac{a}{b^2} (\pi - \psi)
%\end{aligned} \end{equation*} }
%
%\textcolor{blue}{ \begin{equation*} \begin{aligned}
%a = 1 + \frac{\rho^2}{R^2}; b = - \frac{2 \rho}{ R } \\
%\frac{\pi R}{\rho} I_1 = \left( 1 - \frac{a^2}{b^2} \right)
%\int_{\psi}^{\pi} \frac{d \phi}{a + b \cos \phi} +
%\frac{\sin \psi}{b} + \frac{a}{b^2} (\pi - \psi) = \\
%= \left( 1 - \left( \frac{1 + \frac{\rho^2}{R^2}} 
%{ \frac{2 \rho}{R} } \right)^2 \right) 
%\int_{\psi}^{\pi} \frac{d \phi}{1 + \frac{\rho^2}{R^2} -  
%\frac{2 \rho}{R} \cos \phi} -
%\frac{\sin \psi}{\frac{2 \rho}{ R }} + \left( 1 + \frac{\rho^2}{R^2} \right) 
%\frac{\pi - \psi}{\frac{4 \rho^2}{R^2}} = \\
%\left( R^2 - \left( \frac{R^2 + \rho^2}{2 \rho} \right)^2 \right) 
%\int_{\psi}^{\pi} \frac{d \phi}{R^2 + \rho^2 - 2 \rho R \cos \phi} -
%\frac{R}{2 \rho} \sin \psi + \frac{\rho^2 + R^2}{4 \rho^2} (\pi - \psi)  
%\end{aligned} \end{equation*} }
%
%\textcolor{blue}{ \begin{equation*} \begin{aligned}
%\frac{4 \rho^2}{4 \rho^2} R^2 - \left( \frac{R^2 + \rho^2}{2 \rho} \right)^2 =
%\frac{4 \rho^2 R^2 - R^4 - 2 \rho^2 R^2 - \rho^4}{4 \rho^2} =
%- \frac{\left( \rho^2 - R^2 \right)^2}{4 \rho^2} 
%\end{aligned} \end{equation*} }
%
%\textcolor{blue}{ \begin{equation*} \begin{aligned}
%\pi I_1 (S_2) = - \frac{\left( \rho^2 - R^2 \right)^2}{4 \rho^2} 
%\int_{\psi}^{\pi} \frac{d \phi}{R^2 + \rho^2 - 2 \rho R \cos \phi} - \\
%- \frac{R}{2 \rho} \sin \psi + \frac{\rho^2 + R^2}{4 \rho^2}  (\pi - \psi)
%\end{aligned} \end{equation*} }

Тригонометричними перетвореннями зведемо поточний вид $ I_1 $ до табличного 
інтегралу.

\begin{equation*} \begin{aligned}
I_{1} \{ S_2 \} = - \frac{\left( \rho^2 - R^2 \right)^2}{4 \pi \rho^2} 
\int_{\psi}^{\pi} \frac{d \phi}{R^2 + \rho^2 - 2 \rho R \cos \phi} - 
\frac{R}{\rho} \frac{\sin \psi}{2 \pi} +  
\frac{\rho^2 + R^2}{4 \rho^2} \frac{\pi - \psi}{\pi}
\end{aligned} \end{equation*}

%\textcolor{blue}{ \begin{equation*} \begin{aligned}
%\int_{0}^{\pi} \frac{\sin^2{\phi}}{a + b \cos \phi} d \phi =  
%\left( 1 - \frac{a^2}{b^2} \right)
%\int_{0}^{\pi} \frac{d \phi}{a + b \cos \phi} -
%\frac{\sin \pi - \sin 0}{b} + \frac{a}{b^2} (\pi - 0) = \\
%= \left( 1 - \frac{a^2}{b^2} \right)
%\int_{0}^{\pi} \frac{d \phi}{a + b \cos \phi} +
%\frac{a}{b^2} \pi
%\end{aligned} \end{equation*} }

\begin{equation*} \begin{aligned}
I_{1} \{ S_3 \} = - \frac{\left( \rho^2 - R^2 \right)^2}{4 \pi \rho^2} 
\int_{0}^{\pi} \frac{d \phi}{R^2 + \rho^2 - 2 \rho R \cos \phi} + 
\frac{\rho^2 + R^2}{4 \rho^2}
\end{aligned} \end{equation*}

Таблична формула для неозначеного випадку інтегралу може буде знайдена в 
\cite[ст. 181]{imp:ElementFunc1983}.

\begin{equation} \label{eq:caseTableIntegral}
\int \frac{d x}{a + b \cos{x}} = \begin{cases}
\frac{2}{\sqrt{a^2-b^2}} \arctan \frac{\sqrt{a^2-b^2} \tan \frac{x}{2}}
{a + b}, a^2 > b^2 \\
\frac{1}{\sqrt{b^2-a^2}} \ln 
\frac{\sqrt{b^2-a^2} \tan \frac{x}{2} + a + b}
{\sqrt{b^2-a^2} \tan \frac{x}{2} - a - b}, a^2 < b^2
\end{cases}
\end{equation}

Помітимо, що випадок $ a^2 > b^2 $ відповідає області $ \rho > R $, a 
$ a^2 < b^2 $, навпаки, для прожекторної зони випромінювання. Як 
згадувалось раніше, аналогічна методика для отримання перехідної функції 
кругової апертури застосовувалась в дисертаційному дослідженні Думіна О.М.. 
На відміну від цього дослідження, здобувачем розглядається випадок не лише 
для$ \rho > R $, а і для $ \rho < R $, тобто для прожекторної зони, де 
прояв нелінійної природи поширення електромагнітних хвиль найбільший. 
Цікавість до області $ \rho < R $ також викликана тим, що напрямлені 
антени імпульсного випромінювання на практиці найчастіше використовуються 
саме за сценарієм, коли приймач чи випромінювач знаходяться в прожекторній 
зоні.
%
%\textcolor{blue}{ \begin{equation*} \begin{aligned}
%a^2 > b^2  \Rightarrow  
%\left( R^2 + \rho^2 \right)^2 > 4 \rho^2 R^2 \\
%R^4 + 2 \rho^2 R^2 + \rho^4 - 4 \rho^2 R^2 > 0 \Rightarrow 
%\left( \rho^2 - R^2 \right)^2 > 0
%\end{aligned} \end{equation*} }
%
%\textcolor{blue}{ Далі знадобиться: }
%
%\textcolor{blue}{ \begin{equation*} \begin{aligned}
%a^2 - b^2 = - \left( b^2 - a^2 \right) = 
%R^4 + 2 \rho^2 R^2 + \rho^4 - 4 \rho^2 R^2 = \left( \rho^2 - R^2 \right)^2 \\
%\lim_{\alpha \to 0} \tan{\alpha} = 0 \Rightarrow
%\lim_{\alpha \to 0} \arctan \left( a \tan{\alpha} \right) = 0 \\ 
%\lim_{\alpha \to \pi/2} \tan{\alpha} = \infty \Rightarrow
%\lim_{\alpha \to \pi/2} \arctan \left( a \tan{\alpha} \right) = \frac{\pi}{2}
%\end{aligned} \end{equation*} }
%
%\textcolor{blue}{ \begin{equation*} \begin{aligned}
%\int_{\psi}^{\pi} \frac{d \phi}{R^2 + \rho^2 - 2 \rho R \cos \phi} =
%\left. \frac{2}{ |\rho^2 - R^2| } \arctan \left( \frac{ |\rho^2 - R^2| }
%{\left( \rho - R \right)^2} \tan \frac{\phi}{2} \right)
%\right|_{\psi}^{\pi} = \\ = \frac{2}{ |\rho^2 - R^2| } \left.
%\arctan \left( \frac{\rho + R}{ |\rho - R| } \tan \frac{\phi}{2} \right)
%\right|_{\psi}^{\pi} = \\ = \frac{2}{ |\rho^2 - R^2| } \left( \frac{\pi}{2} -
%\arctan \left( \frac{\rho + R}{ |\rho - R| } \tan \frac{\psi}{2} \right) \right)
%\end{aligned} \end{equation*} }

\begin{equation*} \begin{aligned}
I_1 \{ S_2 \} = \frac{ | \rho^2 - R^2 | }{2 \pi \rho^2} \left(
\arctan \left( \frac{\rho + R}{ | \rho - R | } \tan \frac{\psi}{2} \right) -  
\frac{\pi}{2} \right) - \frac{R}{\rho} \frac{\sin \psi}{2 \pi} + 
\frac{\rho^2 + R^2}{4 \rho^2} \frac{\pi - \psi}{\pi}
\end{aligned} \end{equation*}
%
%\textcolor{blue}{ \begin{equation*} \begin{aligned}
%\int_{0}^{\pi} \frac{d \phi}{R^2 + \rho^2 - 2 \rho R \cos \phi} =
%\left. \frac{2}{ | \rho^2 - R^2 | } \arctan \left( \frac{ | \rho^2 - R^2 | }
%{\left( \rho - R \right)^2} \tan \frac{\phi}{2} \right)
%\right|_{0}^{\pi} = \\ = \frac{2}{ | \rho^2 - R^2 | } \left.
%\arctan \left( \frac{\rho + R}{ | \rho - R | } \tan \frac{\phi}{2} \right)
%\right|_{0}^{\pi} = \frac{2}{ | \rho^2 - R^2 | } \frac{\pi}{2}
%\end{aligned} \end{equation*} }

\begin{equation*} \begin{aligned}
I_1 \{ S_3 \} = \frac{\rho^2 + R^2}{4 \rho^2} - 
\frac{ |\rho^2 - R^2| }{4 \rho^2} = \begin{cases}
1/2 , \rho < R \\
R^2 / 2 \rho^2, \rho > R
\end{cases}
\end{aligned} \end{equation*}

Для того щоб отримати значення інтегралу на осі аплікат повернемось до 
початкового виду $ I_1 $ з \eqref{eq:int1start}. Користуючись асимптотичною 
властивістю функції Бесселя \eqref{eq:limJ1toZ} побачимо що інтеграл 
зведеться до випадку \eqref{eq:intJJtable}.
%
%\textcolor{blue} {\begin{equation*} \begin{aligned}
%\left. I_1 \right|_{\rho = 0} = R \int\limits_{0}^{\infty} d \nu
%J_1 \left( \nu R \right) \frac{J_1 \left( \nu \rho \right) }{\nu \rho}
%J_0 \left( \nu \sqrt{c^2 t^2 - z^2} \right) = \\
%= \frac{R}{2} \int\limits_{0}^{\infty} d \nu
%J_1 \left( \nu R \right) J_0 \left( \nu \sqrt{c^2 t^2 - z^2} \right) = 
%\left. \frac{I_2}{2} \right|_{\rho = 0}
%\end{aligned} \end{equation*} }
%
%\textcolor{blue} {\begin{equation*}
%\left. I_1 \right|_{\rho = 0} = \frac{1}{2} \begin{cases}
%0, 0 < R < \sqrt{c^2t^2 - z^2} \\
%\frac{R}{2} \left( c^2t^2 - z^2 \right)^{-1/2}, 0 < R = \sqrt{c^2t^2 - z^2} \\ 
%1, 0 < \sqrt{c^2t^2 - z^2} < R 
%\end{cases}
%\end{equation*} }

\begin{equation}
\left. I_1 \right|_{\rho = 0} = \frac{1}{2} \begin{cases}
0, 0 < R < \sqrt{c^2t^2 - z^2} \\
1/2, 0 < R = \sqrt{c^2t^2 - z^2} \\ 
1, 0 < \sqrt{c^2t^2 - z^2} < R 
\end{cases}
\end{equation}

На останок, спростимо тригонометричні вирази, що містять $ \psi $. Розглянемо 
$ \psi = \arccos f(r,t) $, де $ f(r,t) $ задовільна функція координат. 
Тоді $ f(r,t) = \cos \psi $. Зазначимо, що з означення відомо, що 
$ \psi \in \left[ 0, \pi \right] $, тому $ \sin \psi \geq 0 $. Таким чином:

\begin{equation*} \begin{aligned}
\sin \psi = \sqrt{1 - \cos^2 \psi } = \sqrt{1 - f^2(r,t)}
\end{aligned} \end{equation*}

Згадуючи введене означення для $ \psi $ зашипимо, що

%\textcolor{blue}{ \begin{equation*} \begin{aligned}
%\psi = \arccos \left( \frac{\rho^2 + R^2}{2 \rho R} - 
%\frac{c^2 t^2 - z^2}{2 \rho R} \right)
%\end{aligned} \end{equation*} }
%
%\textcolor{blue}{ \begin{equation*} \begin{aligned}
%\sin \psi = \sqrt{1 - \left( \frac{\rho^2 + R^2}{2 \rho R} - 
%\frac{c^2 t^2 - z^2}{2 \rho R} \right)^2} = 
%\sqrt{1 - \frac{\left( \rho^2 + R^2 - c^2 t^2 + z^2 \right)^2}
%{4 \rho^2 R^2} } = \\ = \sqrt{\frac{4 \rho^2 R^2}{4 \rho^2 R^2} - 
%\frac{\left( \rho^2 + R^2 - c^2 t^2 + z^2 \right)^2}{4 \rho^2 R^2} } =
%\sqrt{\frac{4 \rho^2 R^2 - \left( \rho^2 + R^2 - c^2 t^2 + z^2 \right)^2}
%{4 \rho^2 R^2}} = \\
%= \frac{1}{2 \rho R} \sqrt{4 \rho^2 R^2 - \left( \rho^2 + R^2 \right)^2 +
%2 \left( \rho^2 + R^2 \right) \left( c^2 t^2 - z^2 \right) - 
%\left( c^2 t^2 - z^2 \right)^2} = \\
%= \frac{1}{2 \rho R} \sqrt{- \left( \rho^2 - R^2 \right)^2 +
%2 \left( \rho^2 + R^2 \right) \left( c^2 t^2 - z^2 \right) - 
%\left( c^2 t^2 - z^2 \right)^2} = \\
%= \frac{c^2 t^2 - z^2}{2 \rho R} \sqrt{2 \frac{\rho^2 + R^2 }{c^2 t^2 - z^2} - 
%\left( \frac{\rho^2 - R^2 }{c^2 t^2 - z^2} \right)^2 - 1}
%\end{aligned} \end{equation*} }
%
%\textcolor{red}{ \begin{equation*} \begin{aligned}
%4 \rho^2 R^2 - (\rho^2 + R^2 - c^2 t^2 + z^2)^2 = \\
%= 4 \rho^2 R^2 - (\rho^2 + R^2)^2 - (c^2 t^2 - z^2)^2 + 
%2 (\rho^2 + R^2) (c^2 t^2 - z^2) = \\
%= - (\rho^2 - R^2)^2 - (c^2 t^2 - z^2)^2 \pm 
%2 R^2 (c^2 t^2 - z^2) + 2 (\rho^2 + R^2) (c^2 t^2 - z^2) = \\
%= 4 R^2 (c^2 t^2 - z^2) - (\rho^2 - R^2)^2 - (c^2 t^2 - z^2)^2 +
%2 (\rho^2 - R^2) (c^2 t^2 - z^2) = ?
%\end{aligned} \end{equation*} }

\begin{equation*} \begin{aligned}
\frac{R}{\rho} \frac{\sin \psi}{2 \pi} = 
\frac{\sqrt{4 \rho^2 R^2 - (\rho^2 + R^2 - c^2t^2 + z^2)^2}}{4 \pi \rho^2}
\end{aligned} \end{equation*}

%\textcolor{blue}{ \begin{equation*} \begin{aligned}
%\tan \frac{\psi}{2} = \pm \sqrt{ \frac{1 - \cos \psi}{1 + \cos \psi} } = 
%\sqrt{ \frac{1- \frac{\rho^2 + R^2}{2 \rho R} + \frac{c^2 t^2 - z^2}{2 \rho R}}
%{1 + \frac{\rho^2 + R^2}{2 \rho R} - \frac{c^2 t^2 - z^2}{2 \rho R}} } =
%\sqrt{ \frac{c^2t^2 - z^2 - \left( \rho - R \right)^2}
%{\left( \rho + R \right)^2 - \left( c^2t^2 - z^2 \right)} }
%\end{aligned} \end{equation*} }
%
%\textcolor{blue}{ \begin{equation*} \begin{aligned}
%\frac{\rho + R}{ |\rho - R| } \tan \frac{\psi}{2} = 
%\sqrt{ \frac{ \frac{c^2t^2 - z^2}{\left( \rho - R \right)^2} - 1}
%{ 1 - \frac{c^2t^2 - z^2}{\left( \rho + R \right)^2} } } = 
%\sqrt{ \left( \frac{\rho + R}{\rho - R} \right)^2
%\frac{c^2t^2 - z^2 - \left( \rho - R \right)^2}
%{\left( \rho + R \right)^2 - \left( c^2t^2 - z^2 \right)} }
%\end{aligned} \end{equation*} }

\begin{equation*} \begin{aligned}
\arctan \left( \frac{\rho + R}{ |\rho - R| } \tan \frac{\psi}{2} \right) - 
\frac{\pi}{2} = - \arctan \sqrt{ \left( \frac{\rho - R}{\rho + R} \right)^2
\frac{\left( \rho + R \right)^2 - \left( c^2t^2 - z^2 \right)} 
{\left( c^2t^2 - z^2 \right) - \left( \rho - R \right)^2} }
\end{aligned} \end{equation*}

\begin{equation*} \begin{aligned}
\pi - \psi = \arccos \left( \frac{c^2 t^2 - z^2 - \rho^2 - R^2}{2 \rho R} \right)
\end{aligned} \end{equation*}

Користуючись такими спрощеннями, можемо записати вираз в явному вигляді для 
області $ S_2 $

\begin{equation*} \begin{aligned}
I_1 \{ S_2 \} = \frac{\rho^2 + R^2}{4 \pi \rho^2} \arccos 
\left( \frac{c^2 t^2 - z^2 - \rho^2 - R^2}{2 \rho R} \right) - \\
- \frac{\sqrt{4 \rho^2 R^2 - (\rho^2 + R^2 - c^2t^2 + z^2)^2}}{4 \pi \rho^2} - \\
- \frac{ |\rho^2 - R^2| }{2 \pi \rho^2} 
\arctan \sqrt{ \frac{(\rho - R)^2}{(\rho + R)^2} \cdot
\frac{\left( \rho + R \right)^2 - \left( c^2t^2 - z^2 \right)} 
{\left( c^2t^2 - z^2 \right) - \left( \rho - R \right)^2} }
\end{aligned} \end{equation*}

%%%%%%%%%%%%%%%%%%%%%%%%%%%%%%%%%%%%%%%%%%%%%%%%%%%%%%%%%%%%%%%%%%%%%%%%%%%%%%%
\section{Отримання інтегралу $ I_2 $ в явному виді} \label{sec:i2anal}

\begin{equation} \label{eq:int2start}
I_2 = R \int \limits_{0}^{\infty} d \nu J_1 \left( \nu R \right) 
J_0 \left( \nu \rho \right) J_0 \left( \nu \sqrt{c^2t^2 - z^2} \right)
\end{equation}

Це табличний інтеграл, що може бути знайдений в 
\cite[ст. 228]{imp:SpecFunc1983}.

\begin{equation} \begin{aligned} \label{eq:intJ0J0J1tabel}
\int \limits_{0}^{\infty} d x J_0 \left( ax \right) 
J_0 \left( bx \right) J_1 \left( cx \right) = \begin{cases}
0, 0 < c < | a - b | \\ 
1/c, c > a + b
\end{cases} a, b > 0 \\
= \frac{1}{\pi c} \arccos \frac{a^2 + b^2 - c^2}{2ab},
| a - b | < c < a + b; a,b > 0
\end{aligned} \end{equation}

Фізичні властивості змінних в \eqref{eq:int2start} відповідають умові 
$ a,b,c > 0 $. Запишемо значення інтегралу відносно інших умов.

\begin{equation}
I_2 = \begin{cases}
0, 0 < R < | f_{-} \left( r, t \right) | \\
\frac{1}{\pi} \arccos \frac{c^2t^2 - z^2 + \rho^2 - R^2}
{2 \rho \sqrt{c^2t^2 - z^2}}, | f_{-} \left( r, t \right) | < R < 
f_{+} \left( r, t \right) \\ 1, f_{+} \left( r, t \right) < R \\
\end{cases}
\end{equation}

Тут для спрощення введено наступні переозначення:

\begin{equation*} \begin{aligned}
f_{-} \left( r, t \right) = \rho - \sqrt{c^2t^2 - z^2} \\
f_{+} \left( r, t \right) = \rho + \sqrt{c^2t^2 - z^2}
\end{aligned} \end{equation*}

Якщо $ \rho = 0 $, то область визначення інтегралу $ I_2 $ по формулі 
\eqref{eq:int2start} схропується і інтеграл стає невизначеним. У цьому випадку 
розглянемо інтеграл за допомогою формули \eqref{eq:intJJtable}.

%\textcolor{blue}{ \begin{equation*}
%I_2 \left( \rho = 0 \right) = \begin{cases}
%0, 0 < R < \sqrt{c^2t^2 - z^2} \\
%\frac{R}{2} \left( c^2t^2 - z^2 \right)^{-1/2}, 0 < R = \sqrt{c^2t^2 - z^2} \\ 
%1, 0 < \sqrt{c^2t^2 - z^2} < R 
%\end{cases}
%\end{equation*} }

\begin{equation}
I_2 \left( \rho = 0 \right) = \begin{cases}
0, 0 < R < \sqrt{c^2t^2 - z^2} \\
1/2, 0 < R = \sqrt{c^2t^2 - z^2} \\ 
1, 0 < \sqrt{c^2t^2 - z^2} < R 
\end{cases}
\end{equation}

\chapter{Комплексі функції Ломмеля двох змінних}
\label{ch:lommel}

%%%%%%%%%%%%%%%%%%%%%%%%%%%%%%%%%%%%%%%%%%%%%%%%%%%%%%%%%%%%%%%%%%%%%%%%%%%%%%%
\section{Визначення та лінійні властивості}

В \cite{Boersma1961} приводиться визначення через функцію Бесселя.
%
\begin{equation}
U_n \left[ W, Z \right] = \sum \limits_{m = 0}^{\infty} (-1)^m
\left( \frac{W}{Z} \right)^{n + 2m} J_{n + 2m} (Z)
\end{equation}

В \cite{Boersma1961} також можна знайти наступну властивість.
%
\begin{equation}
U_n \left[ W, Z \right] + U_{n+2} \left[ W, Z \right] = 
\left( \frac{W}{Z} \right)^n J_n (Z)
\end{equation}
%
\textcolor{lightgray} { \begin{equation*} \begin{aligned}
W_\pm = \pm i (\nu ct - \nu z) \\
Z = \sqrt{\nu^2 c^2t^2 - \nu^2 z^2}
\end{aligned} \end{equation*} }
%
\textcolor{lightgray}{ \begin{equation*}
U_0 \left[ W, Z \right] = \sum \limits_{m = 0}^{\infty} (-1)^m
\left( \frac{W}{Z} \right)^{2m} J_{2m} (Z) = J_0 (Z) - \frac{W^2}{Z^2} J_2 (Z) +
\frac{W^4}{Z^4} J_4 (Z) - ...
\end{equation*} }
%
\textcolor{lightgray}{ \begin{equation*}
U_2 \left[ W, Z \right] = \sum \limits_{m = 0}^{\infty} (-1)^m
\left( \frac{W}{Z} \right)^{2 + 2m} J_{2 + 2m} (Z) = 
\frac{W^2}{Z^2} J_2 (Z) - \frac{W^4}{Z^4} J_4 (Z) + \frac{W^6}{Z^6} J_6 (Z) - ...
\end{equation*} }
%
\textcolor{lightgray}{ \begin{equation*} \begin{aligned}
U_0 [W, Z] - U_2 [W, Z] = J_0(Z) - 2 \left( \frac{W^2}{Z^2} J_2 (Z) - 
\frac{W^4}{Z^4} J_4 (Z) + \frac{W^6}{Z^6} J_6 (Z) - ... \right)
\end{aligned} \end{equation*} }
%
\textcolor{lightgray}{ \begin{equation*} \begin{aligned}
U_0 [W, Z] - U_2 [W, Z] = J_0(Z) - 2 U_2(W,Z)
\end{aligned} \end{equation*} }
%
\textcolor{lightgray}{ \begin{equation*} \begin{aligned}
\left( \frac{W}{Z} \right)^{2n} = \left( 
\frac{- i \nu (\mathit{V}t - z)}
{\nu \sqrt{\mathit{V}^2t^2 - z^2}} \right)^{2n} = 
(-i)^{2n} \nu^{2n} \frac{(\mathit{V}t - z)^{2n}}
{(\mathit{V}t - z)^n (\mathit{V}t + z)^n} = \\
= (-i)^{2n} \left( \frac{\mathit{V}t - z}{\mathit{V}t + z} \right)^n = 
(-1)^{n} \left( \frac{\mathit{V}t - z}{\mathit{V}t + z} \right)^n = 
\left( - \frac{\mathit{V}t - z}{\mathit{V}t + z} \right)^n
\end{aligned} \end{equation*} }
%
\textcolor{lightgray}{ \begin{equation*} \begin{aligned}
U_0 [W, Z] - U_2 [W, Z] = J_0(Z) + 2 \left[ 
\frac{\mathit{V}t - z}{\mathit{V}t + z} J_2(Z) + \left( 
\frac{\mathit{V}t - z}{\mathit{V}t + z} \right)^2 J_4(Z) + ... \right]
\end{aligned} \end{equation*} }
%
\begin{equation} \begin{aligned}
U_0 [W, Z] - U_2 [W, Z] = J_0(Z) + 2 \sum_{m=1}^{\infty} \left( 
\frac{\mathit{V}t - z}{\mathit{V}t + z} \right)^m J_{2m} (Z)
\end{aligned} \end{equation}

%%%%%%%%%%%%%%%%%%%%%%%%%%%%%%%%%%%%%%%%%%%%%%%%%%%%%%%%%%%%%%%%%%%%%%%%%%%%%%%
\section{Інтегродиференціальні властивості}

Функція Ломмеля типова для нестаціонарних задач. В \cite[ст. 41]{Borisov1991} 
приведено корисні інтегродиференціальні.
%
\begin{equation} \begin{aligned}
\int \limits_{\xi}^{\tau} ds e^{-i \gamma s} J_0(\sqrt{s^2 - \xi^2 }) = 
\frac{e^{-i \gamma \tau}}{\sqrt{\gamma^2 - 1}} \left( U_1(W_+,Z) + \right. \\ 
\left. + i U_2(W_+,Z) - U_1(W_-,Z) - i U_2(W_-,Z) \right)
\end{aligned} \end{equation}

Тут $ W_\pm = (\gamma \pm \sqrt{\gamma^2 - 1}) (\tau - \xi) $ a 
$ Z = \sqrt{\tau^2 - \xi^2} $. Також для використання цієї формули повинна
виконуватись умова $ \tau - \xi > 0 $.
%
\textcolor{red}{ \begin{equation}
\left. \begin{array}{c}
U_{2n} (W_+, Z) = U_{2n} (W_-, Z) \\
U_{2n+1} (W_+, Z) = - U_{2n+1} (W_-, Z)
\end{array} \right| n \in \Z
\end{equation} }
%
\begin{equation} 
\partder{}{Z} U_n (W,Z) = - \frac{Z}{W} U_{n+1} (W,Z)
\end{equation}
%
\begin{equation}
2 \partder{}{W} U_n (W,Z) = U_{n-1} (W,Z) + 
\left( \frac{Z}{W} \right)^2 U_{n+1} (W,Z)
\end{equation}

%%%%%%%%%%%%%%%%%%%%%%%%%%%%%%%%%%%%%%%%%%%%%%%%%%%%%%%%%%%%%%%%%%%%%%%%%%%%%%%
\section{Інтеграл 3}

\begin{equation} 
I_3 = R \int \limits_{0}^{\infty} \frac{d \nu}{\rho \nu} 
J_1(\nu \rho) J_1(\nu R) (U_0[ W_-, Z ] - U_2[ W_-, Z ])
\end{equation}
%
\begin{equation} \label{eq:intergal3}
I_3 = I_1 - 2 R \int_{0}^{\infty} \frac{d \nu}{\nu \rho} 
J_1(\nu \rho) J_1(\nu R) U_2[ W_-, Z ]
\end{equation}
%
На осі випромінювання, тобто при $ \rho = 0 $. 
%
\textcolor{lightgray}{ \begin{equation*} \begin{aligned}
\left. 2 R \int_{0}^{\infty} \frac{d \nu}{\nu \rho} 
J_1(\nu \rho) J_1(\nu R) \sum_{m=1}^{\infty} \left( 
\frac{ct - z}{ct + z} \right)^m J_{2m} (Z) 
\right|_{\rho = 0} = \\ = R \frac{ct - z}{ct + z} \int_{0}^{\infty} 
d \nu J_1(\nu R) J_2 (\nu \sqrt{c^2t^2 + z^2}) + \\ 
+ R \left( \frac{ct - z}{ct + z} \right)^2 
\int_{0}^{\infty} d \nu J_1(\nu R) J_4 (\nu \sqrt{c^2t^2 + z^2}) + ...
\end{aligned} \end{equation*} }
%
\begin{equation*} \begin{aligned}
I_3 = I_1 + R \sum_{m=1}^{\infty} \left( \frac{ct - z}{ct + z} \right)^m 
\int_{0}^{\infty} d \nu J_1(\nu R) J_{2m} (\nu \sqrt{c^2t^2 + z^2})
\end{aligned} \end{equation*}
%
За допомогою табличного інтегралу 2.12.31.1 з 
\cite[ст. 209]{SpecFunc1983}, що має вид:
%
\begin{equation*} \begin{aligned}
\int_0^\infty J_\mu (bx) J_\nu (cx) dx = A_{\mu,\nu}^1
\end{aligned} \end{equation*}
%
\begin{equation*} \begin{aligned}
A_{\mu,\nu}^1 \left( 0 < c < b \right) = b^{-\nu-1} c^{\nu} 
\Gamma \left[ \begin{array}{l} 
(\nu+\mu+1)/2 \\ (\mu-\nu-1)/2 + 1, \nu + 1 \end{array} \right] \cdot
\\ \cdot F \left(  \frac{\nu+\mu+1}{2}, \frac{\nu-\mu+1}{2}; 
\nu+1; \frac{c^2}{b^2} \right)
\end{aligned} \end{equation*}
%
\begin{equation*} \begin{aligned}
A_{\mu,\nu}^1 \left( 0 < b < c \right) = b^{\mu} c^{-\mu-1} 
\Gamma \left[ \begin{array}{l} 
(\nu+\mu+1)/2 \\ (\nu-\mu-1)/2 + 1, \mu + 1 \end{array} \right] \cdot
\\ \cdot F \left(  \frac{1+\mu+\nu}{2}, \frac{1+\mu-\nu}{2}; 
\mu+1; \frac{b^2}{c^2} \right)
\end{aligned} \end{equation*}
%
\textcolor{lightgray}{ \begin{equation*} \begin{aligned}
\begin{array}{lr} \forall a,k \in \Z: &
(a)_k = \frac{\Gamma(a+k)}{\Gamma(a)} = \frac{(a+k-1)!}{(a-1)!} \end{array}
\end{aligned} \end{equation*} }
%
Тут через $ F $ означено гіпергеометричну функцію Гаусса.
%
\textcolor{lightgray}{ \begin{equation*} \begin{aligned}
F (a_1, a_2; b_1; z) = \sum_{k=0}^\infty 
\frac{(a_1)_k (a_2)_k}{(b_1)_k} \frac{z^k}{k!} 
\end{aligned} \end{equation*} }
%
\begin{equation*} \begin{aligned}
F (a_1, a_2; b_1; z) = \sum_{k=0}^\infty 
\frac{\Gamma(a_1+k) \Gamma(a_2+k) \Gamma(b_1)}
{\Gamma(a_1) \Gamma(a_2) \Gamma(b_1+k)} \frac{z^k}{k!}
\end{aligned} \end{equation*}
%
Таким чином, $ I_3 $ можна записати в явному виді для $ \rho = 0 $.
%
\textcolor{lightgray}{ \begin{equation*} \begin{aligned}
b = R, c = \sqrt{c^2t^2-z^2} \\
\mu = 1, \nu = \{ 2m \}_{m=1}^{\infty}
\end{aligned} \end{equation*} }
%
\textcolor{lightgray}{ \begin{equation*} \begin{aligned}
\left. A_{1,2m}^1 \right|^{\tau < R} =
\frac{(c^2t^2-z^2)^m}{R^{2m+1}}
\Gamma \left[ \begin{array}{l} 1+m \\ 1-m, 2m + 1 \end{array} \right]
F \left( m+1, m; 2m+1; \frac{c^2t^2-z^2}{R^2} \right)
\end{aligned} \end{equation*} }
%
\textcolor{lightgray}{ \begin{equation*} \begin{aligned}
\left. A_{1,2m}^1 \right|^{\tau < R} = \frac{(c^2t^2-z^2)^m}{R^{2m+1}}
\frac{\Gamma(1+m)}{\Gamma(1-m) \Gamma(2m+1)}
F \left( m+1, m; 2m+1; \frac{c^2t^2-z^2}{R^2} \right)
\end{aligned} \end{equation*} }
%
\textcolor{lightgray}{ \begin{equation*} \begin{aligned}
\left. A_{1,2m}^1 \right|^{\tau < R} = \frac{1}{R}
\frac{\Gamma(1+m)}{\Gamma(1-m) \Gamma(2m+1)} \cdot \\
\cdot \sum_{k=0}^\infty \frac{1}{k!} 
\frac{ \Gamma(m+k+1) \Gamma(m+k) \Gamma(2m+1) }
{ \Gamma(m+1) \Gamma(m) \Gamma(2m+k+1) }
\left( \frac{c^2t^2-z^2}{R^2} \right)^{k+m}
\end{aligned} \end{equation*} }
%
\begin{equation*} \begin{aligned}
\left. A_{1,2m}^1 \right|^{c^2t^2 - z^2 < R} \simeq
\frac{1}{ \Gamma(1-m) \Gamma(m) } = 
\frac{\sin(m \pi)}{\pi} = 0, \forall m \in \Z
\end{aligned} \end{equation*}
%
\textcolor{lightgray}{ \begin{equation*} \begin{aligned}
\left. A_{1,2m}^1 \right|^{\tau > R} = \frac{R}{c^2t^2-z^2}
\Gamma \left[ \begin{array}{l} m+1 \\ m, 2 \end{array} \right]
F \left( 1+m, 1-m; 2; \frac{R^2}{c^2t^2-z^2} \right)
\end{aligned} \end{equation*} }
%
\textcolor{lightgray}{ \begin{equation*} \begin{aligned}
\left. A_{1,2m}^1 \right|^{\tau > R} = \frac{R}{c^2t^2-z^2}
\frac{\Gamma(m+1)}{\Gamma(m) \Gamma(2)}
F \left( 1+m, 1-m; 2; \frac{R^2}{c^2t^2-z^2} \right)
\end{aligned} \end{equation*} }
%
\textcolor{lightgray}{ \begin{equation*} \begin{aligned}
\left. A_{1,2m}^1 \right|^{\tau > R} = \frac{mR}{c^2t^2-z^2}
\sum_{k=0}^\infty  \frac{\left( 1-m^2 \right)^k}{2^k k!} 
\left( \frac{R^2}{c^2t^2-z^2} \right)^{k}
\end{aligned} \end{equation*} }
%
\textcolor{lightgray}{ \begin{equation*} \begin{aligned}
\left. A_{1,2m}^1 \right|^{\tau > R} = \frac{m}{R}
\sum_{k=0}^\infty  \frac{\left( 1-m^2 \right)^k}{2^k k!} 
\left( \frac{R^2}{c^2t^2-z^2} \right)^{k+1}
\end{aligned} \end{equation*} }
%
\begin{equation}
\left. I_3 \right|^{\rho=0} = \left. I_1 \right|^{\rho=0} +
\sum_{m=1}^{\infty} \left( \frac{ct - z}{ct + z} \right)^m A_{1,2m}^1
\end{equation}
%
\begin{equation}
\left. A_{1,2m}^1 \right|^{\tau > R} = m
\sum_{k=0}^\infty  \frac{\left( 1-m^2 \right)^k}{2^k k!} 
\left( \frac{R^2}{c^2t^2-z^2} \right)^{k+1}
\end{equation}
%
\begin{equation} 
\left. A_{1,2m}^1 \right|^{\tau < R} = \frac{C_{2m-1}^{m-1}}{2}
\sum_{k=0}^\infty \frac{1}{k!} \left( \frac{m^2+m}{2m+1} \right)^k
\left( \frac{c^2t^2-z^2}{R^2} \right)^{k+m}
\end{equation}

%%%%%%%%%%%%%%%%%%%%%%%%%%%%%%%%%%%%%%%%%%%%%%%%%%%%%%%%%%%%%%%%%%%%%%%%%%%%%%%
\section{Інтеграл 4}

\begin{equation}
I_4 = R \int \limits_{0}^{\infty} d \nu J_0(\nu \rho) J_1(\nu R) 
(U_0[ W_-, Z ] - U_2[ W_-, Z ])
\end{equation}
%
\begin{equation}
I_4 = I_2 - 2 R \int \limits_{0}^{\infty} d \nu 
J_0(\nu \rho) J_1(\nu R) U_2[ W_-, Z ]
\end{equation}
%
По аналогії з \eqref{eq:intergal3}, використаємо формулу 2.12.31.1 з 
\cite[ст. 209]{SpecFunc1983} для пошуку значення інтегралу при $ \rho = 0 $.
%
\textcolor{lightgray}{ \begin{equation*} \begin{aligned}
I_4 = I_2 - 2 R \int \limits_{0}^{\infty} d \nu J_1(\nu R) U_2[ W_-, Z ]
\end{aligned} \end{equation*} }
%
\textcolor{lightgray}{ \begin{equation*} \begin{aligned}
I_4 = I_2 - 2 R \sum_{m=1}^{\infty} \left( \frac{ct - z}{ct + z} \right)^m 
\int_{0}^{\infty} d \nu J_1(\nu R) J_{2m} (\nu \sqrt{c^2t^2 + z^2})
\end{aligned} \end{equation*} }
%
\begin{equation}
\left. I_4 \right|^{\rho=0} = \left. I_2 \right|^{\rho=0} -
2 \sum_{m=1}^{\infty} \left( \frac{ct - z}{ct + z} \right)^m A_{1,2m}^1
\end{equation}

%%%%%%%%%%%%%%%%%%%%%%%%%%%%%%%%%%%%%%%%%%%%%%%%%%%%%%%%%%%%%%%%%%%%%%%%%%%%%%%
\section{Інтеграл 5}

\begin{equation}
I_5 = i R \int \limits_{0}^{\infty} d \nu J_1 \left( \nu R \right) 
J_1 \left( \nu \rho \right)
U_1 \left[ - i \nu \left( ct - z \right), \nu \sqrt{c^2t^2 - z^2} \right]
\end{equation}


\end{document}
