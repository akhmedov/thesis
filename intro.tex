\chapter*{Вступ}

%%%%%%%%%%%%%%%%%%%%%%%%%%%%%%%%%%%%%%%%%%%%%%%%%%%%%%%%%%%%%%%%%%%%%%%%%%%%%%%
\paragraph{Обґрунтування вибору теми дослідження}

Дисертаційну роботу присвячено дослідженню процесів випромінювання, поширення 
і прийому нестаціонарних електромагнітних хвиль. Поширення хвиль досліджується 
з урахуванням їх нелінійної взаємодії з середовищем. 

В якості джерела поля розглядається лінзова антена імпульсного випромінювання 
з круговою апертурою. Хоча, антена і знайшла широке застосування в задачах 
метрології і зондування, її характеристики, особливо у ближній зоні, 
ще недостатньо вивчені.

Відомо, що напрямленість лінзових антен у купі з ефектом електромагнітного 
снаряду призводить до концентрації енергії в ближній зоні випромінювача. Це 
явище викликає необхідність враховувати нелінійну взаємодію поля з середовищем 
поширення при значній амплітуді та крутизні фронту збуджуючого імпульсу.

Таким чином, створення моделі випромінювання імпульсів довільної 
форми лінзовою антеною з круговою апертурою без спрощень на лінійність 
середовища дозволить точніше моделювати випромінювання імпульсів з 
крутим фронтами та великою амплітудою. Уточнена модель процесу 
випромінювання дозволить поліпшити застосування антени в прикладних 
задачах. Серед таких задач -- визначення ефективної площини 
розсіювання та передача інформації з цифровим кодуванням.

Інший фактор, що впливає на ефективність застосування короткоімпульсних
систем, -- це врахування залежності імпульсних характеристик 
надширокосмугових антен від напрямку спостереження, особливо у ближній зоні. 
Виокремлення корисної інформації з електромагнітного імпульсу ускладнене 
залежністю його форми від точки спостереження. Отже, 
розвиток методик виокремлення кількісних та якісних характеристик, що несуть 
корисну інформацію, з нестаціонарного електромагнітного поля дозволить 
покращити робочі характеристики імпульсної радіоелектроніки.

%%%%%%%%%%%%%%%%%%%%%%%%%%%%%%%%%%%%%%%%%%%%%%%%%%%%%%%%%%%%%%%%%%%%%%%%%%%%%%%
\paragraph{Зв'язок роботи з науковими програмами, планами, темами}

Дисертаційну роботу виконано на кафедрі прикладної електродинаміки факультету 
радіофізики, біомедичної електроніки та комп’ютерних систем Харківського 
національного університету імені~В.~Н.~Каразіна відповідно до планів 
науково-дослідних робіт: ``Моделювання та дослідження 
нелінійних нанорозмірних систем із нестаціонарними та гармонійними 
збудженнями для перетворення полів та створення елементів спінтроники'' 
(номер державної реєстрації: 0114U002585, здобувач -- виконавець), 
``Імпульсні та синусоїдальні поля у нелінійних і шаруватих електродинамічних 
структурах та наносистемах як перетворювачах полів і моделей елементів 
спінтроніки'' (номер державної реєстрації: 0117U004851, здобувач -- виконавець),
``Електромагнітні поля імпульсних джерел та наноосциляторів в однорідних, 
шаруватих та нелінійних середовищах'' (номер державної реєстрації: 0120U102309, 
здобувач -- виконавець).

Здобувачем, в рамках даного дослідження, було пройдено стажування в 
Университеті Мурсії (Іспанія) на факультеті математики по програмі 
академічної мобільності для здобувачів ``Erasmus+'' в 2017 році.

%%%%%%%%%%%%%%%%%%%%%%%%%%%%%%%%%%%%%%%%%%%%%%%%%%%%%%%%%%%%%%%%%%%%%%%%%%%%%%%
\paragraph{Мета і завдання дослідження}

Метою роботи є дослідження поведінки нестаціонарного поля у ближній зоні антен, 
у тому числі, з урахуванням нелінійної взаємодії із середовищем, а також 
вдосконалення методик обробки прийнятих імпульсних надширокосмугових хвиль за 
рахунок використання особливостей поля ближньої зони.

Задачі дослідження:

\begin{enumerate}

\item Отримання перехідної функції лінзової антени імпульсного 
випромінювання, як явної функції від просторових координат та часу, що 
справедлива для довільної точки спостереження;

\item Аналіз енергетичних та нестаціонарних властивостей поля лінзової антени 
імпульсного випромінювання у ближній зоні для імпульсів різної форми;

\item Уточнення перехідної функції лінзової антени імпульсного 
випромінювання для випадку поліноміальної нелінійності середовища;

\item Розвиток методики виокремлення корисної інформації з нестаціонарного 
електромагнітного поля за рахунок застосування новітніх методів науки аналізу 
даних, що дозволить враховувати ефекти ближньої зони в реальному часі.

\end{enumerate}

Об’єкт дослідження -- електромагнітне поле, що випромінюється лінзовою антеною 
імпульсного випромінювання.

Предмет дослідження -- вплив ефектів ближньої зони випромінювання і нелінійних 
ефектів взаємодії поля із середовищем на часову залежність
електромагнітних імпульсів та їх інформаційну ємність.

%%%%%%%%%%%%%%%%%%%%%%%%%%%%%%%%%%%%%%%%%%%%%%%%%%%%%%%%%%%%%%%%%%%%%%%%%%%%%%%
\paragraph{Методи дослідження}

Теоретичну основу дисертації становлять наукові праці вітчизняних та 
зарубіжних дослідників. Методологічну основу дисертації складають 
загальнонаукові та спеціальні наукові методи пізнання, серед яких:

\begin{enumerate}

\item Метод еволюційних рівнянь у часовій області. Метод теоретичної 
радіофізики, що застосовано для зведення системи рівнянь Максвела до 
системи рівнянь відносно скалярних функцій, шляхом неповного розділення 
змінних за методикою Рімана-Вольтера.

\item Метод функції Рімана. Метод розв'язання неоднорідних 
диференціальних рівнянь, що застосовано для розв'язку системи еволюційних 
рівнянь відносно коефіцієнтів розкладу електромагнітного поля по модовому 
базису.

\item Метод теорії збурень. Метод лінеаризації математичних задач, що 
застосовано для врахування слабкої нелінійності середовища, яку представлено 
у вигляді поліноміального розкладу вектору поляризації за ступенями 
напруженості електричного поля.

\item Метод зворотного поширення помилки. Метод машинного навчання,
що застосовано для розв'язання задачі тренування, а саме для 
оптимізації параметрів запропонованої моделі виокремлення корисної 
інформації з надширокосмугового радіосигналу.

\end{enumerate}

%%%%%%%%%%%%%%%%%%%%%%%%%%%%%%%%%%%%%%%%%%%%%%%%%%%%%%%%%%%%%%%%%%%%%%%%%%%%%%%
\paragraph{Наукова новизна отриманих результатів}

Отримано закон поширення нестаціонарних електромагнітних 
хвиль породжених круговою апертурою у нелінійному Керрівському 
середовищі, що дозволяє оцінити вплив нелінійних ефектів на поодинокий 
імпульс. Проаналізовано вплив перетворення мод у нелінійному середовищі 
при самодії поодинокого надширокосмугового електромагнітного імпульсу та 
встановлено укручення фронту імпульсу на осі випромінювання. 
Врахування нелінійних ефектів проведено на основі аналізу енергетичних 
властивостей поля у ближній зоні, де густина енергії найбільша.

В ході побудови нелінійної моделі випромінювання автором у першому 
наближені побудовано перехідну функцію антен імпульсного випромінювання. 
Тобто, отримано закон випромінювання нестаціонарних хвиль у вигляді 
явної функцію всіх просторових координат та часу без наближення 
дальньої зони, що дозволяє розв’язувати задачі моделювання поля антени 
типу LIRA в режимі реального часу.

Запропоновано авторську методику виокремлення корисної інформації з 
нестаціонарної імпульсної хвилі, що полягає в застосуванні аналогових 
рекурентних штучних нейронних мереж задля досягнення підвищеної точності 
у ближній зоні. Продемонстровано, що запропонована методика дозволяє 
досягти покращення якості радіоканалу у ближній, проміжній та дальній 
зонах імпульсних антен. Проведено моделювання задачі цифрової 
комунікації за запропонованою методикою обробки прийнятного 
сигналу, де кодування виконано надширокосмуговими наносекундними 
імпульсами різної форми.

%%%%%%%%%%%%%%%%%%%%%%%%%%%%%%%%%%%%%%%%%%%%%%%%%%%%%%%%%%%%%%%%%%%%%%%%%%%%%%%
\paragraph{Практичне значення отриманих результатів}

Дисертаційна робота відноситься до основних наукових напрямів сучасної 
радіофізики та визначає тенденції її подальшого розвитку. Напрямок
досліджень -- теорія випромінювання та приймання нестаціонарних 
нелінійних електромагнітних хвиль.

Створено модель антен імпульсного випромінювання у ближній зоні, що 
покращує результати попередніх досліджень і дозволяє якісніше моделювати 
процес випромінювання та приймання електромагнітних хвиль з застосуванням 
такої антени. Врахування нелінійної природи електромагнітного поля дозволяє
моделювати поширення високоамплітудних надкоротких імпульсів в різноманітних
задачах радіофізики. Описано нові нелінійні ефекти самодії нестаціонарного 
імпульсного поля крізь середовище.

До державного підприємства ``Український інститут інтелектуальної 
власності'' подано заявку наотримання патенту на  винахід ``Спосіб виокремлення 
корисної інформації з надширокосмугових (НШС) електромагнітних хвиль'' 
з номером a202004038. Патентним повіреним України з реєстраційним номером 
464 проведено аналіз патентоздатності, новизни та технічного рівня об'єкту 
захисту інтелектуальної власності. Використання розробленої методики дозволяє 
застосовувати математичні процеси високої складності в реальному часі 
при незначному споживанні енергії. Гнучкість методики  сприяє 
запровадженою моделей економіки замкненого циклу при виробництві.

Деякі отримані в дисертації результати включено до навчального курсу 
«Випромінювання надширокосмугових сигналів», що викладається на 
факультеті радіофізики, біомедичної електроніки і комп’ютерних систем
студентам IV--V курсів кафедри прикладної електродинаміки.


%%%%%%%%%%%%%%%%%%%%%%%%%%%%%%%%%%%%%%%%%%%%%%%%%%%%%%%%%%%%%%%%%%%%%%%%%%%%%%%
\paragraph{Особистий внесок здобувача}

В працях \cite{my:Telecom2018, my:UKRCON2017, my:UKRCON2019} здобувач провів
аналіз існуючих моделей поля лінзової антени імпульсного випромінювання 
при довільному нестаціонарному збуджені і створив модель, що не має недоліків,
які спостерігались в аналогічних дослідженнях. Отриманий розв'язок описує 
поведінку поля в усіх точках простору. Здобувачем проведено порівняння з 
існуючими моделями і не виявлено розбіжностей, що свідчать про помилковість.

В праці \cite{my:Vesnik2017-2} побудовано поперечні розподіли енергії, а
в праці \cite{imp:Vesnik2018} теоретично обґрунтовано енергетичні згустки, 
що спостерігаються.

В працях \cite{my:Vesnik2015, my:Vesnik2017, my:Vesnik2017-2, my:MMET2014, 
my:UWBUSIS2014, my:ICATT2015, my:UWBUSIS2016, my:KPI2016, my:DIPED2019} 
здобувач провів аналітичну роботу по врахуванню нелінійної взаємодії 
нестаціонарного електромагнітного поля з тривимірним середовищем, 
подібним за нелінійними властивостями до атмосфери землі. Також здобувач
провів числове моделювання процесу нелінійної взаємодії та проаналізував 
його результати.

Здобувачем запропоновано \cite{my:UWBUSIS2018} і теоретично обґрунтовано
\cite{my:TKEA2020} авторську методику виокремлення корисної інформації 
з нестаціонарної електромагнітної хвилі. Здобувачем проведено числове 
моделювання \cite{my:TKEA2020}, що демонструє працездатність 
та переваги нової концепції.

%%%%%%%%%%%%%%%%%%%%%%%%%%%%%%%%%%%%%%%%%%%%%%%%%%%%%%%%%%%%%%%%%%%%%%%%%%%%%%%
\paragraph{Апробація матеріалів дисертації}

Основні результати, що складають в дисертацію були представлені 
на конференціях і семінарах міжнародного рівня:

\begin{enumerate}

	\item 15th International Conference on Mathematical Methods in 
	Electromagnetic Theory (MMET) (Dnipropetrovsk, 2014)

	\item Proc. 7th International Conference on Ultrawideband and 
	Ultrashort Impulse Signals (UWBUSIS) (Kharkiv, 2015) (автор отримав 
	винагороду за кращу доповідь секції)

	\item International Conference on Antenna Theory and Techniques
	(ICATT) (Kharkiv, 2015)

	\item Ultrawideband and Ultrashort Impulse Signals 
	(UWBUSIS) (Odessa, 2016)

	\item Radio Electronics and Info Communications (Kiev, 2016)

	\item 2017 IEEE First Ukraine Conference on Electrical and Computer 
	Engineering (UKRCON) (Kiev, 2017)

	\item Ultrawideband and Ultrashort Impulse Signals (UWBUSIS) 
	(Odessa, 2018) (робота зайняла 1е місце на конкурсі робіт молодих вчених)

	\item 2017 IEEE 2nd Ukraine Conference on Electrical and Computer 
	Engineering (UKRCON) (Lviv, 2019)

	\item 2019 XXIVth International Seminar/Workshop on Direct and Inverse
	Problems of Electromagnetic and Acoustic Wave Theory (DIPED) (Lviv, 2019)
	
	\item 2020 IEEE Ukrainian Microwave Week (UkrMW) (Kharkiv, 2020)
\end{enumerate} 

\paragraph{Публікації}

Матеріали дисертації опубліковано у 18 наукових працях, серед яких 7
статей у наукових фахових виданнях (зокрема з них 1 стаття, що входять до 
наукометричної бази даних Scopus), і 9 тез доповідей на конференціях 
міжнародного рівня. Також на основі даних, отриманих в процесі виконання 
дисертаційної роботи, автором було подано заяву на отримання патенту на 
винахід до Українського інституту інтелектуальної власності.

%%%%%%%%%%%%%%%%%%%%%%%%%%%%%%%%%%%%%%%%%%%%%%%%%%%%%%%%%%%%%%%%%%%%%%%%%%%%%%%
\paragraph{Обсяг і структура дисертації}

\textcolor{red}{Дисертація складається зі вступу, 4 розділів, висновків, списку використаної 
літератури та 4-х додатків. Загальний обсяг дисертації становить ... сторінок, з 
яких ... сторінок основного тексту. Список використаної літератури на 13-и 
сторінках включає в себе 122 найменування. Всього у дисертації 42 рисунка, 
дві таблиці.}

%%%%%%%%%%%%%%%%%%%%%%%%%%%%%%%%%%%%%%%%%%%%%%%%%%%%%%%%%%%%%%%%%%%%%%%%%%%%%%%
% \paragraph{Подяка}

% Проведення наукового дослідження, а також написання цієї кваліфікаційної роботи
% було б неможливе без заохочення і фінансової підтримки, що було мені надано.

% Я з гордістю висловлюю подяку колись моїм викладачам, а тепер і колегам

% На різних етапах мого дослідження долучались...

% заохочення і неоціненний інтелектуальний внесок

% Подяка контреб'юторам проекту Maxwell

% Моїй сім'ї -- найдорожчим людям, без яких я -- ніхто.

