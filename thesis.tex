\documentclass{vakthesis}

\usepackage[T2A]{fontenc}
\usepackage[utf8]{inputenc}
\usepackage[english,russian,ukrainian]{babel}

\usepackage[intlimits]{amsmath}
\allowdisplaybreaks
\usepackage{amsthm}
\usepackage{amssymb}

% Таблиці зі стовпчиками, що розтягуються
\usepackage{tabularx}
\usepackage{geometry}
\geometry{hmargin={30mm,15mm},lines=29,vcentering}

% експорт зображень
\usepackage{graphicx}
\graphicspath{ {Application/} }

\usepackage{eqparbox} 
\usepackage[x11names]{xcolor} 

% нумкрация теорем/лемм
\theoremstyle{plain}
\newtheorem{theorem}{Теорема}[chapter]
\newtheorem{lemma}{Лема}[chapter]
\newtheorem{corollary}{Наслідок}[chapter]
\theoremstyle{definition}
\newtheorem{definition}{Означення}[chapter]
\newtheorem{example}{Приклад}[chapter]
\theoremstyle{remark}
\newtheorem{remark}{Зауваження}[chapter]

\DeclareMathOperator{\Rea}{Re}
\DeclareMathOperator{\Ima}{Im}
\DeclareMathOperator{\sinc}{sinc} % \sinc t = \frac{\sin t}{t}
% \DeclareMathOperator{\min}{min}

\newcommand{\N}{\mathbb{N}}
\newcommand{\Z}{\mathbb{Z}}
\newcommand{\Q}{\mathbb{Q}}
\newcommand{\R}{\mathbb{R}}

\newcommand{\rot}{\mathop{\mathrm{rot}}\nolimits} % rot{A}

\newcommand{\crossprod}[2]{ \left[  #1 \times #2 \right] } % вектор произвед
\newcommand{\dotprod}[2]{ \left<  #1 \cdot #2 \right> } % скалрное произвед
\newcommand{\triple}[3]{ \left<  #1 , #2 , #3 \right> } % скалрное произвед

\newcommand{\vect}[1]{ \overrightarrow{\mathbf #1} } % bold and with arrow
\newcommand{\func}[2]{ #1 \left( #2 \right) } % функциональная зависимость

\newcommand{\partder}[2]{ \frac{\partial #1}{\partial #2}}
\newcommand{\derivat}[2]{ \frac{d #1}{d #2}}

\newcommand{\mybibappendix}{
	
	\begin{center} 
		\textit{\textbf{Наукові праці у наукових фахових виданнях України:}}
	\end{center}
	
	\newcounter{ItemsInMyWriting}
	
	\begin{enumerate}
		
		\item Dumin O.M., Tretyakov O.A., \textbf{Akhmedov R.D.}, Dumina O.O. 
		Evolutionary Approach for the Problem of Electromagnetic Fiead 
		Propagation Through Nonlinear Medium // Вісник Харківського національного 
		університету імені В.Н. Каразіна, Серія ``Радіофізика та електроніка''. 
		2015. випуск 24. С. 23--28.
		
		\textit{Внесок здобувача: Аналітична робота по доказу та виведенню 
			математичних співвідношень. Аналіз отриманих результатів.}
		
		\item Думін О. М., \textbf{Ахмедов Р. Д.}, Міжмодове перетворення нестаціонарного 
		електромагнітного поля в нелінійному необмеженому середовищі // Вісник 
		Харківського національного університету імені В.Н. Каразіна, Серія 
		``Радіофізика та електроніка''. 2017, випуск 26. С. 42--47.
		
		\textit{Внесок здобувача: Отримання модового розкладу поля у відкритому 
			просторі методом еволюційних рівнянь. Статистична обробка отриманих результатів. }
		
		\item Думін О. М., \textbf{Ахмедов Р. Д.}, Випромінювання та розповсюдження 
		електромагнітного снаряду в нелінійному середовищі // Вісник 
		Харківського національного університету імені В.Н. Каразіна, Серія 
		``Радіофізика та електроніка''. 2017, випуск 27. С. 37--42.
		
		\textit{Внесок здобувача: Застосування теорії збурень для врахування 
			нелінійних складових поляризації}
		
		\item Думін О., \textbf{Ахмедов Р.}, Черкасов Д., Імпульсне випромінювання 
		антени з круговою апертурою в ближній зоні // Вісник Харківського 
		національного університету імені В. Н. Каразіна. Серія ``Радіофізика та 
		електроніка''. 2018. випуск 28. C. 30--33.
		
		\textit{Внесок здобувача: Підготовка графічних матеріалів до публікації. 
			Аналітична робота над математичним апаратом методу еволюційних рівнянь.}
		
		\item Думін О.М., \textbf{Ахмедов Р.Д.}, Черкасов Д.В., Поширення імпульсної 
		електромагнітної хвилі в керрівському середовищі // Вісник Харківського 
		національного університету імені В.Н. Каразіна. ``Радіофізика та 
		електроніка''. 2018. Вип. 29. С.11--16.
		
		\textit{Внесок здобувача: Розв'язання системи рівнянь Максвела з урахуванням
			нелінійних властивостей середовища для неоднорідності у вигляді плаского 
			диску з електричним струмом.}
		
		\item \textbf{Ахмедов Р.Д.}, Виокремлення корисної інформації з 
		надширокосмугової хвилі у ближній зоні випромінювання. ``Технология и 
		конструирование в электронной аппаратуре'', 2020, No 3-4, с. 3—10. DOI: 
		http://dx.doi.org/10.15222/TKEA2020.3-4.03
		
		\textit{Внесок здобувача: Розробка авторської методики виділення корисної 
			інформації з імпульсної надширокосмугової електромагнітної хвилі, проведення 
			числових симуляцій процесу випромінювання-поширювання-приймання імпульсів з
			урахуванням розробленої методики.}
		
		\setcounter{ItemsInMyWriting}{\value{enumi}}
	\end{enumerate}
	
%	\begin{center}
%		\textit{\textbf{Патенти:}}
%	\end{center}
%		
%	\begin{enumerate}
%		\setcounter{enumi}{\value{ItemsInMyWriting}}
%		
%		\item \textbf{Ахмедов Р. Д.}, Спосіб виділення корисної інформації з 
%		надширокосмугових (НШС) електромагнітних хвиль // Український інститут 
%		інтелектуальної власності. Київ. 2020.
%		
%		\textit{Внесок здобувача: Розроблено методику виділення корисної інформації 
%			з нестаціонарних імпульсних електромагнітних хвиль. Проведено порівняльну 
%			характеристику різних.}
%		
%		\setcounter{ItemsInMyWriting}{\value{enumi}}
%	\end{enumerate}
		
	\begin{center} 
		\textit{\textbf{Наукові праці у фахових виданнях, що входять до 
				міжнародних наукометричних баз:}}
	\end{center}
	
	\begin{enumerate}
		\setcounter{enumi}{\value{ItemsInMyWriting}}
		
		\item \textbf{Akhmedov R.}, Dumin O., Katrich V., Impulse radiation of antenna 
		with circular aperture // Telecommunications and Radio Engineering. Kharkiv. 
		2018. Vol. 77. P. 1767--1784.
		
		\textit{Внесок здобувача: Розв'язання задачі випромінювання імпульсу довільної
			геометричної форми лінзовою імпульсною антеною з круговою апертурою. Аналітична
			робота по отриманню перехідної функції для ближньої зони, як явної функції 
			від просторових координат та часу.}
		
		\setcounter{ItemsInMyWriting}{\value{enumi}}
	\end{enumerate}
		
	% \begin{center} 
	% \textit{\textbf{Наукові праці у фахових закордонних виданнях:}}
	% \end{center}
	
	\begin{center} 
		\textit{\textbf{Наукові праці апробаційного характеру (тези доповідей на 
				наукових конференціях) за темою дисертації:}}
	\end{center}
	
	\begin{enumerate}
		\setcounter{enumi}{\value{ItemsInMyWriting}}
		
		\item Dumin O.M., Katrich V.A., \textbf{Akhmedov R.D.}, Tretyakov O.A., 
		Dumina O.O., Evolutionary Approach for the Problems of Transient 
		Electromagnetic Field Propagation in Nonlinear Medium // 15th International 
		Conference on Mathematical Methods in Electromagnetic Theory (MMET).
		Dnipropetrovsk. 2014.
		
		\item Dumin O.M., Tretyakov O.A., \textbf{Akhmedov R.D.}, Stadnik Yu.B., 
		Katrich V.A., Dumina, O.O., Modal Basis Method for Propagation of 
		Transient Electromagnetic Fields in Nonlinear Medium // Proc. 7th 
		International Conference on Ultrawideband and Ultrashort Impulse Signals 
		(UWBUSIS). Kharkiv. 2014.
		
		\item Dumin O.M., Tretyakov O.A., \textbf{Akhmedov R.D.}, and Dumina O.O., 
		Transient Electromagnetic Field Propagation through Nonlinear Medium in 
		time domain // International Conference on Antenna Theory and Techniques, 
		21 -- 24 April, 2015. Kharkiv. 2015.
		
		\item Dumin O.M., \textbf{Akhmedov R.D.}, Dumina O.O., Propagation of 
		Transient Field Radiated from Plane Disk in Nonlinear Medium // 
		Ultrawideband and Ultrashort Impulse Signals, 5--11 September 2016. 
		Odessa. 2016.
		
		\item Dumin O., \textbf{Akhmedov R.}, Dumina O., Transient Field 
		Radiation of Plane Disk into Nonlinear Medium // Radio Electronics and 
		Info Communications, 11--16 September 2016. Kiev. 2016.
		
		\item Dumin O., \textbf{Akhmedov R.}, Katrich V., Dumina O., Transient 
		Radiation of Circle with Uniform Current Distribution // 2017 IEEE First 
		Ukraine Conference on Electrical and Computer Engineering (UKRCON), 
		May 29 -- June 2 2017. Kiev. 2017.
		
		\item \textbf{Akhmedov R.}, Dumin O., Ultrashort Impulse Radiation from 
		Plane Disk with Uniform Current Distribution // Ultrawideband and 
		Ultrashort Impulse Signals, 4--7 September 2018. Odessa. 2018.
		
		\item Dumin O., \textbf{Akhmedov R.}, Dumina O., Cherkasov D., Near Zone 
		of Plane Disk with Uniform Transient Current Distribution // 2017 IEEE 2nd 
		Ukraine Conference on Electrical and Computer Engineering (UKRCON), 
		June 2 -- June 9 2019. Lviv. 2019.
		
		\item Dumin O., \textbf{Akhmedov R.}, Katrich V., Cherkasov D., 
		Impulse Electromagnetic Wave Propagation in Kerr Medium // 2019 XXIVth 
		International Seminar/Workshop on Direct and Inverse Problems of 
		Electromagnetic and Acoustic Wave Theory (DIPED). Lviv. 2019.
		
		\item  \textbf{R. Akhmedov}, Neural Radio in DS-UWB IoT Applications // 2020 
		IEEE Ukrainian Microwave Week (UkrMW), Kharkiv, Ukraine, 2020, pp. 1073-1078, 
		doi: 10.1109/UkrMW49653.2020.9252611.
		
		\setcounter{ItemsInMyWriting}{\value{enumi}}
	\end{enumerate}
	
} % Локальние определения

%\includeonly{ch1,bib,app3,app1}

% информация о подключенных пакетах в логах
%\listfiles

\begin{document}

% \title{Випромінювання нестаціонарних полів та їх розповсюдження в 
% sнелінійному просторі}

% \title{Імпульсний підводний радіозв'язок}
% \title{Дивракція та інтерференція випромінювання плаского диску}
% \title{Передача інформації пласким диском на межі розподілу двох середовищ}
% \title{Інформаційна емність імпульсних систем передачі даних}
\title{Розповсюдження та інформаційна емність імпульсного 
надширокосмугового випромінювання}

\author{Ахмедов Ролан Джавадович}

\supervisor{Думін Олександр Миколайович}
{кандидат фізико-математичних наук, доцент}

\speciality{01.04.03}

\udc{537.87}

\institution
{Харкивський національний університет імені В. Н. Каразіна}
{Харків}

\date{2018}

\maketitle

\tableofcontents

\chapter*{Вступ}

%%%%%%%%%%%%%%%%%%%%%%%%%%%%%%%%%%%%%%%%%%%%%%%%%%%%%%%%%%%%%%%%%%%%%%%%%%%%%%%
\paragraph{Актуальність теми}

Час та частота - це абстракції, що описують одне явище природи - зміна
енергетичних взаємодій в системі. Для макроскопічної електродинаміки, зокрема,
існують підходи, що застосовують обидві абстракції. Вибір абстракції, тобто 
підходу, визначаються характером задачі, що вирішується. Історично склалось, що
більш широке розповсюдження отримала частотно-орієнтована методологія, яка в 
повній мірі виправдала себе. Однак новітні технології все частіше потребують 
розв'язків, які методи частотної області надати не зможуть. Наприклад, 
розв'язання задач розповсюдження та випромінювання в нестаціонарних 
неоднорідних середовищах, що характеризуються нелінійними та анізотропними 
ефектами. Це відновило інтерес до методів часової області.

%%%%%%%%%%%%%%%%%%%%%%%%%%%%%%%%%%%%%%%%%%%%%%%%%%%%%%%%%%%%%%%%%%%%%%%%%%%%%%%
\paragraph{Зв'язок роботи з науковими програмами}

Erasmus+

%%%%%%%%%%%%%%%%%%%%%%%%%%%%%%%%%%%%%%%%%%%%%%%%%%%%%%%%%%%%%%%%%%%%%%%%%%%%%%%
\paragraph{Мета та задача дослідження}

%%%%%%%%%%%%%%%%%%%%%%%%%%%%%%%%%%%%%%%%%%%%%%%%%%%%%%%%%%%%%%%%%%%%%%%%%%%%%%%
\paragraph{Методи дослідження}

\textcolor{red}{
Мотивуючись цим, вибираємо часовий метод для аналізу імпульсного випромінювання,
а саме метод еволюційних рівнянь. В якості основи для метода є вилучення 
поперечних компонент поля методом Рімана-Вольтера, також відомого у вітчизняній 
літературі як метод еволюційних рівнянь. Це дозволяє звести задачу розв'язання 
системи рівнянь Максвела до розв'язання диференціального рівняння другого 
порядку в часних похідних. У випадку вакуумного середовища це рівняння 
зводиться до рівняння Клейна-Гордона.}

В данній роботі використано метод теорії збуджень для побудови рекурентного
ітеративного методу врахування нелінійності.

Напрям пошуку солітоноподібних розвя'зків задачі випромінювання спрямовано 
згідно досліджень, що передбачають їх появу в нелінійний оптиці. Той факт,
що за основні ідеї та положення нелінійної оптики близькі до положень 
нелінійної радіофізики в теоретичномц плані дозволяє говорити про правильний 
напрям досліджень.

\textcolor{red}{Рекурентні нейронні мережі}

\textcolor{red}{Метод перехідної функції}

%%%%%%%%%%%%%%%%%%%%%%%%%%%%%%%%%%%%%%%%%%%%%%%%%%%%%%%%%%%%%%%%%%%%%%%%%%%%%%%
\paragraph{Наукова новизна отриманих результатів}

\textcolor{red}{Розв'язок декількох задач випромінювання в часовій області для 
лінійного та нелінійного простору}

\textcolor{red}{Адаптація методу еволюційних рівнянь для чисельного розв'язку
та побудова відповідного програмного комплексу, що розповсюджується під 
ліцензією GPL, як один з проектів GNU спів-товариства}

\textcolor{red}{Авторський метод синтезу протоколів передачі інформації}

%%%%%%%%%%%%%%%%%%%%%%%%%%%%%%%%%%%%%%%%%%%%%%%%%%%%%%%%%%%%%%%%%%%%%%%%%%%%%%%
\paragraph{Практичне значення отриманих результатів}

\begin{enumerate} 
	\item Шведський проект по аналізу властивостей надпровідникових матеріалів
	\item Передача інформації імпульсами на велику відстань
	\item Анаітичний розв'язок для лінзевих антен збуджених TEM рупором
	\item Оцінка нелінійних явищ що супроводжують випромінювання LIRA-и
	\item Новий підхід до прийому надширокосмугового сигналу без АЦП
\end{enumerate} 

%%%%%%%%%%%%%%%%%%%%%%%%%%%%%%%%%%%%%%%%%%%%%%%%%%%%%%%%%%%%%%%%%%%%%%%%%%%%%%%
\paragraph{Особистий вклад дисертанта}

Аналітичний розв'язок задачі плаского диску у всіх точках
Розв'язок задачі плаского диску слабкого нелінійного простору
Авторський метод синтезу протоколів передачі інформації

%%%%%%%%%%%%%%%%%%%%%%%%%%%%%%%%%%%%%%%%%%%%%%%%%%%%%%%%%%%%%%%%%%%%%%%%%%%%%%%
\paragraph{Апробація результатів дослідження}

\begin{enumerate} 
	\item Erasmus+
	\item Upsala Univ.
	\item Young scientist award UWBUSIS 2018
\end{enumerate} 

%%%%%%%%%%%%%%%%%%%%%%%%%%%%%%%%%%%%%%%%%%%%%%%%%%%%%%%%%%%%%%%%%%%%%%%%%%%%%%%
\paragraph{Публікації}

%%%%%%%%%%%%%%%%%%%%%%%%%%%%%%%%%%%%%%%%%%%%%%%%%%%%%%%%%%%%%%%%%%%%%%%%%%%%%%%
\paragraph{Зміст роботи}

%%%%%%%%%%%%%%%%%%%%%%%%%%%%%%%%%%%%%%%%%%%%%%%%%%%%%%%%%%%%%%%%%%%%%%%%%%%%%%%
\paragraph{Подяка}

\chapter{Метод еволюційних рівнянь}
\label{ch:evolution}

%%%%%%%%%%%%%%%%%%%%%%%%%%%%%%%%%%%%%%%%%%%%%%%%%%%%%%%%%%%%%%%%%%%%%%%%%%%%%%
\section{Матеріальні рівняння середовища}

Електромагнітні властивості середовища можна математично описати шляхом
визначення векторів електричної $ \vect{D} $ та магнітної $ \vect{B} $ 
індукції за допомогою математичних рівнянь.
%
\begin{equation} \label{eq:MInduct}
\vect{D} = \epsilon_0 \vect{E} + \func{\vect{P}}{\vect{E},\vect{H}}
\end{equation}
%
\begin{equation} \label{eq:EInduct} 
\vect{B} = \mu_0 \vect{H} + \mu_0 \func{\vect{M}}{\vect{E},\vect{H}}
\end{equation}

Нелінійне середовище характеризується нелінійною залежністю поляризації
$ \vect{P} $ і намагніченості $ \vect{M} $ від векторів напруження 
електромагнітного поля. 

\textcolor{red}{Коли Р залежить від Н? Чи можна її не враховувати 
надалі. Приклади.}

\textcolor{red}{Яка фізична остова лежить у відмінності розмірностей доданих?}

В загальному випадку вектор поляризації має довільній вид та залежать від 
магнітної та електричної складової поля, а у лінійному випадку має вид
$ \epsilon \vect{E} $. Відносна діелектрична проникність середовища 
$ \epsilon $ взагалі є матриця, кожний з елементів якої, залежить від 
повного переліку незалежних координат та часу. Розглянемо шарувате середовище, 
як середу для розповсюдження і припустимо що фронт хвильового пакету проходить 
через шари під прямим кутом, тоді $ \epsilon $ є скалярна функція, що залежить 
лише від поздовжньої координати та часу. Аналогічні міркування можна провести 
і для вектора намагніченості.

\textcolor{red}{Уточнити чи не є $ \epsilon $ тензором для нелінійних 
компонент.}

Для задач слабкої нелінійності оптичної фізики використовується ряд Тейлора, 
так як при \textcolor{red}{не надто сильних полях} вклад більших степенів,
дійсно, мінімізується за рахунок невеликого відхилення від лінійної
функції \textcolor{red}{[джерело]}.
%
\begin{equation} \label{eq:polar}
\vect{P} = \epsilon_0 \left( \epsilon - 1 \right) \vect{E} + 
\vect{P^\prime}  = \epsilon_0 \left( \epsilon - 1 \right)
\vect{E} + \sum\limits_{i=2}^\infty  {\chi^e}_i \vect{E}^i 
\end{equation}
%
\textcolor{lightgray}{ \begin{equation*} \begin{aligned}
\vect{D} = \epsilon_0 \epsilon \vect{E} + \vect{P^\prime}
\end{aligned} \end{equation*} }
%
\begin{equation} \label{eq:magnit}
\vect{M} = \left( \mu - 1 \right) \vect{H} + 
\vect{M^\prime} = \left( \mu - 1 \right)
\vect{H} + \sum\limits_{i=2}^\infty  {\chi^m}_i \vect{H}^i 
\end{equation}
%
\textcolor{lightgray}{ \begin{equation*} \begin{aligned}
\vect{B} = \mu_0 \mu  \vect{H} + \mu_0 \vect{M^\prime}
\end{aligned} \end{equation*} }

Тут перший додаток має особливий фізичний смисл -- це лінійна складова поля та 
складова з найбільшим абсолютнім значенням коефіцієнту при векторі 
напруженості. Фізично, коефіцієнт є відносною проникністю середовища для 
відповідного степеню поля. Всі додатки крім першого це нелінійні складові поля,
кожен з яких має свій фізичний смисл.  Позначмо суму нелінійних складових 
векторів поляризації та намагніченості $ \vect{P^\prime} $ та 
$ \vect{M^\prime} $ відповідно.

\textcolor{red}{Смисл перших 5и додатків (таблиця).}

%%%%%%%%%%%%%%%%%%%%%%%%%%%%%%%%%%%%%%%%%%%%%%%%%%%%%%%%%%%%%%%%%%%%%%%%%%%%%%
\section{Рівняння Максвела}

\textcolor{red}{Закон Ампера}
\begin{equation} \label{eq:AmpereLow}
\crossprod{\nabla}{\vect{H}} = 
\frac{\partial \vect{D}}{\partial t} + \vect{J^\sigma} + \vect{J^e}
\end{equation}
%
\textcolor{red}{Закон индукції Фарадея}
\begin{equation} \label{eq:FaradayInduction}
-\crossprod{\nabla}{\vect{E}} =
\frac{\partial \vect{B}}{\partial t} + \vect{J^{h}}
\end{equation}
%
\textcolor{red}{Теорема Гаусса}
\begin{equation} \label{eq:GaussTheorem}
\dotprod{\nabla}{\vect{D}} = \rho^\sigma + \rho^e
\end{equation}
%
\textcolor{red}{Теорема Гаусса для магнітного поля}
\begin{equation} \label{eq:GaussMagnetic}
\dotprod{\nabla}{\vect{B}} = \rho^h
\end{equation}

%%%%%%%%%%%%%%%%%%%%%%%%%%%%%%%%%%%%%%%%%%%%%%%%%%%%%%%%%%%%%%%%%%%%%%%%%%%%%%
\subsection{Узагальнене джерело поля для задач випромінювання}

Додатки з нелінійними складовими векторів поляризації та намагніченості мають
розмірність густин струму, відповідно. Введемо узагальнений електричний 
$ \vect{J} $ та $ \vect{I} $ магнітній струми таким чином, щоб ці додатки 
не заважали майбутнім міркуванням. Ця дія відповідає фізичному змісту цих 
додатків та не порушує математичної консеквентності, що буде обумовлено далі.
%
\begin{equation*}
\vect{J} = \partder{\vect{P^\prime}}{t} + 
\vect{J^\sigma} + \vect{J^e}
\end{equation*}
%
\begin{equation*}
\vect{I} = \mu_0 \partder{\vect{M^\prime}}{t} + \vect{J^h}
\end{equation*}

Підставляючи поляризацію \eqref{eq:polar} і намагніченість 
\eqref{eq:magnit} до матеріальних рівнянь \eqref{eq:EInduct} и 
\eqref{eq:MInduct} з наступною підставковою в роторні рівняння Максвелла
\eqref{eq:AmpereLow} и \eqref{eq:FaradayInduction} отримаємо наступне: 
%
\textcolor{lightgray}{ \begin{equation*} \begin{aligned}
\crossprod{\nabla}{\vect{H}} = \epsilon_0 \partder{}{t} \left[ 
\vect{E} + \left( \epsilon - 1 \right) \vect{E} \right] + 
\partder{\vect{P^\prime}}{t} + \vect{J^\sigma} + \vect{J^e}= \\
= \epsilon_0 \partder{}{t} \left( \epsilon \vect{E} \right) +
\partder{\vect{P^\prime}}{t} + \vect{J^\sigma} + \vect{J^e} = 
\epsilon_0 \left( \partder{\epsilon}{t} 
\vect{E} + \epsilon \partder{\vect{E}}{t} \right) + 
\partder{\vect{P^\prime}}{t} + \vect{J^\sigma} + \vect{J^e}
\end{aligned} \end{equation*} }
%
\begin{equation} \label{eq:rotHfromE}
\crossprod{\nabla}{\vect{H}} = 
\epsilon_0 \partder{}{t} \left( \epsilon \vect{E} \right) + \vect{J}
\end{equation}
%
\begin{equation} \label{eq:rotEfromH} 
- \crossprod{\nabla}{\vect{E}} = 
\mu_0 \partder{}{t} \left( \mu \vect{H} \right) + \vect{I}
\end{equation}

Схожа ситуація і для джерел що представлена зарядами. Нехай наступні вирази
опишуть узагальнену електричну $ \varrho $ (ро) та магнітну $ g $ густини 
заряду.
%
\begin{equation*}
\varrho = \rho^\sigma + \rho^e - \dotprod{\nabla}{\vect{P^\prime}}
\end{equation*}
%
\begin{equation*}
g = \rho^h - \mu_0 \dotprod{\nabla}{\vect{M^\prime}}
\end{equation*}

Підставляючи поляризацію та намагніченість до відповідних формулювань теореми 
Гаусса отримаємо її вигляд для задачі слабкої нелінійності в анізотропному 
середовищі.
%
\textcolor{lightgray}{ \begin{equation*} \begin{aligned}
\dotprod{\nabla}{ \left( \epsilon_0 \epsilon \vect{E} + 
\vect{P^\prime} \right) } = \rho^\sigma + \rho^e \\
\dotprod{\nabla}{ \epsilon_0 \epsilon \vect{E} } = \rho^\sigma + \rho^e -
\dotprod{\nabla}{ \vect{P^\prime} }
\end{aligned} \end{equation*} }
%
\begin{equation} \label{eq:divE} 
\epsilon_0 \dotprod{\nabla}{ \epsilon \vect{E} } = \varrho
\end{equation}
%
\begin{equation} \label{eq:divH}
\mu_0 \dotprod{\nabla}{ \mu \vect{H} } = g
\end{equation}

%%%%%%%%%%%%%%%%%%%%%%%%%%%%%%%%%%%%%%%%%%%%%%%%%%%%%%%%%%%%%%%%%%%%%%%%%%%%%%
\subsection{Виключення поздовжних компонент поля}

Диференціальні рівняння першого порядку \eqref{eq:divE}, \eqref{eq:divH} та 
векторні другого \eqref{eq:rotHfromE}, \eqref{eq:rotEfromH} формують систему 
рівнянь Максвелла відносно невідомих векторних величин $ \vect{E} $ і 
$ \vect{H} $.

Для спрощення цієї системи пропонується використати метод розділення змінних
Фур'є. Аналогічно до методу функції Гріна з класичної електродинаміки, 
спрощення відбувається шляхом зменшення кількості невідомих 
\textcolor{red}{Джерело}, вилучаючи їх з рівняння. 

Метод Функції Гріна як і будь-який метод частотної області, вибирає саме час, 
як змінну для виключення, обмежуючи себе розгляданням квазі-стаціонарних 
процесів. Метод еволюційних рівнянь, в свою чергу, пропонує виключення 
просторової змінної. З трьох просторових координат можна виділити одну -- вісь 
розповсюдження поля. 

Виключення саме цієї просторової залежності зумовлено тісним зв'язком 
координати розповсюдження з координатою часу через принцип причинності. Його 
сутність в термінології спеціальної теорії відносності полягає в тому, що дві 
події можуть бути причинно зв'язані одна з одної тоді, і тільки коли, інтервал 
між ними часоподібний, що напряму слідує з того, що ніяка взаємодія не може 
розповсюджуватись швидше за світло. \cite[ст. 22]{imp:LandauII}. В 
електродинамічному сенсі це означає, що поле не може розповсюдитись далі у 
вільному просторі, ніж може пройти світло за той самий час та по тій самій осі 
випромінювання. Математично це можна записати, як $ ct - z > 0 $, де $ z $
поздовжна просторова координата розповсюдження, а $ c = 2,998 \cdot 10^8 $ м/с 
-- фундаментальна константа, швидкість світла в вакуумі.

З рівнянь Максвела відокремимо векторну компоненту 
$ \vect{z_0} $ для всіх величин. Як відомо, буль-який вектор можна розписати, 
як суму добутків ортів та відповідних проекцій: так оператор $ \nabla $ можна 
записати як $ \nabla_\perp + \vect{z_0} \partder{}{z} $, а довільний вектор
$ \vect{A} $, як $ \vect{A_\perp} + \vect{z_0} A_z $. Користуючись визначенням 
векторного добутку лінійної комбінації векторів, з \eqref{eq:rotHfromE} 
отримаємо два незалежні рівняння.
%
\textcolor{lightgray}{ \begin{equation*} \begin{aligned}
\rot{\vect{A}} = \crossprod{\nabla}{\vect{A}} = \crossprod
{\left( \nabla_\perp + \vect{z_0} \partder{}{z} \right)}
{\left( \vect{A_\perp} + \vect{z_0} A_z \right)} = \\
= \crossprod{\nabla_\perp}{\vect{A_\perp}} + 
\crossprod{\nabla_\perp}{\vect{z_0} A_z} +
\crossprod{\vect{z_0} \partder{}{z}}{\vect{A_\perp}} +
\crossprod{ \vect{z_0} \partder{}{z} }{ \vect{z_0} A_z } = \\
= \crossprod{\nabla_\perp}{\vect{A_\perp}} + 
\crossprod{\nabla_\perp}{\vect{z_0}} A_z +
\partder{}{z} \crossprod{\vect{z_0}}{\vect{A_\perp}}
\end{aligned} \end{equation*} }
%
\textcolor{lightgray}{ \begin{equation*} \begin{aligned}
\crossprod{\nabla}{\vect{H}} = 
\crossprod{\nabla_\perp}{\vect{H_\perp}} + 
\crossprod{\nabla_\perp}{\vect{z_0}} H_z +
\partder{}{z} \crossprod{\vect{z_0}}{\vect{H_\perp}} = \\
= \epsilon_0 \partder{}{t} \left( \epsilon  \vect{E_\perp} + 
\epsilon \vect{z_0} E_z \right) + \vect{J_\perp} + \vect{z_0} J_z
\end{aligned} \end{equation*} }
%
\begin{equation} \label{eq:rotHt} 
\crossprod{\nabla_\perp}{\vect{z_0}} H_z +
\partder{}{z} \crossprod{\vect{z_0}}{\vect{H_\perp}} =
\epsilon_0 \partder{}{t} \left( \epsilon  \vect{E_\perp} \right) + 
\vect{J_\perp}
\end{equation}
%
\textcolor{lightgray}{ \begin{equation*} \begin{aligned}
\dotprod{\vect{z_0}}{\crossprod{\nabla_\perp}{\vect{H_\perp}}} =
\triple{\vect{z_0}}{\nabla_\perp}{\vect{H_\perp}}
\end{aligned} \end{equation*} }
%
\begin{equation} \label{eq:rotHz}
\triple{\vect{z_0}}{\nabla_\perp}{\vect{H_\perp}} = 
\epsilon_0 \partder{}{t} \left( \epsilon  E_z \right) + J_z 
\end{equation}

Аналогічні міркування можна провести по відношенню до закону індукції Фарадея 
записаного по формі \eqref{eq:rotEfromH}.
%
\textcolor{lightgray}{ \begin{equation*} \begin{aligned}
- \crossprod{\nabla}{\vect{E}} = 
- \crossprod{\nabla_\perp}{\vect{E_\perp}} - 
\crossprod{\nabla_\perp}{\vect{z_0}} E_z -
\partder{}{z} \crossprod{\vect{z_0}}{\vect{E_\perp}} = \\
= \mu_0 \partder{}{t} \left( \mu  \vect{H_\perp} + \mu \vect{z_0} H_z \right) + 
\vect{I_\perp} + \vect{z_0} I_z
\end{aligned} \end{equation*} }
%
\begin{equation} \label{eq:rotEt} 
- \crossprod{\nabla_\perp}{\vect{z_0}} E_z -
\partder{}{z} \crossprod{\vect{z_0}}{\vect{E_\perp}} = 
\mu_0 \partder{}{t} \left( \mu  \vect{H_\perp} \right) + \vect{I_\perp}
\end{equation}
%
\textcolor{lightgray}{ \begin{equation*} \begin{aligned}
- \dotprod{\vect{z_0}}{\crossprod{\nabla_\perp}{\vect{E_\perp}}} = 
- \triple{\vect{z_0}}{\nabla_\perp}{\vect{E_\perp}}
\end{aligned} \end{equation*} }
%
\begin{equation} \label{eq:rotEz}
- \triple{\vect{z_0}}{\nabla_\perp}{\vect{E_\perp}} =
\mu_0 \partder{}{t} \left(\mu H_z \right) + I_z
\end{equation}

З теореми Гауса \eqref{eq:divE} та її інтерпретації до магнітного поля 
\eqref{eq:divH} також виключимо $ \vect{z_0} $ компоненту, тепер, користуючись 
комутативними та асоціативними властивостями скалярного добутку векторів.
%
\textcolor{lightgray}{ \begin{equation*} \begin{aligned}
\dotprod{\nabla}{\vect{A}} = \dotprod
{\left( \nabla_\perp + \vect{z_0} \partder{}{z} \right)}
{\left( \vect{A_\perp} + \vect{z_0} A_z \right)} = \\
= \dotprod{\nabla_\perp}{\vect{A_\perp}} + 
\dotprod{\nabla_\perp}{\vect{z_0} A_z}  +
\dotprod{\vect{z_0} \partder{}{z}}{\vect{A_\perp}} +
\dotprod{\vect{z_0} \partder{}{z}}{\vect{z_0} A_z} = \\
= \dotprod{\nabla_\perp}{\vect{A_\perp}} +
\dotprod{\nabla_\perp}{\vect{z_0}} A_z +
\dotprod{\vect{z_0}}{\partder{\vect{A_\perp}}{z}} +
\dotprod{\vect{z_0}}{\vect{z_0}} \partder{A_z}{z} = \\
= \dotprod{\nabla_\perp}{\vect{A_\perp}} + \partder{A_z}{z}
\end{aligned} \end{equation*} }
%
\begin{equation} \label{eq:divEt} 
\epsilon_0 \partder{}{z} \left( \epsilon E_z \right) = 
\varrho - \epsilon_0 \epsilon \dotprod{\nabla_\perp}{\vect{E_\perp}}
\end{equation}
%
\begin{equation} \label{eq:divHt}
\mu_0 \partder{}{z} \left( \mu H_z \right) = 
g - \mu_0 \mu \dotprod{\nabla_\perp}{\vect{H_\perp}}
\end{equation}

Як зазначалось раніше, в данні роботі розглядається пошарово неоднорідне 
середовище, а отже $ \epsilon = \epsilon(z,t) $ и $ \mu = \mu(z,t) $. Тому
маємо змогу винести показники проникності середовища з під оператора у 
від'ємнику.

Як видно з рівнянь \eqref{eq:rotHt} -- \eqref{eq:divHt}, поздовжні компоненти 
поля однозначно розділились, але саме поле залишилось електромагнітним, що 
типово для нестаціонарних задач. Для стаціонарних задач, в просторовому та 
електродинамічному планах, поле розділилось б на чисто електричне та чисто
магнітне за рахунок нульових похідних від часу.

Випишемо в окрему систему тільки ті рівняння, що містять $ H_z $ та виразимо з 
них саму цю компоненту. Тепер, діючи на рівняння \eqref{eq:rotHt} операторами 
$ \mu_0 \partder{}{t} \mu $ і $ \mu_0 \partder{}{z} \mu $ виключимо поздовжну 
магнітну компоненту з рівнянь Максвелла, підставивши, відповідно, 
\eqref{eq:divHt} та \eqref{eq:rotEz}. В результаті маємо систему не з трьох 
рівнянь, а вже з двох відносно поперечних компонент поля. Результат зашипимо не 
в вигляді системи, а як чотиривимірне векторне рівняння.
%
\textcolor{lightgray}{ \begin{equation*} \begin{aligned}
\begin{cases} 
\crossprod{\nabla_\perp}{\vect{z_0}} H_z =
\epsilon_0 \partder{}{t} \left( \epsilon \vect{E_\perp} \right) -
\partder{}{z} \crossprod{\vect{z_0}}{\vect{H_\perp}} + \vect{J_\perp} \\
- \triple{\vect{z_0}}{\nabla_\perp}{\vect{E_\perp}} =
\mu_0 \partder{}{t} \left(\mu H_z \right) + I_z \\ 
\mu_0 \partder{}{z} \left( \mu H_z \right) = 
g - \mu_0 \mu \dotprod{\nabla_\perp}{\vect{H_\perp}}
\end{cases}
\end{aligned} \end{equation*} }
%
\textcolor{lightgray}{ \begin{equation*} \begin{aligned}
\begin{cases} 
\left. \crossprod{\nabla_\perp}{\vect{z_0}} H_z = \vect{F_H} 
\right| \cdot \mu_0 \partder{}{z} \mu \\
\left. \crossprod{\nabla_\perp}{\vect{z_0}} H_z = \vect{F_H} 
\right| \cdot \mu_0 \partder{}{t} \mu \\
\mu_0 \partder{}{z} \left( \mu H_z \right) = 
g - \mu_0 \mu \dotprod{\nabla_\perp}{\vect{H_\perp}} \\
\mu_0 \partder{}{t} \left(\mu H_z \right) =
\triple{\nabla_\perp}{\vect{z_0}}{\vect{E_\perp}} - I_z
\end{cases}
\end{aligned} \end{equation*} }
%
\textcolor{lightgray}{ \begin{equation*} \begin{aligned}
\begin{cases} 
\crossprod{\nabla_\perp}{\vect{z_0}} \left(
g - \mu_0 \mu \dotprod{\nabla_\perp}{\vect{H_\perp}} \right) =
\mu_0 \partder{}{z} \left( \mu \vect{F_H} \right) \\
\crossprod{\nabla_\perp}{\vect{z_0}} \left(
\triple{\nabla_\perp}{\vect{z_0}}{\vect{E_\perp}} - I_z \right) = 
\mu_0 \partder{}{t} \left( \mu \vect{F_H} \right)
\end{cases}
\end{aligned} \end{equation*} }
%
\textcolor{lightgray}{ \begin{equation*} \begin{aligned}
\begin{cases} 
- \mu_0 \mu \dotprod{\crossprod{\nabla_\perp}{\vect{z_0}} \nabla_\perp}
{\vect{H_\perp}} = \mu_0 \partder{}{z} \left( \mu \vect{F_H} \right) -
\crossprod{\nabla_\perp}{\vect{z_0}} g \\
\left. \crossprod{\nabla_\perp 
\triple{\nabla_\perp}{\vect{z_0}}{\vect{E_\perp}}
}{\vect{z_0}} = \mu_0 \partder{}{t} \left( \mu \vect{F_H} \right) +
\crossprod{\nabla_\perp}{\vect{z_0}} I_z \right| \times \vect{z_0}
\end{cases}
\end{aligned} \end{equation*} }
%
\textcolor{lightgray}{ \begin{equation*} \begin{aligned}
\crossprod{ \crossprod
{\nabla_\perp \triple{\nabla_\perp}{\vect{z_0}}{\vect{E_\perp}}}
{\vect{z_0}} }{ \vect{z_0} } = \crossprod{ \crossprod{\nabla_\perp \phi}
{\vect{z_0}} }{\vect{z_0}} = \\ = - \crossprod{ \vect{z_0} }{ 
\crossprod{\nabla_\perp \phi}{\vect{z_0}} } = - \dotprod{\nabla_\perp \phi}
{ \dotprod{\vect{z_0}}{\vect{z_0}} } + \dotprod{\vect{z_0}}
{ \dotprod{\vect{z_0}}{\nabla_\perp \phi} } = \\ = - \nabla_\perp \phi = 
- \nabla_\perp \dotprod{\crossprod{\nabla_\perp}{\vect{z_0}}}{\vect{E_\perp}} = 
\dotprod{\nabla_\perp \crossprod{\vect{z_0}}{\nabla_\perp}}{\vect{E_\perp}}
\end{aligned} \end{equation*} }
%
\textcolor{lightgray}{ \begin{equation*} \begin{aligned}
\crossprod {\crossprod{\nabla_\perp}{\vect{z_0}} I_z}{\vect{z_0}} = 
- \crossprod {\vect{z_0}}{\crossprod{\nabla_\perp}{\vect{z_0}} I_z} = \\
= - \dotprod{{\nabla_\perp}}{\dotprod{\vect{z_0}}{\vect{z_0}}} I_z + 
\dotprod{\vect{z_0}}{\dotprod{\vect{z_0}}{{\nabla_\perp}}} I_z = 
- \nabla_\perp I_z 
\end{aligned} \end{equation*} }
%
\textcolor{lightgray}{ \begin{equation*} \begin{aligned}
\begin{cases} 
\dotprod{\crossprod{\vect{z_0}}{\nabla_\perp} \nabla_\perp} {\vect{H_\perp}} = 
\mu^{-1} \partder{}{z} \left( \mu \vect{F_H} \right) +
\left( \mu_0 \mu \right)^{-1} \crossprod{\vect{z_0}}{\nabla_\perp} g \\
\dotprod{\nabla_\perp \crossprod{\vect{z_0}}{\nabla_\perp}}{\vect{E_\perp}}
= - \mu_0 \partder{}{t} \left( \mu \crossprod{\vect{z_0}}{\vect{F_H}} \right) -
\nabla_\perp I_z 
\end{cases}
\end{aligned} \end{equation*} }
%
\begin{equation}
\left( \begin{array}{c} 
\dotprod{\crossprod{\vect{z_0}}{\nabla_\perp} \nabla_\perp} {\vect{H_\perp}} \\
\dotprod{\nabla_\perp \crossprod{\vect{z_0}}{\nabla_\perp}}{\vect{E_\perp}} \\
\end{array} \right) = \left( \begin{array}{c} 
\frac{1}{\mu} \partder{}{z} \left( \mu \vect{F_H} \right) +
\frac{1}{\mu_0 \mu} \crossprod{\vect{z_0}}{\nabla_\perp} g \\
- \mu_0 \partder{}{t} \left( \mu \crossprod{\vect{z_0}}{\vect{F_H}} \right) -
\nabla_\perp I_z 
\end{array} \right)
\end{equation}
%
\begin{equation*}
\vect{F_H} = \epsilon_0 \partder{}{t} \left( \epsilon \vect{E_\perp} \right) - 
\partder{}{z} \crossprod{\vect{z_0}}{\vect{H_\perp}} + \vect{J_\perp}
\end{equation*}

Переозначення $ \vect{F_H} $ не несе фізичного змісту, а введено лише для 
спрощення виду формул. Аналогічним чином виключимо компоненту $ E_z $. Отримане 
векторне рівняння матиме наступний вид.
%
\begin{equation}
\left( \begin{array}{c} 
\dotprod{\nabla_\perp \crossprod{\vect{z_0}}{\nabla_\perp}} {\vect{H_\perp}} \\
\dotprod{\crossprod{\vect{z_0}}{\nabla_\perp} \nabla_\perp}{\vect{E_\perp}} \\
\end{array} \right) = \left( \begin{array}{c} 
- \epsilon_0 \partder{}{t} \left( \epsilon \crossprod{\vect{F_E}}{\vect{z_0}} 
\right) - \nabla_\perp J_z \\
\frac{1}{\epsilon} \partder{}{z} \left( \epsilon \vect{F_E} \right) +
\frac{1}{\epsilon_0 \epsilon} \crossprod{\nabla_\perp \varrho}{\vect{z_0}}
\end{array} \right)
\end{equation}
%
\begin{equation*}
\vect{F_E} = \mu_0 \partder{}{t} \left( \mu  \vect{H_\perp} \right) +
\partder{}{z} \crossprod{\vect{z_0}}{\vect{E_\perp}} + \vect{I_\perp}
\end{equation*}

Ліва частина матричних рівнянь може бути представлена в виді оператора, 
що діє на чотиривимірній вектор. Таким чином для  

%%%%%%%%%%%%%%%%%%%%%%%%%%%%%%%%%%%%%%%%%%%%%%%%%%%%%%%%%%%%%%%%%%%%%%%%%%%%%%
% \section{Побудова модового базису}

%%%%%%%%%%%%%%%%%%%%%%%%%%%%%%%%%%%%%%%%%%%%%%%%%%%%%%%%%%%%%%%%%%%%%%%%%%%%%%
\section{Згортка електромагнітного поля по базису}

\textcolor{red} { \begin{equation}
\vect{H_\perp} = \frac{1}{\sqrt{\mu_0}} \left( 
\sum \limits_{m=-\infty}^{\infty} \int \limits_{0}^{\infty} d \nu
I_m^h \nabla_\perp \Psi_m + \sum \limits_{n=-\infty}^{\infty}
\int \limits_{0}^{\infty} d \chi I_n^e 
\crossprod{\vect{z_0}}{\nabla_\perp \Phi_n} \right)
\end{equation} }
%
\textcolor{red} { \begin{equation} 
\vect{E_\perp} = \frac{1}{\sqrt{\epsilon_0}} \left( 
\sum \limits_{m=-\infty}^{\infty} \int \limits_{0}^{\infty} 
d \nu V_m^h \crossprod{ \nabla_\perp \Psi_m }{ \vect{z_0} } +
\sum \limits_{n=-\infty}^{\infty} \int \limits_{0}^{\infty}
d \chi V_n^e \nabla_\perp \Phi_n \right)
\end{equation} }
%
\textcolor{red} { \begin{equation} 
H_z (r,t) = \frac{1}{\sqrt{\mu_0}} \sum \limits_{m=-\infty}^{\infty}
\int \limits_0^\infty \nu^2 d \nu h_m \Psi_m
\end{equation} }
%
\textcolor{red} { \begin{equation} 
E_z (r,t) = \frac{1}{\sqrt{\epsilon_0}} \sum \limits_{n=-\infty}^{\infty}
\int \limits_0^\infty \chi^2 d \chi e_n \Phi_n
\end{equation} }

%%%%%%%%%%%%%%%%%%%%%%%%%%%%%%%%%%%%%%%%%%%%%%%%%%%%%%%%%%%%%%%%%%%%%%%%%%%%%%
\section{Еволюційні рівняння}

\begin{equation}
\partial_z (\mu h_m) = \mu I_m^h + \frac{\sqrt[-2]{\mu_0}}{2 \pi}
\int_0^{2\pi} d \varphi \int_0^{\infty} \rho d \rho
\Psi_m^* (\nu) g;
\end{equation}
%
\begin{equation}
\partial_{ct} (\mu h_m) = - V_m^h - \frac{\sqrt{\epsilon_0}}{2 \pi}
\int_0^{2\pi} d \varphi \int_0^{\infty} \rho d \rho
\Psi_m^* (\nu) I_z;
\end{equation}
%
\begin{equation}
- \partial_{ct} (\epsilon V_m^h) - \partial_z I_m^h + \nu^2 h_m = 
\frac{\sqrt{\mu_0}}{2 \pi} \int_0^{2\pi} d \varphi 
\int_0^{\infty} \rho d \rho \crossprod{\vect{z_0}}{\vect{J_\perp}}
\nabla_\perp \Psi_m^* (\nu)
\end{equation}
%
\textcolor{lightgray} { \begin{equation*}
\partial_{ct} (\epsilon V_n^h) + \partial_z I_n^h = 
- \frac{\sqrt{\mu_0}}{2 \pi} \int_0^{2\pi} d \varphi 
\int_0^{\infty} \rho d \rho 
\dotprod {\vect{J_\perp}} {\nabla_\perp \Phi_n^* (\chi)}
\end{equation*} }
%
\begin{equation}
\partial_{ct} (\epsilon e_n) = - I_n^e - 
\frac{\sqrt{\mu_0}}{2 \pi} \int_0^{2\pi} d \varphi 
\int_0^{\infty} \rho d \rho \Phi_n^* (\chi) J_z
\end{equation}
%
\begin{equation}
\partial_{z} (\epsilon e_n) = \epsilon V_n^e + 
\frac{\sqrt[-2]{\epsilon_0}}{2 \pi} \int_0^{2\pi} d \varphi 
\int_0^{\infty} \rho d \rho \Phi_n^* (\chi) \varrho
\end{equation}
%
\begin{equation}
- \partial_{ct}(\mu I_n^e) - \partial_z V_n^e + \chi^2 e_n = 
\frac{\sqrt{\epsilon_0}}{2 \pi} \int_0^{2\pi} d \varphi 
\int_0^{\infty} \rho d \rho \crossprod{\vect{I_\perp}}{\vect{z_0}}
\nabla_\perp \Phi_n^* (\chi)
\end{equation}
%
\textcolor{lightgray} { \begin{equation*}
\partial_{ct}(\mu I_m^e) + \partial_z V_m^e = - 
\frac{\sqrt{\epsilon_0}}{2 \pi} \int_0^{2\pi} d \varphi 
\int_0^{\infty} \rho d \rho 
% \dotprod {\vect{I_\perp}} {\nabla_\perp \Phi_m^* (\nu)}
\dotprod {\crossprod {\vect{I_\perp}} {\vect{z_0}} } 
{ \nabla_\perp \Phi_m^* (\nu) }
\end{equation*} }


%%%%%%%%%%%%%%%%%%%%%%%%%%%%%%%%%%%%%%%%%%%%%%%%%%%%%%%%%%%%%%%%%%%%%%%%%%%%%%
% \section{Ітеративний підхід до врахування нелінійності}

\chapter{Випромінювання плаского диску}
\label{ch:pdisk}

%%%%%%%%%%%%%%%%%%%%%%%%%%%%%%%%%%%%%%%%%%%%%%%%%%%%%%%%%%%%%%%%%%%%%%%%%%%%%%%
\section{Постановка задачі}
%
\begin{equation}
\vect{j_0} \left( r, t \right) = \vect{J} = \vect{x_0} A_0 H(t) \delta(z) 
\left(  H(\rho) - H(\rho - R) \right)
\end{equation}
%
\begin{figure}[htbp] \begin{center}
\includegraphics[scale=0.55]{PlaneDisk}
\caption{Геометрія випромінювача} \label{fig:pdisk}
\end{center} \end{figure}
%
% \caption{Геометрія випромінювача} \label{fig:pdisk}
% \end{figure}
%
\textcolor{lightgray} { \begin{equation*} \begin{aligned}
\begin{cases}
\vect{\rho_0} = \vect{x_0} \cos \varphi + \vect{y_0} \sin \varphi \\
\vect{\varphi_0} = - \vect{x_0} \sin \varphi + \vect{y_0} \cos \varphi
\end{cases} \Rightarrow \mathbf{A} = \left( \begin{array}{cc}
\cos \varphi & \sin \varphi \\
- \sin \varphi & \cos \varphi
\end{array} \right)
\end{aligned} \end{equation*} }
%
\textcolor{lightgray} { \begin{equation*} \begin{aligned}
\vect{j_0} \left( \vect{\rho_0}, \vect{\varphi_0} \right) = 
\mathbf{A} \vect{j_0} \left( \vect{x_0}, \vect{y_0} \right) = \\
= H(t) \delta(z) (  H(\rho) - H(\rho - R) ) 
( \vect{\rho_0} \cos \varphi - \vect{\varphi_0} \sin \varphi )
\end{aligned} \end{equation*} }
%
\begin{equation}
j_m \left( r, t; \nu \right) = \frac{\sqrt{\mu_0}}{2\pi} 
\int \limits_{0}^{2\pi} d \varphi \int \limits_{0}^{\infty} \rho d \rho 
\vect{j_0} \crossprod{ \nabla_\perp \Psi_m^* }{ \vect{z_0} }
\end{equation}
%
\textcolor{lightgray} { \begin{equation*} \begin{aligned}
\crossprod{ \nabla_\perp \Psi_m^* }{ \vect{z_0} } = 
- \sqrt{\nu} e^{-im\varphi} \left( 
\vect{\varphi_0} \frac{J_{m-1} (\nu \rho) - J_{m+1} (\nu \rho)}{2} + 
\right. \\ + \left. i m \vect{\rho_0} \frac{J_m (\nu \rho)}
{\rho \nu} \right) = - \sqrt{\nu} e^{-im\varphi} \left( 
\vect{\varphi_0} \frac{J_{m-1} (\nu \rho) - J_{m+1} (\nu \rho)}{2} + 
\right. \\ + \left. i \vect{\rho_0} \frac{J_{m-1} (\nu \rho) + 
J_{m+1} (\nu \rho)}{2} \right)
\end{aligned} \end{equation*} }
%
\textcolor{lightgray} { \begin{equation*} \begin{aligned}
\vect{j_0} \crossprod{ \nabla_\perp \Psi_m^* }{ \vect{z_0} } = 
- \sqrt{\nu} ( \cos m \varphi - i \sin m \varphi ) 
H(t) \delta(z) ( H(\rho) - H(\rho - R) ) \cdot \\ \cdot \left( 
i \frac{J_{m-1} (\nu \rho) + J_{m+1} (\nu \rho)}{2} \cos \varphi
- \frac{J_{m-1} (\nu \rho) - J_{m+1} (\nu \rho)}{2} \sin \varphi
\right)
\end{aligned} \end{equation*} }
%
\textcolor{red}{ Среда распространения и начальные условия }
%
\textcolor{lightgray} { \begin{equation*} \begin{aligned}
j_m = \frac{\sqrt{\mu_0}}{2\pi} \sqrt{\nu} \delta(z) H(t) \cdot \\
\cdot \Big( \int \limits_{0}^{2\pi} d \varphi \sin \varphi 
( \cos m \varphi - i \sin m \varphi) \int \limits_{0}^{R} 
\frac{J_{m-1} (\nu \rho) - J_{m+1} (\nu \rho)}{2} \rho d \rho - \\
- i \int \limits_{0}^{2\pi} d \varphi \cos \varphi 
( \cos m \varphi - i \sin m \varphi) \int \limits_{0}^{R} 
\frac{J_{m-1} (\nu \rho) + J_{m+1} (\nu \rho)}{2} \rho d \rho \Big)
\end{aligned} \end{equation*} }
%
\textcolor{red} { \begin{equation*} \begin{aligned}
\int \limits_{0}^{2\pi} d \varphi \sin \varphi 
( \cos m \varphi - i \sin m \varphi) = i\pi ( \delta_{m,-1} - \delta_{m,1} )
\end{aligned} \end{equation*} }
%
\textcolor{red} { \begin{equation*} \begin{aligned}
\int \limits_{0}^{2\pi} d \varphi \cos \varphi 
( \cos m \varphi - i \sin m \varphi) = \pi ( \delta_{m,-1} + \delta_{m,1} )
\end{aligned} \end{equation*} }
%
\textcolor{lightgray} { \begin{equation*} \begin{aligned}
j_m = \frac{\sqrt{\mu_0}}{2\pi} \sqrt{\nu} \delta(z) H(t) 
i\pi ( \delta_{m,-1} - \delta_{m,1} ) \int \limits_{0}^{R} 
\frac{J_{m-1} (\nu \rho) - J_{m+1} (\nu \rho)}{2} \rho d \rho - \\
- \frac{\sqrt{\mu_0}}{2\pi} \sqrt{\nu} \delta(z) H(t) 
i\pi ( \delta_{m,-1} + \delta_{m,1} ) \int \limits_{0}^{R} 
\frac{J_{m-1} (\nu \rho) + J_{m+1} (\nu \rho)}{2} \rho d \rho =
\end{aligned} \end{equation*} }
%
\textcolor{lightgray} { \begin{equation*} \begin{aligned}
= i \frac{\sqrt{\mu_0 \nu}}{4} \delta(z) H(t)
\delta_{m,-1} \int \limits_{0}^{R} \left( J_{-2} (\nu \rho) - 
J_0 (\nu \rho) \right) \rho d \rho - \\
- i \frac{\sqrt{\mu_0 \nu}}{4} \delta(z) H(t)
\delta_{m,1} \int \limits_{0}^{R} \left( J_{0} (\nu \rho) - 
J_2 (\nu \rho) \right) \rho d \rho - \\
- i \frac{\sqrt{\mu_0 \nu}}{4} \delta(z) H(t)
\delta_{m,-1} \int \limits_{0}^{R} \left( J_{-2} (\nu \rho) +  
J_0 (\nu \rho) \right) \rho d \rho - \\
- i \frac{\sqrt{\mu_0 \nu}}{4} \delta(z) H(t)
\delta_{m,1} \int \limits_{0}^{R} \left( J_{0} (\nu \rho) +
J_2 (\nu \rho) \right) \rho d \rho =
\end{aligned} \end{equation*} }
%
\textcolor{lightgray} { \begin{equation*} \begin{aligned}
= - i \frac{\sqrt{\mu_0 \nu}}{2} \delta(z) H(t) 
(\delta_{m,1} + \delta_{m,-1}) 
\int \limits_{0}^{R} \left( J_{0} (\nu \rho) + 
J_2 (\nu \rho) \right) \rho d \rho - \\
- i \frac{\sqrt{\mu_0 \nu}}{2} \delta(z) H(t) 
(\delta_{m,1} + \delta_{m,-1}) 
\int \limits_{0}^{R} \left( J_{0} (\nu \rho) -
J_2 (\nu \rho) \right) \rho d \rho = \\
= - i \frac{\sqrt{\mu_0 \nu}}{2} \delta(z) H(t) 
(\delta_{m,1} + \delta_{m,-1}) 
\int \limits_{0}^{R} J_{0} (\nu \rho) \rho d \rho
\end{aligned} \end{equation*} }
%
\textcolor{lightgray} { \begin{equation*} \begin{aligned}
\int \limits_{0}^{R} J_{0} (\nu \rho) \rho d \rho = 
\frac{1}{\nu^2} \int \limits_{0}^{R} J_{0} (\nu \rho) \nu \rho d \nu \rho =
\left. \frac{\rho J_1 (\nu \rho) }{\nu} \right|_{0}^{R} = 
\frac{R J_1 (\nu R)}{\nu}
\end{aligned} \end{equation*} }
%
\begin{equation} 
j_m (z, t; \nu) = - i R A_0 \frac{\sqrt{\mu_0}}{2} \delta(z) H(t) 
\frac{\delta_{m,1} + \delta_{m,-1}}{\sqrt{\nu}} J_1 (\nu R)
\end{equation}

%%%%%%%%%%%%%%%%%%%%%%%%%%%%%%%%%%%%%%%%%%%%%%%%%%%%%%%%%%%%%%%%%%%%%%%%%%%%%%%
\section{Лінійні коефіцієнти}

\begin{equation} \begin{aligned}
V_m^h  = - \mu \partial_{ct} (h_m)
\end{aligned} \end{equation}
%
\begin{equation} \begin{aligned}
I_m^h  = \partial_z (h_m)
\end{aligned} \end{equation}
%
\textcolor{lightgray} { \begin{equation*} \begin{aligned}
- \epsilon \partial_{ct} (V_m^h) - \partial_z I_m^h + \nu^2 h_m = 
\frac{\sqrt{\mu_0}}{2 \pi} \int_0^{2\pi} d \varphi 
\int_0^{\infty} \rho d \rho \crossprod{\vect{z_0}}{\vect{J_\perp}}
\nabla_\perp \Psi_m^* (\nu) 
\end{aligned} \end{equation*} }
%
\textcolor{red} { \begin{equation*} \begin{aligned}
\crossprod{\vect{z_0}}{\vect{J_\perp}} \nabla_\perp \Psi_m^* (\nu) =
\vect{J_\perp} \crossprod{\nabla_\perp \Psi_m^* (\nu)}{\vect{z_0}}
\end{aligned} \end{equation*} }
%
\begin{equation*} \begin{aligned}
\epsilon \partial_{ct} \left( \mu \partial_{ct} h_m \right) -
\mu^{-1} \partial_z \left( \mu  \partial_z h_m \right) + 
\nu^2 h_m = j_m (z,t,\nu) \\
\partial_{ct} =  \frac{1}{c} \partder{}{t}; 
\epsilon = \epsilon (z,t);
\mu = \mu (z,t)
\end{aligned} \end{equation*}
%
\begin{equation} \begin{aligned}
\left( \frac{\sqrt{\epsilon \mu}}{c} \right)^2 
\frac{\partial^2 h_m}{\partial t^2} - 
\frac{\partial^2 h_m}{\partial z^2} + \nu^2 h_m = j_m (z,t,\nu)
\end{aligned} \end{equation}
%
\begin{equation} \begin{aligned}
\mathit{V} = \frac{c}{\sqrt{\epsilon \mu}} 
\end{aligned} \end{equation}
%
\textcolor{red}{Функція Рімана: перевірити принцип причинності}
%
\begin{equation*}
G(t,t',z,z') = \frac{\mathit{V}}{2} H \left( \mathit{V} (t-t') - (z-z') \right)
J_0 \left( \nu \sqrt{\mathit{V}^2 (t-t')^2 - (z-z')^2} \right)
\end{equation*}
%
\begin{equation}
h_m (z, t; \nu) = \iint_S j_m (t',z') G(t,t',z,z') dt' dz'
\end{equation}
%
\textcolor{lightgray} { \begin{equation*} \begin{aligned}
h_m (z, t; \nu) = - i \mathit{V} R \frac{\sqrt{\mu_0}}{4} 
\frac{\delta_{m,1} + \delta_{m,-1}}{\sqrt{\nu}} J_1 (\nu R)
\int \limits_{0}^{\infty} \delta(z) \cdot \\ \cdot
\int \limits_{t - \frac{z}{\mathit{V}}}^{0} 
J_0 \left( \nu \sqrt{\mathit{V}^2 (t-t')^2 - (z-z')^2} \right) dt' dz' = 
i \mathit{V} R \frac{\sqrt{\mu_0}}{4} 
\frac{\delta_{m,1} + \delta_{m,-1}}{\sqrt{\nu}} J_1 (\nu R)
\cdot \\ \cdot \int \limits_{0}^{\infty} \delta(z)
\int \limits_{0}^{t - \frac{z}{\mathit{V}}} 
J_0 \left( \nu \sqrt{\mathit{V}^2 (t-t')^2 - (z-z')^2} \right) dt' dz
\end{aligned} \end{equation*} }
%
\textcolor{lightgray} { \begin{equation*} \begin{aligned}
h_m (z, t; \nu) = i \mathit{V} R \frac{\sqrt{\mu_0}}{4} 
\frac{\delta_{m,1} + \delta_{m,-1}}{\sqrt{\nu}} J_1 (\nu R)
\int \limits_{0}^{t - \frac{z}{\mathit{V}}} 
J_0 \left( \nu \sqrt{\mathit{V}^2 (t-t')^2 - z^2} \right) dt'
\end{aligned} \end{equation*} }
%
\textcolor{red}{Перевірити вірність наступної нотожності
%
\begin{equation}
h_m = \frac{i R A_0}{4} \frac{\delta_{m,1} + \delta_{m,-1}}
{\sqrt{\nu} \sqrt{\epsilon_0 \epsilon \mu}} J_1 (\nu R) 
\int \limits_{0}^{t - \frac{z}{\mathit{V}}} 
J_0 \left( \nu \sqrt{\mathit{V}^2 (t-t')^2 - z^2} \right) dt'
\end{equation} }
%
\textcolor{red} { \begin{equation*} \begin{aligned}
\partder{}{t} \int \limits_{0}^{t - \frac{z}{\mathit{V}}} 
J_0 \left( \nu \sqrt{\mathit{V}^2 (t-t')^2 - z^2} \right) dt' =
J_0 \left( \nu \sqrt{\mathit{V}^2 t^2 - z^2} \right)
\end{aligned} \end{equation*} }
%
\textcolor{lightgray} { \begin{equation*} \begin{aligned}
V_m^h = - \frac{\mu}{c} \partder{h_m}{t} = 
\sqrt{\mu_0} \sqrt{\frac{\mu}{\epsilon}} \frac{iR A_0}{4} 
\frac{\delta_{m,1} + \delta_{m,-1}}{\sqrt{\nu}} J_1 (\nu R)
J_0 \left( \nu \sqrt{\mathit{V}^2 t^2 - z^2} \right)
\end{aligned} \end{equation*} }
%
\begin{equation}
V_m^h (z, t; \nu) = - \frac{iR A_0}{4} \sqrt{\frac{\mu_0 \mu}{\epsilon}} 
\frac{\delta_{m,1} + \delta_{m,-1}}{\sqrt{\nu}} J_1 (\nu R)
J_0 \left( \nu \sqrt{\mathit{V}^2 t^2 - z^2} \right)
\end{equation}
%
\textcolor{lightgray} { \begin{equation*} \begin{aligned}
\int \limits_{0}^{t - \frac{z}{\mathit{V}}} 
J_0 \left( \nu \sqrt{\mathit{V}^2 (t-t')^2 - z^2} 
\right) dt' = \left[ \begin{array}{cc} 
\nu \mathit{V} (t-t') = s & t' = t - \frac{ds}{\nu \mathit{V}} \\
dt' = -\frac{ds}{\nu \mathit{V}} & \\
s(0) = \nu \mathit{V} t & s \left( t - \frac{z}{\mathit{V}} \right) = \nu z
\end{array} \right] = \\ = - \frac{1}{\nu \mathit{V}} 
\int_{\nu \mathit{V} t}^{\nu z} ds 
J_0 (\sqrt{s^2 - \nu^2 z^2}) = \frac{1}{\nu \mathit{V}} 
\int_{\nu z}^{\nu \mathit{V} t} ds
J_0 (\sqrt{s^2 - \nu^2 z^2})
\end{aligned} \end{equation*} }
%
\textcolor{lightgray} { \begin{equation*} \begin{aligned}
\int_{\nu z}^{\nu \mathit{V} t} ds e^{-i0s} J_0 (\sqrt{s^2 - \nu^2 z^2}) = \\ 
= \frac{1}{i} (U_1[W_+,Z] + i U_2[W_+,Z] - U_1[W_-,Z] - i U_2[W_-,Z]) = \\
= \frac{1}{i} (-U_1[W_-,Z] + i U_2[W_+,Z] - U_1[W_-,Z] - i U_2[W_+,Z]) = \\
= \left[ \begin{array}{c} W_\pm = \pm i (\nu \mathit{V} t - \nu z) \\
Z = \sqrt{\nu^2 \mathit{V}^2 t^2 - \nu^2 z^2} \end{array} \right] = 
2i U_1 \left[ -i \nu (\mathit{V}t-z), \nu \sqrt{\mathit{V}^2 t^2-z^2} \right]
\end{aligned} \end{equation*} }
%
\textcolor{lightgray} { \begin{equation*} \begin{aligned}
\int \limits_{0}^{t - \frac{z}{\mathit{V}}} 
J_0 \left( \nu \sqrt{\mathit{V}^2 (t-t')^2 - z^2} 
\right) dt' = \frac{2i}{\nu \mathit{V}} U_1 
\left[ -i \nu (\mathit{V}t-z), \nu \sqrt{\mathit{V}^2t^2-z^2} \right]
\end{aligned} \end{equation*} }
%
\textcolor{lightgray} { \begin{equation*} \begin{aligned}
h_m (z, t; \nu) = \mathit{V} \sqrt{\mu_0} \frac{iR A_0}{4} 
\frac{\delta_{m,1} + \delta_{m,-1}} {\sqrt{\nu}} J_1 (\nu R) 
\frac{2i}{\nu \mathit{V}} U_1 \left[ W_-, Z \right]
\end{aligned} \end{equation*} }
%
\begin{equation}
h_m (z, t; \nu) = - \sqrt{\mu_0} \frac{R A_0}{2} 
\frac{\delta_{m,1} + \delta_{m,-1}}
{\nu^{3/2}} J_1 (\nu R) U_1 \left[ W_-, Z \right]
\end{equation}
%
\textcolor{lightgray} { \begin{equation*} \begin{aligned}
I_{m}^{h} = \partder{h_m}{z} = 
- \sqrt{\mu_0} \frac{R A_0}{2} 
\frac{\delta_{m,1} + \delta_{m,-1}}
{\nu^{3/2}} J_1 (\nu R) \partder{}{z} U_1 [ W_-, Z ]
\end{aligned} \end{equation*} }
%
\textcolor{lightgray} { \begin{equation*} \begin{aligned}
\begin{array}{lcr}
\derivat{W_-}{z} = i \nu & &
\derivat{Z}{z} = \frac{\nu}{2 \sqrt{\mathit{V}^2 t^2 - z^2}} (-2z) = 
- \frac{\nu z}{\sqrt{\mathit{V}^2 t^2 - z^2}} \\
\end{array}
\end{aligned} \end{equation*} }
%
\textcolor{lightgray} { \begin{equation*} \begin{aligned}
\left( \frac{Z}{W} \right)^2 = 
\left( - \frac{ \sqrt{\mathit{V}^2 t^2-z^2}}{i(\mathit{V} t-z)} \right)^2 =
\left( \frac{ i \sqrt{\mathit{V}^2 t^2-z^2}}{\mathit{V}t-z} \right)^2 =
- \frac{\mathit{V}^2 t^2-z^2}{(\mathit{V} t-z)^2} = 
- \frac{\mathit{V}t+z}{\mathit{V}t-z}
\end{aligned} \end{equation*} }
%
\textcolor{lightgray} { \begin{equation*} 
\partder{}{Z} U_n (W,Z) = - \frac{Z}{W} U_{n+1} (W,Z)
\end{equation*} }
%
\textcolor{lightgray} { \begin{equation*}
2 \partder{}{W} U_n (W,Z) = U_{n-1} (W,Z) + 
\left( \frac{Z}{W} \right)^2 U_{n+1} (W,Z)
\end{equation*} }
%
\textcolor{lightgray} { \begin{equation*} \begin{aligned}
\partder{}{z} U_1 \left[ -i \nu (ct-z), \nu \sqrt{c^2t^2-z^2} \right] =
\partder{}{z} U_1[W,Z] = \partder{U_1}{W} \derivat{W}{z} + 
\partder{U_1}{Z} \derivat{Z}{z} = \\
= \frac{i \nu}{2} \left( U_0 - \frac{ct+z}{ct-z} U_2 \right) -
\frac{\nu z}{\sqrt{c^2t^2 - z^2}} 
\left( - \frac{i \sqrt{c^2t^2-z^2}}{ct-z} \right) U_2 = \\
= \frac{i \nu}{2} U_0 - \frac{i \nu}{2} \frac{ct+z}{ct-z} U_2 +
\frac{i \nu z}{ct-z} U_2 = \\ = \frac{i \nu}{2} U_0 - \frac{i \nu}{2} U_2
\left( \frac{ct}{ct-z} + \frac{z}{ct-z} - \frac{2z}{ct-z} \right) = 
\frac{i \nu}{2} (U_0[W_-,Z] - U_2[W_-,Z])
\end{aligned} \end{equation*} }
%
\begin{equation}
I_{m}^{h} = - \sqrt{\mu_0} \frac{iR A_0}{4} 
\frac{\delta_{m,1} + \delta_{m,-1}}{\sqrt{\nu}} 
J_1 (\nu R) \left( U_0 [ W_-, Z ] - U_2 [ W_-, Z ] \right)
\end{equation}

%%%%%%%%%%%%%%%%%%%%%%%%%%%%%%%%%%%%%%%%%%%%%%%%%%%%%%%%%%%%%%%%%%%%%%%%%%%%%%%
\section{Лінійне поле}

\textcolor{lightgray} { \begin{equation*} \begin{aligned}
\vect{E_\perp} = \frac{1}{\sqrt{\epsilon_0}} \left( 
\sum \limits_{m=-\infty}^{\infty} \int \limits_{0}^{\infty} 
d \nu V_m^h \crossprod{ \nabla_\perp \Psi_m }{ \vect{z_0} } +
\sum \limits_{n=-\infty}^{\infty} \int \limits_{0}^{\infty}
d \chi V_n^e \nabla_\perp \Phi_n \right)
\end{aligned} \end{equation*} }
%
\textcolor{lightgray} { \begin{equation*} \begin{aligned}
\crossprod{ \nabla_\perp \Psi_m }{ \vect{z_0} } = 
- e^{im\varphi} \left( \vect{\varphi_0} \sqrt{\nu} 
\frac{J_{m-1} (\nu \rho) - J_{m+1} (\nu \rho)}{2} - 
i m \vect{\rho_0} \frac{J_m (\nu \rho)}{ \rho \sqrt{\nu}} \right)
\end{aligned} \end{equation*} }
%
\textcolor{lightgray} { \begin{equation*} \begin{aligned}
\vect{E_\perp} = \frac{1}{\sqrt{\epsilon_0}} \int_{0}^{\infty} 
V_{-1}^h \crossprod{ \nabla_\perp \Psi_{-1}  }{ \vect{z_0} } +
\frac{1}{\sqrt{\epsilon_0}} \int \limits_{0}^{\infty} 
V_{1}^h \crossprod{ \nabla_\perp \Psi_{1} }{ \vect{z_0} } = \\
= \frac{i R A_0}{4} \sqrt{\frac{\mu_0 \mu}{\epsilon_0 \epsilon}} 
e^{- i \varphi} \int_{0}^{\infty} \frac{J_1 (\nu R)}{\sqrt{\nu}} 
J_0 \left( \nu \sqrt{c^2 t^2 - z^2} \right) \cdot \\
\cdot \left( \vect{\varphi_0} \sqrt{\nu} 
\frac{J_2 (\nu \rho) - J_0 (\nu \rho)}{2} +
i \vect{\rho_0} \frac{J_1 (\nu \rho)}{ \rho \sqrt{\nu}} \right) - \\
+ \frac{i R A_0}{4} \sqrt{\frac{\mu_0 \mu}{\epsilon_0 \epsilon}}
e^{i \varphi} \int \limits_{0}^{\infty} \frac{J_1 (\nu R)}{ \sqrt{\nu}}
J_0 \left( \nu \sqrt{c^2 t^2 - z^2} \right) \cdot \\
\cdot \left( \vect{\varphi_0} \sqrt{\nu}
\frac{J_0 (\nu \rho) - J_2 (\nu \rho)}{2} - 
i \vect{\rho_0} \frac{J_1 (\nu \rho)}{ \rho \sqrt{\nu}} \right)
\end{aligned} \end{equation*} }
%
\textcolor{lightgray} { \begin{equation*} \begin{aligned}
E_\varphi = \frac{i R A_0}{8} \sqrt{\frac{\mu_0 \mu}{\epsilon_0 \epsilon}} 
e^{-i \varphi} \int \limits_{0}^{\infty} J_1 (\nu R)
J_0 \left( \nu \sqrt{c^2 t^2 - z^2} \right)
\left( J_2 (\nu \rho) - J_0 (\nu \rho) \right) + \\
+ \frac{i R A_0}{8} \sqrt{\frac{\mu_0 \mu}{\epsilon_0 \epsilon}} 
e^{i \varphi} \int \limits_{0}^{\infty} J_1 (\nu R)
J_0 \left( \nu \sqrt{c^2 t^2 - z^2} \right)
\left( J_0 (\nu \rho) - J_2 (\nu \rho) \right) = \\
= \frac{i R A_0}{4} \sqrt{\frac{\mu_0 \mu}{\epsilon_0 \epsilon}}
\frac{e^{i \varphi} - e^{-i \varphi} }{2} \int \limits_{0}^{\infty} 
J_1 (\nu R) J_0 \left( \nu \sqrt{c^2 t^2 - z^2} \right) 
\left( J_0 (\nu \rho) - J_2 (\nu \rho) \right) =
\end{aligned} \end{equation*} }
%
\textcolor{lightgray} { \begin{equation*} \begin{aligned}
= \frac{R A_0}{4} \sqrt{\frac{\mu_0 \mu}{\epsilon_0 \epsilon}} 
\frac{e^{i \varphi} - e^{-i \varphi} }{2i} \int \limits_{0}^{\infty} 
J_1 (\nu R) J_0 \left( \nu \sqrt{c^2 t^2 - z^2} \right) 
\left( J_2 (\nu \rho) - J_0 (\nu \rho) \right) = \\
= \frac{R A_0}{4} \sqrt{\frac{\mu_0 \mu}{\epsilon_0 \epsilon}} \sin \varphi 
\int \limits_{0}^{\infty} J_1 (\nu R) 
J_0 \left( \nu \sqrt{c^2 t^2 - z^2} \right) 
\left( J_2 (\nu \rho) - J_0 (\nu \rho) \right)
\end{aligned} \end{equation*} }
%
\textcolor{lightgray} { \begin{equation*} \begin{aligned}
J_2 (\nu \rho) - J_0 (\nu \rho) = \frac{2}{\nu \rho} J_1 (\nu \rho) - 
2 J_0 (\nu \rho)
\end{aligned} \end{equation*} }
%
\textcolor{lightgray} { \begin{equation*} \begin{aligned}
E_\varphi = \frac{R A_0}{2} \sqrt{\frac{\mu_0 \mu}{\epsilon_0 \epsilon}}
\sin \varphi \int \limits_{0}^{\infty} J_1 (\nu R) 
J_0 \left( \nu \sqrt{c^2 t^2 - z^2} \right) 
\left( \frac{J_1 (\nu \rho)}{\nu \rho} - J_0 (\nu \rho) \right)
\end{aligned} \end{equation*} }
%
\textcolor{lightgray} { \begin{equation*} \begin{aligned}
E_\rho = \frac{i R A_0}{4} \sqrt{\frac{\mu_0 \mu}{\epsilon_0 \epsilon}}  
e^{- i \varphi} \int \limits_{0}^{\infty} \frac{J_1 (\nu R)}{\sqrt{\nu}} 
J_0 \left( \nu \sqrt{c^2 t^2 - z^2} \right) 
\left( - i \frac{J_1 (\nu \rho)}{\rho \sqrt{\nu}} \right) + \\
+ \mu \frac{i R A_0}{4} \sqrt{\frac{\mu_0}{\epsilon_0}}  e^{i \varphi}
\int \limits_{0}^{\infty} \frac{J_1 (\nu R)}{\sqrt{\nu}}
J_0 \left( \nu \sqrt{c^2 t^2 - z^2} \right) 
\left( - i \frac{J_1 (\nu \rho)}{ \rho \sqrt{\nu}} \right) = \\
= \mu \frac{R A_0}{2} \sqrt{\frac{\mu_0 \mu}{\epsilon_0 \epsilon}} 
\frac{e^{i \varphi} + e^{-i \varphi}}{2}
\int \limits_{0}^{\infty} \frac{J_1 (\nu R)}{\sqrt{\nu}}
J_0 \left( \nu \sqrt{c^2 t^2 - z^2} \right) 
\frac{J_1 (\nu \rho)}{ \rho \sqrt{\nu}} = \\
= \mu \frac{R A_0}{2} \sqrt{\frac{\mu_0 \mu}{\epsilon_0 \epsilon}} 
\cos \varphi \int \limits_{0}^{\infty} \frac{d \rho}{\nu \rho} 
J_1 (\nu \rho) J_1 (\nu R) J_0 \left( \nu \sqrt{c^2 t^2 - z^2} \right)
\end{aligned} \end{equation*} }
%
\begin{equation} \label{eq:linear_e_cyl}
\vect{E} \left( r, t \right) = \frac{A_0}{2} 
\sqrt{\frac{\mu_0 \mu}{\epsilon_0 \epsilon}}
\Big( \vect{\rho_0} I_1 \cos \varphi - 
\vect{ \varphi_0 } \left( I_2 - I_1 \right) \sin \varphi \Big)
\end{equation}
%
\begin{equation*}
I_1 = R \int \limits_{0}^{\infty} \frac{d \rho}{\nu \rho} J_1 (\nu \rho) 
J_1 (\nu R) J_0 \left( \nu \sqrt{\frac{c^2 t^2}{\epsilon \mu} - z^2} \right)
\end{equation*}
%
\begin{equation*}
I_2 = R \int_{0}^{\infty} d \rho J_1 (\nu R) 
J_0 (\nu \rho) J_0 \left( \nu \sqrt{\frac{c^2 t^2}{\epsilon \mu} - z^2} \right)
\end{equation*}
%
\textcolor{red} { Правильній перехід до декартових векторних кординат? } 
%
\textcolor{lightgray} { \begin{equation*} \begin{aligned}
\mathbf{A} = \left( \begin{array}{cc}
\cos \varphi & \sin \varphi \\
- \sin \varphi & \cos \varphi
\end{array} \right) \begin{array}{ccc}
	& \det A = 1 		&	\\
	& A^{-1} = A^{T}	&
\end{array} 
\mathbf{A^{-1}} = \left( \begin{array}{cc}
\cos \varphi & - \sin \varphi \\
\sin \varphi & \cos \varphi
\end{array} \right) 
\end{aligned} \end{equation*} }
%
\textcolor{lightgray} { \begin{equation*} \begin{aligned}
\vect{E} = 
\mathbf{A^{-1}} \vect{E} \left( \vect{\rho_0}, \vect{\varphi_0} \right) = 
\frac{A_0}{2} \sqrt{\frac{\mu_0 \mu}{\epsilon_0 \epsilon}}
\left( \begin{array}{cc} \cos \varphi & - \sin \varphi \\
\sin \varphi & \cos \varphi \end{array} \right)
\left( \begin{array}{c} I_1 \cos \varphi \\
- (I_2 - I_1) \sin \varphi \end{array} \right) = \\
= \frac{A_0}{2} \sqrt{\frac{\mu_0 \mu}{\epsilon_0 \epsilon}}
\left( \begin{array}{c} I_1 \cos^2 \varphi + (I_2 - I_1) \sin^2 \varphi \\
I_1 \sin \varphi \cos \varphi - (I_2 - I_1) 
\sin \varphi \cos \varphi \end{array} \right)
\end{aligned} \end{equation*} }
%
\begin{equation*} \begin{aligned}
\vect{E} \left( \vect{x_0}, \vect{y_0} \right) = \frac{A_0}{2} 
\sqrt{\frac{\mu_0 \mu}{\epsilon_0 \epsilon}} \left( \begin{array}{c} 
I_1 \cos^2 \varphi + (I_2 - I_1) \sin^2 \varphi \\
I_1 \sin \varphi \cos \varphi - (I_2 - I_1) 
\sin \varphi \cos \varphi \end{array} \right)
\end{aligned} \end{equation*}
%
\begin{figure}[h] \begin{center}
\includegraphics[scale=0.7]{LinearElectric}
\caption{Ефект електромагнітного снаряду ($ z = 2 $ м)} \label{fig:emp_rho}
\end{center} \end{figure}
%
\begin{figure}[h] \begin{center}
\includegraphics[scale=0.7]{LinearElectric2}
\caption{Ефект електромагнітного снаряду ($ \rho = 0.2 $ м)} \label{fig:emp_z}
\end{center} \end{figure}
%
\begin{figure}[h] \begin{center}
\includegraphics[scale=0.7]{LinearPulsShape}
\caption{Кутова залежнысть формы імпульсу ($ \rho = R/2 .. 2R $ м)} 
\label{fig:emp_shape}
\end{center} \end{figure}
%
\textcolor{lightgray} { \begin{equation*} \begin{aligned}
\vect{H_\perp} = \frac{1}{\sqrt{\mu_0}} \left( 
\sum \limits_{m=-\infty}^{\infty} \int \limits_{0}^{\infty} d \nu
I_m^h \nabla_\perp \Psi_m + \sum \limits_{n=-\infty}^{\infty}
\int \limits_{0}^{\infty} d \chi I_n^e 
\crossprod{\vect{z_0}}{\nabla_\perp \Phi_n} \right)
\end{aligned} \end{equation*} }
%
\textcolor{lightgray} { \begin{equation*} \begin{aligned}
\nabla_\perp \Psi_m = e^{i m \varphi} \left( \vect{\rho_0} 
\sqrt{\nu} \frac{ J_{m-1}(\nu \rho) - J_{m+1}(\nu \rho) }{2} +
i m \vect{\varphi_0} \frac{J_m(\nu \rho)}{\sqrt{\nu} \rho} \right)
\end{aligned} \end{equation*} }
%
\textcolor{lightgray} { \begin{equation*} \begin{aligned}
\vect{H_\perp} = \frac{1}{\sqrt{\mu_0}} \left( 
\int \limits_{0}^{\infty} d \nu I_{-1}^h \nabla_\perp \Psi_{-1} +
\int \limits_{0}^{\infty} d \nu I_1^h \nabla_\perp \Psi_1 \right) = \\
= - \frac{A_0}{\sqrt{\mu_0}} \int \limits_{0}^{\infty} d \nu
\sqrt{\mu_0} \frac{iR}{4} J_1 (\nu R)
\frac{ U_0 [ W_-, Z ] - U_2 [ W_-, Z ] }{\sqrt{\nu}}  
e^{- i \varphi} \cdot \\ \cdot \left( \vect{\rho_0} 
\sqrt{\nu} \frac{ J_{2}(\nu \rho) - J_{0}(\nu \rho) }{2} +
i \vect{\varphi_0} \frac{J_1(\nu \rho)}{\sqrt{\nu} \rho} \right) -
\frac{A_0}{\sqrt{\mu_0}} \int \limits_{0}^{\infty} d \nu 
\sqrt{\mu_0} \frac{iR}{4} J_1 (\nu R) \cdot \\
\cdot \frac{ U_0 [ W_-, Z ] - U_2 [ W_-, Z ] }{\sqrt{\nu}} 
e^{i \varphi} \left( \vect{\rho_0} 
\sqrt{\nu} \frac{ J_{0}(\nu \rho) - J_{2}(\nu \rho) }{2} +
i \vect{\varphi_0} \frac{J_1(\nu \rho)}{\sqrt{\nu} \rho} \right)
\end{aligned} \end{equation*} }
%
\textcolor{lightgray} { \begin{equation*} \begin{aligned}
H_\varphi = \frac{R A_0}{4} 
\frac{e^{i \varphi} + e^{- i \varphi}}{\rho} \int \limits_{0}^{\infty} 
\frac{d\nu}{\nu} (U_0[ W_-, Z ] - U_2[ W_-, Z ]) J_1(\nu R) J_1(\nu \rho) = \\
= \frac{R}{2} \cos \varphi \int \limits_{0}^{\infty}
\frac{d\nu}{\nu \rho} (U_0[ W_-, Z ] - U_2[ W_-, Z ]) 
J_1(\nu R) J_1(\nu \rho)
\end{aligned} \end{equation*} }
%
\textcolor{lightgray} { \begin{equation*} \begin{aligned}
H_\rho = \frac{R A_0}{4} \frac{e^{i \varphi} - e^{- i \varphi}}{2i}
\int \limits_{0}^{\infty} d \nu (J_{0}(\nu \rho) - J_{2}(\nu \rho))
J_1(\nu R) (U_0[ W_-, Z ] - U_2[ W_-, Z ]) = \\
= \frac{R}{2} \sin \varphi \int \limits_{0}^{\infty} d \nu 
(J_0(\nu \rho) - \frac{J_1(\nu \rho)}{\nu \rho})
J_1(\nu R) (U_0[ W_-, Z ] - U_2[ W_-, Z ]) = \\
\end{aligned} \end{equation*} }
%
\textcolor{lightgray} { \begin{equation*} \begin{aligned}
\vect{H_\perp} \left( r, t \right) = \frac{A_0}{2} \left( 
\vect{\rho_0} \left( I_4 - I_3 \right) \sin \varphi +
\vect{\varphi_0} I_3 \cos \varphi  \right)
\end{aligned} \end{equation*} }
%
\textcolor{lightgray} { \begin{equation*} \begin{aligned}
H_z (r,t) = \frac{1}{\sqrt{\mu_0}} \sum \limits_{m=-\infty}^{\infty}
\int \limits_0^\infty \nu^2 d \nu h_m \Psi_m
\end{aligned} \end{equation*} }
%
\textcolor{lightgray} { \begin{equation*} \begin{aligned}
\Psi_m (\nu) = \frac{J_m(\nu \rho)}{\sqrt{\nu}} e^{im \varphi} 
\end{aligned} \end{equation*} }
%
\textcolor{lightgray} { \begin{equation*} \begin{aligned}
H_z (r,t) = 
\frac{1}{\sqrt{\mu_0}} \int \limits_0^\infty \nu^2 d \nu h_{1} \Psi_{1} +
\frac{1}{\sqrt{\mu_0}} \int \limits_0^\infty \nu^2 d \nu h_{-1} \Psi_{-1}
\end{aligned} \end{equation*} }
%
\textcolor{lightgray} { \begin{equation*} \begin{aligned}
H_z (r,t) = R A_0 \frac{e^{im \varphi}-e^{-im \varphi}}{2} \int_0^\infty 
d \nu J_1(\nu \rho) J_1 (\nu R)
U_1 \left[ -i \nu (ct-z), \nu \sqrt{c^2t^2-z^2} \right]
\end{aligned} \end{equation*} }
%
\textcolor{lightgray} { \begin{equation*} \begin{aligned}
H_z (r,t) = - R A_0 \sin \varphi \int_0^\infty 
d \nu J_1(\nu \rho) J_1 (\nu R) U_1 [ W_-, Z ] = \\
= - i R A_0 \sin \varphi \int_{0}^{\infty} J_1 \left( \nu R \right)
J_1 \left( \nu \rho \right) U_1 [ W_-, Z ]
\end{aligned} \end{equation*} }
%
\textcolor{lightgray} { \begin{equation*} \begin{aligned}
H_z \left( r, t \right) = - A_0 I_5 \sin \varphi
\end{aligned} \end{equation*} }
%
\begin{equation} \label{eq:linear_h_cyl}
\vect{H} (r, t) = \frac{A_0}{2} \Big( 
\vect{\rho_0} \left( I_4 - I_3 \right) \sin \varphi +
\vect{\varphi_0} I_3 \cos \varphi -
2 \vect{z_0} I_5 \sin \varphi \Big)
\end{equation}
%
\begin{equation*}
I_3 = R \int \limits_{0}^{\infty}
\frac{d\nu}{\nu \rho} J_1(\nu R) J_1(\nu \rho)
\Big( U_0[ W_-, Z ] - U_2[ W_-, Z ] \Big) 
\end{equation*}
%
\begin{equation*}
I_4 = R \int \limits_{0}^{\infty} d\nu J_1(\nu R) J_0(\nu \rho)
\Big( U_0[ W_-, Z ] - U_2[ W_-, Z ] \Big) 
\end{equation*}
%
\begin{equation*}
I_5 = i R \int \limits_0^\infty 
d \nu J_1(\nu \rho) J_1 (\nu R)
U_1 \left[ -i \nu \left( \frac{ct}{\sqrt{\epsilon \mu}} - z \right), 
\nu \sqrt{\frac{c^2t^2}{\epsilon \mu}-z^2} \right]
\end{equation*}
%
\begin{figure}[h] \begin{center}
\includegraphics[scale=0.7]{LinearMagnetic}
\caption{Магнітно-статичне поле ($ z = 2 $ м)} \label{fig:emp_h_rho}
\end{center} \end{figure}

За визначенням дальньої зони - це область простору, де $ E_\varphi = H_\rho $ та
$ H_\varphi = E_\rho $. Дня лінійного поля плаского диску ці умови виконується, 
коли $ U_0(W_-,Z) - U_2(W_-,Z) = J_0(Z) $. Остання тотожність математично вірна
тоді і тільки коли 
%
\begin{equation} \label{eq:FraunhoferDistance}
\left. \lim_{z \to \infty} \left( \frac{ct-z}{ct+z} \right)^m 
\right|_{m > 1} = 0
\end{equation}

Остання тотожність є умовою дальньої зони для антени що породжує 
нестаціонарне поле. Варто зазначити, що при рості $ z $ росте і $ t $, 
відповідно до принципу причинності, тобто умова $ ct - z > 0 $ виконується.

%%%%%%%%%%%%%%%%%%%%%%%%%%%%%%%%%%%%%%%%%%%%%%%%%%%%%%%%%%%%%%%%%%%%%%%%%%%%%%%%
\section{Вторинне джерело поля}

При взаємодії з середовищем поле змінюється. Така повединка називається ефектом 
самодії. Джерелом такого поля стає весь простір його розповсюдження. Назвемо 
таке джерело вторинним. Випадки коли вторинним джерелом можна знехтувати, за 
рахунок його незначного вливу порівняно з первинним джерелом, називається 
лінійною задачею випромінювання. Прикладами такого поля є 
\eqref{eq:linear_h_cyl} та \eqref{eq:linear_e_cyl}. Тепер розглянемо 
середовище, де вплив самодії все ще незначний порівнянно з лінійним полем, але
викликає розбіжності в емпіричних показниках та данних теоретичної моделі. 
Таке середовище називають слабним нелінійним середовищем. Вторинне джерело 
поля в такому середовищі є функцією амплатуди поля. В якості такого джерела 
розглянемо вторинній електричний струм
%
\begin{equation*} 
\vect{J^\prime} = \partder{}{t} \vect{P^\prime} \left( \vect{E} \right) +
\sigma \vect{E}
\end{equation*}

Остання рівність складається з двох додаків. Перший додаток відповідає саме 
нелінійним властивостям поля. Останній додаток описує провідникові властивості 
середи розповлюдження, які теж мжуть бути описані вторинним джерелом. 
%
\begin{equation*} 
\vect{J^\prime} = \vect{\rho_0} \left( \partder{}{t}
P_\rho^\prime \left( \vect{E} \right) + \sigma E_\rho \right) + 
\vect{\varphi_0} \left( \partder{}{t} P_\varphi^\prime \left( \vect{E} \right) + 
\sigma E_\varphi \right) + \vect{z_0} \partder{}{t} P_z^\prime 
\left( \vect{E} \right) 
\end{equation*}
%
\textcolor{lightgray}{ \begin{equation*} \label{eq:linear_e_cyl}
\vect{E} \left( r, t \right) = \frac{A_0}{2} 
\sqrt{\frac{\mu_0 \mu}{\epsilon_0 \epsilon}}
\Big( \vect{\rho_0} I_1 \cos \varphi - 
\vect{ \varphi_0 } \left( I_2 - I_1 \right) \sin \varphi \Big)
\end{equation*} }
%
\textcolor{lightgray}{ \begin{equation*}
I_1 = \begin{cases}
0, 0 < R < | f_{-} \left( r, t \right) | \\
f_1, | f_{-} \left( r, t \right) | < R < f_{+} \left( r, t \right) \\ 
1/2, f_{+} \left( r, t \right) < R
\end{cases}
\end{equation*} }
%
\textcolor{lightgray}{ \begin{equation*} \begin{aligned}
f_1 = \frac{\rho^2 + R^2}{4 \pi \rho^2} \arccos 
\frac{c^2 t^2 - z^2 - \rho^2 - R^2}{2 \rho R}  -
\frac{\sqrt{4 \rho^2 R^2 - (\rho^2 + R^2 - c^2t^2 + z^2)^2}}{4 \pi \rho^2} - \\
- \frac{ |\rho^2 - R^2| }{2 \pi \rho^2} 
\arctan \sqrt{ \frac{(\rho - R)^2}{(\rho + R)^2} \cdot
\frac{\left( \rho + R \right)^2 - \left( c^2t^2 - z^2 \right)} 
{\left( c^2t^2 - z^2 \right) - \left( \rho - R \right)^2} }
\end{aligned} \end{equation*} }
%
\textcolor{lightgray}{ \begin{equation*}
I_2 = \begin{cases}
0, 0 < R < | f_{-} \left( r, t \right) | \\
\frac{1}{\pi} \arccos \frac{c^2t^2 - z^2 + \rho^2 - R^2}
{2 \rho \sqrt{c^2t^2 - z^2}}, | f_{-} \left( r, t \right) | < R < 
f_{+} \left( r, t \right) \\ 1, f_{+} \left( r, t \right) < R
\end{cases}
\end{equation*} }
%
Струм для області $ S_1 $, яка відповідає області статичного поля.
%
\begin{equation*} \begin{aligned}
\vect{J^\prime} \left\{ S_1 \right\} = 
\vect{\rho_0} \partder{}{t} P_\rho^\prime \left( \Theta \right) + 
\vect{\varphi_0} \partder{}{t} P_\varphi^\prime \left( \Theta \right) + 
\vect{z_0} \partder{}{t} P_z^\prime \left( \Theta \right)
\end{aligned} \end{equation*}
%
\textcolor{red}{ Згідно з неоднозначностю векторного потенціалу 
\cite[ст. 77]{imp:LandauII} $ \vect{J^\prime} \left\{ S_1 \right\}  = 0 $. }
%
\begin{equation*} \begin{aligned}
\vect{J^\prime} \left\{ S_2 \right\} = \vect{\rho_0} \left( \partder{}{t}
P_\rho^\prime \left( \vect{E} \right) + \frac{\sigma A_0 I_1}{2} 
\sqrt{\frac{\mu_0 \mu}{\epsilon_0 \epsilon}} \cos \varphi \right) + \\
+ \vect{\varphi_0} \left( \partder{}{t} 
P_\varphi^\prime \left( \vect{E} \right) + \frac{\sigma A_0 (I_2 - I_1)}{2} 
\sqrt{\frac{\mu_0 \mu}{\epsilon_0 \epsilon}} \sin \varphi \right) + 
\vect{z_0} \partder{}{t} P_z^\prime \left( \vect{E} \right)
\end{aligned} \end{equation*}
%
\begin{equation*} \begin{aligned}
\vect{J^\prime} \left\{ S_3 \right\} = \vect{\rho_0} \left( \partder{}{t}
P_\rho^\prime \left( \vect{E} \right) + \frac{\sigma A_0}{4} 
\sqrt{\frac{\mu_0 \mu}{\epsilon_0 \epsilon}} \cos \varphi \right) + \\
+ \vect{\varphi_0} \left( \partder{}{t} 
P_\varphi^\prime \left( \vect{E} \right) + \frac{\sigma A_0}{4} 
\sqrt{\frac{\mu_0 \mu}{\epsilon_0 \epsilon}} \sin \varphi \right) + 
\vect{z_0} \partder{}{t} P_z^\prime \left( \vect{E} \right)
\end{aligned} \end{equation*}

%%%%%%%%%%%%%%%%%%%%%%%%%%%%%%%%%%%%%%%%%%%%%%%%%%%%%%%%%%%%%%%%%%%%%%%%%%%%%%%%
\section{Нелінійність Керра}
%
\textcolor{red}{ Чи правильно, що нелінійна ппроникність поля залежеть від 
амплтуди джерела? }
%
\begin{equation*}
\vect{P^\prime} \left( \vect{E} \right) = 
\chi_3^e \left( A_0 \right) \vect{E}^3 
\left( A_0, R, \epsilon, \mu, vt, \vect{r} \right)
\end{equation*}
%
\begin{equation*}
\vect{P^\prime} \left( \vect{E} \right) = 
\chi_3^e \left( A_0 \right)  
\dotprod{ \crossprod{ \vect{E} }{ \vect{E} } }{ \vect{E} }
\end{equation*}
%
\textcolor{lightgray}{ \begin{equation*} \begin{aligned}
\vect{P^\prime} \left( \vect{E} \right) = 
\frac{ {A_0}^3 \chi_3^e }{ 8 } \left( \frac{\mu_0 \mu}
{\epsilon_0 \epsilon} \right)^{3/2} \left( {I_1}^2 \cos^2 \varphi + 
\left( I_2 - I_1 \right)^2 \sin^2 \varphi \right) \cdot \\ 
\cdot \Big( \vect{\rho_0} I_1 \cos \varphi - 
\vect{ \varphi_0 } \left( I_2 - I_1 \right) \sin \varphi \Big)
\end{aligned} \end{equation*} }

%%%%%%%%%%%%%%%%%%%%%%%%%%%%%%%%%%%%%%%%%%%%%%%%%%%%%%%%%%%%%%%%%%%%%%%%%%%%%%%%
\section{Нелінійні коефіцієнти}
%
\textcolor{lightgray}{ \begin{equation*} \begin{aligned}
f_1 = \frac{\rho^2 + R^2}{4 \pi \rho^2} \arccos 
\frac{c^2 t^2 - z^2 - \rho^2 - R^2}{2 \rho R}  -
\frac{\sqrt{4 \rho^2 R^2 - (\rho^2 + R^2 - c^2t^2 + z^2)^2}}{4 \pi \rho^2} - \\
- \frac{ |\rho^2 - R^2| }{2 \pi \rho^2} 
\arctan \sqrt{ \frac{(\rho - R)^2}{(\rho + R)^2} \cdot
\frac{\left( \rho + R \right)^2 - \left( c^2t^2 - z^2 \right)} 
{\left( c^2t^2 - z^2 \right) - \left( \rho - R \right)^2} }
\end{aligned} \end{equation*} }
%
\textcolor{lightgray}{ \begin{equation*} \begin{aligned}
\partder{I_1 \left\{ S_2 \right\}}{t} = \frac{\rho^2 + R^2}{4 \pi \rho^2}
\partder{}{t} \arccos \frac{c^2 t^2 - z^2 - \rho^2 - R^2}{2 \rho R} - \\
- \partder{}{t} \frac{\sqrt{4 \rho^2 R^2 - (\rho^2 + R^2 - c^2t^2 + z^2)^2}}
{4 \pi \rho^2} - \\ - \frac{ |\rho^2 - R^2| }{2 \pi \rho^2} \partder{}{t} 
\arctan \sqrt{ \frac{(\rho - R)^2}{(\rho + R)^2} \cdot
\frac{\left( \rho + R \right)^2 - \left( c^2t^2 - z^2 \right)} 
{\left( c^2t^2 - z^2 \right) - \left( \rho - R \right)^2} }
\end{aligned} \end{equation*} }
%
\textcolor{lightgray}{ \begin{equation*} \begin{aligned}
\partder{}{t} \arccos \frac{c^2 t^2 - z^2 - \rho^2 - R^2}{2 \rho R} = 
- \frac{2 c^2 t}
{ \sqrt{4 \rho^2 R^2 - \left(c^2 t^2 - z^2 - \rho^2 - R^2 \right)^2} }
\end{aligned} \end{equation*} }
%
\textcolor{lightgray}{ \begin{equation*} \begin{aligned}
\partder{}{t} \frac{\sqrt{4 \rho^2 R^2 - (\rho^2 + R^2 - c^2t^2 + z^2)^2}}
{4 \pi \rho^2} = \frac{1}{8 \pi \rho^2} 
\frac{ 4 c^2 t (\rho^2 + R^2 - c^2 t^2 + z^2) }
{ \sqrt{4 \rho^2 R^2 - (\rho^2 + R^2 - c^2t^2 + z^2)^2} } = \\
= \frac{c^2 t}{2 \pi \rho^2} \frac{\rho^2 + R^2 - c^2 t^2 + z^2}
{ \sqrt{4 \rho^2 R^2 - (\rho^2 + R^2 - c^2t^2 + z^2)^2} }
\end{aligned} \end{equation*} }
%
\textcolor{red}{ \begin{equation*} \begin{aligned}
\partder{}{t} \arctan \sqrt{ \frac{(\rho - R)^2}{(\rho + R)^2}
\frac{\left( \rho + R \right)^2 - \left( c^2t^2 - z^2 \right)} 
{\left( c^2t^2 - z^2 \right) - \left( \rho - R \right)^2} } = 
\frac{ \partder{}{t} \sqrt{ \frac{(\rho - R)^2}{(\rho + R)^2} \frac
{\left( \rho + R \right)^2 - \left( c^2t^2 - z^2 \right)} 
{\left( c^2t^2 - z^2 \right) - \left( \rho - R \right)^2}} }
{1 + \frac{(\rho - R)^2}{(\rho + R)^2}
\frac{\left( \rho + R \right)^2 - \left( c^2t^2 - z^2 \right)} 
{\left( c^2t^2 - z^2 \right) - \left( \rho - R \right)^2} } = \\
% = \frac{1}{2} \frac{ \sqrt{ \frac{(\rho - R)^2}{(\rho + R)^2} \frac 
% {\left( c^2t^2 - z^2 \right) - \left( \rho - R \right)^2}
% {\left( \rho + R \right)^2 - \left( c^2t^2 - z^2 \right)} } }
% {1 + \frac{(\rho - R)^2}{(\rho + R)^2}
% \frac{\left( \rho + R \right)^2 - \left( c^2t^2 - z^2 \right)} 
% {\left( c^2t^2 - z^2 \right) - \left( \rho - R \right)^2} } 
% \frac{(\rho - R)^2}{(\rho + R)^2} \partder{}{t} \frac
% {\left( \rho + R \right)^2 - \left( c^2t^2 - z^2 \right)} 
% {\left( c^2t^2 - z^2 \right) - \left( \rho - R \right)^2}} = \\
= \frac{1}{2} \frac{(\rho - R)^2}{(\rho + R)^2}
\frac{ - 2 c^2 t \left( (c^2 t^2 - z^2) - (\rho - R)^2 \right) -
2 c^2 t \left( (\rho + R)^2 - (c^2 t^2 - z^2) \right) }
{\left( (c^2 t^2 - z^2) - (\rho - R)^2 \right)^2} \frac{?}{?} = \\
= c^2 t \frac{(\rho - R)^2}{(\rho + R)^2} \frac
{ \left( \rho + R \right)^2 - \left( \rho - R \right)^2 }
{ \left( (c^2t^2 - z^2) - (\rho - R)^2 \right)^2 }
\frac{ \sqrt{ \frac{(\rho - R)^2}{(\rho + R)^2} \frac 
{\left( c^2t^2 - z^2 \right) - \left( \rho - R \right)^2}
{\left( \rho + R \right)^2 - \left( c^2t^2 - z^2 \right)} } }
{1 + \frac{(\rho - R)^2}{(\rho + R)^2}
\frac{\left( \rho + R \right)^2 - \left( c^2t^2 - z^2 \right)} 
{\left( c^2t^2 - z^2 \right) - \left( \rho - R \right)^2} } = \\
= \frac{(\rho - R)^2}{(\rho + R)^2} \frac
{ 4 c^2 t \rho R }{ \left( (c^2t^2 - z^2) - (\rho - R)^2 \right)^2 }
\frac{ \sqrt{ \frac{(\rho - R)^2}{(\rho + R)^2} \frac 
{\left( c^2t^2 - z^2 \right) - \left( \rho - R \right)^2}
{\left( \rho + R \right)^2 - \left( c^2t^2 - z^2 \right)} } }
{1 + \frac{(\rho - R)^2}{(\rho + R)^2}
\frac{\left( \rho + R \right)^2 - \left( c^2t^2 - z^2 \right)} 
{\left( c^2t^2 - z^2 \right) - \left( \rho - R \right)^2} }
\end{aligned} \end{equation*} }
%
\textcolor{lightgray}{ \begin{equation*} \begin{aligned}
\partder{}{t} \arctan \sqrt{ \frac{(\rho - R)^2}{(\rho + R)^2}
\frac{\left( \rho + R \right)^2 - \left( c^2t^2 - z^2 \right)} 
{\left( c^2t^2 - z^2 \right) - \left( \rho - R \right)^2} } = 
\partder{}{t} \arctan \sqrt{ \frac
{1 - \frac{c^2t^2 - z^2}{\left( \rho + R \right)^2} } 
{ \frac{c^2t^2 - z^2}{ \left( \rho - R \right)^2 } - 1} } = \\
= \frac{1}{1 + \frac{1 - \frac{c^2t^2 - z^2}{\left( \rho + R \right)^2} } 
{ \frac{c^2t^2 - z^2}{ \left( \rho - R \right)^2 } - 1} } \frac{1}{2}
\sqrt{ \frac{ \frac{c^2t^2 - z^2}{ \left( \rho - R \right)^2 } - 1 }
{1 - \frac{c^2t^2 - z^2}{\left( \rho + R \right)^2} } } 
\frac{ - \frac{2 c^2 t}{\left( \rho + R \right)^2} 
\left( \frac{c^2t^2 - z^2}{ \left( \rho - R \right)^2} - 1 \right) - 
\frac{ 2 c^2 t }{ \left( \rho - R \right)^2 } 
\left( 1 - \frac{c^2t^2 - z^2}{\left( \rho + R \right)^2} \right) }
{\left( \frac{c^2t^2 - z^2}{ \left( \rho - R \right)^2 } - 1 \right)^2} = \\
= - c^2 t \frac{ \frac{c^2t^2-z^2}{(\rho-R)^2} - 1 }
{ \frac{c^2t^2-z^2}{(\rho-R)^2} - 1 + 1 - 
\frac{c^2t^2 - z^2}{\left( \rho + R \right)^2} }
\sqrt{ \frac{ \frac{c^2t^2 - z^2}{ \left( \rho - R \right)^2 } - 1}
{1 - \frac{c^2t^2 - z^2}{\left( \rho + R \right)^2} } } \frac
{ \frac{c^2t^2 - z^2}{ \left( \rho^2 - R^2 \right)^2 } - \frac{1}{(\rho+R)^2} + 
\frac{1}{(\rho-R)^2} - \frac{ c^2t^2 - z^2 }{ \left( \rho^2 - R^2 \right)^2 } }
{ \left( \frac{c^2t^2 - z^2}{ \left( \rho - R \right)^2} - 1 \right)^2 } = \\
= - \frac{4 \rho R c^2 t}{ \left( \rho^2 - R^2 \right)^2 } 
\frac{ 1 }{ \frac{c^2t^2-z^2}{(\rho-R)^2} - 
\frac{c^2t^2 - z^2}{\left( \rho + R \right)^2} }
\sqrt{ \frac{ \frac{c^2t^2 - z^2}{ \left( \rho - R \right)^2 } - 1}
{1 - \frac{c^2t^2 - z^2}{\left( \rho + R \right)^2} } } \frac
{ 1 }{ \frac{c^2t^2 - z^2}{ \left( \rho - R \right)^2} - 1 } = \\
= - \frac{c^2 t}{ c^2 t^2 - z^2 } \frac{1} { 
\sqrt{ 1 - \frac{c^2t^2 - z^2}{(\rho + R)^2 } } 
\sqrt{ \frac{c^2t^2 - z^2}{ (\rho - R)^2 } - 1} }
\end{aligned} \end{equation*} }
%
\textcolor{lightgray}{ \begin{equation*} \begin{aligned}
\partder{ I_1 \{ S_2 \} }{t} = - \frac{c^2 t}{2 \pi \rho^2}
\frac{\rho^2 + R^2}
{ \sqrt{4 \rho^2 R^2 - \left(c^2 t^2 - z^2 - \rho^2 - R^2 \right)^2} } - \\
- \frac{c^2 t}{2 \pi \rho^2} \frac{\rho^2 + R^2 - c^2 t^2 + z^2}
{ \sqrt{4 \rho^2 R^2 - (\rho^2 + R^2 - c^2t^2 + z^2)^2} } + \\ 
+ \frac{ c^2 t }{2 \pi \rho^2} \frac{|\rho^2 - R^2|}{ c^2 t^2 - z^2 } \frac{1} 
{ \sqrt{ 1 - \frac{c^2t^2 - z^2}{(\rho + R)^2 } } 
\sqrt{ \frac{c^2t^2 - z^2}{ (\rho - R)^2 } - 1} }
\end{aligned} \end{equation*} }
%
\begin{equation*} \begin{aligned}
\partder{ I_1 \{ S_2 \} }{t} = 
\frac{ c^2 t }{2 \pi \rho^2} \frac{|\rho^2 - R^2|}{ c^2 t^2 - z^2 } \frac{1} 
{ \sqrt{ 1 - \frac{c^2t^2 - z^2}{(\rho + R)^2 } } 
\sqrt{ \frac{c^2t^2 - z^2}{ (\rho - R)^2 } - 1} } - \\
- \frac{c^2 t}{2 \pi \rho^2} \frac{\rho^2 + R^2}
{ \sqrt{4 \rho^2 R^2 - \left(c^2 t^2 - z^2 - \rho^2 - R^2 \right)^2} } - \\
- \frac{c^2 t}{2 \pi \rho^2} \frac{\rho^2 + R^2 - c^2 t^2 + z^2}
{ \sqrt{4 \rho^2 R^2 - (\rho^2 + R^2 - c^2t^2 + z^2)^2} }
\end{aligned} \end{equation*}
%
\begin{equation*} \begin{aligned}
\partder{ I_1 \{ S_{1,3} \} }{t} = 0
\end{aligned} \end{equation*}
%
\textcolor{lightgray}{ \begin{equation*} \begin{aligned}
\partder{ I_2 \{ S_2 \} }{t} = \frac{1}{\pi} \partder{}{t} \arccos 
\frac{c^2t^2 - z^2 + \rho^2 - R^2}{2 \rho \sqrt{c^2t^2 - z^2}} = \\
= - \frac{1}{\pi} \frac{1} { \sqrt{ 1 - \frac{ (c^2t^2 - z^2 + \rho^2 - R^2)^2 }
{4 \rho^2 (c^2t^2 - z^2)^2} } } \partder{}{t} \frac{c^2t^2 - z^2 + \rho^2 - R^2}
{2 \rho \sqrt{c^2t^2 - z^2}} = \\
= - \frac{1}{\pi} \frac{1} { \sqrt{ 1 - \frac{ (c^2t^2 - z^2 + \rho^2 - R^2)^2 }
{4 \rho^2 (c^2t^2 - z^2)^2} } } 
\frac{2c^2t 2 \rho \sqrt{c^2t^2 - z^2} - 
\frac{2c^2t \rho}{\sqrt{c^2t^2 - z^2}} (c^2t^2 - z^2 + \rho^2 - R^2) }
{4 \rho^2 (c^2t^2 - z^2)} = \\
= - \frac{2 c^2 t \rho}{\pi} \frac{1} 
{ \sqrt{ 1 - \frac{ (c^2t^2 - z^2 + \rho^2 - R^2)^2 }
{4 \rho^2 (c^2t^2 - z^2)} } } 
\frac{2 \sqrt{c^2t^2 - z^2} - 
\frac{c^2t^2 - z^2 + \rho^2 - R^2}{\sqrt{c^2t^2 - z^2}} }
{4 \rho^2 (c^2t^2 - z^2)} = \\
= - \frac{c^2 t}{2 \pi \rho (c^2t^2 - z^2)} 
\frac{ 2 \sqrt{c^2t^2 - z^2} - 
\frac{c^2t^2 - z^2 + \rho^2 - R^2}{\sqrt{c^2t^2 - z^2}} } 
{ \sqrt{ 1 - \frac{ (c^2t^2 - z^2 + \rho^2 - R^2)^2 }
{4 \rho^2 (c^2t^2 - z^2)^2} } } = \\
= - \frac{c^2 t}{\pi \sqrt{c^2t^2 - z^2} } 
\frac{ 2 (c^2t^2 - z^2) - (c^2t^2 - z^2 + \rho^2 - R^2) } 
{ \sqrt{ 4 \rho^2 (c^2t^2 - z^2)^2 - (c^2t^2 - z^2 + \rho^2 - R^2)^2 } } = \\
= - \frac{c^2 t}{\pi \sqrt{c^2t^2 - z^2} } \frac{ c^2t^2 - z^2 -  \rho^2 + R^2 } 
{ \sqrt{ 4 \rho^2 (c^2t^2 - z^2)^2 - (c^2t^2 - z^2 + \rho^2 - R^2)^2 } }
\end{aligned} \end{equation*} }
%
\begin{equation*} \begin{aligned}
\partder{ I_2 \{ S_2 \} }{t} = 
- \frac{c^2 t}{\pi \sqrt{c^2t^2 - z^2} } \frac{ c^2t^2 - z^2 -  \rho^2 + R^2 } 
{ \sqrt{ 4 \rho^2 (c^2t^2 - z^2)^2 - (c^2t^2 - z^2 + \rho^2 - R^2)^2 } }
\end{aligned} \end{equation*}
%
\begin{equation*} \begin{aligned}
\partder{ I_2 \{ S_{1,3} \} }{t} = 0
\end{aligned} \end{equation*}
%
\textcolor{lightgray}{ \begin{equation*} \begin{aligned}
\partder{I_1 \left\{ S_2 \right\}}{t} = \frac{\rho^2 + R^2}{4 \pi \rho^2}
\partder{}{t} \arccos \frac{c^2 t^2 - z^2 - \rho^2 - R^2}{2 \rho R} - \\
- \partder{}{t} \frac{\sqrt{4 \rho^2 R^2 - (\rho^2 + R^2 - c^2t^2 + z^2)^2}}
{4 \pi \rho^2} - \\ - \frac{ |\rho^2 - R^2| }{2 \pi \rho^2} \partder{}{t} 
\arctan \sqrt{ \frac{(\rho - R)^2}{(\rho + R)^2} \cdot
\frac{\left( \rho + R \right)^2 - \left( c^2t^2 - z^2 \right)} 
{\left( c^2t^2 - z^2 \right) - \left( \rho - R \right)^2} }
\end{aligned} \end{equation*} }
%
\textcolor{lightgray}{ \begin{equation*} \begin{aligned}
\vect{P^\prime} \left( \vect{E} \right) = 
\frac{ {A_0}^3 \chi_3^e }{ 8 } \left( \frac{\mu_0 \mu}
{\epsilon_0 \epsilon} \right)^{3/2} \left( {I_1}^2 \cos^2 \varphi + 
\left( I_2 - I_1 \right)^2 \sin^2 \varphi \right) \cdot \\ 
\cdot \Big( \vect{\rho_0} I_1 \cos \varphi - 
\vect{ \varphi_0 } \left( I_2 - I_1 \right) \sin \varphi \Big)
\end{aligned} \end{equation*} }
%
\begin{equation*} \begin{aligned}
P_z^\prime \left( \vect{E} \right) = 0
\end{aligned} \end{equation*}
%
\textcolor{lightgray} { \begin{equation*} \begin{aligned}
\partder{P_\rho^\prime}{t}   = \frac{ {A_0}^3 \chi_3^e }{ 8 } 
\left( \frac{\mu_0 \mu} {\epsilon_0 \epsilon} \right)^{3/2} \left(
\left( {I_1}^2 \cos^2 \varphi + ( I_2 - I_1 )^2 \sin^2 \varphi \right)
\partder{I_1}{t} \cos \varphi + \right. \\
\left. + I_1 \cos \varphi \left( 2 I_1 \partder{I_1}{t} \cos^2 \varphi + 
2 ( I_2 - I_1 ) \left( \partder{I_2}{t} - \partder{I_1}{t} \right) 
\sin^2 \varphi \right) \right) = \\ = \frac{ {A_0}^3 \chi_3^e }{ 8 } 
\left( \frac{\mu_0 \mu} {\epsilon_0 \epsilon} \right)^{3/2} \left(
\partder{I_1}{t} {I_1}^2 \cos^3 \varphi + \partder{I_1}{t} ( I_2 - I_1 )^2 
\cos \varphi \sin^2 \varphi + \right. \\
\left. + 2 {I_1}^2 \partder{I_1}{t} \cos^3 \varphi + 
2 I_1 ( I_2 - I_1 ) \left( \partder{I_2}{t} - \partder{I_1}{t} \right) 
\cos \varphi \sin^2 \varphi \right) = \\ = \frac{ {A_0}^3 \chi_3^e }{ 8 } 
\left( \frac{\mu_0 \mu} {\epsilon_0 \epsilon} \right)^{3/2} \left(
\partder{I_1}{t} {I_1}^2 \cos^3 \varphi + \partder{I_1}{t} ( I_2 - I_1 )^2 
\cos \varphi \sin^2 \varphi + \right. \\
\left. + 2 {I_1}^2 \partder{I_1}{t} \cos^3 \varphi + 
2 I_1 ( I_2 - I_1 ) \left( \partder{I_2}{t} - \partder{I_1}{t} \right) 
\cos \varphi \sin^2 \varphi \right) 
\end{aligned} \end{equation*} }
%
\begin{equation*} \begin{aligned}
\partder{P_\rho^\prime}{t} = \frac{ {A_0}^3 \chi_3^e }{ 8 } 
\left( \frac{\mu_0 \mu} {\epsilon_0 \epsilon} \right)^{3/2} \left(
3 {I_1}^2 \partder{I_1}{t} \cos^3 \varphi + \right. \\
+ \left. ( I_2 - I_1 ) \cos \varphi \sin^2 \varphi \left( 
\partder{I_1}{t} ( I_2 - I_1 ) + 2 I_1 \left( \partder{I_2}{t} - 
\partder{I_1}{t} \right) \right) \right)
\end{aligned} \end{equation*}
%
\textcolor{lightgray} { \begin{equation*} \begin{aligned}
\partder{P_\varphi^\prime}{t}   = - \frac{ {A_0}^3 \chi_3^e }{ 8 } 
\left( \frac{\mu_0 \mu} {\epsilon_0 \epsilon} \right)^{3/2} \left(
\left( {I_1}^2 \cos^2 \varphi + ( I_2 - I_1 )^2 \sin^2 \varphi \right)
\left( \partder{I_2}{t} - \partder{I_1}{t} \right) \sin \varphi + \right. \\
\left. + (I_2 - I_1) \sin \varphi \left( 2 I_1 \partder{I_1}{t} \cos^2 \varphi + 
2 ( I_2 - I_1 ) \left( \partder{I_2}{t} - \partder{I_1}{t} \right) 
\sin^2 \varphi \right) \right) = \\ = - \frac{ {A_0}^3 \chi_3^e }{ 8 } 
\left( \frac{\mu_0 \mu} {\epsilon_0 \epsilon} \right)^{3/2} \left(
{I_1}^2 \left( \partder{I_2}{t} - \partder{I_1}{t} \right) 
\sin \varphi \cos^2 \varphi + \right. \\ \left. 
+ ( I_2 - I_1 )^2 \left( \partder{I_2}{t} - \partder{I_1}{t} \right) 
\sin^3 \varphi + 2 I_1 \partder{I_1}{t} (I_2 - I_1) 
\sin \varphi \cos^2 \varphi + \right. \\ 
+ \left. 2 ( I_2 - I_1 )^2 \left( \partder{I_2}{t} - \partder{I_1}{t} \right) 
\sin^3 \varphi \right)
\end{aligned} \end{equation*} }
%
\begin{equation*} \begin{aligned}
\partder{P_\varphi^\prime}{t} = - \frac{ {A_0}^3 \chi_3^e }{ 8 } 
\left( \frac{\mu_0 \mu} {\epsilon_0 \epsilon} \right)^{3/2} \left(
3 ( I_2 - I_1 )^2 \left( \partder{I_2}{t} - \partder{I_1}{t} \right)
\sin^3 \varphi \right. + \\
+ \left. I_1 \sin \varphi \cos^2 \varphi \left( 
I_1 \left( \partder{I_2}{t} - \partder{I_1}{t} \right) + 
2 \partder{I_1}{t} (I_2 - I_1) \right) \right)
\end{aligned} \end{equation*}
%
\textcolor{lightgray} { \begin{equation*}
j_m \left( r, t; \nu \right) = \frac{\sqrt{\mu_0}}{2\pi} 
\int \limits_{0}^{2\pi} d \varphi \int \limits_{0}^{\infty} \rho d \rho 
\vect{j_0} \crossprod{ \nabla_\perp \Psi_m^* }{ \vect{z_0} }
\end{equation*} }
%
\textcolor{lightgray} { \begin{equation*} \begin{aligned}
\vect{J^\prime} \left\{ S_2 \right\} = \vect{\rho_0} \left( \partder{}{t}
P_\rho^\prime \left( \vect{E} \right) + \frac{\sigma A_0 I_1}{2} 
\sqrt{\frac{\mu_0 \mu}{\epsilon_0 \epsilon}} \cos \varphi \right) + \\
+ \vect{\varphi_0} \left( \partder{}{t} 
P_\varphi^\prime \left( \vect{E} \right) + \frac{\sigma A_0 (I_2 - I_1)}{2} 
\sqrt{\frac{\mu_0 \mu}{\epsilon_0 \epsilon}} \sin \varphi \right)
\end{aligned} \end{equation*} }
%
\textcolor{lightgray} { \begin{equation*} \begin{aligned}
\crossprod{ \nabla_\perp \Psi_m^* }{ \vect{z_0} } =
- \sqrt{\nu} e^{-im\varphi} \left( \vect{\varphi_0} \frac{J_{m-1} (\nu \rho) - 
J_{m+1} (\nu \rho)}{2} + i m \vect{\rho_0} \frac{J_m (\nu \rho)}
{\rho \nu} \right) = \\ 
= - \sqrt{\nu} e^{-im\varphi} \left( \vect{\varphi_0} \frac{J_{m-1} (\nu \rho) - 
J_{m+1} (\nu \rho)}{2} + i \vect{\rho_0} \frac{J_{m-1} (\nu \rho) + 
J_{m+1} (\nu \rho)}{2} \right)
\end{aligned} \end{equation*} }
%
\textcolor{lightgray} { \begin{equation*} \begin{aligned}
\vect{j_0} \crossprod{ \nabla_\perp \Psi_m^* }{ \vect{z_0} } = 
- i e^{-im\varphi} m \frac{J_m (\nu \rho)}{\rho \sqrt{\nu}}
\left( \partder{}{t} P_\rho^\prime \left( \vect{E} \right) + 
\frac{\sigma A_0 I_1}{2} \sqrt{\frac{\mu_0 \mu}{\epsilon_0 \epsilon}} 
\cos \varphi \right) - \\
- \sqrt{\nu} e^{-im\varphi} \frac{J_{m-1} (\nu \rho) - J_{m+1} (\nu \rho)}{2}
\left( \partder{}{t} P_\varphi^\prime \left( \vect{E} \right) + 
\frac{\sigma A_0 (I_2 - I_1)}{2} \sqrt{\frac{\mu_0 \mu}{\epsilon_0 \epsilon}} 
\sin \varphi \right)
\end{aligned} \end{equation*} }
%
\begin{equation*} \begin{aligned}
j_m = - \frac{\sqrt{\mu_0}}{2\pi} 
\int_{0}^{2\pi} d \varphi \int \limits_{0}^{\infty} \rho d \rho 
\left( i e^{-im\varphi} m \frac{J_m (\nu \rho)}{\rho \sqrt{\nu}}
\left( \partder{}{t} P_\rho^\prime \left( \vect{E} \right) + 
\frac{\sigma A_0 I_1}{2} \sqrt{\frac{\mu_0 \mu}{\epsilon_0 \epsilon}} 
\cos \varphi \right) \right. + \\
+ \left. \sqrt{\nu} e^{-im\varphi} 
\frac{J_{m-1} (\nu \rho) - J_{m+1} (\nu \rho)}{2}
\left( \partder{}{t} P_\varphi^\prime \left( \vect{E} \right) + 
\frac{\sigma A_0 (I_2 - I_1)}{2} \sqrt{\frac{\mu_0 \mu}{\epsilon_0 \epsilon}} 
\sin \varphi \right) \right)
\end{aligned} \end{equation*}
%
\textcolor{lightgray} { \begin{equation*} \begin{aligned}
\int_{0}^{2 \pi} d \varphi e^{-im \varphi} \partder{P_\rho^\prime}{t} = 
\frac{ {A_0}^3 \chi_3^e }{ 8 } \int_{0}^{2\pi} d \varphi
e^{-im\varphi} \left( \frac{\mu_0 \mu} {\epsilon_0 \epsilon} \right)^{3/2} 
\left( 3 {I_1}^2 \partder{I_1}{t} \cos^3 \varphi + \right. \\
+ \left. ( I_2 - I_1 ) \cos \varphi \sin^2 \varphi \left( 
\partder{I_1}{t} ( I_2 - I_1 ) + 2 I_1 \left( \partder{I_2}{t} - 
\partder{I_1}{t} \right) \right) \right) = \\
= \frac{ {A_0}^3 \chi_3^e }{ 8 } 
\left( \frac{\mu_0 \mu} {\epsilon_0 \epsilon} \right)^{3/2}
\left( \frac{3 \pi}{4} {I_1}^2 \partder{I_1}{t} \left( \delta_{m,-3} - 
\delta_{m,3} + 3 \delta_{m,-1} + 3 \delta_{m,1} \right) + \right. \\
+ \frac{\pi }{4} \left. ( I_2 - I_1 ) \left( \delta_{m,1} - 
\delta_{m,-3} + \delta_{m,-1} - \delta_{m,3} \right) \left( 
\partder{I_1}{t} ( I_2 - I_1 ) + 2 I_1 \left( \partder{I_2}{t} - 
\partder{I_1}{t} \right) \right) \right)
\end{aligned} \end{equation*} }
%
\textcolor{lightgray} { \begin{equation*} \begin{aligned}
\int_{0}^{2\pi} e^{-i m \varphi} \cos^3 \varphi d \varphi = 
\frac{\pi}{4} \delta_{m,-3} - \frac{\pi}{4} \delta_{m,3} + 
\frac{3 \pi}{4} \delta_{m,-1} + \frac{3 \pi}{4} \delta_{m,1}
\end{aligned} \end{equation*} }
%
\textcolor{lightgray} { \begin{equation*} \begin{aligned}
\int_{0}^{2\pi} e^{-i m \varphi} \cos \varphi \sin^2 \varphi d \varphi = 
\frac{\pi \delta_{m,1} }{4} - \frac{\pi \delta_{m,-3} }{4} + 
\frac{\pi \delta_{m,-1} }{4} - \frac{\pi \delta_{m,3} }{4}
\end{aligned} \end{equation*} }
%
\begin{equation*} \begin{aligned}
\int_{0}^{2\pi} d \varphi e^{-im \varphi} \partder{P_\rho^\prime}{t} = 
\frac{ {A_0}^3 \chi_3^e }{ 8 } 
\left( \frac{\mu_0 \mu} {\epsilon_0 \epsilon} \right)^{3/2}
\left( \frac{3 \pi}{4} {I_1}^2 \partder{I_1}{t} \left( \delta_{m,-3} - 
\delta_{m,3} + 3 \delta_{m,-1} + 3 \delta_{m,1} \right) + \right. \\
+ \frac{\pi }{4} \left. ( I_2 - I_1 ) \left( \delta_{m,1} - 
\delta_{m,-3} + \delta_{m,-1} - \delta_{m,3} \right) \left( 
\partder{I_1}{t} ( I_2 - I_1 ) + 2 I_1 \left( \partder{I_2}{t} - 
\partder{I_1}{t} \right) \right) \right)
\end{aligned} \end{equation*}
%
\textcolor{lightgray} { \begin{equation*} \begin{aligned}
\int_{0}^{2\pi} e^{-i m \varphi} \sin^3 \varphi d \varphi = 
\frac{3 \pi i}{4} \delta_{m,-1} - \frac{3 \pi i}{4} \delta_{m,1} - 
\frac{\pi i}{4} \delta_{m,-3} - \frac{\pi i}{4} \delta_{m,3}
\end{aligned} \end{equation*} }
%
\textcolor{lightgray} { \begin{equation*} \begin{aligned}
\int_{0}^{2\pi} e^{-i m \varphi} \sin \varphi \cos^2 \varphi d \varphi = 
\frac{\pi i }{4} \delta_{m,-3} - \frac{\pi i }{4} \delta_{m,1} + 
\frac{\pi i }{4} \delta_{m,3} + \frac{\pi i }{4} \delta_{m,-1}
\end{aligned} \end{equation*} }
%
\textcolor{lightgray} { \begin{equation*} \begin{aligned}
\int_{0}^{2 \pi} d \varphi e^{-im \varphi} \partder{P_\varphi^\prime}{t} = \\
= - \frac{ {A_0}^3 \chi_3^e }{ 8 } \int_{0}^{2 \pi} d \varphi e^{-im \varphi}
\left( \frac{\mu_0 \mu} {\epsilon_0 \epsilon} \right)^{3/2} \left(
3 ( I_2 - I_1 )^2 \left( \partder{I_2}{t} - \partder{I_1}{t} \right)
\sin^3 \varphi \right. + \\
+ \left. I_1 \sin \varphi \cos^2 \varphi \left( 
I_1 \left( \partder{I_2}{t} - \partder{I_1}{t} \right) + 
2 \partder{I_1}{t} (I_2 - I_1) \right) \right) = 
- \frac{ {A_0}^3 \chi_3^e }{ 8 } \codt \\ 
\cdot \left( \frac{\mu_0 \mu} {\epsilon_0 \epsilon} \right)^{3/2} \left(
\frac{3 \pi i}{4} ( I_2 - I_1 )^2 \left( \partder{I_2}{t} - 
\partder{I_1}{t} \right) \left( 3 \delta_{m,-1} - 3 \delta_{m,1} - 
\delta_{m,-3} - \delta_{m,3} \right) \right. + \\
+ \left. \frac{\pi i}{4} I_1 \left(  \delta_{m,-3} - \delta_{m,1} + 
\delta_{m,3} + \delta_{m,-1} \right) \left( 
I_1 \left( \partder{I_2}{t} - \partder{I_1}{t} \right) + 
2 \partder{I_1}{t} (I_2 - I_1) \right) \right)
\end{aligned} \end{equation*} }
%
\begin{equation*} \begin{aligned}
\int_{0}^{2 \pi} d \varphi e^{-im \varphi} \partder{P_\varphi^\prime}{t} = 
- \frac{ {A_0}^3 \chi_3^e }{ 8 }
\left( \frac{\mu_0 \mu} {\epsilon_0 \epsilon} \right)^{3/2} \cdot \\ 
\cdot \left( \frac{3 \pi i}{4} ( I_2 - I_1 )^2 \left( \partder{I_2}{t} - 
\partder{I_1}{t} \right) \left( 3 \delta_{m,-1} - 3 \delta_{m,1} - 
\delta_{m,-3} - \delta_{m,3} \right) \right. + \\
+ \left. \frac{\pi i}{4} I_1 \left(  \delta_{m,-3} - \delta_{m,1} + 
\delta_{m,3} + \delta_{m,-1} \right) \left( 
I_1 \left( \partder{I_2}{t} - \partder{I_1}{t} \right) + 
2 \partder{I_1}{t} (I_2 - I_1) \right) \right)
\end{aligned} \end{equation*}
%
\textcolor{lightgray} { \begin{equation*} \begin{aligned}
\int \limits_{0}^{2\pi} d \varphi \sin \varphi 
\left( \cos m \varphi - i \sin m \varphi \right) = 
i \pi \left( \delta_{m,-1} - \delta_{m,1} \right)
\end{aligned} \end{equation*} }
%
\textcolor{lightgray} { \begin{equation*} \begin{aligned}
\int \limits_{0}^{2\pi} d \varphi \cos \varphi 
( \cos m \varphi - i \sin m \varphi) = \pi ( \delta_{m,-1} + \delta_{m,1} )
\end{aligned} \end{equation*} }

%%%%%%%%%%%%%%%%%%%%%%%%%%%%%%%%%%%%%%%%%%%%%%%%%%%%%%%%%%%%%%%%%%%%%%%%%%%%%%%%
\section{Нелінійна поправка}

\chapter{Дифракція на межі розподілу двох середовищ}
\label{ch:diffraction}

%%%%%%%%%%%%%%%%%%%%%%%%%%%%%%%%%%%%%%%%%%%%%%%%%%%%%%%%%%%%%%%%%%%%%%%%%%%%%%%
% \section{}
\chapter{Передача інформації у часовому просторі короткоімпульсними 
електромагнітними полями}
\label{ch:neuron}

%%%%%%%%%%%%%%%%%%%%%%%%%%%%%%%%%%%%%%%%%%%%%%%%%%%%%%%%%%%%%%%%%%%%%%%%%%%%%%%
\section{Недоліки класичного імпульсного радіо}

Велика інформаційна ємність імпульсного надширокосмугового випромінювання, 
а також низьке споживання енергії для випромінювання за рахунок 
короткотривалих збуджень робить його привабленим для технічного використання.
Новою сферою застосування послідовної надширокосмугової радіоелектроніки 
(DS-UWB) сьогодні стає інтернет речей. Основними чинниками для цього стали 
високий рівень інформаційної безпеки, порівняно низький рівень споживання 
електроенергії та стійкість до вузькосмугових завад. 
В задачах інтернету речей, робота таких пристроїв на маленькій відстані є, 
скоріш, перевагою, а ніж недоліком. Цей чинник зменшує радіо-забруднення 
приміщення за рахунок низьких потужностей та напрямленності випромінювання.

В оглядовій роботі \cite{imp:ChannelImplementation} приведені основні схеми
роботи імпульсного радіо послідовної передачі (без застосування 
стробоскопічного принципу), що використовуються в різноманітних сферах, 
починаючи з телекомунікації і закінчуючи радарними задачами, які можна 
узагальнити наступною схемою (Рис.~\ref{fig:emp_radio}).

\begin{figure}[htbp] \begin{center}
\includegraphics[scale=0.45]{classical_radio}
\caption{Класична схема імпульсного радіо} \label{fig:emp_radio}
\end{center} \end{figure}

Існуючі принципи аналогової обробки прийнятого імпульсного радіосигналу 
успадковані від схемотехніки, що використовувались для гармонійних сигналів 
\cite{imp:ComunicationsOverview}: для обробки отриманого з антени 
електричного струму використовується послідовна фільтрація та підсилення з 
подальшим оцифровуванням за допомогою АЦП і цифрової обробки в модулях FPGA 
(Рис.~\ref{fig:emp_radio}).

Кожен з послідовних етапів аналогової обробки, направлений на покращення 
окремої характеристики сигналу, неминуче впливає і на інші його характеристики,
що накопичує похибку та губить частину інформації про сигнал:

\begin{enumerate}
	\item фільтрація разом з інформацією про шум частково зачіпає спектр, 
	а відповідно і форму корисного сигналу і не прибирає шум повністю;
	\item кавзі-лінійне підсилювання незначним чином впливає на 
	форму імпульсу за рахунок нелінійності амплітудно-частотної 
	характеристики, а також підсилює не відфільтрований шум;
	\item аналогово-цифрове перетворення сигналу губить частину інформації 
	сигналу та шуму за рахунок дискретизації, що також ускладнює числову 
	обробку.
\end{enumerate}

Таким чином, на числову обробку потрапляє дещо видозміненій від початкового
сигнал. Також сам алгоритм числової обробки в модулі FPGA, зазвичай,
вибирається простим, через умову обробки сигналу в квазі-реальному часі.

\textcolor{red}{
Серед методів обробки сигналу в модулях FPGA, з лінійною складністю по часу 
виконання, можна виділити сімейство методів корелятивного порівняння, що 
базується на фільтрі Калмана. Ці методи, фактично, порівнюють отриманий сигнал 
з еталонним і надають коефіцієнт відповідності, чого достатньо для бінарної
класифікації наявності сигналу. }

В ближній зоні антени форма сигналу сильно залежить від напрямку 
спостереження \cite{imp:Wu1985, imp:Sodin1992-10, my:Telecom2018}. Це 
призводить до падіння точності роботи описаного підходу порівняння з 
еталоном. Також, погіршання якості роботи кореляційних алгоритмів впаде 
при прийманні сильних електромагнітних імпульсних хвиль, через неілнійну 
природу розвовсюдження в просторі та в компонентах антенно-фідерної стстеми.

Імпульсні надширокосмугові радіотехнічні пристрої мають теоретичні переваги 
над вузькосмуговими в плані інформаційної ємності, але на практиці, не 
вдається використовувати ці переваги повною мірою через складність обробки 
надширокосмугових сигналів \cite{imp:ChannelLimitations}. 

Для перетворення електромагнітної хвилі у вільному просторі в надширокосмуговий 
електричний струм у дротах прийнято використовувати масштабно-часове 
перетворення, яке може бути технічно виконано різними способами 
\cite{imp:Astanin1989}, а процес виділення корисної інформації з сигналу без 
її втрати може бути покращено, як буде показано далі.

%%%%%%%%%%%%%%%%%%%%%%%%%%%%%%%%%%%%%%%%%%%%%%%%%%%%%%%%%%%%%%%%%%%%%%%%%%%%%%%
\section{Імпульсний радіоприймач на базі апаратної нейронної мережі}

Всі задачі випромінювання, поширення чи дифракції хвилі формалізуються
у вигляді деяких параметричних рівнянь відносно компонентів струму чи 
компонентів поля. Коли аналітичне розв'язання такого рівняння знайти не 
вдається, а числовий метод розв'язання має значну обчислювальну складність, 
доцільно використати штучні нейронні мережі (ШНМ) прямого поширення, графові 
моделі машинного навчання або імпульсні нейронні мережі у купі з методами їх 
навчання для пошуку розв'язків.

Рівень оптимізації сучасних програмних інструментів машинного навчання, таких 
як CUDA та Tensorflow, а також рівень розвитку апаратних інструментів GPU/ASIC
дозволяють аналізувати цифрові часові послідовності за час порядку 
десятків мілісекунд, що дозволяє використовувати такі інструменти при роботі 
з сигналом у квазі-реальному часі, опрацьовуючи сигнал після АЦП. Недоліком 
такого методу може стати висока ціна кінцевих виробів, а також високий 
рівень споживання енергії. З іншого боку, галузь що швидко розвивається - 
аналогові штучні нейронні мережі виконані за технологією CMOS. На такі 
пристрої можна подати сигнал напряму з антенно-фідерної системи. Такі 
пристрої вже широко застосовуються радіотехніками в галузі когнітивного 
радіо \cite{imp:Husseini2010}, а також адаптивних вузькосмугових антенних 
систем \cite{imp:Zbynek2002}. В цих задачах, апаратні ШНМ мережі 
використовується для оптимізації деяких параметрів прийому-передачі сигналу 
в режимі реального часу.

Останнім часом, технічний розвиток в галузі апаратних штучних нейронних мереж 
дозволив втілювати їх різноманітні топологічні особливості в електронних 
аналогових пристроях. Проаналізуємо можливість застосування цих технологій для 
задач класифікації отриманого сигналу (sequence-to-label) та визначення його 
присутності в кожен момент часу (sequence-to-sequence). Результат розв'язання 
таких задач аналізу даних дозволить оцінити практичну цінність 
такого методу обробки радіосигналу в різних задачах 
прикладної електродинаміки: радіолокації, телекомунікації, вимірювання, 
зондування тощо. 

Важливим напрямком розвитку CMOS технології для покращення технічних 
характеристик імпульсного радіо стає зменшення техпроцесу. Наразі існують
готові прилади CMOS LSTM з техпроцесом 180нм. Використаннм техпроцесу 2нм,
що активно розвивається, можна збільшити швидкість обробки сигналів у 
порівнянні з класичними схемами імпульсного радіо з аналогічною цифровою 
обробкою у $ 10^9 $ разів.

\begin{figure}[htbp] \begin{center}
\includegraphics[scale=0.45]{neuron_radio}
\caption{Схема імпульсного радіо на нейронній схемотехніці} 
\label{fig:neural_radio}
\end{center} \end{figure}

Штучна нейронна мережа тут є електричним колом, внутрішня передавальна 
функція якого визначається лінійною комбінацією деякого набору матричних 
характеристик, які визначаються різноманітними методами оптимізації. Таким 
чином, задача обробки прийнятого радіосигналу зводиться до пошуку необхідних 
матричних параметрів. Для цього гарно підходять градієнтні методи навчання 
з учителем, де конкретна імплементація процесу тренування та набір 
тренувальних даних залежатиме від типу задачі, що розв'язується.

В якості нейронної архітектури для кіл обробки радіосигналу розглянемо
схему encoder-decoder \cite{imp:Ying2017}. 
В цій архітектурі нейронна мережа топологічно розбивається на 
дві частини. Перша частина трансформує вхідну часову послідовність в деякий 
набір параметрів, які однозначно характеризують вхідний сигнал. Тобто, 
encoder проектує вхідний сигнал на деякий ознаковий
простір. Друга частина мережі, decoder, перетворює набір ознак в ту якісну 
або кількісну характеристику, яку передбачає постанова задачі. Наприклад,
для задачі телекомунікації - це інформаційне повідомлення, або для радарної 
задачі - це положення та тип цілі. Такий підхід забезпечить повторне 
використання попередньо-навчених encoder-ів для різноманітних задач 
радіофізики. Технічна можливість додати декілька різних декодерів в 
електричні кола радіоприймальної схеми зробить цю концепцію промислово 
придатною.

Вихідний сигнал, що продукує апаратна нейронна мережа, визначається 
активаційною функцією вихідного нейрону. Цифровий вихідний сигнал можна 
отримати ступеневою активаційною функцією (персептрон Розенблата) та зручно 
використовувати описаний пристрій, як мереживний інтерфейс комп'ютера чи 
джерело керуючого сигналу для робототехніки.

Прикладом застосування аналогових нейронних мереж у радіо є покращення 
характеристик роботи CDMA телефонії, шляхом послідовної лінійної фільтрації
з подальшою обробкою повнозв'язною аналоговою нейронною мережею прямого 
поширення \textcolor{red}{[ПОСИЛАННЯ]}. Принципова відмінність нейронного
радіо від прототипів полягає у відмові від обробки кожної характеристики 
сигналу окремо, і делегації його аналізу, як цілого, пристрою, системотехніка 
якого реалізує деякий математичний граф. В досліджені 
\cite{imp:Zhang2009} задачу розрізнення збуджень різного типу 
виріщують за допомогою лінійного фільтру і, як буде показано далі, цю ж 
задачу краще розв'язує енкодер штучної нейронної мережі за рахунок 
врахування складної природи поширення імпульсних електромагнітних 
хвиль.

Як і класична схема імпульсного радіо, запропонована схема передбачає 
можливість застосовувати одну і ту ж антену для передачі та прийму сигналу. 
Також використання нейропроцесору не виключає застосування традиційних 
методів аналогової обробки: підсилення, фільтрація.

Важливим аспектом є те, що використання ПЗУ дозволить програмувати нейронний 
процесор шляхом встановлення нових параметрів аналогових нейронів керуючими 
напругами, що дає змогу змінювати технічні характеристики пристроїв без 
втручання в його схемотехніку. Але практика не завжди потребує такої 
гнучкості: частіше, для всього періоду використання вбудованої електроніки, 
її призначення та порядок роботи залишаються постійним, а отже керуюча 
напруга може задаватись резисторами, що значно здешевить пристрій.

Порівняння рівня споживання електроенергії аналогового нейропроцесора та 
аналогічного GPU/ASIC \cite{imp:AnalogLSTM} показало кращу енергоефективність 
для першого. Відтак, використовуючи аналогового нейропроцесор замість АЦП та
FPGA можна скоротити споживання енергії надшироосмуговим радіо.

%%%%%%%%%%%%%%%%%%%%%%%%%%%%%%%%%%%%%%%%%%%%%%%%%%%%%%%%%%%%%%%%%%%%%%%%%%%%%%%
\section{Формування тренувальних даних}

Розглянемо задачу односторонньої передачі інформації через нейронне радіо. 
В якості передавальної антени розглядається антена типу LIRA. Для спрощення, 
в якості приймальної антени розглянемо ідеальний надширокосуговий вимірювач 
напруженності електричного поля, який не змінює форму отриманого сигналу,
тобто гарантує відсутність впливу приймальної антени на отриманий сигнал. 
Сигнал з приймальної антено-фідерної системи подається на деяку апаратну 
нейронну мережу. Напруженість електричного поля, породженого випромінювальною 
антеною $ \vect{E}_{tx} $, можна визначити в довільній точці спостереження при 
довільному збуджені з урахуваннями ефектів ближньої зони, користуючись 
згорткою \eqref{eq:duhamel} по перехідній функції 
$ \vect{E}_0 $. Тоді, отриманий з антенно-фідерної системи сигнал, буде 
пропорційний до компонентів напруженості електричного поля випромінювальної 
антени. Здійснюючи гіпотетичне вимірювання в такій площині, що спостерігається 
$ OX $ компонента напруженості поля, а приймальна лінія ідеально узгоджена і 
немає втрат, отриманий сигнал матиме вигляд:

\begin{equation}
f_{rx} \left( \vect{r}, t \right) = 
\int_0^t \derivat{f_{tx}}{\tau} \vect{E_0} (\vect{r}, t - \tau) d \tau,
\end{equation}
%
де $ \vect{E_0} $ - перехідна функція передавальної антени типу LIRA, а 
$ f_{tx} (t) $ - форма сигналу, що збуджує передавальну антену.

Метою даного моделювання є пошук оптимальної нейронної архітектури, а також 
її вагових коефіцієнтів, що дозволить співвіднести прийнятий сигнал з деяким 
типом збудження на передавачі в умовах завад, та з урахуванням деяких ефектів 
ближньої зони.

Тренувальний набір даних для цієї задачі складатиметься з пар часових 
послідовностей - струм збудження передавальної антени $ f_{tx} (t) $ та струм, 
що буде отримано приймачем $ f_{rx} (t) $ при різних розташуваннях 
приймача $ \vect{r} $ відносно системи координат передавача, введеної як на 
Рис.~\ref{fig:pdisk}. Для максимально правильного функціонування мережі в 
заданих умовах, набір тренувальних даних повинен містити вичерпну інформацію про 
поведінку поля у всій області функціонування антенної системи, а тобто містити 
вимірювання в ближній і дальній зонах. Користуючись визначенням дальньої зони,
отримуємо максимальне віддалення від джерела, де необхідно проводити 
вимірювання - $ 0 \leq z \leq 8R $. Користуючись направленістю антен типу 
LIRA, обмежимо радіус поперечного зрізу циліндричної області, де проводяться 
вимірювання $ 0 \leq \rho \leq R $, а користуючись симетрією джерела розглянемо 
не весь зріз, а лише його першу чверть $ 0 \leq \varphi \leq \pi / 2 $.

Таким чином, просторовий об'єм, в якому необхідно провести вимірювання 
складає $ \pi R^2 / 2 $. З огляду на низьку сходимість деяких алгоритмів 
навчання заповнимо отримана область великою кількістю (10000) випадково 
розміщених і рівномірно розподілених уявних вимірювань для тренувального 
набору даних, а також в чотири рази меншим тестувальним набором - 
2500 зразків. Для наближення моделі до реальних умов, до кожного з семплів 
додамо деяку випадкову заваду. Такий підхід відомий в літературі, як 
задача аналізу AWGN каналу, де енергія шуму визначається як квадрат 
середнього відхилення, що можна записати математично:

\begin{equation} \label{eq:snr}
SNR = 10 * \log_{10} \left( \frac{W}{\sigma^2} \right),
\end{equation}
%
де $ W $ - енергія електромагнітної хвилі, а $ \sigma $ - другий статистичний 
момент моделі білого гаусівського шуму. У виразі не враховано вплив постійної 
фонової напруженості поля, яка фігурує в моделі білого шуму для каналу зв'язку 
через відсутність його впливу на часову залежність прийнятого сигналу.

\begin{figure}[htbp] \begin{center}
\includegraphics[scale=0.4]{P4AWGN}
\caption{Енергія моделі білого шуму в залежності від параметру} \label{fig:P4AWGN}
\end{center} \end{figure}

Через напрямлені властивості антени, навіть при постійному рівні завад, 
отриманий набір даних складатиметься зі зразків з різним значенням SNR: більшим 
на осі та меншим на периферії (Рис.~\ref{fig:SNR}). Для якісного навчання 
необхідно, щоб розподіл тренувальних даних за значенням SNR відповідав 
імовірнісному розподілу прийнятих сигналів за значенням SNR в умовах 
реального використання. Припустимо, що користувачі пристроїв будуть 
намагатись вести прийомо-передачу максимізуючи SNR, тоді реальний імовірнісний 
розподіл прийнятих сигналів матиме вигляд гаусівського, через статистичні 
відхилення від ідеальних параметрів та в умовах завад.

Для якісного процесу навчання необхідно, щоб тренувальний набір даних не 
лише містив всю палітру значень SNR при кожній епосі навчання, а ще і 
послідовно зменшував його середнє значення кожного разу, досягаючи 
локального мінімуму цільової функції.

\begin{figure}[htbp] \begin{center}
\includegraphics[scale=0.9]{Dataset_SNR}
\caption{Зашумленість набору тренувальних даних} \label{fig:SNR}
\end{center} \end{figure}

На Рис.~\ref{fig:SNR} зображено гістограму, де висота стовбчика ілюструє 
кількість зразків в датасеті з відповідним значенням SNR. Для генерації 
тренувальних даних застосовано рандомайзер mt19937 в реалізації стандартної
бібліотеки шаблонів С++ для 64-х розрядних систем за стандартом ISO C++17.

Для перевірки можливостей нейронного радіо вирізняти різні види сигналів 
розглянемо відразу ряд збуджувальних імпульсів:

\begin{equation} \label{eq:type_void}
f_0 = void = 0;
\end{equation}
%
\begin{equation} \label{eq:type_sinc}
f_1 = sinc = \sinc(t-\tau/2);
\end{equation}
%
\begin{equation} \label{eq:type_gauss}
f_2 = gauss = \exp(- (t-\tau/2)^2 );
\end{equation}
%
\begin{equation} \label{eq:type_gauss_perp}
f_3 = gauss\_perp = \partder{}{t} \exp(- (t-\tau/2)^2 );
\end{equation}
%
тоді отриманий набір даних матиме 4 класи, де окремим класом є білий шум. Для
дотримання збалансованості даних в процесі навчання, до рандомайзера додамо 
випадковий рівномірно розподілений дискретний параметр, що відповідатиме 
за тип збудження: \textit{void}, \textit{sinc}, \textit{gauss}, 
\textit{gauss\_perp}. Для максимізації якості навчання розглянемо 
лише датасет, де вибірки за типом збудження будуть кількісно збалансовані. 

Процес навчання штучної нейронної мережі може проходити як на комп'ютері так і 
на апаратному модулі за рахунок створення спеціальних програмних драйверів
до апаратної частини. В межах даного дослідження 
достатньо навчання на комп'ютері, як на ресурсах CPU так і на ресурсах GPU.
Для цього набір тренувальних даних необхідно дискретизувати. Частоту 
дискретизації вибираємо, користуючись критерієм Найквіста.

Тобто кожен зразок тренувального набору складатиметься з часової
послідовності, що відповідає прийнятому сигналу та метаданих, що описують 
переданий сигнал: значення енергетичного SNR, тип збудження, 
точка спостереження, ефективна тривалість збудження та інше. Для цього
зручно використати формати зберігання даних JSON та TFRecord.

\textcolor{red}{TODO: порівняти методи навчання NLP з DeepUWB}

%%%%%%%%%%%%%%%%%%%%%%%%%%%%%%%%%%%%%%%%%%%%%%%%%%%%%%%%%%%%%%%%%%%%%%%%%%%%%%%
\section{Моделювання детекції сигналу для повнозв'язних моделей}

Розглянемо декілька моделей нейронних мереж та визначимо недоліки та переваги
кожної з них. Сімейство моделей, що розглядається звузимо лише до тих, які 
просто виготовляти. Перші моделі штучних нейронних мереж прямого 
поширення, що були запропоновані - повнозв'язні з одним прихованим шаром.

\textcolor{red}{TODO: плавні активаційні функції через аналогову імплементацію}

\begin{figure}[htbp] \begin{center}
\includegraphics[scale=0.65]{simple_radio}
\caption{Імпульсне радіо на основі багатошарового перцептрону} 
\label{fig:mp_radio}
\end{center} \end{figure}

Такі мережі є популярним інструментом розв'язання задач різного типу 
\cite{imp:Kussul2004}, в основному, через досить простий алгоритм навчання.
Для подачі часової послідовності (сигналу) на її вхід обов'язковим стає 
застосування методики ковзного вікна шляхом його затримки, що зменшує частоту
дискретизації вихідного сигналу з антени та втрачає частину 
даних. Такий підхід забезпечує подачу до нейронної мережі ковзного вікна даних 
та не враховує результат обробки минулого вікна при переході до наступного.

Можна підрахувати, що розмірність вхідного шару визначається частотою 
дискретизації. Для імпульсів переданих антеною типу LIRA з одиничним 
електричним розміром вхідний шар складатиметься не менше ніж з 300 штучних 
нейронів. Так як сигнал може знаходитись в будь-якій частині вікна прихований 
шар повинен містити не меншу кількість нейронів для збереження якісного 
результату в кожен момент часу. З іншого боку, для більшості задач 
електродинаміки важливо визначити, що сигнал взагалі був в якомусь з вікон,
що накладаються, тому кількість нейронів в другому шарі можна дещо зменшити.
Останній вихідний шар, що виконує функцію декодеру матиме розмірність 
якісної або кількісної характеристики, яку передбачає постанова задачі.
У випадку, що розглядається - це тип збудження, який може приймати чотири 
дискретні значення. Таким чином, кількість параметрів, що підлягають 
тренуванню - $ 76400 $. Для розв'язання електромагнітних задач на кшталт
зондування, де необхідно враховувати перевідбиття, розмір вікна спостереження
почне залежати від постанови задачі, що призведе до зростання кількості 
тренувальних параметрів до величини порядку $ 10^7 $.

\begin{figure}[htbp] \begin{center}
\includegraphics[scale=0.9]{FC_S2L_loss}
\caption{Зміна значення цільової функції повнозв'язної моделі  
в процесі тренування} \label{fig:fcnn_loss}
\end{center} \end{figure}

Не дивлячись на величезну кількість тренувальних параметрів, навчання 
проходить досить швидко за рахунок малої глибини моделі. 
На Рис.~\ref{fig:fcnn_loss} зображено зміну значень цільової функції в 
процесі тренування. Її пораховано на тренувальних та на тестувальних даних 
та зображено окремими кривими. Перетин тренувальної та тестувальної 
цільової функції після третьої епохи є індикатором перенавчання моделі. 
Для тренування використовувалась техніка Dropeout (виключення окремих 
нейронв з деяких етапів навчання), що в купі з перенавчанням є фактором, 
що вказує на недостатню інформаційну ємність повнозв'язної моделі для 
розв'язання поставленої задачі.

Задачею радіоприймача є перетворення в реальному часі сигналу з антени в 
деяку послідовність корисних даних. Згідно класифікації задач аналізу даних, 
цю задачу можна розглядати як many-to-many так і many-to-one. Повнозв'язна 
нейронна мережа працює саме за схемою many-to-one, коли деякому вікну, що 
спостерігається, призначається одна якісна характеристика -- вектор 
імовірностей присутності сигналів кожного з видів.

Точність роботи, визначена за метрикою середньої відносної помилки,
справедлива лише до третьої епохи, через проблему перенавчання, що наступає 
коли криві перетинаються. Таким чином, мінімальне значення середньої 
відносної помилки для штучної нейронної мережі пов'язаного типу складе 
$ 89 \% $.

Серед недоліків застосування описаної архітектури в якості нейронного радіо
можна відмітити велику кількість тренувальних параметрів, що ускладнить 
електронну схему пристрою. Цю проблему можна вирішити використанням 
рекурентних штучних нейронних мереж, які одномоментно приймають на вхід 
одне значення часової послідовності і накопичують інформацію про сигнал 
з плином часу за рахунок зміни свого внутрішнього стану.

\textcolor{red}{TODO: Оцінка стійкості апаратних нейронних мереж до 
внутрішніх шумів}

%%%%%%%%%%%%%%%%%%%%%%%%%%%%%%%%%%%%%%%%%%%%%%%%%%%%%%%%%%%%%%%%%%%%%%%%%%%%%%%
\section{Моделювання детекції сигналу для рекурентних моделей}

На відміну від повнозв'язної моделі, рекурентні можуть працювати як за схемою
many-to-one так і за схемою many-to-many, коли в кожен момент часу деякому 
вхідному сигналу співвідноситься якісна або кількісна вихідна характеристика.

У випадку many-to-one вірний прогноз нейронної мережі стосується ковзного 
вікна в цілому, а отже, точність визначення меж сигналу у часі буде визначена 
розміром вікна спостереження, яке в декілька разів ширше за сигнал. Моделі, 
що працюють за схемою many-to-many позбавлені цього недоліку.

Так як режим роботи нейронної мережі, а тобто many-to-one або many-to-many,
визначається алгоритмом тренування, апаратна реалізація пристрою залишається 
однаковою для many-to-one і many-to-many схем, що зручно для прикладного 
застосування.

Проста архітектура та мала кількість тренувальних параметрів є 
перевагами класичної топології рекурентної штучної нейронної мережі.
Її основним недоліком є нестабільність процесу навчання -- проблеми 
exploding gradient і vanishing gradient. Саме для вирішення цих проблем 
були створені архітектури GRU та LSTM \cite{imp:Hochreiter1997}.

\begin{figure}[htbp] \begin{center}
\includegraphics[scale=0.7]{lstm_radio}
\caption{Імпульсне радіо на основі нейронної мережі 
довго-короткотривалої пам'яті} \label{fig:lstm_radio}
\end{center} \end{figure}

На Рис.~\ref{fig:lstm_radio} зображено нейронне радіо з використанням 
рекурентних штучних нейронних мереж. Вихідний шар (decoder) залишився без 
змін - його розмірність визначається практичними потребами. В поточному 
досліді нейрони останнього шару мають сигмоїдальні активаційні функції 
та відповідають імовірностям спостерігати сигнал певного типу. Вхідний 
шар ШНМ є рекурентним, тобто закладається з ланцюжку однакових нейронів.
Описана штучна нейронна мережа має лише 38-116 змінних параметрів в 
залежності від типу рекурентного шару.

Розглянемо в якості вхідного шару LSTM ланцюжок. Його перевага над 
GRU в контексті нейронного радіо -- універсальність: він працює як за схемою
many-to-one так і за схемою many-to-many. Єдиним недоліком LSTM у порівнянні 
з GRU топологією стане більш тривале поширення сигналу крізь такий шар,
що можна нівелювати, використовуючи менший техпроцес.

Спершу розглянемо режим роботи many-to-one, щоб порівняти результат з 
повнозв'язаною нейронною мережею.

\begin{figure}[htbp] \begin{center}
\includegraphics[scale=0.35]{LSTM_S2L_loss}
\caption{Зміна значення цільової функції моделі LSTM 
в процесі тренування} \label{fig:lstm_loss}
\end{center} \end{figure}

На Рис.~\ref{fig:lstm_loss} зображено зміну значень цільової функції в процесі
навчання на тренувальних даних. Помічаємо, що швидкість навчання 
не рівномірна -- спостерігаються ``плато'' зі сталими значеннями цільової 
функції, не приймаючи до уваги випадкових викидів. Подальший аналіз показав, 
що кожен з таких відрізків відповідає за навчання розпізнаванню кожного з 
типу сигналів, що вивчаються. Як можна помітити на Рис.~\ref{fig:lstm_loss},
кожен наступний тип сигналу вивчається довше минулого. Порядок вивчення 
імпульсів теж виявився не випадковим: чим більше осциляцій відносно нуля має 
імпульс, тим довше і пізніше він вивчається.

Бачимо, що застосування нейронних мереж замість лінійної фільтрації 
дозволяє покращити розпізнавальну здатність у ближній зоні антени, де форма 
сигналу досить мінлива \cite{my:UWBUSIS2018}.

Використовуючи рекурентні нейронні мережі можна досягти точності в 
$ 99.7\% $, що значно перевищує результати повнозв'язної нейронної мережі. 
Однак, з рис.~\ref{fig:lstm_loss} видно, що використання рекурентних 
нейронних мереж помітно сповільнює процес навчання: кількість епох 
тренування зросла на два порядки.

Розглянемо тренування за моделлю many-to-many. Топологія мережі залишається 
як на рис.~\ref{fig:lstm_radio}, а дані для тренування доповнимо анотацією 
для кожного моменту часу замість анотування деякого вікна спостереження.

\begin{figure}[htbp] \begin{center}
\includegraphics[scale=0.9]{lstm-seq2seq-loss}
\caption{Зміна значення цільової функції моделі LSTM
в процессі тренування} \label{fig:lstm_seq2seq_loss}
\end{center} \end{figure}

На Рис.~\ref{fig:lstm_seq2seq_loss} зображено зміну значень цільової функції
для задачі маркування послідовності (many-to-many). Тут зображено процес 
навчання штучної нейронної мережі з рис.~\ref{fig:lstm_radio}. Цільова
функція тренувального процесу спрямована на максимізацію здібності визначати
імовірності присутності сигналу певного виду в певний момент спостереження.
При переході від many-to-one до many-to-many тренування сповільнилось.

\begin{figure}[htbp] \begin{center}
\includegraphics[scale=0.9]{seq2seq_example}
\caption{Приклад правильного аналізу} \label{fig:seq2seq_example}
\end{center} \end{figure}

На Рис.~\ref{fig:seq2seq_example} зображено приклад роботи рекурентної 
штучної нейронної мережі для класифікації прийнятого надширокосмугового сигналу 
в кожен момент часу. В представленому зразку даних спостерігається сигнал,
породжений збудженням антени типу LIRA, що має часову залежність у вигляді 
похідної від Гаусіна \eqref{eq:type_gauss_perp} та спостерігається з 
таким відхиленням від осі $ OZ $, що збуджене поле виглядає як поле породжене 
іншим збудженням \eqref{eq:type_gauss}. Також на рисунку зображені 
імовірності приналежності сигналу в кожен момент часу до певного збудження 
чи шуму (відсутності збудження).

Для проміжку часу, що передує видимому імпульсу імовірності приналежності 
сигналу до одного з типів залишаються 
приблизно рівними та складають близько $ 25 \% $. Тобто при спостеріганні 
білого шуму значення виходу нейронної мережі, фактично, не правильне. Не 
дивлячись на це, наявність сигналу можна визначити, аналізуючи всі вихідні 
значення ШНМ: якщо імовірності наявності сигналів кожного з типів (в тому 
числі і його відсутності) рівні, то спостерігається лише шум і цю системн 
похибку можна врахувати. Таку похибку пояснити тим, що нейронна мережа 
намагається виділити в шумі сигнал кожної з вивчених форм, а не знайшовши 
сигналу повертає мінімальний рівноімовірний результат. Така систематична 
похибка легко виявляється та нейтралізується евристичним аналізом або 
додатковим модулем, що виконує операцію XOR (наприклад нейропроцесор з 
трьох нейронів).

На рис.~\ref{fig:lstm_seq2seq_loss} проілюстровано, що навіть у ближній зоні, 
де форма імпульсу може змінюватись настільки, що стає більше схожою на інший 
сигнал, нейронне радіо гарантує стійкий режим роботи. З моменту, коли сигнал 
візуально спостерігається (індекси часової послідовності 
$ \left[ 45, 55 \right]$ ), деякий час імовірність приналежності сигналу до 
деяких типів зростає одночасно. Це можна пояснити через схожість 
градієнта часової послідовності на градієнти сигналів різних типів. Далі, 
мережа визначається з вибором і тримає його весь час тривалості сигналу. 
Стійка детекція сигналу за пороговим значенням значенням $ 0.707 $ займає 
близько $ 70\% $ тривалості сигналу.

Точність роботи мережі на валідаційному датасеті впала до $ 98.9\% $,
що є закономірним при підвищенні точності визначення тривалості сигналу.

\begin{figure}[htbp] \begin{center}
\includegraphics[scale=0.9]{lstm_seq2seq_bad}
\caption{Приклад неправильного аналізу} \label{fig:lstm_seq2seq_bad}
\end{center} \end{figure}

На рис.~\ref{fig:lstm_seq2seq_bad} зображено приклад неправильного
розпізнавання сигналу. Зразок містить сигнал, породжений збудженням антени 
типу LIRA, що має часову залежність у вигляді \eqref{eq:type_sinc}. 
Рекурентна штучна нейронна мережа надала нестійку
детекцію трьох сигналів в хронологічній послідовності: \textit{sinc}, 
\textit{gauss}, \textit{sinc}. Детекція першого сигналу викликана 
одномоментною схожістю сигналу на \textit{sinc}, що викликало ланцюжкову 
реакцію для подальших помилкових детекції сигналів: мережа гадає, що 
помилково детектований сигнал накладається на реальний сигнал і робить 
невірне передбачення класу його приналежності.

%%%%%%%%%%%%%%%%%%%%%%%%%%%%%%%%%%%%%%%%%%%%%%%%%%%%%%%%%%%%%%%%%%%%%%%%%%%%%%%
\section{Моделювання детекції сигналу для графічних моделей}

\textcolor{red}{TODO: HMM для захищеного радіо-вимикача}

\textcolor{red}{TODO: HNN для захищеного радіо-вимикача}

%%%%%%%%%%%%%%%%%%%%%%%%%%%%%%%%%%%%%%%%%%%%%%%%%%%%%%%%%%%%%%%%%%%%%%%%%%%%%%%
\section{Універсальна топологія нейронного радіо та її застосування}

Як показано раніше, повнозв'язна штучна нейронна мережа прямого поширення
дозволяє провести перетворення сигналу на інформацію, що покращить якісні 
характеристики детектора у порівнянні з класичним імпульсним радіо. 
Застосування енкодеру у вигляді рекурентної нейронної мережі робить нейронне 
радіо можливим для практичної реалізації через суттєве зменшення кількості 
штучних нейронів, а також покращить якість роботи, за рахунок топологічного 
врахування принципу причинності і принципу суперпозиції. Також архітектуру 
можна покращити, замінивши повнозв'язний шар графічним.

Така рекурентно-графічна мережа дозволить якісно краще розв'язувати задачі 
імпульсної телефонії для багатокористувацького середовища, де врахування форми 
імпульсу, його тривалості і ефектів ближньої зони особливо критичне. Отримана 
архітектура штучної нейронної мережі також зустрічається і використовується в 
задачах аналізу розпізнавання усної мови на кшталт розпізнавання голосових 
команд \textcolor{red}{[ПОСИЛАННЯ]} або в задачах аналізу даних з сенсорів
присутності людини в приміщенні \textcolor{red}{[ПОСИЛАННЯ]}. 

\textcolor{red}{TODO: BER (bit error rate) calculation in DS-UWB system in AWGN}

Для первинного навчання нейронного радіо застосовуються дані, отримані 
теоретичним моделюванням, замість експериментальних вимірювань задля спрощення 
практичного застосування пристроїв та їх виготовлення. При такому підході,
різниця реальних даних і тренувальних викликає падіння точності детектування.
Шляхом вирішення цієї проблеми є застосування методів переносу навчання, які 
широко використовуються в задачах аналізу часових послідовностей. 

Сутність методів переносу навчання полягає в дотренуванні окремих елементів 
мережі, користуючись експериментально отриманими даними, для адаптації її 
параметрів для реальних умов. Таким чином, первинне тренування на даних
теоретичних моделювань дозволяє суттєво зменшити об'єм емпіричних вимірювань,
необхідних для проведення дотренування у порівнянні з тренуванням без 
первинного наближення \textcolor{red}{[ПОСИЛАННЯ]}.

Запропонована рекурентно-графічна архітектура також дозволяє отримати приріст 
в швидкості передачі даних за рахунок побудови протоколів комунікації, де для
кодування використовуються імпульси різної форми (high radix networking).

\begin{equation}
C = \frac{1}{N_{smp}} \frac{\log_2 \left( 1 + SNR \right)}{1/B + \tau_{RMS}} 
\end{equation}

\begin{figure}[htbp] \begin{center}
\includegraphics[scale=0.7]{channel_capacity}
\caption{Інформаційна ємність імпульсного випромінювання} \label{fig:info_cap}
\end{center} \end{figure}

Для налагодження якісного імпульсного зв'язку використовують послідовності 
імпульсів для кодування одного символу, що дозволяє розв'язати задачу 
імпульсної комунікації в умовах перевідбиттів (multi-path problem).

При використанні запропонованої рекурентно-графічної моделі, графічний 
decoder не вирішує проблему перевідбиттів (multipath): недостатня 
запам'ятовувальна здатність графічної моделі призводить до лінійного 
погіршення якості детектування імпульсу при кількісному збільшенні імпульсів, 
що кодують сигнал. Очевидно, що топологічне ускладнення декодеру, наприклад 
застосування LSTM, дозволить вирішити цю проблему. З іншого боку, замість 
ускладнення декодеру, простіше застосувати FPGA для аналізу цифрового сигналу, 
що повертається нейронним радіо.

Зазвичай надширокосмугова імпульсна радіолокація виконується 
через вимірювання часу надходження відбитого випромінювання. Використання 
нейронної мережі дозволить підвищити точність такого вимірювання через 
визначення не тільки часу, а і азимутального кута прийому. Для цього навчимо 
нейронну мережу розпізнавати напрямок до цілі за формою імпульсу, яка сильно 
змінюється у ближній зоні.

При розв'язанні задач зондування, де інформація про об'єкт зондування 
розташована у після-імпульсних коливаннях, тривалість яких фактично 
нескінченна, а тривалість ROI залижіть від глибини розташовання цілі. 
Очевидним недоліком запропонованої рекурентно-графічної моделі стає лінійне 
погіршення якості роботи енкодеру, який запам'ятовуватиме все триваліші
відгуки середовища на імпульс. Просте подовження рекурентного ланцюжку 
енкодену не вирішить цієї проблеми і спостерігатиметься лінійне погіршення 
якості його роботи.

Для задач розпізнавання геометрично складних цілей в середовищі зі сторонніми
об'єктами енкодер потребує архітектурних ускладнень. Серед методів, які 
дозволяють збільшити запам'ятовувальну здатність енкодеру: Bi-LSTM, 
LSTM-to-LSTM при many-to-many зв'язками та Attention based LSTM.

\textcolor{red}{TODO: denoising autoencoder (NDA)}

\textcolor{red}{TODO: генерація сигналу для передачі декодером мережі для 
розв'язання задач адаптивних антен, або для випромінювання сигналу-відповіді 
на вхідний запит по радіоаналу}

%%%%%%%%%%%%%%%%%%%%%%%%%%%%%%%%%%%%%%%%%%%%%%%%%%%%%%%%%%%%%%%%%%%%%%%%%%%%%%
\section*{Висновки до розділу \ref{ch:neuron}}

Імпульсні надширокосмугові радіотехнічні пристрої мають теоретичні переваги 
над вузькосмуговими в плані інформаційної ємності, але на практиці, не 
вдається використовувати ці переваги повною мірою через складність обробки 
надширокосмугових сигналів \cite{imp:ChannelLimitations}, про що також 
свідчать результати проведених симуляцій. Отже нейронне радіо може стати 
перспективним напрямком розвитку телекомунікаційних систем, після 
впровадження 5G технології.

Одним з напрямків дослідження щодо розвитку нейронного радіо може стати
застосування в якості аналогово модуля рекурентної нейронної мережі з 
комплексними тренувальними параметрами \cite{imp:NIPS2018}. За результатами 
досліджень, така модель краще підходять для аналізу імпульсних часових 
послідовностей, але станом на сьогодні не існує відповідних пристоїв, що 
працюють з аналоговим струмом.

Перспективним напрямком дослідження в області нейронного радіо є 
застосування імпульсних штучних нейронних мереж замість штучної нейронної 
мережі прямого поширення. Тренування таких мереж здійснюється шляхом 
самоорганізації системи під зовнішнім впливом з позитивним підкріпленням. 
Такі мережі простіше виконати у виді аналогової мікросхеми, ніж мережі
прямого поширення. З огляду малого обсягу інструментального 
апарату для навчання таких моделей цей підхід в даному досліджені не 
розглядається, але швидкий розвиток подібних технологій залишає їх 
дослідження перспективним в майбутньому.

Результати цього розділу впроваджують деякі ідеї опубліковані в роботах 
автора \cite{my:Telecom2018, my:UKRCON2019} та відображені в роботах автора
\textcolor{red}{[ПОСИЛАННЯ]}. Представлені матеріали знайшли своє застосування
в проектах з відкритим кодом та відомі як DeepUWB.


\chapter*{Висновки}

\begin{enumerate}
%
\item Побудовано аналітичне розв'язання у вигляді кусково визначеної функції для 
задачі випромінювання круглої апертури при нестаціонарному збуджені у вигляді 
прямокутної функції. Розв'язок отримано без наближення дальної зони та визначено 
для всіх точок спостереження в кожен момент часу. Використання моделі круглої 
апертури, як моделі антен типу LIRA перевірено на експерементальних даних в 
окремих точках та на даних отриманих методом FDTD з комерційного електромагнітного 
симулятора CST Studio.
%
\item Отримане розв'язання задачі випромінювання плаского диску при збуджені у 
вигляді функції Хевісайда в лінійному наближенні має чітку просторово-часову 
зональність та ілюструє твердження Фарадея, що випромінює не антена, а простір 
довколі неї. Отримані області випромінювання наступають послідовно для довільної 
точки спостереженя. Остання за часом настання область $ S_3 $ відповідає 
стаціонарному (усталеному) процесу випромінювання, коли всі точки апертури 
поєднані зі спостерегічем за принципом причинності. Настанню усталеного процесу 
передує область деякого транзитивного процесу $ S_2 $, поки поле від всієї 
апертури не досягне спостерігача. Найпершою для спостерігача просторово-часовою 
областю випромінювання в прожекторній зоні круглої апертури настає область 
електромагнітного снаряду $ S_1 $, де з хвилі у ТЕМ рупора формується ТЕ хвиля 
у вільному просторі.
%
\item 
%
\end{enumerate}
%


%\begin{bibset}{Список використаних джерел}
\bibliographystyle{acm}
% Для сортування літератури за алфавітом використовуйте
%\bibliographystyle{gost71s}
\bibliography{../my,../import}
%\end{bibset}
%GATHER{xampl-mybib.bib}

%\begin{bibset}[a]{Список публікацій автора}
%\bibliographystyle{acm}
%\bibliography{mybib}
%\end{bibset}


\appendix
\chapter{Деякі властивості тригонометричних функції}
\label{ch:trigonometric}

В цьому розділі представлено деякі маловідомі властивості тригонометричних 
функцій для зручності споживання викладеного в кваліфікаційній роботі 
матеріалу.

\textcolor{blue}{
\begin{equation*}
\derivat{}{\varphi} \arccos \varphi = - \frac{1}{ \sqrt{1 - \varphi^2} }
\end{equation*}
%
\begin{equation*}
\derivat{}{\varphi} \arctan \varphi = \frac{1}{1 + \varphi^2}
\end{equation*}
%
\begin{equation*}
\cos \alpha \cos \beta = \frac{1}{2} 
\left(  \cos (\alpha + \beta) + \cos (\alpha - \beta) \right)
\end{equation*}
%
\begin{equation*}
\sin \alpha \cos \beta = \frac{1}{2} 
\left( \sin (\alpha + \beta) + \sin (\alpha - \beta) \right)
\end{equation*}
%
\begin{equation*}
\sin \alpha \sin \beta = \frac{1}{2} 
\left( \cos (\alpha - \beta) - \cos (\alpha + \beta) \right)
\end{equation*}
%
\begin{equation*}
e^{im \varphi} = \cos m \varphi + i \sin m \varphi
\end{equation*}
%
\begin{equation*}
e^{-im \varphi} = \cos m \varphi - i \sin m \varphi
\end{equation*}
%
\begin{equation*}
\sin \varphi = \frac{e^{i \varphi} - e^{- i \varphi}}{2i}
\end{equation*}
%
\begin{equation*}
\cos \varphi = \frac{e^{i \varphi} + e^{- i \varphi}}{2}
\end{equation*}
%
\begin{equation*}
\arctan \frac{1}{x} = \frac{\pi}{2} - \arctan x
\end{equation*}
%
\begin{equation*}
\pi - \arccos x = \arccos (-x)
\end{equation*}
} % textcolor blue
%
\begin{equation}
\arccos x - \arccos y = \mp \arccos \left( 
xy + \sqrt{(1-x^2)(1-y^2)} \right),
\left\{ \begin{array}{c} x \ge y \\ x < y  \end{array} \right\}
\end{equation}
%
\begin{equation}
\arctan x - \arctan y = 
\arctan \frac{x-y}{1+xy}, xy > -1 
\end{equation}
%
\begin{equation}
\arctan x - \arctan y = \pm \pi + \arctan \frac{x-y}{1+xy}, 
\left\{ \begin{array}{c} x > 0 \\ x < 0  \end{array} \right\}, xy < -1 
\end{equation}

\section{Визначення символу Кронакера, через інтеграли}

В роботі часто застосовується визначення символу Кронакера через 
інтеграл на комплексній площині над експонентою з уявним показником.
Тут зібрані деякі не табличні інтеграли, що застосовані в роботі. 

\begin{equation} \begin{aligned} \label{eq:int_exp0}
\int_{0}^{2\pi} e^{\pm i (m-n) \varphi} d \varphi = 2 \pi \delta_{m,n} 
\end{aligned} \end{equation}
%
\begin{equation} \begin{aligned} \label{eq:int_exp1}
\int \limits_{0}^{2\pi} d \varphi \sin \varphi 
\left( \cos m \varphi - i \sin m \varphi \right) = 
i \pi \left( \delta_{m,-1} - \delta_{m,1} \right)
\end{aligned} \end{equation}
%
\textcolor{blue}{ \begin{equation*} \begin{aligned}
\int_{0}^{2\pi} d \varphi \sin \varphi 
\left( \cos m \varphi - i \sin m \varphi \right) = \int_{0}^{2\pi} d \varphi
\left( \sin \varphi \cos m \varphi - i \sin \varphi \sin m \varphi \right) = \\
= \frac{1}{2} \int_{0}^{2\pi} d \varphi \left( \sin (\varphi + m \varphi) + 
\sin (\varphi - m \varphi) - i \cos (\varphi - m \varphi) + 
i \cos (\varphi + m \varphi) \right) = \\
= \frac{i}{2} \int_{0}^{2\pi} d \varphi \left( -i \sin (\varphi + m \varphi) -
i \sin (\varphi - m \varphi) - \cos (\varphi - m \varphi) + 
\cos (\varphi + m \varphi) \right) = \\
= \frac{i}{2} \int_{0}^{2\pi} d \varphi \left( e^{-i (\varphi + m \varphi)} - 
e^{i (\varphi - m \varphi)} \right) = 
i \pi \left( \delta_{m,-1} - \delta_{m,1} \right)
\end{aligned} \end{equation*} }
%
\begin{equation} \begin{aligned} \label{eq:int_exp2}
\int \limits_{0}^{2\pi} d \varphi \cos \varphi 
( \cos m \varphi - i \sin m \varphi) = \pi ( \delta_{m,-1} + \delta_{m,1} )
\end{aligned} \end{equation}
%
\textcolor{blue}{ \begin{equation*} \begin{aligned}
\int_{0}^{2\pi} d \varphi \cos \varphi 
\left( \cos m \varphi - i \sin m \varphi \right) = \int_{0}^{2\pi} d \varphi
\left( \cos \varphi \cos m \varphi - i \cos \varphi \sin m \varphi \right) = \\
= \frac{1}{2} \int_{0}^{2\pi} d \varphi \left( 
\cos (\varphi + m \varphi) + \cos (\varphi - m \varphi) - 
i \sin (m \varphi + \varphi) - i \sin (m \varphi - \varphi) \right) = \\
= \frac{1}{2} \int_{0}^{2\pi} d \varphi \left( 
\cos (\varphi + m \varphi) + \cos (\varphi - m \varphi) - 
i \sin (m \varphi + \varphi) + i \sin (\varphi - m \varphi) \right) = \\
= \frac{1}{2} \int_{0}^{2\pi} d \varphi 
\left( e^{-i (1 + m) \varphi} - e^{i (1 - m) \varphi} \right) = 
\pi \left( \delta_{m,-1} + \delta_{m,1} \right)
\end{aligned} \end{equation*} }
%
\begin{equation} \begin{aligned} \label{eq:int_exp3}
\int_0^{2\pi} e^{-i m \varphi} \cos \varphi \sin^2 \varphi d \varphi = 
\frac{\pi \delta_{m,1} }{4} + \frac{\pi \delta_{m,-1} }{4} - 
\frac{\pi \delta_{m,-3} }{4} - \frac{\pi \delta_{m,3} }{4}
\end{aligned} \end{equation}
%
\textcolor{blue}{ \begin{equation*} \begin{aligned}
e^{-i m \varphi} \cos \varphi \sin^2 \varphi = e^{-i m \varphi} 
\frac{e^{i\varphi} + e^{-i\varphi}}{2} \frac{1 - \cos 2\varphi}{2} = \\
\frac{2e^{-i(m-1)\varphi} + 2e^{-i(m+1)\varphi}}{8} - 
\frac{e^{2i\varphi} + e^{-2i\varphi}}{2} 
\frac{2e^{-i(m-1)\varphi} + 2e^{-i(m+1)\varphi}}{8} = \\
\frac{2e^{-i(m-1)\varphi} + 2e^{-i(m+1)\varphi}}{8} - 
\frac{e^{-i(m-3)\varphi} + e^{-i(m+1)\varphi} + 
e^{-i(m-1)\varphi} + e^{-i(m+3)\varphi}}{8} = \\
= \frac{e^{-i(m-1)\varphi}}{8} + \frac{e^{-i(m+1)\varphi}}{8} -
\frac{e^{-i(m-3)\varphi}}{8} - \frac{e^{-i(m+3)\varphi}}{8}
\end{aligned} \end{equation*} }
%
\begin{equation} \begin{aligned} \label{eq:int_exp4}
\int_{0}^{2\pi} e^{-i m \varphi} \sin^3 \varphi d \varphi = 
\frac{3 \pi i}{4} \delta_{m,-1} - \frac{3 \pi i}{4} \delta_{m,1} - 
\frac{\pi i}{4} \delta_{m,-3} + \frac{\pi i}{4} \delta_{m,3}
\end{aligned} \end{equation}
%
\textcolor{blue}{ \begin{equation*} \begin{aligned}
e^{-i m \varphi} \sin^3 \varphi = e^{-i m \varphi} 
\frac{1 - \cos 2\varphi}{2} \frac{e^{i\varphi} - e^{-i\varphi}}{2i} = \\
= \frac{e^{-i(m-1)\varphi} - e^{-i(m+1)\varphi}}{4i} - 
\frac{e^{-i(m-3)\varphi} + e^{-i(m+1)\varphi} -
e^{-i(m-1)\varphi} - e^{-i(m+3)\varphi}}{8i} = \\
= \frac{3 e^{-i(m-1)\varphi}}{8i} - \frac{3 e^{-i(m+1)\varphi}}{8i} - 
\frac{e^{-i(m-3)\varphi}}{8i} + \frac{e^{-i(m+3)\varphi}}{8i} = \\
= \frac{3i e^{-i(m+1)\varphi}}{8} - \frac{3i e^{-i(m-1)\varphi}}{8} + 
\frac{i e^{-i(m-3)\varphi}}{8} - \frac{i e^{-i(m+3)\varphi}}{8} 
\end{aligned} \end{equation*} }
%
\textcolor{blue}{ \begin{equation*} \begin{aligned}
\int_{0}^{2\pi} e^{-i m \varphi} \sin^3 \varphi d \varphi = 
\frac{i\pi}{4} \left( 3 \delta_{m,-1} - 3 \delta_{m,1} + 
\delta_{m,3} - \delta_{m,-3} \right)
\end{aligned} \end{equation*} }
%
\begin{equation} \begin{aligned} \label{eq:int_exp5}
\int_0^{2\pi} e^{-i m \varphi} \sin \varphi \cos^2 \varphi d \varphi = 
\frac{\pi i }{4} \delta_{m,-1} - \frac{\pi i }{4} \delta_{m,1} -
\frac{\pi i }{4} \delta_{m,3} + \frac{\pi i }{4} \delta_{m,-3}
\end{aligned} \end{equation}
%
\textcolor{blue}{ \begin{equation*} \begin{aligned}
e^{-i m \varphi} \sin \varphi \cos^2 \varphi = 
\cos^2 \varphi e^{-i m \varphi} \frac{e^{i\varphi} - e^{-i\varphi}}{2i} = \\
= \frac{1 + \cos 2\varphi}{2} 
\frac{e^{i(1-m)\varphi} - e^{-i(1+m)\varphi}}{2i} = \\
= \frac{e^{i(1-m)\varphi} - e^{-i(1+m)\varphi}}{4i} + 
\frac{e^{2i\varphi} + e^{-2i\varphi}}{2} 
\frac{e^{i(1-m)\varphi} - e^{-i(1+m)\varphi}}{4i} = \\
\frac{ 2 e^{i(1-m)\varphi} - 2 e^{-i(1+m)\varphi}}{8i} +
\frac{e^{i(3-m)\varphi} + e^{-i(1+m)\varphi} - 
e^{-i(m-1)\varphi} - e^{-i(3+m)\varphi}}{8i} = \\
= -\frac{ i e^{-i (m-1) \varphi} }{8} + \frac{ i e^{-i (m+1) \varphi} }{8} -
\frac{ i e^{-i (m-3) \varphi} }{8} + \frac{ i e^{-i (m+3) \varphi} }{8}
\end{aligned} \end{equation*} }
%
\begin{equation} \begin{aligned} \label{eq:int_exp6}
\int_{0}^{2\pi} e^{-i m \varphi} \cos^3 \varphi d \varphi = 
\frac{\pi}{4} \delta_{m,-3} + \frac{\pi}{4} \delta_{m,3} + 
\frac{3 \pi}{4} \delta_{m,-1} + \frac{3 \pi}{4} \delta_{m,1}
\end{aligned} \end{equation}
%
\textcolor{blue}{ \begin{equation*} \begin{aligned}
e^{-i m \varphi} \cos^3 \varphi = 
\cos^2 \varphi e^{-i m \varphi} \frac{e^{i \varphi} + e^{-i \varphi}}{2} =
\frac{\cos^2 \varphi}{2} 
\left( e^{-i (1+m) \varphi} + e^{i (1-m) \varphi} \right) = \\
= \frac{ 1 + \cos 2 \varphi } { 4 } 
\left( e^{-i (1+m) \varphi} + e^{i (1-m) \varphi} \right) = \\
= \frac{e^{-i(1+m) \varphi} + e^{i(1-m) \varphi}}{4} + 
\frac{e^{-i(1+m) \varphi} + e^{i(1-m) \varphi}}{4}
\frac{e^{2i\varphi} + e^{-2i\varphi}}{2} = \\
= \frac{e^{-i(1+m) \varphi} + e^{i(1-m) \varphi}}{4} +
\frac{ e^{i(1-m) \varphi} + e^{-i(3+m) \varphi} + 
e^{i(3-m) \varphi} + e^{-i(1+m) \varphi} }{8} = \\
= \frac{3 e^{i(1-m) \varphi}}{8} + \frac{e^{-i(3+m) \varphi}}{8} +
\frac{e^{i(3-m) \varphi}}{8} + \frac{ 3 e^{-i(1+m) \varphi} }{8}
\end{aligned} \end{equation*} }
%
\textcolor{blue}{ \begin{equation*} \begin{aligned}
\int_{0}^{2\pi} d \varphi \left( \frac{3 e^{i(1-m) \varphi}}{8} + 
\frac{e^{-i(3+m) \varphi}}{8} + \frac{e^{i(3-m) \varphi}}{8} + 
\frac{ 3 e^{-i(1+m) \varphi} }{8} \right) = \\
= \frac{\pi}{4} \delta_{m,-3} + \frac{\pi}{4} \delta_{m,3} + 
\frac{3 \pi}{4} \delta_{m,-1} + \frac{3 \pi}{4} \delta_{m,1}
\end{aligned} \end{equation*} }

\include{vector}
\chapter{Інтеграли від циліндричної функції Бесселя першого роду}
\label{ch:bessel}

Циліндрична функція Бесселя першого роду -- базисна функція методу еволюційних 
рівнянь. Її властивості широко застосовуються в багатьох дослідженнях присвячених
нестаціонарним сигналам та процесам. Тут зібрано основні властивості 
функції Бесселя, використані в роботі, а також продемонстровано спосіб отримання 
аналітичного розв'язку для деяких не табличних інтегралів з
ядром у вигляді добутку декількох функцій Бесселя.

%%%%%%%%%%%%%%%%%%%%%%%%%%%%%%%%%%%%%%%%%%%%%%%%%%%%%%%%%%%%%%%%%%%%%%%%%%%%%%%
% Визначення на лінійні властивості
\begin{equation}
J_{-n} \left( z \right) = \left( -1 \right)^n J_n \left( z \right)
\end{equation}
%
\begin{equation} \label{eq:bessel_order_change}
J_{n+1} \left( z \right) + J_{n-1} \left( z \right) = 
\frac{2n}{z} J_n \left( z \right)
\end{equation}
% Асимптотичні властивості
\begin{equation} \label{eq:limJ1toZ}
\lim_{z \to 0} \left. \frac{J_1 \left( z \right)}{z} \right. = \frac{1}{2}
\end{equation}
% Інтегродиференціальні властивості
%\textcolor{blue}{
%\begin{equation*}
%2 \derivat{}{z} J_n \left( z \right) = 
%J_{n-1} \left( z \right) - J_{n+1} \left( z \right) 
%\end{equation*}
%%
%\begin{equation*}
%\derivat{}{z} J_n \left( z \right) = 
%J_{n-1} \left( z \right) - \frac{n}{z} J_{n} \left( z \right) 
%\end{equation*}
%%
%\begin{equation*}
%\derivat{}{z} J_n \left( z \right) = 
%\frac{n}{z} J_{n} \left( z \right) - J_{n+1} \left( z \right) 
%\end{equation*}
%%
%\begin{equation*}
%\derivat{}{z} \frac{ J_n \left( z \right) }{ z^n }  = 
%- \frac{ J_{n+1} \left( z \right) }{ z^n }
%\end{equation*}
%%
%\begin{equation*}
%\derivat{}{z} \left( z^n J_n \left( z \right) \right)  = 
%z^n J_{n-1} \left( z \right)
%\end{equation*}
%}

%%%%%%%%%%%%%%%%%%%%%%%%%%%%%%%%%%%%%%%%%%%%%%%%%%%%%%%%%%%%%%%%%%%%%%%%%%%%%%%
\section{Отримання інтегралу $ I_1 $ в явному виді} \label{sec:i1anal}
%

Отриманий інтеграл є азимутально-симетричним амплітудним 
коефіцієнтом для компонентів векторів напруженості поля, 
породженого антенами імпульсного випромінювання. Очевидно,
що значення інтегралу -- функція часу, лінійних просторових 
координат та радіусу апертури.

\begin{equation} \label{eq:int1start}
I_1 = R \int\limits_{0}^{\infty} \frac{d\nu}{\rho \nu} 
J_1 \left( \nu R \right) J_1 \left( \nu \rho \right) 
J_0 \left( \nu \sqrt{c^2 t^2 - z^2} \right)
\end{equation}

Спробуємо знайти аналітичне значення виразу \eqref{eq:int1start}, 
притупивши, що він сходиться. Інтеграли такого виду 
зустрічаються в \cite[ст. 398]{imp:Watson1922}.

\begin{equation} \begin{aligned} \label{eq:intJJJtable}
\int\limits_{0}^{\infty} \frac{d t}{t^{\lambda + \nu}} 
J_\mu \left( at \right) J_\nu \left( bt \right) J_\nu \left( ct \right) =
\frac{ \left( bc/2 \right) ^\nu }
{ \Gamma \left( \nu + 1/2 \right) \Gamma \left( 1/2 \right) } \cdot \\
\cdot \int\limits_{0}^{\infty} \int\limits_{0}^{\pi}
\frac{J_\mu \left( at \right) J_\nu \left( \omega t \right)}
{\omega^\nu t^\lambda} \sin^{2\nu}{\phi} d\phi dt, \\
\omega = \sqrt{b^2 + c^2 - 2bc \cos \phi} \\
\Re \left( \nu \right) > - \frac{1}{2};
\Re \left( \mu + \nu + 2 \right) > \Re \left( \lambda + 1 \right) > 0
\end{aligned} \end{equation}

%\textcolor{blue}{ \begin{equation*} \begin{aligned}
%a = \sqrt{c^2 t^2 - z^2}; b = R; c = \rho; \lambda = 0 \\
%\nu = 1; \mu = 0; \omega = \sqrt{R^2 + \rho^2 - 2 \rho R \cos \phi} \\
%\int\limits_{0}^{\infty} \frac{d\nu}{\nu} 
%J_1 \left( \nu R \right) J_1 \left( \nu \rho \right) 
%J_0 \left( \nu \sqrt{c^2 t^2 - z^2} \right) = 
%\frac{R^2}{ 2 \Gamma \left( 3/2 \right) \Gamma \left( 1/2 \right) } \cdot \\
%\int\limits_{0}^{\pi} 
%\frac{\sin^2{\phi}}{\sqrt{R^2 + \rho^2 - 2 \rho R \cos \phi}}
%\int\limits_{0}^{\infty} d \nu J_1 \left( \nu \omega \right) 
%J_0 \left( \nu \sqrt{c^2 t^2 - z^2} \right) d \phi
%\end{aligned} \end{equation*} }
%
%\textcolor{blue}{ \begin{equation*} \begin{aligned}
%\Gamma \left( 3/2 \right) \Gamma \left( 1/2 \right) = 
%\frac{\sqrt{\pi}}{2} \cdot \sqrt{\pi} = \frac{\pi}{2} 
%\end{aligned} \end{equation*} }
%
%\textcolor{blue}{ \begin{equation*} \begin{aligned}
%I_1 = \frac{R^2}{\pi} \int\limits_{0}^{\pi} 
%\frac{\sin^2{\phi}}{\sqrt{R^2 + \rho^2 - 2 \rho R \cos \phi}}
%\int\limits_{0}^{\infty} d \nu J_1 \left( \nu \omega \right) 
%J_0 \left( \nu \sqrt{c^2 t^2 - z^2} \right) d \phi
%\end{aligned} \end{equation*} }

Використання формули \eqref{eq:intJJJtable} дозволяє спростити $ I_1 $ до 
інтегралу по двом функціям Бесселя в ядрі замість трьох. Використаємо наступну 
формулу з \cite{imp:Golubovic2013} для пошуку рішення нового інтегралу. 

\begin{equation} \begin{aligned} \label{eq:intJJtable}
\int\limits_{0}^{\infty} d \nu
J_n \left( a \nu \right) J_{n-1} \left( b \nu \right) = \begin{cases} 
b^{n-1} / a^n , 0 < b < a \\
1 / 2 b , 0 < a = b \\
0 , 0 < a < b
\end{cases} 
\end{aligned} \end{equation}

%\textcolor{blue}{ \begin{equation*} \begin{aligned}
%\int\limits_{0}^{\infty} d \nu J_1 \left( \nu \omega \right) 
%J_0 \left( \nu \sqrt{c^2 t^2 - z^2} \right) = \begin{cases}
%\left( R^2 + \rho^2 - 2 \rho R \cos \phi \right)^{-1/2}, 0 < b < a \\
%\frac{1}{2} \left( c^2 t^2 - z^2 \right)^{-1/2}, 0 < a = b \\
%0 , 0 < a < b
%\end{cases} 
%\end{aligned} \end{equation*} }
%
%\textcolor{blue}{ \begin{equation*} \begin{aligned}
%\sqrt{R^2 + \rho^2 - 2 \rho R \cos \phi} > \sqrt{c^2 t^2 - z^2} \\
%R^2 + \rho^2 - 2 \rho R \cos \phi > c^2 t^2 - z^2 \\
%\cos \phi < \frac{R^2 + \rho^2}{2 \rho R} - \frac{c^2 t^2 - z^2}{2 \rho R} \\
%\phi > \arccos \left( \frac{\rho^2 + R^2}{2 \rho R} - 
%\frac{c^2 t^2 - z^2}{2 \rho R} \right), 0 \leq \phi \leq \pi
%\end{aligned} \end{equation*} }
%
%\textcolor{blue}{ \begin{equation*}
%\phi > \arccos \left( \frac{\rho^2 + R^2}{2 \rho R} - 
%\frac{c^2 t^2 - z^2}{2 \rho R} \right)
%\end{equation*} }
%
%\textcolor{blue}{ \begin{equation*} \begin{aligned}
%\begin{cases}
%\frac{\rho^2 + R^2}{2 \rho R} - \frac{c^2 t^2 - z^2}{2 \rho R} \leq 1 \\
%\frac{\rho^2 + R^2}{2 \rho R} - \frac{c^2 t^2 - z^2}{2 \rho R} \geq - 1
%\end{cases}
%\begin{cases}
%\rho^2 + R^2 - c^2 t^2 + z^2 \leq 2 \rho R \\
%\rho^2 + R^2 - c^2 t^2 + z^2 \geq - 2 \rho R
%\end{cases}
%\end{aligned} \end{equation*} }
%
%\textcolor{blue}{ \begin{equation*} \begin{aligned}
%\begin{cases}
%0 \leq \left( R - \rho \right)^2 \leq c^2 t^2 - z^2 \\ 
%\left( \rho + R \right)^2 \geq c^2 t^2 - z^2 \geq 0
%\end{cases}
%\begin{cases}
%0 \leq R \leq \rho + \sqrt{c^2 t^2 - z^2} \\
%R \geq \left| \rho - \sqrt{c^2 t^2 - z^2} \right| \geq 0
%\end{cases}
%\begin{cases}
%R \leq f_+(r,t) \\
%R \geq \left| f_-(r,t) \right|
%\end{cases} 
%\end{aligned} \end{equation*} }

\begin{equation*} \begin{aligned}
I_1 \in \begin{cases}
S_1: \{ 0 \leq \phi \leq \psi \}, 0 < R < 
\left| \rho - \sqrt{c^2 t^2 - z^2} \right| \\
S_2: \{ \psi \leq \phi \leq \pi \}, \left| \rho - \sqrt{c^2 t^2 - z^2} \right| \leq 
R \leq \rho + \sqrt{c^2 t^2 - z^2} \\
S_3: \{ 0 \leq \phi \leq \pi \}, R > \rho + \sqrt{c^2 t^2 - z^2}
\end{cases} 
\end{aligned} \end{equation*}

\begin{equation*} \begin{aligned}
I_1 \{ S_1 \} = 0
\end{aligned} \end{equation*}

%\textcolor{blue}{ \begin{equation*} \begin{aligned}
%I_1 = \frac{R^2}{\pi} \int\limits_{0}^{\pi} 
%\frac{\sin^2{\phi}}{\sqrt{R^2 + \rho^2 - 2 \rho R \cos \phi}}
%\int\limits_{0}^{\infty} d \nu J_1 \left( \nu \omega \right) 
%J_0 \left( \nu \sqrt{c^2 t^2 - z^2} \right) d \phi = \\
%= \frac{R^2}{\pi} \int_{\psi}^{\pi}
%\frac{\sin^2{\phi}}{\sqrt{R^2 + \rho^2 - 2 \rho R \cos \phi}}
%\frac{1}{\sqrt{R^2 + \rho^2 - 2 \rho R \cos \phi}} d \phi = \\
%= \frac{R^2}{\pi} \int_{\psi}^{\pi}
%\frac{\sin^2{\phi}}{R^2 + \rho^2 - 2 \rho R \cos \phi} d \phi = 
%\frac{1}{\pi} \int_{\psi}^{\pi}
%\frac{\sin^2{\phi}}{1 + \frac{\rho^2}{R^2} - \frac{2 \rho}{R} \cos \phi} d \phi
%\end{aligned} \end{equation*} }

Згідно властивістю адитивності при розбиттях для інтегралів Рімана, 
значення інтегралу в одній точці не впливає на значення інтегралу у 
визначених межах, а отже:

\begin{equation*} \begin{aligned}
I_1 = \frac{1}{\pi} \int_{\psi}^{\pi}
\frac{\sin^2{\phi}}{1 + \frac{\rho^2}{R^2} - 
\frac{2 \rho}{R} \cos \phi} d \phi \\
\psi = \arccos \left( \frac{\rho^2 + R^2}{2 \rho R} - 
\frac{c^2 t^2 - z^2}{2 \rho R} \right)
\end{aligned} \end{equation*}

%\textcolor{blue}{ \begin{equation*} \begin{aligned}
%\int \frac{\sin^2{\phi}}{a + b \cos \phi} d \phi = 
%\int \frac{1 - \cos^2{\phi}}{a + b \cos \phi} d \phi = 
%\int\frac{d \phi}{a + b \cos \phi}  -
%\int \frac{\cos^2{\phi}}{a + b \cos \phi} d \phi = \\
%= \int \frac{d \phi}{a + b \cos \phi}  - 
%\int \frac{\cos^2{\phi}}{a + b \cos \phi} d \phi -
%\frac{a}{b} \int \frac{\cos \phi}{a + b \cos \phi} d \phi + \\
%+ \frac{a}{b} \int \frac{\cos \phi}{a + b \cos \phi} d \phi = 
%\int \frac{d \phi}{a + b \cos \phi} +
%\frac{a}{b} \int \frac{\cos \phi}{a + b \cos \phi} d \phi - \\
%- \int \frac{\cos^2{\phi} + \frac{a}{b} \cos \phi} {a + b \cos \phi} d \phi =
%\int \frac{d \phi}{a + b \cos \phi} + 
%\frac{a}{b} \int\limits_{0}^{\psi} \frac{\cos \phi}{a + b \cos \phi} d \phi -
%\end{aligned} \end{equation*} }
%
%\textcolor{blue}{ \begin{equation*} \begin{aligned}
%- \frac{1}{b} \int \frac{\cos \phi + a/b} {a/b +  \cos \phi} \cos \phi d \phi = 
%\int \frac{d \phi}{a + b \cos \phi} + 
%\frac{a}{b} \int \frac{\cos \phi}{a + b \cos \phi} d \phi - \\
%- \frac{1}{b} \int \cos \phi d \phi = \int \frac{d \phi}{a + b \cos \phi} - 
%\frac{1}{b} \int \cos \phi d \phi + \frac{a}{b^2} \int
%\frac{a - a + b \cos \phi}{a + b \cos \phi} d \phi = \\ 
%= \int\frac{d \phi}{a + b \cos \phi} - \frac{1}{b} \int \cos \phi d \phi +
%\frac{a}{b^2} \int \frac{a + b \cos \phi}{a + b \cos \phi} d \phi - \\ 
%- \frac{a^2}{b^2} \int \frac{d \phi}{a + b \cos \phi} = 
%\left( 1 - \frac{a^2}{b^2} \right) \int\frac{d \phi}{a + b \cos \phi} - 
%\frac{1}{b} \int \cos \phi d \phi + \frac{a}{b^2} \int d \phi
%\end{aligned} \end{equation*} }
%
%\textcolor{blue}{ \begin{equation*} \begin{aligned}
%\int_{\psi}^{\pi} \frac{\sin^2{\phi}}{a + b \cos \phi} d \phi =  
%\left( 1 - \frac{a^2}{b^2} \right)
%\int_{\psi}^{\pi} \frac{d \phi}{a + b \cos \phi} -
%\frac{\sin \pi - \sin \psi}{b} + \frac{a}{b^2} (\pi - \psi) = \\
%= \left( 1 - \frac{a^2}{b^2} \right)
%\int_{\psi}^{\pi} \frac{d \phi}{a + b \cos \phi} +
%\frac{\sin \psi}{b} + \frac{a}{b^2} (\pi - \psi)
%\end{aligned} \end{equation*} }
%
%\textcolor{blue}{ \begin{equation*} \begin{aligned}
%a = 1 + \frac{\rho^2}{R^2}; b = - \frac{2 \rho}{ R } \\
%\frac{\pi R}{\rho} I_1 = \left( 1 - \frac{a^2}{b^2} \right)
%\int_{\psi}^{\pi} \frac{d \phi}{a + b \cos \phi} +
%\frac{\sin \psi}{b} + \frac{a}{b^2} (\pi - \psi) = \\
%= \left( 1 - \left( \frac{1 + \frac{\rho^2}{R^2}} 
%{ \frac{2 \rho}{R} } \right)^2 \right) 
%\int_{\psi}^{\pi} \frac{d \phi}{1 + \frac{\rho^2}{R^2} -  
%\frac{2 \rho}{R} \cos \phi} -
%\frac{\sin \psi}{\frac{2 \rho}{ R }} + \left( 1 + \frac{\rho^2}{R^2} \right) 
%\frac{\pi - \psi}{\frac{4 \rho^2}{R^2}} = \\
%\left( R^2 - \left( \frac{R^2 + \rho^2}{2 \rho} \right)^2 \right) 
%\int_{\psi}^{\pi} \frac{d \phi}{R^2 + \rho^2 - 2 \rho R \cos \phi} -
%\frac{R}{2 \rho} \sin \psi + \frac{\rho^2 + R^2}{4 \rho^2} (\pi - \psi)  
%\end{aligned} \end{equation*} }
%
%\textcolor{blue}{ \begin{equation*} \begin{aligned}
%\frac{4 \rho^2}{4 \rho^2} R^2 - \left( \frac{R^2 + \rho^2}{2 \rho} \right)^2 =
%\frac{4 \rho^2 R^2 - R^4 - 2 \rho^2 R^2 - \rho^4}{4 \rho^2} =
%- \frac{\left( \rho^2 - R^2 \right)^2}{4 \rho^2} 
%\end{aligned} \end{equation*} }
%
%\textcolor{blue}{ \begin{equation*} \begin{aligned}
%\pi I_1 (S_2) = - \frac{\left( \rho^2 - R^2 \right)^2}{4 \rho^2} 
%\int_{\psi}^{\pi} \frac{d \phi}{R^2 + \rho^2 - 2 \rho R \cos \phi} - \\
%- \frac{R}{2 \rho} \sin \psi + \frac{\rho^2 + R^2}{4 \rho^2}  (\pi - \psi)
%\end{aligned} \end{equation*} }

Тригонометричними перетвореннями зведемо поточний вид $ I_1 $ до табличного 
інтегралу.

\begin{equation*} \begin{aligned}
I_{1} \{ S_2 \} = - \frac{\left( \rho^2 - R^2 \right)^2}{4 \pi \rho^2} 
\int_{\psi}^{\pi} \frac{d \phi}{R^2 + \rho^2 - 2 \rho R \cos \phi} - 
\frac{R}{\rho} \frac{\sin \psi}{2 \pi} +  
\frac{\rho^2 + R^2}{4 \rho^2} \frac{\pi - \psi}{\pi}
\end{aligned} \end{equation*}

%\textcolor{blue}{ \begin{equation*} \begin{aligned}
%\int_{0}^{\pi} \frac{\sin^2{\phi}}{a + b \cos \phi} d \phi =  
%\left( 1 - \frac{a^2}{b^2} \right)
%\int_{0}^{\pi} \frac{d \phi}{a + b \cos \phi} -
%\frac{\sin \pi - \sin 0}{b} + \frac{a}{b^2} (\pi - 0) = \\
%= \left( 1 - \frac{a^2}{b^2} \right)
%\int_{0}^{\pi} \frac{d \phi}{a + b \cos \phi} +
%\frac{a}{b^2} \pi
%\end{aligned} \end{equation*} }

\begin{equation*} \begin{aligned}
I_{1} \{ S_3 \} = - \frac{\left( \rho^2 - R^2 \right)^2}{4 \pi \rho^2} 
\int_{0}^{\pi} \frac{d \phi}{R^2 + \rho^2 - 2 \rho R \cos \phi} + 
\frac{\rho^2 + R^2}{4 \rho^2}
\end{aligned} \end{equation*}

Таблична формула для неозначеного випадку інтегралу може буде знайдена в 
\cite[ст. 181]{imp:ElementFunc1983}.

\begin{equation} \label{eq:caseTableIntegral}
\int \frac{d x}{a + b \cos{x}} = \begin{cases}
\frac{2}{\sqrt{a^2-b^2}} \arctan \frac{\sqrt{a^2-b^2} \tan \frac{x}{2}}
{a + b}, a^2 > b^2 \\
\frac{1}{\sqrt{b^2-a^2}} \ln 
\frac{\sqrt{b^2-a^2} \tan \frac{x}{2} + a + b}
{\sqrt{b^2-a^2} \tan \frac{x}{2} - a - b}, a^2 < b^2
\end{cases}
\end{equation}

Помітимо, що випадок $ a^2 > b^2 $ відповідає області $ \rho > R $, a 
$ a^2 < b^2 $, навпаки, для прожекторної зони випромінювання. Як 
згадувалось раніше, аналогічна методика для отримання перехідної функції 
кругової апертури застосовувалась в дисертаційному дослідженні Думіна О.М.. 
На відміну від цього дослідження, здобувачем розглядається випадок не лише 
для$ \rho > R $, а і для $ \rho < R $, тобто для прожекторної зони, де 
прояв нелінійної природи поширення електромагнітних хвиль найбільший. 
Цікавість до області $ \rho < R $ також викликана тим, що напрямлені 
антени імпульсного випромінювання на практиці найчастіше використовуються 
саме за сценарієм, коли приймач чи випромінювач знаходяться в прожекторній 
зоні.
%
%\textcolor{blue}{ \begin{equation*} \begin{aligned}
%a^2 > b^2  \Rightarrow  
%\left( R^2 + \rho^2 \right)^2 > 4 \rho^2 R^2 \\
%R^4 + 2 \rho^2 R^2 + \rho^4 - 4 \rho^2 R^2 > 0 \Rightarrow 
%\left( \rho^2 - R^2 \right)^2 > 0
%\end{aligned} \end{equation*} }
%
%\textcolor{blue}{ Далі знадобиться: }
%
%\textcolor{blue}{ \begin{equation*} \begin{aligned}
%a^2 - b^2 = - \left( b^2 - a^2 \right) = 
%R^4 + 2 \rho^2 R^2 + \rho^4 - 4 \rho^2 R^2 = \left( \rho^2 - R^2 \right)^2 \\
%\lim_{\alpha \to 0} \tan{\alpha} = 0 \Rightarrow
%\lim_{\alpha \to 0} \arctan \left( a \tan{\alpha} \right) = 0 \\ 
%\lim_{\alpha \to \pi/2} \tan{\alpha} = \infty \Rightarrow
%\lim_{\alpha \to \pi/2} \arctan \left( a \tan{\alpha} \right) = \frac{\pi}{2}
%\end{aligned} \end{equation*} }
%
%\textcolor{blue}{ \begin{equation*} \begin{aligned}
%\int_{\psi}^{\pi} \frac{d \phi}{R^2 + \rho^2 - 2 \rho R \cos \phi} =
%\left. \frac{2}{ |\rho^2 - R^2| } \arctan \left( \frac{ |\rho^2 - R^2| }
%{\left( \rho - R \right)^2} \tan \frac{\phi}{2} \right)
%\right|_{\psi}^{\pi} = \\ = \frac{2}{ |\rho^2 - R^2| } \left.
%\arctan \left( \frac{\rho + R}{ |\rho - R| } \tan \frac{\phi}{2} \right)
%\right|_{\psi}^{\pi} = \\ = \frac{2}{ |\rho^2 - R^2| } \left( \frac{\pi}{2} -
%\arctan \left( \frac{\rho + R}{ |\rho - R| } \tan \frac{\psi}{2} \right) \right)
%\end{aligned} \end{equation*} }

\begin{equation*} \begin{aligned}
I_1 \{ S_2 \} = \frac{ | \rho^2 - R^2 | }{2 \pi \rho^2} \left(
\arctan \left( \frac{\rho + R}{ | \rho - R | } \tan \frac{\psi}{2} \right) -  
\frac{\pi}{2} \right) - \frac{R}{\rho} \frac{\sin \psi}{2 \pi} + 
\frac{\rho^2 + R^2}{4 \rho^2} \frac{\pi - \psi}{\pi}
\end{aligned} \end{equation*}
%
%\textcolor{blue}{ \begin{equation*} \begin{aligned}
%\int_{0}^{\pi} \frac{d \phi}{R^2 + \rho^2 - 2 \rho R \cos \phi} =
%\left. \frac{2}{ | \rho^2 - R^2 | } \arctan \left( \frac{ | \rho^2 - R^2 | }
%{\left( \rho - R \right)^2} \tan \frac{\phi}{2} \right)
%\right|_{0}^{\pi} = \\ = \frac{2}{ | \rho^2 - R^2 | } \left.
%\arctan \left( \frac{\rho + R}{ | \rho - R | } \tan \frac{\phi}{2} \right)
%\right|_{0}^{\pi} = \frac{2}{ | \rho^2 - R^2 | } \frac{\pi}{2}
%\end{aligned} \end{equation*} }

\begin{equation*} \begin{aligned}
I_1 \{ S_3 \} = \frac{\rho^2 + R^2}{4 \rho^2} - 
\frac{ |\rho^2 - R^2| }{4 \rho^2} = \begin{cases}
1/2 , \rho < R \\
R^2 / 2 \rho^2, \rho > R
\end{cases}
\end{aligned} \end{equation*}

Для того щоб отримати значення інтегралу на осі аплікат повернемось до 
початкового виду $ I_1 $ з \eqref{eq:int1start}. Користуючись асимптотичною 
властивістю функції Бесселя \eqref{eq:limJ1toZ} побачимо що інтеграл 
зведеться до випадку \eqref{eq:intJJtable}.
%
%\textcolor{blue} {\begin{equation*} \begin{aligned}
%\left. I_1 \right|_{\rho = 0} = R \int\limits_{0}^{\infty} d \nu
%J_1 \left( \nu R \right) \frac{J_1 \left( \nu \rho \right) }{\nu \rho}
%J_0 \left( \nu \sqrt{c^2 t^2 - z^2} \right) = \\
%= \frac{R}{2} \int\limits_{0}^{\infty} d \nu
%J_1 \left( \nu R \right) J_0 \left( \nu \sqrt{c^2 t^2 - z^2} \right) = 
%\left. \frac{I_2}{2} \right|_{\rho = 0}
%\end{aligned} \end{equation*} }
%
%\textcolor{blue} {\begin{equation*}
%\left. I_1 \right|_{\rho = 0} = \frac{1}{2} \begin{cases}
%0, 0 < R < \sqrt{c^2t^2 - z^2} \\
%\frac{R}{2} \left( c^2t^2 - z^2 \right)^{-1/2}, 0 < R = \sqrt{c^2t^2 - z^2} \\ 
%1, 0 < \sqrt{c^2t^2 - z^2} < R 
%\end{cases}
%\end{equation*} }

\begin{equation}
\left. I_1 \right|_{\rho = 0} = \frac{1}{2} \begin{cases}
0, 0 < R < \sqrt{c^2t^2 - z^2} \\
1/2, 0 < R = \sqrt{c^2t^2 - z^2} \\ 
1, 0 < \sqrt{c^2t^2 - z^2} < R 
\end{cases}
\end{equation}

На останок, спростимо тригонометричні вирази, що містять $ \psi $. Розглянемо 
$ \psi = \arccos f(r,t) $, де $ f(r,t) $ задовільна функція координат. 
Тоді $ f(r,t) = \cos \psi $. Зазначимо, що з означення відомо, що 
$ \psi \in \left[ 0, \pi \right] $, тому $ \sin \psi \geq 0 $. Таким чином:

\begin{equation*} \begin{aligned}
\sin \psi = \sqrt{1 - \cos^2 \psi } = \sqrt{1 - f^2(r,t)}
\end{aligned} \end{equation*}

Згадуючи введене означення для $ \psi $ зашипимо, що

%\textcolor{blue}{ \begin{equation*} \begin{aligned}
%\psi = \arccos \left( \frac{\rho^2 + R^2}{2 \rho R} - 
%\frac{c^2 t^2 - z^2}{2 \rho R} \right)
%\end{aligned} \end{equation*} }
%
%\textcolor{blue}{ \begin{equation*} \begin{aligned}
%\sin \psi = \sqrt{1 - \left( \frac{\rho^2 + R^2}{2 \rho R} - 
%\frac{c^2 t^2 - z^2}{2 \rho R} \right)^2} = 
%\sqrt{1 - \frac{\left( \rho^2 + R^2 - c^2 t^2 + z^2 \right)^2}
%{4 \rho^2 R^2} } = \\ = \sqrt{\frac{4 \rho^2 R^2}{4 \rho^2 R^2} - 
%\frac{\left( \rho^2 + R^2 - c^2 t^2 + z^2 \right)^2}{4 \rho^2 R^2} } =
%\sqrt{\frac{4 \rho^2 R^2 - \left( \rho^2 + R^2 - c^2 t^2 + z^2 \right)^2}
%{4 \rho^2 R^2}} = \\
%= \frac{1}{2 \rho R} \sqrt{4 \rho^2 R^2 - \left( \rho^2 + R^2 \right)^2 +
%2 \left( \rho^2 + R^2 \right) \left( c^2 t^2 - z^2 \right) - 
%\left( c^2 t^2 - z^2 \right)^2} = \\
%= \frac{1}{2 \rho R} \sqrt{- \left( \rho^2 - R^2 \right)^2 +
%2 \left( \rho^2 + R^2 \right) \left( c^2 t^2 - z^2 \right) - 
%\left( c^2 t^2 - z^2 \right)^2} = \\
%= \frac{c^2 t^2 - z^2}{2 \rho R} \sqrt{2 \frac{\rho^2 + R^2 }{c^2 t^2 - z^2} - 
%\left( \frac{\rho^2 - R^2 }{c^2 t^2 - z^2} \right)^2 - 1}
%\end{aligned} \end{equation*} }
%
%\textcolor{red}{ \begin{equation*} \begin{aligned}
%4 \rho^2 R^2 - (\rho^2 + R^2 - c^2 t^2 + z^2)^2 = \\
%= 4 \rho^2 R^2 - (\rho^2 + R^2)^2 - (c^2 t^2 - z^2)^2 + 
%2 (\rho^2 + R^2) (c^2 t^2 - z^2) = \\
%= - (\rho^2 - R^2)^2 - (c^2 t^2 - z^2)^2 \pm 
%2 R^2 (c^2 t^2 - z^2) + 2 (\rho^2 + R^2) (c^2 t^2 - z^2) = \\
%= 4 R^2 (c^2 t^2 - z^2) - (\rho^2 - R^2)^2 - (c^2 t^2 - z^2)^2 +
%2 (\rho^2 - R^2) (c^2 t^2 - z^2) = ?
%\end{aligned} \end{equation*} }

\begin{equation*} \begin{aligned}
\frac{R}{\rho} \frac{\sin \psi}{2 \pi} = 
\frac{\sqrt{4 \rho^2 R^2 - (\rho^2 + R^2 - c^2t^2 + z^2)^2}}{4 \pi \rho^2}
\end{aligned} \end{equation*}

%\textcolor{blue}{ \begin{equation*} \begin{aligned}
%\tan \frac{\psi}{2} = \pm \sqrt{ \frac{1 - \cos \psi}{1 + \cos \psi} } = 
%\sqrt{ \frac{1- \frac{\rho^2 + R^2}{2 \rho R} + \frac{c^2 t^2 - z^2}{2 \rho R}}
%{1 + \frac{\rho^2 + R^2}{2 \rho R} - \frac{c^2 t^2 - z^2}{2 \rho R}} } =
%\sqrt{ \frac{c^2t^2 - z^2 - \left( \rho - R \right)^2}
%{\left( \rho + R \right)^2 - \left( c^2t^2 - z^2 \right)} }
%\end{aligned} \end{equation*} }
%
%\textcolor{blue}{ \begin{equation*} \begin{aligned}
%\frac{\rho + R}{ |\rho - R| } \tan \frac{\psi}{2} = 
%\sqrt{ \frac{ \frac{c^2t^2 - z^2}{\left( \rho - R \right)^2} - 1}
%{ 1 - \frac{c^2t^2 - z^2}{\left( \rho + R \right)^2} } } = 
%\sqrt{ \left( \frac{\rho + R}{\rho - R} \right)^2
%\frac{c^2t^2 - z^2 - \left( \rho - R \right)^2}
%{\left( \rho + R \right)^2 - \left( c^2t^2 - z^2 \right)} }
%\end{aligned} \end{equation*} }

\begin{equation*} \begin{aligned}
\arctan \left( \frac{\rho + R}{ |\rho - R| } \tan \frac{\psi}{2} \right) - 
\frac{\pi}{2} = - \arctan \sqrt{ \left( \frac{\rho - R}{\rho + R} \right)^2
\frac{\left( \rho + R \right)^2 - \left( c^2t^2 - z^2 \right)} 
{\left( c^2t^2 - z^2 \right) - \left( \rho - R \right)^2} }
\end{aligned} \end{equation*}

\begin{equation*} \begin{aligned}
\pi - \psi = \arccos \left( \frac{c^2 t^2 - z^2 - \rho^2 - R^2}{2 \rho R} \right)
\end{aligned} \end{equation*}

Користуючись такими спрощеннями, можемо записати вираз в явному вигляді для 
області $ S_2 $

\begin{equation*} \begin{aligned}
I_1 \{ S_2 \} = \frac{\rho^2 + R^2}{4 \pi \rho^2} \arccos 
\left( \frac{c^2 t^2 - z^2 - \rho^2 - R^2}{2 \rho R} \right) - \\
- \frac{\sqrt{4 \rho^2 R^2 - (\rho^2 + R^2 - c^2t^2 + z^2)^2}}{4 \pi \rho^2} - \\
- \frac{ |\rho^2 - R^2| }{2 \pi \rho^2} 
\arctan \sqrt{ \frac{(\rho - R)^2}{(\rho + R)^2} \cdot
\frac{\left( \rho + R \right)^2 - \left( c^2t^2 - z^2 \right)} 
{\left( c^2t^2 - z^2 \right) - \left( \rho - R \right)^2} }
\end{aligned} \end{equation*}

%%%%%%%%%%%%%%%%%%%%%%%%%%%%%%%%%%%%%%%%%%%%%%%%%%%%%%%%%%%%%%%%%%%%%%%%%%%%%%%
\section{Отримання інтегралу $ I_2 $ в явному виді} \label{sec:i2anal}

\begin{equation} \label{eq:int2start}
I_2 = R \int \limits_{0}^{\infty} d \nu J_1 \left( \nu R \right) 
J_0 \left( \nu \rho \right) J_0 \left( \nu \sqrt{c^2t^2 - z^2} \right)
\end{equation}

Це табличний інтеграл, що може бути знайдений в 
\cite[ст. 228]{imp:SpecFunc1983}.

\begin{equation} \begin{aligned} \label{eq:intJ0J0J1tabel}
\int \limits_{0}^{\infty} d x J_0 \left( ax \right) 
J_0 \left( bx \right) J_1 \left( cx \right) = \begin{cases}
0, 0 < c < | a - b | \\ 
1/c, c > a + b
\end{cases} a, b > 0 \\
= \frac{1}{\pi c} \arccos \frac{a^2 + b^2 - c^2}{2ab},
| a - b | < c < a + b; a,b > 0
\end{aligned} \end{equation}

Фізичні властивості змінних в \eqref{eq:int2start} відповідають умові 
$ a,b,c > 0 $. Запишемо значення інтегралу відносно інших умов.

\begin{equation}
I_2 = \begin{cases}
0, 0 < R < | f_{-} \left( r, t \right) | \\
\frac{1}{\pi} \arccos \frac{c^2t^2 - z^2 + \rho^2 - R^2}
{2 \rho \sqrt{c^2t^2 - z^2}}, | f_{-} \left( r, t \right) | < R < 
f_{+} \left( r, t \right) \\ 1, f_{+} \left( r, t \right) < R \\
\end{cases}
\end{equation}

Тут для спрощення введено наступні переозначення:

\begin{equation*} \begin{aligned}
f_{-} \left( r, t \right) = \rho - \sqrt{c^2t^2 - z^2} \\
f_{+} \left( r, t \right) = \rho + \sqrt{c^2t^2 - z^2}
\end{aligned} \end{equation*}

Якщо $ \rho = 0 $, то область визначення інтегралу $ I_2 $ по формулі 
\eqref{eq:int2start} схропується і інтеграл стає невизначеним. У цьому випадку 
розглянемо інтеграл за допомогою формули \eqref{eq:intJJtable}.

%\textcolor{blue}{ \begin{equation*}
%I_2 \left( \rho = 0 \right) = \begin{cases}
%0, 0 < R < \sqrt{c^2t^2 - z^2} \\
%\frac{R}{2} \left( c^2t^2 - z^2 \right)^{-1/2}, 0 < R = \sqrt{c^2t^2 - z^2} \\ 
%1, 0 < \sqrt{c^2t^2 - z^2} < R 
%\end{cases}
%\end{equation*} }

\begin{equation}
I_2 \left( \rho = 0 \right) = \begin{cases}
0, 0 < R < \sqrt{c^2t^2 - z^2} \\
1/2, 0 < R = \sqrt{c^2t^2 - z^2} \\ 
1, 0 < \sqrt{c^2t^2 - z^2} < R 
\end{cases}
\end{equation}

\chapter{Комплексі функції Ломмеля двох змінних}
\label{ch:lommel}

%%%%%%%%%%%%%%%%%%%%%%%%%%%%%%%%%%%%%%%%%%%%%%%%%%%%%%%%%%%%%%%%%%%%%%%%%%%%%%%
\section{Визначення та лінійні властивості}

В \cite{Boersma1961} приводиться визначення через функцію Бесселя.
%
\begin{equation}
U_n \left[ W, Z \right] = \sum \limits_{m = 0}^{\infty} (-1)^m
\left( \frac{W}{Z} \right)^{n + 2m} J_{n + 2m} (Z)
\end{equation}

В \cite{Boersma1961} також можна знайти наступну властивість.
%
\begin{equation}
U_n \left[ W, Z \right] + U_{n+2} \left[ W, Z \right] = 
\left( \frac{W}{Z} \right)^n J_n (Z)
\end{equation}
%
\textcolor{lightgray} { \begin{equation*} \begin{aligned}
W_\pm = \pm i (\nu ct - \nu z) \\
Z = \sqrt{\nu^2 c^2t^2 - \nu^2 z^2}
\end{aligned} \end{equation*} }
%
\textcolor{lightgray}{ \begin{equation*}
U_0 \left[ W, Z \right] = \sum \limits_{m = 0}^{\infty} (-1)^m
\left( \frac{W}{Z} \right)^{2m} J_{2m} (Z) = J_0 (Z) - \frac{W^2}{Z^2} J_2 (Z) +
\frac{W^4}{Z^4} J_4 (Z) - ...
\end{equation*} }
%
\textcolor{lightgray}{ \begin{equation*}
U_2 \left[ W, Z \right] = \sum \limits_{m = 0}^{\infty} (-1)^m
\left( \frac{W}{Z} \right)^{2 + 2m} J_{2 + 2m} (Z) = 
\frac{W^2}{Z^2} J_2 (Z) - \frac{W^4}{Z^4} J_4 (Z) + \frac{W^6}{Z^6} J_6 (Z) - ...
\end{equation*} }
%
\textcolor{lightgray}{ \begin{equation*} \begin{aligned}
U_0 [W, Z] - U_2 [W, Z] = J_0(Z) - 2 \left( \frac{W^2}{Z^2} J_2 (Z) - 
\frac{W^4}{Z^4} J_4 (Z) + \frac{W^6}{Z^6} J_6 (Z) - ... \right)
\end{aligned} \end{equation*} }
%
\textcolor{lightgray}{ \begin{equation*} \begin{aligned}
U_0 [W, Z] - U_2 [W, Z] = J_0(Z) - 2 U_2(W,Z)
\end{aligned} \end{equation*} }
%
\textcolor{lightgray}{ \begin{equation*} \begin{aligned}
\left( \frac{W}{Z} \right)^{2n} = \left( 
\frac{- i \nu (\mathit{V}t - z)}
{\nu \sqrt{\mathit{V}^2t^2 - z^2}} \right)^{2n} = 
(-i)^{2n} \nu^{2n} \frac{(\mathit{V}t - z)^{2n}}
{(\mathit{V}t - z)^n (\mathit{V}t + z)^n} = \\
= (-i)^{2n} \left( \frac{\mathit{V}t - z}{\mathit{V}t + z} \right)^n = 
(-1)^{n} \left( \frac{\mathit{V}t - z}{\mathit{V}t + z} \right)^n = 
\left( - \frac{\mathit{V}t - z}{\mathit{V}t + z} \right)^n
\end{aligned} \end{equation*} }
%
\textcolor{lightgray}{ \begin{equation*} \begin{aligned}
U_0 [W, Z] - U_2 [W, Z] = J_0(Z) + 2 \left[ 
\frac{\mathit{V}t - z}{\mathit{V}t + z} J_2(Z) + \left( 
\frac{\mathit{V}t - z}{\mathit{V}t + z} \right)^2 J_4(Z) + ... \right]
\end{aligned} \end{equation*} }
%
\begin{equation} \begin{aligned}
U_0 [W, Z] - U_2 [W, Z] = J_0(Z) + 2 \sum_{m=1}^{\infty} \left( 
\frac{\mathit{V}t - z}{\mathit{V}t + z} \right)^m J_{2m} (Z)
\end{aligned} \end{equation}

%%%%%%%%%%%%%%%%%%%%%%%%%%%%%%%%%%%%%%%%%%%%%%%%%%%%%%%%%%%%%%%%%%%%%%%%%%%%%%%
\section{Інтегродиференціальні властивості}

Функція Ломмеля типова для нестаціонарних задач. В \cite[ст. 41]{Borisov1991} 
приведено корисні інтегродиференціальні.
%
\begin{equation} \begin{aligned}
\int \limits_{\xi}^{\tau} ds e^{-i \gamma s} J_0(\sqrt{s^2 - \xi^2 }) = 
\frac{e^{-i \gamma \tau}}{\sqrt{\gamma^2 - 1}} \left( U_1(W_+,Z) + \right. \\ 
\left. + i U_2(W_+,Z) - U_1(W_-,Z) - i U_2(W_-,Z) \right)
\end{aligned} \end{equation}

Тут $ W_\pm = (\gamma \pm \sqrt{\gamma^2 - 1}) (\tau - \xi) $ a 
$ Z = \sqrt{\tau^2 - \xi^2} $. Також для використання цієї формули повинна
виконуватись умова $ \tau - \xi > 0 $.
%
\textcolor{red}{ \begin{equation}
\left. \begin{array}{c}
U_{2n} (W_+, Z) = U_{2n} (W_-, Z) \\
U_{2n+1} (W_+, Z) = - U_{2n+1} (W_-, Z)
\end{array} \right| n \in \Z
\end{equation} }
%
\begin{equation} 
\partder{}{Z} U_n (W,Z) = - \frac{Z}{W} U_{n+1} (W,Z)
\end{equation}
%
\begin{equation}
2 \partder{}{W} U_n (W,Z) = U_{n-1} (W,Z) + 
\left( \frac{Z}{W} \right)^2 U_{n+1} (W,Z)
\end{equation}

%%%%%%%%%%%%%%%%%%%%%%%%%%%%%%%%%%%%%%%%%%%%%%%%%%%%%%%%%%%%%%%%%%%%%%%%%%%%%%%
\section{Інтеграл 3}

\begin{equation} 
I_3 = R \int \limits_{0}^{\infty} \frac{d \nu}{\rho \nu} 
J_1(\nu \rho) J_1(\nu R) (U_0[ W_-, Z ] - U_2[ W_-, Z ])
\end{equation}
%
\begin{equation} \label{eq:intergal3}
I_3 = I_1 - 2 R \int_{0}^{\infty} \frac{d \nu}{\nu \rho} 
J_1(\nu \rho) J_1(\nu R) U_2[ W_-, Z ]
\end{equation}
%
На осі випромінювання, тобто при $ \rho = 0 $. 
%
\textcolor{lightgray}{ \begin{equation*} \begin{aligned}
\left. 2 R \int_{0}^{\infty} \frac{d \nu}{\nu \rho} 
J_1(\nu \rho) J_1(\nu R) \sum_{m=1}^{\infty} \left( 
\frac{ct - z}{ct + z} \right)^m J_{2m} (Z) 
\right|_{\rho = 0} = \\ = R \frac{ct - z}{ct + z} \int_{0}^{\infty} 
d \nu J_1(\nu R) J_2 (\nu \sqrt{c^2t^2 + z^2}) + \\ 
+ R \left( \frac{ct - z}{ct + z} \right)^2 
\int_{0}^{\infty} d \nu J_1(\nu R) J_4 (\nu \sqrt{c^2t^2 + z^2}) + ...
\end{aligned} \end{equation*} }
%
\begin{equation*} \begin{aligned}
I_3 = I_1 + R \sum_{m=1}^{\infty} \left( \frac{ct - z}{ct + z} \right)^m 
\int_{0}^{\infty} d \nu J_1(\nu R) J_{2m} (\nu \sqrt{c^2t^2 + z^2})
\end{aligned} \end{equation*}
%
За допомогою табличного інтегралу 2.12.31.1 з 
\cite[ст. 209]{SpecFunc1983}, що має вид:
%
\begin{equation*} \begin{aligned}
\int_0^\infty J_\mu (bx) J_\nu (cx) dx = A_{\mu,\nu}^1
\end{aligned} \end{equation*}
%
\begin{equation*} \begin{aligned}
A_{\mu,\nu}^1 \left( 0 < c < b \right) = b^{-\nu-1} c^{\nu} 
\Gamma \left[ \begin{array}{l} 
(\nu+\mu+1)/2 \\ (\mu-\nu-1)/2 + 1, \nu + 1 \end{array} \right] \cdot
\\ \cdot F \left(  \frac{\nu+\mu+1}{2}, \frac{\nu-\mu+1}{2}; 
\nu+1; \frac{c^2}{b^2} \right)
\end{aligned} \end{equation*}
%
\begin{equation*} \begin{aligned}
A_{\mu,\nu}^1 \left( 0 < b < c \right) = b^{\mu} c^{-\mu-1} 
\Gamma \left[ \begin{array}{l} 
(\nu+\mu+1)/2 \\ (\nu-\mu-1)/2 + 1, \mu + 1 \end{array} \right] \cdot
\\ \cdot F \left(  \frac{1+\mu+\nu}{2}, \frac{1+\mu-\nu}{2}; 
\mu+1; \frac{b^2}{c^2} \right)
\end{aligned} \end{equation*}
%
\textcolor{lightgray}{ \begin{equation*} \begin{aligned}
\begin{array}{lr} \forall a,k \in \Z: &
(a)_k = \frac{\Gamma(a+k)}{\Gamma(a)} = \frac{(a+k-1)!}{(a-1)!} \end{array}
\end{aligned} \end{equation*} }
%
Тут через $ F $ означено гіпергеометричну функцію Гаусса.
%
\textcolor{lightgray}{ \begin{equation*} \begin{aligned}
F (a_1, a_2; b_1; z) = \sum_{k=0}^\infty 
\frac{(a_1)_k (a_2)_k}{(b_1)_k} \frac{z^k}{k!} 
\end{aligned} \end{equation*} }
%
\begin{equation*} \begin{aligned}
F (a_1, a_2; b_1; z) = \sum_{k=0}^\infty 
\frac{\Gamma(a_1+k) \Gamma(a_2+k) \Gamma(b_1)}
{\Gamma(a_1) \Gamma(a_2) \Gamma(b_1+k)} \frac{z^k}{k!}
\end{aligned} \end{equation*}
%
Таким чином, $ I_3 $ можна записати в явному виді для $ \rho = 0 $.
%
\textcolor{lightgray}{ \begin{equation*} \begin{aligned}
b = R, c = \sqrt{c^2t^2-z^2} \\
\mu = 1, \nu = \{ 2m \}_{m=1}^{\infty}
\end{aligned} \end{equation*} }
%
\textcolor{lightgray}{ \begin{equation*} \begin{aligned}
\left. A_{1,2m}^1 \right|^{\tau < R} =
\frac{(c^2t^2-z^2)^m}{R^{2m+1}}
\Gamma \left[ \begin{array}{l} 1+m \\ 1-m, 2m + 1 \end{array} \right]
F \left( m+1, m; 2m+1; \frac{c^2t^2-z^2}{R^2} \right)
\end{aligned} \end{equation*} }
%
\textcolor{lightgray}{ \begin{equation*} \begin{aligned}
\left. A_{1,2m}^1 \right|^{\tau < R} = \frac{(c^2t^2-z^2)^m}{R^{2m+1}}
\frac{\Gamma(1+m)}{\Gamma(1-m) \Gamma(2m+1)}
F \left( m+1, m; 2m+1; \frac{c^2t^2-z^2}{R^2} \right)
\end{aligned} \end{equation*} }
%
\textcolor{lightgray}{ \begin{equation*} \begin{aligned}
\left. A_{1,2m}^1 \right|^{\tau < R} = \frac{1}{R}
\frac{\Gamma(1+m)}{\Gamma(1-m) \Gamma(2m+1)} \cdot \\
\cdot \sum_{k=0}^\infty \frac{1}{k!} 
\frac{ \Gamma(m+k+1) \Gamma(m+k) \Gamma(2m+1) }
{ \Gamma(m+1) \Gamma(m) \Gamma(2m+k+1) }
\left( \frac{c^2t^2-z^2}{R^2} \right)^{k+m}
\end{aligned} \end{equation*} }
%
\begin{equation*} \begin{aligned}
\left. A_{1,2m}^1 \right|^{c^2t^2 - z^2 < R} \simeq
\frac{1}{ \Gamma(1-m) \Gamma(m) } = 
\frac{\sin(m \pi)}{\pi} = 0, \forall m \in \Z
\end{aligned} \end{equation*}
%
\textcolor{lightgray}{ \begin{equation*} \begin{aligned}
\left. A_{1,2m}^1 \right|^{\tau > R} = \frac{R}{c^2t^2-z^2}
\Gamma \left[ \begin{array}{l} m+1 \\ m, 2 \end{array} \right]
F \left( 1+m, 1-m; 2; \frac{R^2}{c^2t^2-z^2} \right)
\end{aligned} \end{equation*} }
%
\textcolor{lightgray}{ \begin{equation*} \begin{aligned}
\left. A_{1,2m}^1 \right|^{\tau > R} = \frac{R}{c^2t^2-z^2}
\frac{\Gamma(m+1)}{\Gamma(m) \Gamma(2)}
F \left( 1+m, 1-m; 2; \frac{R^2}{c^2t^2-z^2} \right)
\end{aligned} \end{equation*} }
%
\textcolor{lightgray}{ \begin{equation*} \begin{aligned}
\left. A_{1,2m}^1 \right|^{\tau > R} = \frac{mR}{c^2t^2-z^2}
\sum_{k=0}^\infty  \frac{\left( 1-m^2 \right)^k}{2^k k!} 
\left( \frac{R^2}{c^2t^2-z^2} \right)^{k}
\end{aligned} \end{equation*} }
%
\textcolor{lightgray}{ \begin{equation*} \begin{aligned}
\left. A_{1,2m}^1 \right|^{\tau > R} = \frac{m}{R}
\sum_{k=0}^\infty  \frac{\left( 1-m^2 \right)^k}{2^k k!} 
\left( \frac{R^2}{c^2t^2-z^2} \right)^{k+1}
\end{aligned} \end{equation*} }
%
\begin{equation}
\left. I_3 \right|^{\rho=0} = \left. I_1 \right|^{\rho=0} +
\sum_{m=1}^{\infty} \left( \frac{ct - z}{ct + z} \right)^m A_{1,2m}^1
\end{equation}
%
\begin{equation}
\left. A_{1,2m}^1 \right|^{\tau > R} = m
\sum_{k=0}^\infty  \frac{\left( 1-m^2 \right)^k}{2^k k!} 
\left( \frac{R^2}{c^2t^2-z^2} \right)^{k+1}
\end{equation}
%
\begin{equation} 
\left. A_{1,2m}^1 \right|^{\tau < R} = \frac{C_{2m-1}^{m-1}}{2}
\sum_{k=0}^\infty \frac{1}{k!} \left( \frac{m^2+m}{2m+1} \right)^k
\left( \frac{c^2t^2-z^2}{R^2} \right)^{k+m}
\end{equation}

%%%%%%%%%%%%%%%%%%%%%%%%%%%%%%%%%%%%%%%%%%%%%%%%%%%%%%%%%%%%%%%%%%%%%%%%%%%%%%%
\section{Інтеграл 4}

\begin{equation}
I_4 = R \int \limits_{0}^{\infty} d \nu J_0(\nu \rho) J_1(\nu R) 
(U_0[ W_-, Z ] - U_2[ W_-, Z ])
\end{equation}
%
\begin{equation}
I_4 = I_2 - 2 R \int \limits_{0}^{\infty} d \nu 
J_0(\nu \rho) J_1(\nu R) U_2[ W_-, Z ]
\end{equation}
%
По аналогії з \eqref{eq:intergal3}, використаємо формулу 2.12.31.1 з 
\cite[ст. 209]{SpecFunc1983} для пошуку значення інтегралу при $ \rho = 0 $.
%
\textcolor{lightgray}{ \begin{equation*} \begin{aligned}
I_4 = I_2 - 2 R \int \limits_{0}^{\infty} d \nu J_1(\nu R) U_2[ W_-, Z ]
\end{aligned} \end{equation*} }
%
\textcolor{lightgray}{ \begin{equation*} \begin{aligned}
I_4 = I_2 - 2 R \sum_{m=1}^{\infty} \left( \frac{ct - z}{ct + z} \right)^m 
\int_{0}^{\infty} d \nu J_1(\nu R) J_{2m} (\nu \sqrt{c^2t^2 + z^2})
\end{aligned} \end{equation*} }
%
\begin{equation}
\left. I_4 \right|^{\rho=0} = \left. I_2 \right|^{\rho=0} -
2 \sum_{m=1}^{\infty} \left( \frac{ct - z}{ct + z} \right)^m A_{1,2m}^1
\end{equation}

%%%%%%%%%%%%%%%%%%%%%%%%%%%%%%%%%%%%%%%%%%%%%%%%%%%%%%%%%%%%%%%%%%%%%%%%%%%%%%%
\section{Інтеграл 5}

\begin{equation}
I_5 = i R \int \limits_{0}^{\infty} d \nu J_1 \left( \nu R \right) 
J_1 \left( \nu \rho \right)
U_1 \left[ - i \nu \left( ct - z \right), \nu \sqrt{c^2t^2 - z^2} \right]
\end{equation}


\end{document}
