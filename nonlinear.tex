\chapter{Нестаціонарні поля плаского випромінювача у нелінійному середовищі}
\label{ch:nonlinear}

%%%%%%%%%%%%%%%%%%%%%%%%%%%%%%%%%%%%%%%%%%%%%%%%%%%%%%%%%%%%%%%%%%%%%%%%%%%%%%%%
\section{Матеріальні рівняння нелінійного середовища}

Взаємодію поля крізь середовище з самим собою, завдяки теорії суперпозиції,
можна представити у вигляді додаткового стороннього джерела поля, що буде
просторово розподілене в усій області розповсюдження породжувальної
сильної хвилі. Назвемо таке джерело вторинним, а поле у лінійному наближені,
породжувальною хвилею.

Розглянемо модель, де характер взаємодії електромагнітного поля і середовища 
задається в матеріальних рівняннях, а поляризація та намагніченість 
розглядаються, як джерела електромагнітної індукції.

\begin{equation*}
\vect{D} = \epsilon_0 \vect{E} + \vect{P} \left( \vect{E}, \vect{H} \right) =
\epsilon_0 \vect{E} + \epsilon_0 \chi_e \left( \vect{E}, \vect{H} \right)
\end{equation*}

\begin{equation*}
\vect{B} = \mu_0 \vect{H} + \mu_0 \vect{M} \left( \vect{E}, \vect{H} \right) =
\mu_0 \vect{H} + \mu_0 \chi_m \left( \vect{E}, \vect{H} \right)
\end{equation*}

Так як діелектричні середовища проявляють на порядок слабші магнітні
нелінійні властивості, виключимо нелінійну намагніченість з моделі задля
спрощення \textcolor{red}{[ПОСИЛАННЯ]}.

Виключимо з розглядання гіротропні і не-хіральні середовища, тоді взаємний 
вплив магнітної та електричної індукції зникне і вектор поляризації стане 
функцією лише електричної напруженості, а намагніченість - лише магнітної
напруженості. Також, клас середовищ, що розглядається, обмежимо сталими, 
однорідними та ізотропними властивостями. Тоді, користуючись розкладом за 
малим параметром, можна записати вектор поляризації у вигляді нескінченного 
поліному:

\begin{equation} \label{eq:d_voltera}
\vect{D} = \epsilon_0 \vect{E} + \epsilon_0 \chi_e \left( \vect{E} \right) = 
\epsilon_0 \vect{E} + \epsilon_0 \sum_{k=1}^{\infty} \int_0^t
\chi_e^{(k)} (\tau) \vect{E}^k (\tau) d \tau,
\end{equation}
%
де $\chi_e^{(k)} (t) $ коефіцієнти розкладу Вольтера нелінійної функції 
$ \chi_e $ по параметру $ \vect{E} $.

Розклад Вольтера ілюструє затримку у відгуках середовища на помірно сильні 
збудження. Розклад \eqref{eq:d_voltera} може застосовуватись у випадках,
коли породжуюче поле, поширюватись у середовищі, не змінює його квантовий
стан \cite{imp:Smirnov2012}. Такий тип нелінійності називається 
параметричним та відповідає випадку слабкої нелінійності. Тепер припустимо, 
що нелінійні ефекти в середовищі, що розглядається, не є інерційними за часом 
та породжують індукційний відгук миттєво. Тоді, від розкладу в ряд Вольтера 
перейдемо до Тейлорівського ряду, як моделі нелінійності:

\begin{equation} \label{eq:d_teilor}
\vect{D} = \epsilon_0 \vect{E} + 
\epsilon_0 \sum_{k=1}^{\infty} \chi_e^{(k)} \vect{E}^k (t).
\end{equation}

Застосовуючи властивість симетрії вектору поляризації 
%
\begin{equation}
\vect{P} \left( \vect{E} \right) = - \vect{P} \left( - \vect{E} \right)
\end{equation}
%
помічаємо, що лише непарні доданки ряду Тейлора \eqref{eq:d_teilor} можуть 
бути не нульовими. Тепер, відокремлюючи перший, лінійний, доданок з 
поліному, отримаємо наступний вид вектору електричної індукції:

\begin{equation} \label{eq:d_teilor_odd}
\vect{D} = \epsilon_0 \epsilon \vect{E} + 
\epsilon_0 \sum_{k=1}^{\infty} \chi_e^{(2k+1)} \vect{E}^{2k+1} (t).
\end{equation}

Для широкого класу прикладних задач розглядається
лише перший нелінійний доданок розкладу. Таким чином, отримаємо кубічну 
нелінійну складову вектору поляризації, що зустрічається в оптиці під 
назвою нелінійна поляризація Керра:

\begin{equation} \label{eq:d_kerr}
\vect{D} = 
\epsilon_0 \epsilon \vect{E} + \epsilon_0 \chi_e^{(3)} \vect{E}^{3} = 
\epsilon_0 \epsilon \vect{E} + \vect{P}^\prime.
\end{equation}

Хоча, розклад в ряд Тейлора \eqref{eq:d_teilor_odd} не гарантує зменшення 
впливу кожного наступного доданку, тобто $ \chi_e^{(i)} < \chi_e^{(i+1)} $
на практиці врахування лише першого нелінійного доданку найчастіше дає
гарну точність до наближення слабкої нелінійності і врахуванням доданків
вищих порядків нехтують.

\textcolor{red}{TODO: Які нелінійні ефекти спостерігаються в 
керрівському середовищі?}

Перша складова вектору електричної індукції \eqref{eq:d_kerr} відповідає 
полю при лінійному наближенні $ \vect{E} $, що породжене деяким 
струмом $ \vect{J} $, а другий доданок відповідає нелінійному Керрівському 
відгуку середовища, який згідно з принципом суперпозиції, можна розглянути, 
як деяке поле $ \vect{E}^\prime $, породжене додатковим розподілом струму 
зміщення $ \vect{J}^\prime $ (далі вторинне джерело). Тоді, відповідно до
аналогії зі струмом зміщення, індукований нелінійний вторинний струм

\begin{equation} \label{eq:j_kerr}
\vect{J^\prime} = \partder{\vect{P}^\prime}{t} = 
\epsilon_0 \chi_e^{(3)} \partder{\vect{E}^3}{t}.
\end{equation}

Згідно описаної моделі, деяке стороннє джерело $ \vect{J} $ породжує 
імпульсне електромагнітне поле $ \vect{E} $ та $ \vect{H} $ розраховане
з припущенням лінійності електромагнітної індукції. Це лінійне поле, 
поширюватись із втратами крізь середовище, формує струм зміщення 
$ \vect{J}^\prime $, який в свою чергу є джерелом поля $ \vect{E}^\prime $ 
та $ \vect{H}^\prime $. Тоді, нелінійна самодія хвилі крізь середовище, 
згідно принципу суперпозиції $ \vect{E} + \vect{E}^\prime $ та 
$ \vect{H} + \vect{H}^\prime $.

Відокремивши лінійну складову з нелінійного доданку вектору індукції, 
отримаємо діелектричну проникність поля, що характеризує нелінійні 
властивості середовища $ \chi_e^{(3)} \vect{E}^2 $, що залежить від 
напруженості поля.

Вторинне джерело $ \vect{J}^\prime $ забирає енергію породжуючої хвилі та 
формується частиною енергії втрат $ \vect{E} $, що характеризують середовище.

\textcolor{red}{TODO: Розглянута модель, не пояснює нелінійні ефекти в вакуумі}

Вторинне джерело поля не є реальним джерелом, в прямому розумінні. 
Джерело $ \vect{J^\prime} $ наближено моделює нелінійну природу фізичних 
явищ поширення сильних електромагнітних хвиль. Обмеження, що було 
введено при побудові моделі, дозволять розглядати лише таку комбінацію 
середовища і породжувальної хвилі, для яких взаємодія відбувається лише
через вид взаємодії коли поле змінює просторову орієнтацію молекул.

Розглянемо в якості породжувальної хвилі $ \vect{E} $ поле, збуджене 
пласким диском електричного струму з часовою залежністю у вигляді 
функції Хевісайда -- моментальний стрибок амплітуди струму від нуля до 
значення $ A_0 $.

За рахунок отриманого в розділі \ref{ch:linear} розв'язку у наближенні 
лінійної поляризації можемо побудувати вторинний струм $ \vect{J^\prime} $
аналітично. Для розв'язання задачі випромінювання аналітично заданого 
вторинного струму можемо скористатись, як числовим так і аналітичним 
алгоритмом. В цьому випадку надамо перевагу аналітичному методу, а саме 
методу еволюційних рівнянь. У порівнянні з числовим розрахунком, такий підхід
забезпечить розуміння фізики процесу випромінювання.

Ітеративний підхід надає лише наближений опис вливу слабкої нелінійності на
випромінюваня антени типу LIRA. Однією з умов роботи лінзових імпульсних 
антен є те, що матеріал лінзи повинен мати більшу діелектричну проникність, 
ніж вільний простір, в який відбувається випромінювання, а отже нелінійні 
ефекти проявляться при розв'язанні внутрішньої задачі антени більше, ніж 
для зовнішньої. Задачу, можна вважати моделлю випромінювання LIRA, яка 
не враховує нелінійних ефектів в процесі формування квазіпласної хвилі.

\textcolor{blue}{ \begin{equation*} 
\vect{J^\prime} = 
\vect{\rho_0}    \partder{}{t} P_\rho^\prime    \left( \vect{E} \right) + 
\vect{\varphi_0} \partder{}{t} P_\varphi^\prime \left( \vect{E} \right) + 
\vect{z_0}       \partder{}{t} P_z^\prime       \left( \vect{E} \right) 
\end{equation*} }
%
\textcolor{blue}{ \begin{equation*}
\vect{P^\prime} \left( \vect{E} \right) = \epsilon_0 \chi_e^{(3)} 
\dotprod{ \vect{E} }{ \vect{E} } \cdot \vect{E} 
\end{equation*} }
%
\textcolor{blue}{ \begin{equation*} \begin{aligned}
\vect{P^\prime} \left( \vect{E} \right) = 
\frac{ {A_0}^3 \epsilon_0 \chi_e^{(3)} }{ 8 } \left( \frac{\mu_0 \mu}
{\epsilon_0 \epsilon} \right)^{3/2} \left( {I_1}^2 \cos^2 \varphi + 
\left( I_2 - I_1 \right)^2 \sin^2 \varphi \right) \cdot \\ 
\cdot \Big( \vect{\rho_0} I_1 \cos \varphi - 
\vect{ \varphi_0 } \left( I_2 - I_1 \right) \sin \varphi \Big)
\end{aligned} \end{equation*} }
%
\textcolor{blue}{ \begin{equation*} \begin{aligned}
\vect{P^\prime} \left( \vect{E} \right) = 
\frac{ {A_0}^3 \epsilon_0 \chi_e^{(3)} }{ 8 } \left( \frac{\mu_0 \mu}
{\epsilon_0 \epsilon} \right)^{3/2} \left( {I_1}^2 \cos^2 \varphi + 
\left( I_2 - I_1 \right)^2 \sin^2 \varphi \right) \cdot \\ 
\cdot \Big( \vect{\rho_0} I_1 \cos \varphi - 
\vect{ \varphi_0 } \left( I_2 - I_1 \right) \sin \varphi \Big)
\end{aligned} \end{equation*} }
%
\textcolor{blue}{ \begin{equation*}
\vect{E} = \frac{A_0}{2} \sqrt{\frac{\mu_0 \mu}{\epsilon_0 \epsilon}}
\Big( \vect{\rho_0} I_1 \cos \varphi - 
\vect{ \varphi_0 } \left( I_2 - I_1 \right) \sin \varphi \Big)
\end{equation*} }
%
\textcolor{blue}{ \begin{equation*} \begin{aligned}
\vect{E}^2 = \frac{A_0^2}{4} \frac{\mu_0 \mu}{\epsilon_0 \epsilon}
\Big( I_1^2 \cos^2 \varphi + \left( I_2 - I_1 \right)^2 \sin^2 \varphi \Big)
\end{aligned} \end{equation*} }
%
\textcolor{blue}{ \begin{equation*} \begin{aligned}
\partder{ \vect{E}^2 }{t} = \frac{A_0^2}{4} 
\frac{\mu_0 \mu}{\epsilon_0 \epsilon}
\left( 2 I_1 \partder{I_1}{t} \cos^2 \varphi + 
2 ( I_2 - I_1 ) \left( \partder{I_2}{t} - \partder{I_1}{t} \right) 
\sin^2 \varphi \right)
\end{aligned} \end{equation*} }

Користуючись виразом для вторинного струму Керра \eqref{eq:j_kerr} та 
напруженістю електричного поля \eqref{eq:linear_e_cyl}, запишемо компоненти 
струму:

\textcolor{blue} { \begin{equation*} \begin{aligned}
\partder{P_\rho^\prime}{t}   = \frac{ {A_0}^3 \epsilon_0 \chi_e^{(3)} }{ 8 } 
\left( \frac{\mu_0 \mu} {\epsilon_0 \epsilon} \right)^{3/2} \left(
\left( {I_1}^2 \cos^2 \varphi + ( I_2 - I_1 )^2 \sin^2 \varphi \right)
\partder{I_1}{t} \cos \varphi + \right. \\
\left. + I_1 \cos \varphi \left( 2 I_1 \partder{I_1}{t} \cos^2 \varphi + 
2 ( I_2 - I_1 ) \left( \partder{I_2}{t} - \partder{I_1}{t} \right) 
\sin^2 \varphi \right) \right) = \\ 
= \frac{ {A_0}^3 \epsilon_0 \chi_e^{(3)} }{ 8 } 
\left( \frac{\mu_0 \mu} {\epsilon_0 \epsilon} \right)^{3/2} \left(
\partder{I_1}{t} {I_1}^2 \cos^3 \varphi + \partder{I_1}{t} ( I_2 - I_1 )^2 
\cos \varphi \sin^2 \varphi + \right. \\
\left. + 2 {I_1}^2 \partder{I_1}{t} \cos^3 \varphi + 
2 I_1 ( I_2 - I_1 ) \left( \partder{I_2}{t} - \partder{I_1}{t} \right) 
\cos \varphi \sin^2 \varphi \right)
\end{aligned} \end{equation*} }
%
\begin{equation*} \begin{aligned}
\partder{P_\rho^\prime}{t} = \frac{ {A_0}^3 \epsilon_0 \chi_e^{(3)} }{ 8 } 
\left( \frac{\mu_0 \mu} {\epsilon_0 \epsilon} \right)^{3/2} \left(
3 {I_1}^2 \partder{I_1}{t} \cos^3 \varphi + \right. \\
+ \left. ( I_2 - I_1 ) \cos \varphi \sin^2 \varphi \left( 
\partder{I_1}{t} ( I_2 - I_1 ) + 2 I_1 \left( \partder{I_2}{t} - 
\partder{I_1}{t} \right) \right) \right);
\end{aligned} \end{equation*}
%
\textcolor{blue} { \begin{equation*} \begin{aligned}
\partder{P_\varphi^\prime}{t}   = 
- \frac{ {A_0}^3 \epsilon_0 \chi_e^{(3)} }{ 8 } 
\left( \frac{\mu_0 \mu} {\epsilon_0 \epsilon} \right)^{3/2} \left(
\left( {I_1}^2 \cos^2 \varphi + ( I_2 - I_1 )^2 \sin^2 \varphi \right)
\left( \partder{I_2}{t} - \partder{I_1}{t} \right) \sin \varphi + \right. \\
\left. + (I_2 - I_1) \sin \varphi \left( 2 I_1 \partder{I_1}{t} \cos^2 \varphi + 
2 ( I_2 - I_1 ) \left( \partder{I_2}{t} - \partder{I_1}{t} \right) 
\sin^2 \varphi \right) \right) = \\ 
= - \frac{ {A_0}^3 \epsilon_0 \chi_e^{(3)} }{ 8 } 
\left( \frac{\mu_0 \mu} {\epsilon_0 \epsilon} \right)^{3/2} \left(
{I_1}^2 \left( \partder{I_2}{t} - \partder{I_1}{t} \right) 
\sin \varphi \cos^2 \varphi + \right. \\ \left. 
+ ( I_2 - I_1 )^2 \left( \partder{I_2}{t} - \partder{I_1}{t} \right) 
\sin^3 \varphi + 2 I_1 \partder{I_1}{t} (I_2 - I_1) 
\sin \varphi \cos^2 \varphi + \right. \\ 
+ \left. 2 ( I_2 - I_1 )^2 \left( \partder{I_2}{t} - \partder{I_1}{t} \right) 
\sin^3 \varphi \right)
\end{aligned} \end{equation*} }
%
\begin{equation*} \begin{aligned}
\partder{P_\varphi^\prime}{t} = 
- \frac{ {A_0}^3 \epsilon_0 \chi_e^{(3)} }{ 8 } 
\left( \frac{\mu_0 \mu} {\epsilon_0 \epsilon} \right)^{3/2} \left(
3 ( I_2 - I_1 )^2 \left( \partder{I_2}{t} - \partder{I_1}{t} \right)
\sin^3 \varphi \right. + \\
+ \left. I_1 \sin \varphi \cos^2 \varphi \left( 
I_1 \left( \partder{I_2}{t} - \partder{I_1}{t} \right) + 
2 \partder{I_1}{t} (I_2 - I_1) \right) \right).
\end{aligned} \end{equation*}

Так як поздовжня напруженість електричного поля відсутня, то

\begin{equation*} \begin{aligned}
\partder{P_z^\prime}{t} = 0.
\end{aligned} \end{equation*}

Вирази для поперечних компонентів поля стають шматочно-визначеними
через свої залежності від $ I_1 $ та $ I_2 $, а також від їх похідних в 
кожному з доданків, згрупованих по залежностях від азимутального кута.
Область визначення інтегралів $ I_1 $ та $ I_2 $ відома та має вигляд 
$ S_1 \cup S_2 \cup S_3 $, де межі кожної з під-областей задежать від часу 
\eqref{eq:s1zone}-\eqref{eq:s3zone}. Записавши інтеграли $ I_1 $ та $ I_2 $ 
через визначення функції Хевісайда побачимо, що похідні від інтегралів 
міститимуть дельта-функції в особливих точках на межах часово-просторових 
областей випромінювання $ S_1 $, $ S_2 $, $ S_3 $. Користуючись неоднозначністю 
векторного потенціалу \cite[ст. 77]{imp:LandauII}, звільнимося від дельта-функцій 
в виразі для похідних від $ I_1 $ та $ I_2 $, тоді

\begin{equation*} \begin{aligned}
\frac{1}{v} \partder{I_\alpha}{t} = 
\frac{1}{v} \partder{ I_\alpha \{ S_{2} \} }{t} 
\Big( H \left( vt^2 - z^2 - (\rho - R)^2 \right)  - 
H \left( vt^2 - z^2 - (\rho + R)^2 \right) \Big),
\end{aligned} \end{equation*}
%
де $ v = c/\sqrt{\epsilon \mu} $ - швидкість світла в середовищі при 
лінійному наближенні вектору поляризації. Таким чином, область часу-простору,
де розподілений вторинний струм, обмежена лише $ S_2 $, тобто в усіх точках 
спостереження, що відповідають співвідношенню

\begin{equation*} \begin{aligned}
S^\prime \in S_2 \subset (\rho-R)^2 < vt^2 - z^2 < (\rho+R)^2.
\end{aligned} \end{equation*}

Тоді явні вирази для похідних будуть:

\textcolor{blue}{ \begin{equation*} \begin{aligned}
I_1 \left\{ S_2 \right\} = \frac{\rho^2 + R^2}{4 \pi \rho^2} \arccos 
\frac{c^2 t^2 - z^2 - \rho^2 - R^2}{2 \rho R}  -
\frac{\sqrt{4 \rho^2 R^2 - (\rho^2 + R^2 - c^2t^2 + z^2)^2}}{4 \pi \rho^2} - \\
- \frac{ |\rho^2 - R^2| }{2 \pi \rho^2} 
\arctan \sqrt{ \frac{(\rho - R)^2}{(\rho + R)^2} \cdot
\frac{\left( \rho + R \right)^2 - \left( c^2t^2 - z^2 \right)} 
{\left( c^2t^2 - z^2 \right) - \left( \rho - R \right)^2} }
\end{aligned} \end{equation*} }
%
\textcolor{blue}{ \begin{equation*} \begin{aligned}
\partder{I_1 \left\{ S_2 \right\}}{t} = \frac{\rho^2 + R^2}{4 \pi \rho^2}
\partder{}{t} \arccos \frac{c^2 t^2 - z^2 - \rho^2 - R^2}{2 \rho R} - \\
- \partder{}{t} \frac{\sqrt{4 \rho^2 R^2 - (\rho^2 + R^2 - c^2t^2 + z^2)^2}}
{4 \pi \rho^2} - \\ - \frac{ |\rho^2 - R^2| }{2 \pi \rho^2} \partder{}{t} 
\arctan \sqrt{ \frac{(\rho - R)^2}{(\rho + R)^2} \cdot
\frac{\left( \rho + R \right)^2 - \left( c^2t^2 - z^2 \right)} 
{\left( c^2t^2 - z^2 \right) - \left( \rho - R \right)^2} }
\end{aligned} \end{equation*} }
%
\textcolor{blue}{ \begin{equation*} \begin{aligned}
\partder{}{t} \arccos \frac{c^2 t^2 - z^2 - \rho^2 - R^2}{2 \rho R} = 
- \frac{2 c^2 t}
{ \sqrt{4 \rho^2 R^2 - \left(c^2 t^2 - z^2 - \rho^2 - R^2 \right)^2} }
\end{aligned} \end{equation*} }
%
\textcolor{blue}{ \begin{equation*} \begin{aligned}
- \partder{}{t} \left( \rho^2 + R^2 - c^2t^2 + z^2 \right)^2 = 
- 2 (\rho^2 + R^2 -c^2t^2 + z^2) (-2 c^2 t)
\end{aligned} \end{equation*} }
%
\textcolor{blue}{ \begin{equation*} \begin{aligned}
\partder{}{t} \frac{\sqrt{4 \rho^2 R^2 - (\rho^2 + R^2 - c^2t^2 + z^2)^2}}
{4 \pi \rho^2} = \frac{1}{8 \pi \rho^2} 
\frac{ 4 c^2 t (\rho^2 + R^2 - c^2 t^2 + z^2) }
{ \sqrt{4 \rho^2 R^2 - (\rho^2 + R^2 - c^2t^2 + z^2)^2} } = \\
= \frac{c^2 t}{2 \pi \rho^2} \frac{\rho^2 + R^2 - c^2 t^2 + z^2}
{ \sqrt{4 \rho^2 R^2 - (\rho^2 + R^2 - c^2t^2 + z^2)^2} }
\end{aligned} \end{equation*} }
%
\textcolor{blue}{ \begin{equation*} \begin{aligned}
\partder{}{t} \arctan \sqrt{ \frac{x}{y} } = 
\frac{1}{1 + \frac{x}{y}} \frac{1}{2} 
\sqrt \frac{y}{x} \partder{}{t} \frac{x}{y}
\end{aligned} \end{equation*} }
%
\textcolor{blue}{ \begin{equation*} \begin{aligned}
- \frac{1}{(\rho + R)^2} + \frac{1}{(\rho - R)^2} = 
\frac{- (\rho-R)^2 + (\rho+R)^2 }{ (\rho^2 - R^2)^2 }
\end{aligned} \end{equation*} }
%
\textcolor{blue}{ \begin{equation*} \begin{aligned}
\partder{}{t} \arctan \sqrt{ \frac{(\rho - R)^2}{(\rho + R)^2}
\frac{\left( \rho + R \right)^2 - \left( c^2t^2 - z^2 \right)} 
{\left( c^2t^2 - z^2 \right) - \left( \rho - R \right)^2} } = 
\partder{}{t} \arctan \sqrt{ \frac
{1 - \frac{c^2t^2 - z^2}{\left( \rho + R \right)^2} } 
{ \frac{c^2t^2 - z^2}{ \left( \rho - R \right)^2 } - 1} } = \\
= \frac{1}{1 + \frac{1 - \frac{c^2t^2 - z^2}{\left( \rho + R \right)^2} } 
{ \frac{c^2t^2 - z^2}{ \left( \rho - R \right)^2 } - 1} } \frac{1}{2}
\sqrt{ \frac{ \frac{c^2t^2 - z^2}{ \left( \rho - R \right)^2 } - 1 }
{1 - \frac{c^2t^2 - z^2}{\left( \rho + R \right)^2} } } 
\frac{ - \frac{2 c^2 t}{\left( \rho + R \right)^2} 
\left( \frac{c^2t^2 - z^2}{ \left( \rho - R \right)^2} - 1 \right) - 
\frac{ 2 c^2 t }{ \left( \rho - R \right)^2 } 
\left( 1 - \frac{c^2t^2 - z^2}{\left( \rho + R \right)^2} \right) }
{\left( \frac{c^2t^2 - z^2}{ \left( \rho - R \right)^2 } - 1 \right)^2} = \\
= - c^2 t \frac{ \frac{c^2t^2-z^2}{(\rho-R)^2} - 1 }
{ \frac{c^2t^2-z^2}{(\rho-R)^2} - 1 + 1 - 
\frac{c^2t^2 - z^2}{\left( \rho + R \right)^2} }
\sqrt{ \frac{ \frac{c^2t^2 - z^2}{ \left( \rho - R \right)^2 } - 1}
{1 - \frac{c^2t^2 - z^2}{\left( \rho + R \right)^2} } } \frac
{ \frac{c^2t^2 - z^2}{ \left( \rho^2 - R^2 \right)^2 } - \frac{1}{(\rho+R)^2} + 
\frac{1}{(\rho-R)^2} - \frac{ c^2t^2 - z^2 }{ \left( \rho^2 - R^2 \right)^2 } }
{ \left( \frac{c^2t^2 - z^2}{ \left( \rho - R \right)^2} - 1 \right)^2 } = \\
= - \frac{4 \rho R c^2 t}{ \left( \rho^2 - R^2 \right)^2 } 
\frac{ 1 }{ \frac{c^2t^2-z^2}{(\rho-R)^2} - 
\frac{c^2t^2 - z^2}{\left( \rho + R \right)^2} }
\sqrt{ \frac{ \frac{c^2t^2 - z^2}{ \left( \rho - R \right)^2 } - 1}
{1 - \frac{c^2t^2 - z^2}{\left( \rho + R \right)^2} } } \frac
{ 1 }{ \frac{c^2t^2 - z^2}{ \left( \rho - R \right)^2} - 1 } = \\
= - \frac{c^2 t}{ c^2 t^2 - z^2 } \frac{1} { 
\sqrt{ 1 - \frac{c^2t^2 - z^2}{(\rho + R)^2 } } 
\sqrt{ \frac{c^2t^2 - z^2}{ (\rho - R)^2 } - 1} }
\end{aligned} \end{equation*} }
%
\textcolor{blue}{ \begin{equation*} \begin{aligned}
\partder{ I_1 \{ S_2 \} }{t} = - \frac{c^2 t}{2 \pi \rho^2}
\frac{\rho^2 + R^2}
{ \sqrt{4 \rho^2 R^2 - \left(c^2 t^2 - z^2 - \rho^2 - R^2 \right)^2} } - \\
- \frac{c^2 t}{2 \pi \rho^2} \frac{\rho^2 + R^2 - c^2 t^2 + z^2}
{ \sqrt{4 \rho^2 R^2 - (\rho^2 + R^2 - c^2t^2 + z^2)^2} } + \\ 
+ \frac{ c^2 t }{2 \pi \rho^2} \frac{|\rho^2 - R^2|}{ c^2 t^2 - z^2 } \frac{1} 
{ \sqrt{ 1 - \frac{c^2t^2 - z^2}{(\rho + R)^2 } } 
\sqrt{ \frac{c^2t^2 - z^2}{ (\rho - R)^2 } - 1} }
\end{aligned} \end{equation*} }
%
\begin{equation} \begin{aligned} \label{eq:i1_partder}
\frac{1}{v} \partder{ I_1 \{ S_2 \} }{t} = \frac{ vt }{2 \pi \rho^2} 
\frac{ (\rho^2 - R^2)^2  (v^2 t^2 - z^2)^{-1} } 
{ \sqrt{ (\rho + R)^2 - v^2t^2 + z^2 } 
\sqrt{ v^2t^2 - z^2 - (\rho - R)^2 } } - \\
- \frac{vt}{2 \pi \rho^2} \frac{2 (\rho^2 + R^2) - (v^2 t^2 - z^2)}
{ \sqrt{4 \rho^2 R^2 - (v^2t^2 - z^2 - \rho^2 - R^2)^2} };
\end{aligned} \end{equation}
%
\textcolor{blue}{ \begin{equation*} \begin{aligned}
\partder{ I_2 \{ S_2 \} }{t} = \frac{1}{\pi} \partder{}{t} \arccos 
\frac{c^2t^2 - z^2 + \rho^2 - R^2}{2 \rho \sqrt{c^2t^2 - z^2}} = \\
= - \frac{1}{\pi} \frac{1} { \sqrt{ 1 - \frac{ (c^2t^2 - z^2 + \rho^2 - R^2)^2 }
{4 \rho^2 (c^2t^2 - z^2)^2} } } \frac{1}{2 \rho} \partder{}{t} 
\frac{c^2t^2 - z^2 + \rho^2 - R^2} {\sqrt{c^2t^2 - z^2}} = \\
= - \frac{1}{2 \rho \pi} \frac{1} 
{ \sqrt{ 1 - \frac{ (c^2t^2 - z^2 + \rho^2 - R^2)^2 }
{4 \rho^2 (c^2t^2 - z^2)} } } \frac{2c^2t \sqrt{c^2t^2 - z^2} - 
\frac{c^2t}{\sqrt{c^2t^2 - z^2}} (c^2t^2 - z^2 + \rho^2 - R^2)
}{c^2t^2 - z^2} = \\ = - \frac{c^2 t}{2 \pi \rho} \frac{1} 
{ \sqrt{ 1 - \frac{ (c^2t^2 - z^2 + \rho^2 - R^2)^2 }
{4 \rho^2 (c^2t^2 - z^2)} } } \frac{2 \sqrt{c^2t^2 - z^2} - 
\frac{c^2t^2 - z^2 + \rho^2 - R^2}{\sqrt{c^2t^2 - z^2}}}{c^2t^2 - z^2} = \\
= - \frac{c^2 t}{2 \pi \rho (c^2t^2 - z^2)} \frac{ 2 \sqrt{c^2t^2 - z^2} - 
\frac{c^2t^2 - z^2 + \rho^2 - R^2}{\sqrt{c^2t^2 - z^2}} } 
{ \sqrt{ 1 - \frac{ (c^2t^2 - z^2 + \rho^2 - R^2)^2 }
{4 \rho^2 (c^2t^2 - z^2)} } } = \\
= - \frac{c^2 t}{\pi (c^2t^2 - z^2) } 
\frac{ 2 (c^2t^2 - z^2) - (c^2t^2 - z^2 + \rho^2 - R^2) } 
{ \sqrt{ 4 \rho^2 (c^2t^2 - z^2) - (c^2t^2 - z^2 + \rho^2 - R^2)^2 } } = \\
= - \frac{c^2 t}{\pi (c^2t^2 - z^2) } \frac{ c^2t^2 - z^2 -  \rho^2 + R^2 } 
{ \sqrt{ 4 \rho^2 (c^2t^2 - z^2) - (c^2t^2 - z^2 + \rho^2 - R^2)^2 } }
\end{aligned} \end{equation*} }
%
\begin{equation} \begin{aligned} \label{eq:i2_partder}
\frac{1}{v} \partder{ I_2 \{ S_2 \} }{t} = 
- \frac{vt}{\pi (v^2t^2 - z^2) } \frac{ v^2t^2 - z^2 - \rho^2 + R^2 } 
{ \sqrt{ 4 \rho^2 (v^2t^2 - z^2) - (v^2t^2 - z^2 + \rho^2 - R^2)^2 } }.
\end{aligned} \end{equation}

З рис.~\ref{fig:jx_secondary} випливає, що проекція вторинного струму обернена 
за знаком відносно тієї ж проекції компоненти напруженості електричного поля.

\begin{figure}[h] \begin{center}
\includegraphics[scale=0.5]{Jperp_A1}
\caption{Нелінійність амплітуди вторинного джерела}
\label{fig:jx_secondary}
\end{center} \end{figure}

У виразах \eqref{eq:i1_partder}, \eqref{eq:i2_partder} очевидна нелінійна 
залежність струму від $ A_0 $.

%%%%%%%%%%%%%%%%%%%%%%%%%%%%%%%%%%%%%%%%%%%%%%%%%%%%%%%%%%%%%%%%%%%%%%%%%%%%%%%%
\section{Енергетичний розподіл поля плаского диску в лінійному наближенні}

Вторинний електричний струм розподілений в усьому напівпросторі $ z > 0 $,
але за рахунок згасання енергії з відстанню в межах $ [1/R, 1/R^2] $ (в 
залежності від напрямку спостереження) можемо обмежити область, де треба 
враховувати нелінійні ефекти за рахунок високої концентрації енергії. Для 
визначення параметричних меж застосування введеної моделі нелінійності та 
оцінки границі зони, де нелінійні ефекти треба враховувати, розглянемо 
енергетичні характеристики поля в ближній зоні.

Тепер розглянемо енергетичний розподіл від поля плаского диску при різних
часових залежностей $ f(t) $ стороннього струму. Побудова класичної 
енергетичної діаграми спрямованості -- мало інформативне дослідження 
нелінійних ефектів: в даній роботі важливим є енергетичний розподіл в 
ближній зоні.

Розглянемо густину енергії електромагнітного поля $ \vect{E} $, збудженого 
пласким диском електричного струму з довільною часовою залежністю
\cite{imp:Schantz2018}, нехтуючи при цьому енергетичним внеском 
$ \vect{H} $. Тоді густину енергії $ W $ можна визначити наступним чином: 


\begin{equation} \label{eq:energy}
W = \frac{\epsilon_0}{2} \int_0^\infty \fanc{\vect{E}}{\vect{r},t}^2 dt.
\end{equation}

Користуючись властивостями перехідної функції LIRA, можемо обмежити
область інтегрування за часом:

\begin{equation} \label{eq:energy}
W = \frac{\epsilon_0}{2} \frac{\mu_0 \mu}{\epsilon_0 \epsilon}
\int_{ct_1}^{c\tau_0+ct_3} \left( E_\rho^2 + E_\varphi^2 \right) dt.
\end{equation}

Вираз \eqref{eq:energy} справедливий для довільної часової залежності 
збуджувального струму. Скористаємось ним для визначення енергетичних 
характеристик перехідної функції, де часова залежність має вигляд функції
Хевісайда $ H(t) $:

\textcolor{blue}{ \begin{equation*} \begin{aligned}
\vect{E}^2 = \frac{A_0^2}{4} \frac{\mu_0 \mu}{\epsilon_0 \epsilon}
\Big( I_1^2 \cos^2 \varphi + \left( I_2 - I_1 \right)^2 \sin^2 \varphi \Big)
\end{aligned} \end{equation*} }
%
\begin{equation} \label{eq:energy_tr}
W_{tr} = \frac{\epsilon_0 A_0^2}{8} \frac{\mu_0 \mu}{\epsilon_0 \epsilon}
\int_{ct_1}^{ct_3}  \Big( I_1^2 \cos^2 \varphi + 
\left( I_2 - I_1 \right)^2 \sin^2 \varphi \Big) dt.
\end{equation}

Розглядаючи \ref{eq:energy_tr} для $ \rho = 0 $, помічаємо, що енергія 
випромінювання плаского диску, завжди лежить у наступних межах для 
сигналів з часовою залежністю $ f(t) $ з областю значень 
$ \left[ -1, 1 \right] $ та тривалістю $ \tau_0 $:

\textcolor{blue}{ \begin{equation*}
\left. W \right|^{\rho=0} = \frac{\epsilon_0 A_0^2}{32} 
\frac{\mu_0 \mu}{\epsilon_0 \epsilon} \Big( \sqrt{R^2+z^2} - z \Big)
\end{equation*} }
%
\textcolor{blue}{ \begin{equation*}
\int_{ct_1}^{c\tau_0+ct_3} 
\left( \int_0^t f(\tau) d \tau < \tau_0 \right) dt < \tau_0 R
\end{equation*} }
%
\begin{equation}
0 \leq W_{max} \left( \tau_0, f(t), \vect{r} \right) < 
\frac{\epsilon_0 \tau_0 R A_0^2}{32} \frac{\mu_0 \mu}{\epsilon_0 \epsilon},
\end{equation}
%
де $ W_{max} $ -- густина енергії, $ \tau_0 $ -- ефективна тривалість імпульсу 
за визначеною метрикою, $ A_0 $ -- максимальна амплітуда сигналу, $ R $ -- 
радіус апертури, а вираз $ \frac{\mu_0 \mu}{\epsilon_0 \epsilon} $ -- імпеданс 
середовища поширення хвилі.

\begin{figure}
\subfloat[$ f(t) = rect(t) $]{\includegraphics[width = 2in]{Wrect_XY}} 
\subfloat[$ f(t) = sinc(t) $]{\includegraphics[width = 2in]{Wsinc_XY}}
\subfloat[$ f(t) = gauss(t) $]{\includegraphics[width = 2in]{Wgauss_XY}}
\caption{Поперечний розподіл густини енергії поля LIRA з різними типами збуджень}
\label{fig:trans_dist}
\end{figure}

На рис.~\ref{fig:trans_dist} зображено поперечні розподіли густини енергії на 
відстані $ R $ від апертури LIRA при різній часовій залежності збуджувального 
струму. Сторони квадратних зрізів рівні та мають розмір $ 2R $. Як видно з 
рисунків, форма збудження має значний вплив на розподіл густини енергії у 
ближній зоні, де вклад нелінійної поправки найбільший, а отже поширення хвилі 
у нелінійному середовищі залежить від форми збуджувального сигналу.

% обмежувати область врахування вторинного струму на основі енергетичних 
% характеристик породжувальної хвилі треба з урахуванням часової залежності 
% струму.

Розглянемо збудження з часовою залежністю у вигляді $ H(t) $. Для того, щоб 
оцінити межі області найбільшого впливу нелінійності на напруженість 
електромагнітного поля побудуємо поздовжній розподіл густини енергії. 

\begin{figure}
\subfloat[Площина $ OXZ $]{\includegraphics[width = 3in]{Wtr_xz}} 
\subfloat[Площина $ OYZ $]{\includegraphics[width = 3in]{Wtr_yz}}
\caption{Поздовжній розподіл густини енергії поля LIRA}
\label{fig:long_dist}
\end{figure}

На рис.~\ref{fig:long_dist} зображено поздовжній розподіл густини енергії 
поля LIRA при часовій залежності збудження у вигляді $ H(t) $ та фокальним 
радіусом 1 метр. Густину енергії побудовано в двох ортогональних площинах.
З рисунків видно, що більша частина енергії зосереджена в прожекторній зоні 
поблизу до апертури ($ z < 2R $) у фігурі, що нагадує конус. Можна помітити, 
що довкола осі випромінювання спостерігається промінь ширина та інтенсивність 
якого повільно зменшуються. Імовірно, це область, де може спостерігатись 
тривимірна солітоноподібна хвиля.

\textcolor{red}{TODO: можна строго порiвняти дiаграму спрямованостi отриману в
часовiй областi з дiаграмою Баума, а також оцiнити її вiдхилення в ближнiй
зонi}

%%%%%%%%%%%%%%%%%%%%%%%%%%%%%%%%%%%%%%%%%%%%%%%%%%%%%%%%%%%%%%%%%%%%%%%%%%%%%%%%
\section{Перетворення мод Керрівською нелінійністю}

Застосуємо метод еволюційних рівнянь для розв'язання задачі випромінювання 
вторинним струмом \eqref{eq:j_kerr}. Для цього, спершу, запишемо модовий
розподіл джерела, який є правою частиною рівняння Клейна-Гордона. 

\textcolor{blue} { \begin{equation*}
j_m \left( r, t; \nu \right) = \frac{\sqrt{\mu_0}}{2\pi} 
\int \limits_{0}^{2\pi} d \varphi \int \limits_0^\infty \rho d \rho 
\vect{j_0} \crossprod{ \nabla_\perp \Psi_m^* }{ \vect{z_0} }
\end{equation*} }
%
\textcolor{blue} { \begin{equation*} 
\vect{J^\prime} = 
\vect{\rho_0}    \partder{}{t} P_\rho^\prime    \left( \vect{E} \right) + 
\vect{\varphi_0} \partder{}{t} P_\varphi^\prime \left( \vect{E} \right) + 
\vect{z_0}       \partder{}{t} P_z^\prime       \left( \vect{E} \right) 
\end{equation*} }
%
\textcolor{blue} { \begin{equation*} \begin{aligned}
\crossprod{ \nabla_\perp \Psi_m^* }{ \vect{z_0} } =
- \vect{\rho_0} i m e^{-im\varphi} \frac{J_m (\nu \rho)}{\rho \sqrt{\nu}}
- \vect{\varphi_0} \sqrt{\nu} e^{-im\varphi} 
\frac{J_{m-1} (\nu \rho) - J_{m+1} (\nu \rho)}{2}
\end{aligned} \end{equation*} }
%
\textcolor{blue} { \begin{equation*} \begin{aligned}
\vect{J^\prime} \crossprod{ \nabla_\perp \Psi_m^* }{ \vect{z_0} } = 
- i e^{-im\varphi} m \frac{J_m (\nu \rho)}{\rho \sqrt{\nu}}
\partder{}{t} P_\rho^\prime \left( \vect{E} \right) - \\
- \sqrt{\nu} e^{-im\varphi} \frac{J_{m-1} (\nu \rho) - J_{m+1} (\nu \rho)}{2}
\partder{}{t} P_\varphi^\prime
\end{aligned} \end{equation*} }

\begin{equation*} \begin{aligned}
j_m = - \frac{\sqrt{\mu_0}}{2\pi} 
\int_{0}^{2\pi} d \varphi \int \limits_{0}^{\infty} \rho d \rho
e^{-im\varphi} \left( i  m \frac{J_m (\nu \rho)}{\rho \sqrt{\nu}}
\partder{j_\rho^\prime}{t} + \sqrt{\nu}
\frac{J_{m-1} (\nu \rho) - J_{m+1} (\nu \rho)}{2}
\partder{j_\varphi^\prime}{t} \right)
\end{aligned} \end{equation*}

Спершу, спростимо вираз $ j_m $ отримавши уявну область 
значень замість комплексної. Згрупувавши доданки за тригонометричними 
функціями,знайдемо інтеграли за азимутальним кутом $ \varphi $, 
користуючись аналітичними інтегралами \eqref{eq:int_exp3}, 
\eqref{eq:int_exp4}, \eqref{eq:int_exp5}, \eqref{eq:int_exp6}. Отримаємо 
вирази для інтегралів від компонентів вектору вторинного нелінійного струму 
з ядром інтегралу у вигляді комплексної експоненти.

\textcolor{blue} { \begin{equation*} \begin{aligned}
\int_0^{2\pi} e^{-i m \varphi} \cos^3 \varphi d \varphi = 
\frac{\pi}{4} \delta_{m,-3} + \frac{\pi}{4} \delta_{m,3} + 
\frac{3 \pi}{4} \delta_{m,-1} + \frac{3 \pi}{4} \delta_{m,1}
\end{aligned} \end{equation*} }
%
\textcolor{blue} { \begin{equation*} \begin{aligned}
\int_0^{2\pi} e^{-i m \varphi} \cos \varphi \sin^2 \varphi d \varphi = 
\frac{\pi \delta_{m,1} }{4} + \frac{\pi \delta_{m,-1} }{4} - 
\frac{\pi \delta_{m,-3} }{4} - \frac{\pi \delta_{m,3} }{4}
\end{aligned} \end{equation*} }
%
\textcolor{blue} { \begin{equation*} \begin{aligned}
\int_{0}^{2 \pi} d \varphi e^{-im \varphi} \partder{P_\rho^\prime}{t} = 
\frac{ {A_0}^3 \epsilon_0 \chi_e^{(3)} }{ 8 } \int_{0}^{2\pi} d \varphi
e^{-im\varphi} \left( \frac{\mu_0 \mu} {\epsilon_0 \epsilon} \right)^{3/2} 
\left( 3 {I_1}^2 \partder{I_1}{t} \cos^3 \varphi + \right. \\
+ \left. ( I_2 - I_1 ) \cos \varphi \sin^2 \varphi \left( 
\partder{I_1}{t} ( I_2 - I_1 ) + 2 I_1 \left( \partder{I_2}{t} - 
\partder{I_1}{t} \right) \right) \right) = \\
= \frac{ {A_0}^3 \epsilon_0 \chi_e^{(3)} }{ 8 } 
\left( \frac{\mu_0 \mu} {\epsilon_0 \epsilon} \right)^{3/2}
\left( \frac{3 \pi}{4} {I_1}^2 \partder{I_1}{t} \left( \delta_{m,-3} + 
\delta_{m,3} + 3 \delta_{m,-1} + 3 \delta_{m,1} \right) + \right. \\
+ \frac{\pi }{4} \left. ( I_2 - I_1 ) \left( \delta_{m,1} + 
\delta_{m,-1} - \delta_{m,-3} - \delta_{m,3} \right) \left( 
\partder{I_1}{t} ( I_2 - I_1 ) + 2 I_1 \left( \partder{I_2}{t} - 
\partder{I_1}{t} \right) \right) \right)
\end{aligned} \end{equation*} }
%
\begin{equation*} \begin{aligned}
\frac{\epsilon_0 \chi_e^{(3)}}{2 \pi} \int_{0}^{2\pi} d \varphi 
e^{-im \varphi} \partder{P_\rho^\prime}{t} = 
\frac{ {A_0}^3 \epsilon_0 \chi_e^{(3)}}{ 64 } 
\left( \frac{\mu_0 \mu} {\epsilon_0 \epsilon} \right)^{3/2} \cdot \\ 
\cdot \left( 3 {I_1}^2 \partder{I_1}{t} \left( \delta_{m,-3} + 
\delta_{m,3} + 3 \delta_{m,-1} + 3 \delta_{m,1} \right) + \right. \\
+ \left. ( I_2 - I_1 ) \left( \delta_{m,1} + \delta_{m,-1} - 
\delta_{m,-3} - \delta_{m,3} \right) \left( 
\partder{I_1}{t} ( I_2 - I_1 ) + 2 I_1 \left( \partder{I_2}{t} - 
\partder{I_1}{t} \right) \right) \right)
\end{aligned} \end{equation*}
%
\textcolor{blue} { \begin{equation*} \begin{aligned}
\int_{0}^{2\pi} e^{-i m \varphi} \sin^3 \varphi d \varphi = 
\frac{3 \pi i}{4} \delta_{m,-1} - \frac{3 \pi i}{4} \delta_{m,1} - 
\frac{\pi i}{4} \delta_{m,-3} + \frac{\pi i}{4} \delta_{m,3}
\end{aligned} \end{equation*} }
%
\textcolor{blue} { \begin{equation*} \begin{aligned}
\int_0^{2\pi} e^{-i m \varphi} \sin \varphi \cos^2 \varphi d \varphi = 
\frac{\pi i }{4} \delta_{m,-1} - \frac{\pi i }{4} \delta_{m,1} -
\frac{\pi i }{4} \delta_{m,3} + \frac{\pi i }{4} \delta_{m,-3}
\end{aligned} \end{equation*} }
%
\textcolor{blue} { \begin{equation*} \begin{aligned}
\int_{0}^{2 \pi} d \varphi e^{-im \varphi} \partder{P_\varphi^\prime}{t} = \\
= - \frac{ {A_0}^3 \epsilon_0 \chi_e^{(3)} }{ 8 } 
\int_{0}^{2 \pi} d \varphi e^{-im \varphi}
\left( \frac{\mu_0 \mu} {\epsilon_0 \epsilon} \right)^{3/2} \left(
3 ( I_2 - I_1 )^2 \left( \partder{I_2}{t} - \partder{I_1}{t} \right)
\sin^3 \varphi \right. + \\
+ \left. I_1 \sin \varphi \cos^2 \varphi \left( 
I_1 \left( \partder{I_2}{t} - \partder{I_1}{t} \right) + 
2 \partder{I_1}{t} (I_2 - I_1) \right) \right) = 
- \frac{ {A_0}^3 \epsilon_0 \chi_e^{(3)} }{ 8 } \cdot \\ 
\cdot \left( \frac{\mu_0 \mu} {\epsilon_0 \epsilon} \right)^{3/2} \left(
\frac{3 \pi i}{4} ( I_2 - I_1 )^2 \left( \partder{I_2}{t} - 
\partder{I_1}{t} \right) \left( 3 \delta_{m,-1} - 3 \delta_{m,1} - 
\delta_{m,-3} + \delta_{m,3} \right) \right. + \\
+ \left. \frac{\pi i}{4} I_1 \left( \delta_{m,-1} - \delta_{m,1} - 
\delta_{m,3} + \delta_{m,-3} \right) \left( 
I_1 \left( \partder{I_2}{t} - \partder{I_1}{t} \right) + 
2 \partder{I_1}{t} (I_2 - I_1) \right) \right)
\end{aligned} \end{equation*} }
%
\begin{equation*} \begin{aligned}
\frac{\epsilon_0 \chi_e^{(3)}}{2 \pi} \int_{0}^{2 \pi} d \varphi 
e^{-im \varphi} \partder{P_\varphi^\prime}{t} = 
- \frac{ {A_0}^3 \epsilon_0 \chi_e^{(3)}  i}{ 64 }
\left( \frac{\mu_0 \mu} {\epsilon_0 \epsilon} \right)^{3/2} \cdot \\ 
\cdot \left( 3 ( I_2 - I_1 )^2 \left( \partder{I_2}{t} - 
\partder{I_1}{t} \right) \left( 3 \delta_{m,-1} - 3 \delta_{m,1} - 
\delta_{m,-3} + \delta_{m,3} \right) \right. + \\
+ \left. I_1 \left( \delta_{m,-1} - \delta_{m,1} - 
\delta_{m,3} + \delta_{m,-3} \right) \left( 
I_1 \left( \partder{I_2}{t} - \partder{I_1}{t} \right) + 
2 \partder{I_1}{t} (I_2 - I_1) \right) \right)
\end{aligned} \end{equation*}

Як видно з останніх виразів, інтегрування за кутом $ \varphi $ дає дискретний 
модовий розподіл вторинного струму.  Також помічаємо, що у розподілах присутні 
лише моди з номерами $ \pm 1 $ та $ \pm 3 $, а вклад вторинного струму в 
інші моди відсутній. Випишемо окремо кожну з ненульових мод розподілу 
вторинного струму. Для цього введемо нові змінні:

\begin{equation} \begin{aligned} \label{eq:alpha}
\alpha = 3 {I_1}^2 \partder{I_1}{t};
\end{aligned} \end{equation}

\begin{equation} \begin{aligned} \label{eq:beta}
\beta = ( I_2 - I_1 ) \left( \partder{I_1}{t} ( I_2 - I_1 ) + 
2 I_1 \left( \partder{I_2}{t} - \partder{I_1}{t} \right) \right);
\end{aligned} \end{equation}

\begin{equation} \begin{aligned} \label{eq:gamma}
\gamma = 3 ( I_2 - I_1 )^2 \left( \partder{I_2}{t} - \partder{I_1}{t} \right);
\end{aligned} \end{equation}

\begin{equation} \begin{aligned} \label{eq:lambda}
\lambda = I_1^2 \left( \partder{I_2}{t} - 
\partder{I_1}{t} \right) + 2 I_1 \partder{I_1}{t} (I_2 - I_1).
\end{aligned} \end{equation}

Запишемо модовий розподіл струму через нові позначення. Спростивши вираз 
можна отримати:

\textcolor{blue} { \begin{equation*} \begin{aligned}
j_m = - \frac{\sqrt{\mu_0}}{2\pi} 
\int_0^{2\pi} d \varphi \int \limits_0^\infty \rho d \rho
e^{-im\varphi} \left( i m \frac{J_m (\nu \rho)}{\rho \sqrt{\nu}}
j_\rho^\prime + \sqrt{\nu}
\frac{J_{m-1} (\nu \rho) - J_{m+1} (\nu \rho)}{2}
j_\varphi^\prime \right)
\end{aligned} \end{equation*} }
%
\textcolor{blue} { \begin{equation*} \begin{aligned}
j_m = - \frac{\sqrt{\mu_0}}{2\pi} 
\int_0^{2\pi} d \varphi \int \limits_0^\infty \rho d \rho 
e^{-im\varphi} \cdot \\ \cdot 
\left( i \sqrt{\nu} \frac{J_{m-1} (\nu \rho) + J_{m+1} (\nu \rho)}{2}
j_\rho^\prime + \sqrt{\nu} 
\frac{J_{m-1} (\nu \rho) - J_{m+1} (\nu \rho)}{2}
j_\varphi^\prime \right)
\end{aligned} \end{equation*} }
%
\textcolor{blue} { \begin{equation*} \begin{aligned}
j_m = - \frac{\sqrt{\mu_0} \sqrt{\nu}}{4\pi} 
\int_0^{2\pi} d \varphi \int \limits_0^\infty \rho d \rho 
e^{-im\varphi} \Big(
J_{m-1} (\nu \rho) ( i j_\rho^\prime + j_\varphi^\prime ) +
J_{m+1} (\nu \rho) ( i j_\rho^\prime - j_\varphi^\prime ) \Big)
\end{aligned} \end{equation*} }
%
\begin{equation} \begin{aligned}
j_1 = \frac{i A_0^3 \sqrt{\mu_0} \epsilon_0 \chi_e^{(3)} \sqrt{\nu}}{128}
\left( \frac{\mu_0 \mu}{\epsilon_0 \epsilon} \right)^{3/2}
\int_0^\infty \rho d \rho \cdot \\ \cdot
\Big( J_0 (\nu \rho) ( 3 \alpha + \beta + 3 \gamma + \lambda) + 
J_2 (\nu \rho) ( 3 \alpha + \beta - 3 \gamma - \lambda ) \Big)
\end{aligned} \end{equation}
%
\textcolor{blue} { \begin{equation*} \begin{aligned}
j_{-1} = \frac{i A_0^3 \sqrt{\mu_0} \epsilon_0 \chi_e^{(3)} \sqrt{\nu}}{128}
\left( \frac{\mu_0 \mu}{\epsilon_0 \epsilon} \right)^{3/2}
\int_0^\infty \rho d \rho \cdot \\ \cdot
\Big( J_2 (\nu \rho) ( 3 \alpha + \beta - 3 \gamma - \lambda ) + 
J_0 (\nu \rho) ( 3 \alpha + \beta + 3 \gamma + \lambda ) \Big)
\end{aligned} \end{equation*} }
%
\begin{equation} \begin{aligned}
j_{-1} = j_{1}
\end{aligned} \end{equation}
%
\textcolor{blue} { \begin{equation*} \begin{aligned}
j_{3} = - \frac{i A_0^3 \sqrt{\mu_0} \epsilon_0 \chi_e^{(3)} \sqrt{\nu}}{128}
\left( \frac{\mu_0 \mu}{\epsilon_0 \epsilon} \right)^{3/2}
\int_0^\infty \rho d \rho \cdot \\ \cdot
\Big( J_2 (\nu \rho) ( \alpha - \beta - \gamma + \lambda) + 
J_4 (\nu \rho) ( \alpha - \beta + \gamma - \lambda) \Big)
\end{aligned} \end{equation*} }
%
\begin{equation*} \begin{aligned}
j_{-3} = - \frac{i A_0^3 \sqrt{\mu_0} \epsilon_0 \chi_e^{(3)} \sqrt{\nu}}{128}
\left( \frac{\mu_0 \mu}{\epsilon_0 \epsilon} \right)^{3/2}
\int_0^\infty \rho d \rho \cdot \\ \cdot
\Big( J_4 (\nu \rho) ( \alpha - \beta + \gamma - \lambda) + 
J_2 (\nu \rho) ( \alpha - \beta - \gamma + \lambda) \Big)
\end{aligned} \end{equation*}
%
\begin{equation} \begin{aligned}
j_{-3} = j_{3}
\end{aligned} \end{equation}

\begin{figure}[htbp] \begin{center}
\includegraphics[scale=1.0]{normed_j1}
\caption{Нормована швидкість зміни моди $ j_1 $ у часі} \label{fig:mode1}
\end{center} \end{figure}

\begin{figure}[htbp] \begin{center}
\includegraphics[scale=1.0]{normed_j3}
\caption{Нормована швидкість зміни моди $ j_3 $ у часі} \label{fig:mode3}
\end{center} \end{figure}

На рис.~\ref{fig:mode1} і на рис.~\ref{fig:mode3} зображено величину модового 
струму, нормовану на час $ j_m / ct $ для різних значень $ ct-z $ та $ \nu $.
$ j_m / ct $ нормування вибрано задля того, щоб зручно показати модовий 
струм, як функцію всіх його змінних на одному графіку. Якщо розглянути вторинне
електричне поле як суперпозицію двох окремих хвиль (з кутовою модою $ m = 1 $ 
та $ m = 3 $), помічаємо, що внесок хвилі при $ m = 1 $ на порядок більший.

Для чисельного розрахунку невласного інтегралу по $ \rho $, що містяться в 
модових розподілах струму $ j_m $, зручно звузити межі інтегрування, 
користуючись областю визначення підінтегральної функцій $ S_2 $:

\textcolor{blue} { \begin{equation*} \begin{aligned}
(\rho - R)^2 \leq v^2t^2 - z^2 \leq (\rho + R)^2
\end{aligned} \end{equation*} }
%
\textcolor{blue} { \begin{equation*} \begin{aligned}
| \rho - R | - \sqrt{v^2t^2 - z^2} \leq 0 \leq \rho + R - \sqrt{v^2t^2 - z^2}
\end{aligned} \end{equation*} }
%
\textcolor{blue} { \begin{equation*} \begin{aligned}
- R - \sqrt{v^2t^2 - z^2} \leq - \rho \leq R - \sqrt{v^2t^2 - z^2}
\end{aligned} \end{equation*} }
%
\textcolor{blue} { \begin{equation*} \begin{aligned}
R + \sqrt{v^2t^2 - z^2} \geq \rho \geq - R + \sqrt{v^2t^2 - z^2}
\end{aligned} \end{equation*} }
%
\begin{equation} \begin{aligned}
\left| \sqrt{v^2t^2 - z^2} - R \right| \leq \rho \leq \sqrt{v^2t^2 - z^2} + R.
\end{aligned} \end{equation}

Задля відокремлення розмірних коефіцієнтів перевизначимо модовий розподіл 
струму:

\begin{equation} \begin{aligned}
j_m = - \frac{i A_0^3 \sqrt{\mu_0} \epsilon_0 \chi_e^{(3)} \sqrt{\nu}}{128}
\left( \frac{\mu_0 \mu}{\epsilon_0 \epsilon} \right)^{3/2} \hat{j_m}.
\end{aligned} \end{equation}

Так як поздовжня компонента вторинного струму $ J_z $ відсутня, рівняння 
Клейна-Гордона відносно поздовжнього електричного еволюційного коефіцієнту 
є однорідним, що згідно методу функції Рімана дає нульовий розв'язок для
цього коефіцієнту. Отже, як і у лінійному наближенні, електромагнітне поле 
з урахуванням ефектів слабкої нелінійності залишається ТЕ типу:

\textcolor{blue}{ \begin{equation*}
- \partial_{ct}(\mu I_n^e) - \partial_z V_n^e + \chi^2 e_n = 0
\end{equation*} }
%
\textcolor{blue}{ \begin{equation*}
\frac{\epsilon \mu}{ \sqrt{\epsilon_0 \mu_0}} 
\frac{\partial^2 e_n}{\partial t^2} - 
\frac{\partial^2 e_n}{\partial z^2} + \chi^2 e_n = 
- \frac{\sqrt{\mu_0}}{2 \pi c} 
\int_0^{2\pi} d \varphi 
\int_0^\infty \rho d \rho \Phi_n^* (\chi) \partder{J_z}{t} = 0
\end{equation*} }
%
\textcolor{blue}{ \begin{equation*}
e_n (z, t; \chi) = \iint_S j_n (t',z', \chi) G(t,t',z,z') dt' dz' = 0
\end{equation*} }
%
\textcolor{blue}{ \begin{equation*}
I_n^e = - \partial_{ct} (\epsilon e_n) - 
\frac{\sqrt{\mu_0}}{2 \pi} \int_0^{2\pi} d \varphi 
\int_0^{\infty} \rho d \rho \Phi_n^* (\chi) J_z
\end{equation*} }
%
\textcolor{blue}{ \begin{equation*}
\partial_{z} e_n = V_n^e
\end{equation*} }
%
\begin{equation} \label{eq:e_evolution}
e_n (z, t; \nu) = V_n^e (z, t; \nu) = I_n^e (z, t; \nu) = 0
\end{equation}

Виведення еволюційних коефіцієнтів починаємо з поздовжнього 
магнітного коефіцієнту $ h_m $. Для розглянутої фізичної моделі ізотропного
та стаціонарного середовища без втрат коефіцієнт $ h_m $ є розв'язком 
рівняння Клейна-Гордона. В лінійному випадку це рівняння містить електричну 
і магнітну сприйнятливості; для нелінійної нотації скористаємось поняттям 
ефективної сприйнятливості та методикою її врахування \cite{imp:Ziolkowski1993}.

\begin{equation} \label{eq:klein_gordon_nl}
\frac{(\epsilon + \chi_e^{(3)}) \mu}{c^2} 
\frac{\partial^2 h_m}{\partial t^2} - 
\frac{\partial^2 h_m}{\partial z^2} + 
\nu^2 h_m = j_m (t',z'; \nu),
\end{equation}
%
де $ j_m (t',z'; \nu) $ -- m-та мода дискретного розподілу стороннього 
джерела, $ \chi_e^{(3)} $ -- відносна нелінійна електрична сприйнятливість 
середовища при кубічній нелінійності, а величина $ \epsilon + \chi_e^{(3)} $ в 
літературі зустрічається, як ефективна нелінійна Керрівська сприйнятливість 
\cite{imp:Ziolkowski1993}. Тоді, розв'язком \eqref{eq:klein_gordon_nl} за 
методом функції Рімана буде \eqref{eq:klein_gordon_sol} з ядром у вигляді 
функції Рімана

\begin{equation*}
G(t,t',z,z') = \frac{v}{2} H \left( v' (t-t') - (z-z') \right)
J_0 \left( \nu \sqrt{v'^2 (t-t')^2 - (z-z')^2} \right),
\end{equation*}
%
де $ v' $ - швидкість світла в середовищі з урахуванням нелінійного 
Керрівського сповільнення

\begin{equation}
v' = \frac{c}{\sqrt{ \left(\epsilon + \chi_e^{(3)}\right) \mu}} = 
\left( \epsilon_0 
\left( \epsilon + \chi_e^{(3)} \right) \mu_0 \mu \right)^{-1/2}.
\end{equation}

Для отримання нелінійних поправок до напруженості електричного поля достатньо 
поперечного модового коефіцієнту $ V_m^h $, який лінійно залежить від $ h_m $

\textcolor{blue}{ \begin{equation*} 
h_m (z, t; \nu) = \iint_S j_m (t',z') G(t,t',z,z') dt' dz',
\end{equation*} }
%
\textcolor{blue}{ \begin{equation*}
V_m^h = - \mu \partial_{ct} (h_m)
\end{equation*} }
%
\textcolor{blue} { \begin{equation*} \begin{aligned} 
V_m^h = - \mu \partial_{ct} \int_0^\infty \int_0^\infty j_m (t',z') G dz' dt'
\end{aligned} \end{equation*} }
%
\textcolor{blue} { \begin{equation*} \begin{aligned} 
V_m^h = - \frac{v' \mu}{2c}
\partder{}{t} \int_0^\infty dz' \int_0^\infty 
dt' H \left( v' (t-t') - (z-z') \right) \cdot \\
\cdot J_0 \left( \nu \sqrt{v'^2 (t-t')^2 - (z-z')^2} \right) j_m (t',z')
\end{aligned} \end{equation*} }
%
\textcolor{blue} { \begin{equation*} \begin{aligned} 
V_m^h = - \frac{1}{2} \sqrt{\frac{\mu}{\epsilon + \chi_e^{(3)}}}
\partder{}{t} \int_0^\infty dz' \int_0^\infty 
dt' H \left( v' (t-t') - (z-z') \right) \cdot \\
\cdot J_0 \left( \nu \sqrt{v'^2 (t-t')^2 - (z-z')^2} \right) j_m (t',z')
\end{aligned} \end{equation*} }
%
\textcolor{blue} { \begin{equation*} \begin{aligned} 
H \left( v' (t-t') - z + z' \right) = 
H \left( - v't' - ( z - v't - z' ) \right) = \\
= H \left( v't' + ( z - v't - z' ) \right) = 
H \left( v't' - ( vt - z + z' ) \right)
\end{aligned} \end{equation*} }
%
\begin{equation} \begin{aligned} 
V_m^h = \frac{i A_0^3 \epsilon_0 \chi_e^{(3)}}{2^8}
\sqrt{\frac{\mu_0 \mu}{\epsilon + \chi_e^{(3)}}} 
\left( \frac{\mu_0 \mu}{\epsilon_0 \epsilon} \right)^{3/2} \sqrt{\nu} 
\cdot \\ \cdot \partder{}{t} \int_0^\infty dz' \int_0^{v't - z + z'} dt'
J_0 \left( \nu \sqrt{v'(t-t')^2 - (z-z')^2} \right) \hat{j_m} (vt',z')
\end{aligned} \end{equation}

Тепер, користуючись правилом інтегрування Лейбніца \cite{imp:NumRecipes2007}, 
спростимо отриманий вираз, взявши аналітично похідну за часом:
%
\textcolor{blue} { \begin{equation*} \begin{aligned}
\partder{}{\tau} 
J_0 \left( \nu \sqrt{\Delta \tau^2 - \Delta z^2} \right) = 
- \nu \Delta \tau 
\frac{J_1 \left( \nu \sqrt{\Delta \tau^2 - \Delta z^2} \right)}
{\sqrt{\Delta \tau^2 - \Delta z^2}}
\end{aligned} \end{equation*} }
%
\textcolor{blue} { \begin{equation*} \begin{aligned}
\partder{}{\theta} \int_{a(\theta)}^{b(\theta)} f(x,\theta) dx = 
\int_{a(\theta)}^{b(\theta)} \partder{f}{\theta} dx + 
f\big( b(\theta), \theta \big) \partder{b}{\theta} -
f\big( a(\theta), \theta \big) \partder{a}{\theta}
\end{aligned} \end{equation*} }
%
\textcolor{blue} { \begin{equation*} \begin{aligned}
\partder{}{t} \int_0^{vt - z + z'} dt'
J_0 \left( \nu \sqrt{v'(t-t')^2 - (z-z')^2} \right) \hat{j_m} (v't',z') = \\
= \int_0^{v't - z + z'} dv't' \partder{J_0}{v't} \hat{j_m} (vt',z') + \\
+ \left. 
J_0 \left( \nu \sqrt{v'(t-t')^2 - (z-z')^2} \right) \hat{j_m} (v't',z')
\right|^{v't' = v't - z + z'}
\end{aligned} \end{equation*} }
%
\textcolor{blue} { \begin{equation*} \begin{aligned}
\left. v'(t-t')^2 - (z-z')^2 \right|^{v't' = v't - z + z'} = 
v't^2 - (v't - z + z')^2 - (z-z')^2 = \\
= v't^2 - (v't - z)^2 + 2 z' (v't - z) + z'^2 - (z-z')^2 = \\
= 2 v't z - z^2 + 2 z' (v't - z) + z'^2 - z^2 + 2 z z' - z'^2 = \\
= 2 v't z - 2 z^2 + 2 z' (v't - z) + 2 z z' = 
2 \big( z (v't - z) + z' (v't - z) + z z' \big) = \\
2 \big( z (v't - z) + z' (v't - z + z) \big) = 
2 \big( z (v't - z) + v't z' \big) = \\
= 2 \big( v't z - z^2 + v't z' \big) = 2 v't (z + z') - 2 z^2
\end{aligned} \end{equation*} }
%
\textcolor{blue} { \begin{equation*} \begin{aligned}
\partder{}{t} \int_0^{v't - z + z'} dt'
J_0 \left( \nu \sqrt{v'(t-t')^2 - (z-z')^2} \right) \hat{j_m} (v't',z') = \\
= - \nu v (t-t') 
\frac{J_1 \left( \nu \sqrt{v'(t-t')^2 - (z-z')^2} \right)}
{\sqrt{v'(t-t')^2 - (z-z')^2}} % + \\
+ J_0 \left( \nu \sqrt{2 v't (z + z') - 2 z^2} \right) 
\hat{j_m} (v't - z + z',z')
\end{aligned} \end{equation*} }
%
\begin{equation} \begin{aligned} \label{eq:vmh_nl}
V_m^h = \frac{i A_0^3 \epsilon_0 \chi_e^{(3)}}{2^8}
\sqrt{\frac{\mu_0 \mu}{\epsilon + \chi_e^{(3)}}} 
\left( \frac{\mu_0 \mu}{\epsilon_0 \epsilon} \right)^{3/2} \sqrt{\nu}
\int_0^\infty dz' \cdot \\ \cdot 
\left\{ J_0 \left( \nu \sqrt{2 v't (z + z') - 2 z^2} \right) 
\hat{j_m} (v't - z + z',z') - \right. \\ 
\left. - \nu \int_0^\infty v'(t-t') 
\frac{J_1 \left( \nu \sqrt{v'(t-t')^2 - (z-z')^2} \right)}
{\sqrt{v'(t-t')^2 - (z-z')^2}} \hat{j_m} (v't',z') dt' \right\}
\end{aligned} \end{equation}

\begin{equation} \begin{aligned} \label{eq:vmh_norm}
V_m^h = \frac{i A_0^3 \epsilon_0 \chi_e^{(3)}}{2^8}
\sqrt{\frac{\mu_0 \mu}{\epsilon + \chi_e^{(3)}}} 
\left( \frac{\mu_0 \mu}{\epsilon_0 \epsilon} \right)^{3/2} 
\sqrt{\nu} \hat{V_m^h}
\end{aligned} \end{equation}

Як видно з \eqref{eq:vmh_nl}, функція $ \func{V_m^h}{\nu; z, t} $ парна відносно 
порядкового номеру моди, тому:

\begin{equation} \begin{aligned} \label{eq:vp1_vm1}
V_1^h = V_{-1}^h
\end{aligned} \end{equation}

\begin{equation} \begin{aligned} \label{eq:vp3_vm3}
V_3^h = V_{-3}^h
\end{aligned} \end{equation}

\textcolor{red} { \begin{equation*} \begin{aligned}
\frac{J_1 \left( \nu \sqrt{\Delta \tau^2 - \Delta z^2} \right)}
{\nu \sqrt{\Delta \tau^2 - \Delta z^2}} =
\frac{J_0 \left( \nu \sqrt{\Delta \tau^2 - \Delta z^2} \right) +
J_2 \left( \nu \sqrt{\Delta \tau^2 - \Delta z^2} \right)}{2}
\end{aligned} \end{equation*} }


%%%%%%%%%%%%%%%%%%%%%%%%%%%%%%%%%%%%%%%%%%%%%%%%%%%%%%%%%%%%%%%%%%%%%%%%%%%%%%%%
\section{Числове моделювання нелінійного поля}

Для розв'язання задачі випромінювання у вільний простір, згідно методу 
модового базису, необхідно підставити в розклад компонентів поля знайдені 
еволюційні коефіцієнти. Як було доведено в \eqref{eq:e_evolution}, всі електричні 
еволюційні коефіцієнти нульові, а отже, за визначенням поздовжня електрична
компонента відсутня як і у лінійному випадку:

\begin{equation} \label{eq:ez_kerr}
E'_z = \frac{1}{\sqrt{\epsilon_0}} \sum_{n=-\infty}^{\infty}
\int_0^\infty \chi^2 d \chi e_n (\nu | v't, z) \Phi_n (\nu | \rho, \phi) = 0,
\end{equation}
%
де $ \Phi_n $ -- базисна функція розкладу \cite{imp:Tretyakov2010}.

З \eqref{eq:ez_kerr} очевидно, що пласка TE хвиля залишається TE при поширення 
крізь нелінійне середовище при слабких ефектах самодії навіть при 
нестаціонарному збудженні.

Поперечні електричні компоненти поля, в свою чергу, визначені наступним 
розкладом \cite{imp:Dumin2000}:

\begin{equation} \begin{aligned}
\vect{E_\perp} = \frac{1}{\sqrt{\epsilon_0}} \left( 
\sum \limits_{m=-\infty}^{\infty} \int \limits_{0}^{\infty} 
d \nu V_m^h \crossprod{ \nabla_\perp \Psi_m }{ \vect{z_0} } +
\sum \limits_{n=-\infty}^{\infty} \int \limits_{0}^{\infty}
d \chi V_n^e \nabla_\perp \Phi_n \right),
\end{aligned} \end{equation}
%
де $ \Psi_m $ та $ \Phi_n  $ є базисні функції розкладу поля, а $ V_m^h $
та $ V_n^e $ - еволюційні коефіцієнти, відомі з виразів \eqref{eq:e_evolution}
та \eqref{ eq:vmh_nl}. Користуючись властивостями симетрії 
\eqref{eq:vp1_vm1} та \eqref{eq:vp3_vm3}, не важко помітити, що
%
\textcolor{blue} { \begin{equation*} \begin{aligned}
\crossprod{ \nabla_\perp \Psi_m }{ \vect{z_0} } = 
- e^{im\varphi} \left( \vect{\varphi_0} \sqrt{\nu} 
\frac{J_{m-1} (\nu \rho) - J_{m+1} (\nu \rho)}{2} - 
i m \vect{\rho_0} \frac{J_m (\nu \rho)}{ \rho \sqrt{\nu}} \right)
\end{aligned} \end{equation*} }
%
\textcolor{blue} { \begin{equation*} \begin{aligned}
E'_\rho = \frac{1}{\sqrt{\epsilon_0}} \sum_{m=-\infty}^{\infty} 
i m e^{im\varphi} \int_{0}^{\infty} \frac{d \nu}{\sqrt{\nu}} 
V_m^h \frac{J_m(\nu \rho)}{\rho}
\end{aligned} \end{equation*} }
%
\textcolor{blue} { \begin{equation*} \begin{aligned}
\sum_{m=-\infty}^\infty m e^{im \varphi} V_m^h J_m(\nu \rho) = \\ =
  e^{  i \varphi} V_{ 1}^h J_{ 1}(\nu \rho) - 
  e^{- i \varphi} V_{-1}^h J_{-1}(\nu \rho) + \\ +
3 e^{ 3i \varphi} V_{ 3}^h J_{ 3}(\nu \rho) - 
3 e^{-3i \varphi} V_{-3}^h J_{-3}(\nu \rho) = \\ =
\left( e^{ i\varphi} + e^{- i\varphi} \right) V_1^h J_1(\nu \rho) + 
3 \left( e^{3i\varphi} + e^{-3i\varphi} \right) V_3^h J_3(\nu \rho) = \\
= 2 \cos \varphi V_1^h J_1(\nu \rho) + 
6 \cos 3 \varphi V_3^h J_3(\nu \rho)
\end{aligned} \end{equation*} }
%
\begin{equation} \begin{aligned}
E'_\rho = \frac{2 i \cos \varphi}{\sqrt{\epsilon_0}}
\int_0^\infty \sqrt{\nu} d \nu V_1^h \frac{J_1(\nu \rho)}{\nu \rho} +
\frac{6 i \cos 3 \varphi}{\sqrt{\epsilon_0}}
\int_0^\infty \sqrt{\nu} d \nu V_3^h \frac{J_3(\nu \rho)}{\nu \rho}.
\end{aligned} \end{equation}

Тепер, користуючись раніше введеним нормуванням еволюційного коефіцієнта 
\eqref{eq:vmh_norm}, запишемо вираз для нелінійної поправки до напруженості 
електричного поля $ E'_\rho $ в зручному для порівняння з лінійним наближенням
\eqref{eq:linear_e_cyl} вигляді:

\textcolor{blue} { \begin{equation*} \begin{aligned}
E'_\rho = - \frac{A_0^3 \epsilon_0 \chi_e^{(3)}}{2^7}
\sqrt{\frac{\mu_0 \mu}{\epsilon_0 \left( \epsilon + \chi_e^{(3)} \right)}} 
\left( \frac{\mu_0 \mu}{\epsilon_0 \epsilon} \right)^{3/2} \cdot \\
\cdot \int_0^\infty \nu d \nu \left(
\cos \varphi \hat{V_1^h} \frac{J_1(\nu \rho)}{\nu \rho} +
3 \cos 3\varphi \hat{V_3^h} \frac{J_3(\nu \rho)}{\nu \rho} 
\right)
\end{aligned} \end{equation*} }

\textcolor{blue} { \begin{equation*} \begin{aligned}
\epsilon_0 \chi_e^{(3)}
\sqrt{\frac{\mu_0 \mu}{\epsilon_0 \left( \epsilon + \chi_e^{(3)} \right)}} 
\left( \frac{\mu_0 \mu}{\epsilon_0 \epsilon} \right)^{3/2} 
\frac{\sqrt{\epsilon}}{\sqrt{\epsilon}} =
\frac{\epsilon_0 \sqrt{\epsilon} \chi_e^{(3)}}
{\sqrt{\epsilon + \chi_e^{(3)}}} 
\left( \frac{\mu_0 \mu}{\epsilon_0 \epsilon} \right)^2
\end{aligned} \end{equation*} }
%
\textcolor{blue} { \begin{equation*} \begin{aligned}
\frac{\sqrt{\epsilon}}
{\sqrt{\epsilon + \chi_e^{(3)}}} =
\frac{\sqrt{\epsilon \mu}}{c} 
\frac{c}{\sqrt{\mu \left( \epsilon + \chi_e^{(3)} \right)}} = 
\frac{v'}{v}
\end{aligned} \end{equation*} }

\begin{equation} \begin{aligned} \label{eq:erho_kerr}
E'_\rho = - \frac{\epsilon_0 \chi_e^{(3)} A_0^3}{2^7}
\frac{v'}{v}
\left( \frac{\mu_0 \mu}{\epsilon_0 \epsilon} \right)^2
\left(\hat{E}_\rho^{(1)} \cos \varphi +
\hat{E}_\rho^{(3)} \cos 3 \varphi \right),
\end{aligned} \end{equation}
%
де відношення $ v'/v < 1 $ є амплітудним коефіцієнтом що залежить від 
сповільнення хвилі при врахування нелінійності Керра - функція, подібна 
за змістом та значенням до інтегральних виразів $ I_1 $ та $ I_2 $ з 
розв'язку у наближенні лінійного поширення \eqref{eq:linear_e_cyl}:

\begin{equation} \begin{aligned} \label{eq:erho_norm}
\hat{E}_\rho^{(m)} = \hat{E}_\rho^{(m)} (vt,\rho,z) = 
\int_0^\infty \nu d \nu \frac{m J_m(\nu \rho)}{\nu \rho} 
\hat{V_m^h} (\nu | vt,\rho,z).
\end{aligned} \end{equation}

Також серед множників помічаємо квадрат імпедансу вільного простору 
$ (\mu_0 \mu) / (\epsilon_0 \epsilon) $, куб максимальної амплітуди 
нестаціонарного струму збедження $ A_0^3 $ та абсолютну нелінійну 
сприйнятливість $ \epsilon_0 \chi_e^{(3)} $.

З виразу \eqref{eq:erho_kerr} добре видно один з проявів амплітудної
самодії електромагнітного випромінювання. Як видно з лінійного розв'язку,
збільшення коефіцієнту заломлення середовища ``розтягує'' нестаціонарний 
імпульс. Іншим проявом керрівської нелінійності в виразі \eqref{eq:erho_kerr} 
є поява амплітудного множника, що залежить від швидкості розповсюдження 
хвилі в середовищі. Варто зазначити, що цей ефект малозначний та складає 
лише $ 0.01\% $ від поля поправки, а тому ним можна знехтувати в випадку, 
що розглядається. Варто перевірити його внесок при сильній нелінійній 
самодії. Даний ефект можна сприймати як поправку до імпедансу вільного 
простору при врахуванні нелінійної сприйнятливості третього порядку:

В компоненті $ \hat{E}_\rho^{(m)} $, як і в інших компонентах поля енергія 
розподіляється поміж чотирьох мод $ m = \{ \pm 1, \pm 3 \} $. Поява мод вищих 
порядків викликає відтік енергії на паразитні пелюстки діаграми діаграми 
напрямленості по азимутальному куту $ \varphi $. Тобто, як видно з 
\eqref{eq:erho_kerr}, окрім кутової залежності $ \cos \varphi $ з'являється
ще і $ \cos 3 \varphi $. Цікаво, що такий самій ефект спостерігається і при 
поширенні пласких хвиль з урахуванням керрівської слабкої нелінійності 
\textcolor{red}{[ПОСИЛАННЯ]}.

\begin{equation*} \begin{aligned}
\frac{v'}{v}
\left( \frac{\mu_0 \mu}{\epsilon_0 \epsilon} \right)^2 = 
\sqrt{\frac{\mu_0 \mu}{\epsilon_0 \left( \epsilon + \chi_e^{(3)} \right)}}
\sqrt[3]{\frac{\mu_0 \mu}{\epsilon_0 \epsilon}} 
\end{aligned} \end{equation*}

Розв'язок відносно нелінійної поправки \eqref{eq:erho_kerr} містить кутову 
залежність та константні коефіцієнти в явному вигляді. Залежність від інших 
$ v't, \rho, z $ змінних приховано в невласному кратному інтегралі дійсної 
області значень $ \hat{E}_\rho^{(m)} $:

\textcolor{blue} { \begin{equation*} \begin{aligned}
\hat{V_m^h} = \int_0^z dz' 
\left\{ J_0 \left( \nu \sqrt{2 vt (z + z') - 2 z^2} \right) 
\hat{j_m} (v't - z + z',z') - \right. \\ 
\left. - \nu \int_0^{v't - z + z'} dvt' v (t-t') 
\frac{J_1 \left( \nu \sqrt{v'(t-t')^2 - (z-z')^2} \right)}
{\sqrt{v'(t-t')^2 - (z-z')^2}} \hat{j_m} (vt',z')  \right\}
\end{aligned} \end{equation*} }

\textcolor{blue} { \begin{equation*} \begin{aligned}
j_1 = \frac{i A_0^3 \sqrt{\mu_0} \epsilon_0 \chi_e^{(3)} \sqrt{\nu}}{128}
\left( \frac{\mu_0 \mu}{\epsilon_0 \epsilon} \right)^{3/2}
\int_0^\infty \rho d \rho \cdot \\ \cdot
\Big( J_0 (\nu \rho) ( 3 \alpha + \beta + 3 \gamma + \lambda) - 
J_2 (\nu \rho) ( 3 \alpha + \beta - 3 \gamma - \lambda ) \Big)
\end{aligned} \end{equation*} }

\begin{equation} \begin{aligned} \label{eq:vmh_final}
\hat{V_m^h} = \int_{0}^{\infty} dz'
\left\{ J_0 \left( \nu \sqrt{2 v't (z + z') - 2 z^2} \right) 
\int_{0}^{\infty} \rho' d \rho'
f (\rho',v't - z + z',z') - \right. \\ 
\left. - \nu \int_{0}^{\infty} dv't' v (t-t') 
\frac{J_1 \left( \nu \sqrt{v'(t-t')^2 - (z-z')^2} \right)}
{\sqrt{v'(t-t')^2 - (z-z')^2}} 
\int_{0}^{\infty} \rho' d\rho'
f_m (\rho',v't',z')  \right\},
\end{aligned} \end{equation}
%
де $ f ( \nu | v't', \rho', z') $ - лінійна комбінація функцій виду
$ I_i I_j \partder{I_i}{vt} $ введених раніше \eqref{eq:alpha} - 
\eqref{eq:lambda}, як $ \alpha, \beta, \gamma, \lambda $:

\begin{equation} \begin{aligned}
f_1 ( \nu | v't', \rho', z') = 
J_0 (\nu \rho') (3 \alpha + \beta + 3 \gamma + \lambda) + \\
+ J_2 (\nu \rho') (3 \alpha + \beta - 3 \gamma - \lambda);
\end{aligned} \end{equation}

\begin{equation} \begin{aligned}
f_3 ( \nu | v't', \rho', z') = 
J_2 (\nu \rho') (\alpha + \beta + \gamma - \lambda) + \\
+ J_4 (\nu \rho') (\alpha - \beta - \gamma + \lambda).
\end{aligned} \end{equation}

Інтегрування за штрихованими змінними є математичною формалізацією 
суперпозиції поля точкових джерел розподілених по ближній зоні апертури
в межах 
$ 0 \leq \rho' \leq \rho $, $ 0 \leq z' \leq z $ та $ 0 \leq t' \leq t $. 
Користуючись цим, а також областю визначення підінтегральних функцій та 
принципом причинності функції Рімана $ v'(t-t')-(z-z') > 0 $, можна обмежити 
область інтегрування в останньому виразі:

\begin{equation} \begin{aligned}
0 \leq v't' \leq v't - z + z';
\end{aligned} \end{equation}

\begin{equation} \begin{aligned}
0 \leq z' \leq \min(z,2R),
\end{aligned} \end{equation}
%
де верхня межа значень $ z' $ також обмежена за рахунок попереднього аналізу 
енергетичних властивостей випромінювання в ближній зоні. Також, для окремих 
інтегралів за змінними $ \rho' $ область інтегрування буде різною. Для 
інтегралу в першому доданку \eqref{eq:vmh_final}

\begin{equation} \begin{aligned}
\left| \sqrt{v'^2t'^2 - z'^2} - R \right| \leq \rho' \leq 
\sqrt{v'^2t'^2 - z'^2} + R,
\end{aligned} \end{equation}
%
а в другому при $ vt' = vt - z + z' $, відповідно:

\textcolor{blue} { \begin{equation*} \begin{aligned}
\left. v'^2 t'^2 - z'^2 \right|^{vt' = v't - z + z'} = 
(v't - z + z')^2 - z'^2 = (v't - z)^2 + 2 z' (v't - z) = \\
(v't - z) (v't - z + 2 z') = (v't - z) (v't + z') + (v't - z) (z + z')
\end{aligned} \end{equation*} }

\begin{equation} \begin{aligned}
\left| \sqrt{(v't - z) (v't - z + 2 z')} - R \right| \leq \rho' \leq 
\sqrt{(v't - z) (v't - z + 2 z')} + R.
\end{aligned} \end{equation}

Аналітичний розрахунок нормованого еволюційного коефіцієнту $ \hat{V_m^h} $
є недоцільним та може виявитись взагалі неможливим, тому залишаються лише
числові квадратурні методи, які дадуть гарну точність в випадку,
кратних визначених інтегралів. Зовнішнім інтегралом в \eqref{eq:erho_kerr} є 
інтеграл за неперервним спектральним параметром $ \nu $. На жаль, фізичного
обґрунтування для обмеження цієї області інтегрування немає, а отже,
доцільно розділити чисельне розв'язання на два етапи:

\begin{enumerate}
	\item числовий розрахунок нормованого еволюційного коефіцієнту 
	$ \hat{V_m^h} $ для деякої обмеженої зверху області значень 
	параметру $ \nu $;
	\item аналіз частотних характеристик під-інтегральної функції за 
	$ \nu $ та обмеження області інтегрування на основі аналізу, що 
	дозволить розрахувати абсолютну похибку чисельного методу.
\end{enumerate}

Замість евристичного аналізу, можна застосувати ітеративним метод 
квадратурного інтегрування, де верхня межа інтегрування визначається 
шляхом мінімізації відхилень значень інтегралу на сусідніх ітераціях.

\begin{figure}[htbp] \begin{center}
\includegraphics[scale=0.99]{KerrEvo205}
\caption{Нормований еволюційний коефіцієнт $ \hat{V_m^h} $ для 
різних $ m $ при $ v't' = 2.05, z' = 2 $.} 
\label{fig:KerrEvo205}
\end{center} \end{figure}

\begin{figure}[htbp] \begin{center}
\includegraphics[scale=0.99]{KerrEvo215}
\caption{Нормований еволюційний коефіцієнт $ \hat{V_m^h} $ для 
різних $ m $ при $ v't' = 2.15, z' = 2 $.} 
\label{fig:KerrEvo215}
\end{center} \end{figure}

\begin{figure}[htbp] \begin{center}
\includegraphics[scale=0.99]{KerrEvo220}
\caption{Нормований еволюційний коефіцієнт $ \hat{V_m^h} $ для 
різних $ m $ при $ v't' = 2.20, z' = 2 $.} 
\label{fig:KerrEvo220}
\end{center} \end{figure}

\begin{figure}[htbp] \begin{center}
\includegraphics[scale=0.99]{KerrEvo225}
\caption{Нормований еволюційний коефіцієнт $ \hat{V_m^h} $ для 
різних $ m $ при $ v't' = 2.25, z' = 2 $.} 
\label{fig:KerrEvo225}
\end{center} \end{figure}

На рис.~\ref{fig:KerrEvo205}--\ref{fig:KerrEvo225} зображено чисельно 
розрахований нормований еволюційний коефіцієнт $ \hat{V_m^h} $ як функцію 
спектрального параметру $ \nu $ для окремих точок модового розкладу розподілу 
вторинно струму. Енергія отриманого поля-поправки для перехідної функції 
кругової апертури розподіляється поміж двох значень дискретного спектрального 
параметру $ m = \{ 1, 3\} $, отже на графіку зобразимо нормований еволюційний 
коефіцієнт $ \hat{V_m^h} $ для кожного з них.

З графіків спостерігаємо на порядок менший енергетичний внесок моди $ m = 3 $, 
що у купі з відсутністю механізму міжмодовогопереходу енергії для даного типу 
хвилі дозволяє спростити вираз нелінійної поправки знехтувавши внеском моди 
$ m = 3 $. На графіках спостерігається чітка періодичність еволюційного 
коефіцієнту відносно $ \nu $. На відміну від лінійного випадку спостерігається 
перемноження періодів коливань, тобто замість одного періоду $ v^2t^2 - z^2 $ 
отримаємо ще і його квадрат. Відметімо, що при відвалені від події джерела 
більший період коливання зростає, а амплітуда коливань зменшується при 
великих значеннях $ \nu $. Тоді, можна стверджувати, що даний інтеграл 
сходиться в довільній точці спостереження та числовий розрахунок 
поля-поправки спрощується.

Застосуємо аналогічний підхід для виведення поправки до $ \varphi $ проекції 
вектору напруженості електричного поля:

\textcolor{blue} { \begin{equation*} \begin{aligned}
E_\varphi = - \frac{1}{2 \sqrt{\epsilon_0}} \sum_{m=-\infty}^{\infty} 
e^{im\varphi} \int_{0}^{\infty} \sqrt{\nu} d \nu 
V_m^h \left( J_{m-1} (\nu \rho) - J_{m+1} (\nu \rho) \right)
\end{aligned} \end{equation*} }

\textcolor{blue} { \begin{equation*} \begin{aligned}
\sum_{m=-\infty}^\infty e^{im \varphi}
V_m^h \left( J_{m-1} (\nu \rho) - J_{m+1} (\nu \rho) \right) = \\
e^{  i \varphi} V_{ 1}^h \left( J_0 (\nu \rho) - J_2 (\nu \rho) \right) +
e^{- i \varphi} V_{-1}^h \left( J_2 (\nu \rho) - J_0 (\nu \rho) \right) + \\
e^{ 3i \varphi} V_{ 3}^h \left( J_2 (\nu \rho) - J_4 (\nu \rho) \right) +
e^{-3i \varphi} V_{-3}^h \left( J_4 (\nu \rho) - J_2 (\nu \rho) \right) = \\
= \left( e^{  i \varphi} - e^{- i \varphi} \right)
\left( J_0 (\nu \rho) - J_2 (\nu \rho) \right) +
\left( e^{ 3i \varphi} - e^{-3i \varphi} \right) 
\left( J_2 (\nu \rho) - J_4 (\nu \rho) \right) = \\
= 2i \sin \varphi \left( J_0 (\nu \rho) - J_2 (\nu \rho) \right) +
2i \sin 3 \varphi \left( J_2 (\nu \rho) - J_4 (\nu \rho) \right)
\end{aligned} \end{equation*} }

\textcolor{blue} { \begin{equation*} \begin{aligned}
E_\varphi =
- \frac{i \sin \varphi}{\sqrt{\epsilon_0}} \int_0^\infty d \nu
\sqrt{\nu} V_1^h \left( J_0 (\nu \rho) - J_2 (\nu \rho) \right) - \\
- \frac{i \sin 3 \varphi}{\sqrt{\epsilon_0}} \int_0^\infty d \nu
\sqrt{\nu} V_3^h \left( J_2 (\nu \rho) - J_4 (\nu \rho) \right)
\end{aligned} \end{equation*} }

\begin{equation} \begin{aligned} \label{eq:ephi_kerr}
E'_\varphi = \frac{\epsilon_0 \chi_e^{(3)} A_0^3}{2^7}
\frac{v_{NL}}{v_{LN}}
\left( \frac{\mu_0 \mu}{\epsilon_0 \epsilon} \right)^2
\left(\hat{E}_\varphi^{(1)} \sin \varphi +
\hat{E}_\varphi^{(3)} \sin 3 \varphi \right)
\end{aligned} \end{equation}
%
де 

\begin{equation} \begin{aligned} \label{eq:ephi_norm}
\hat{E}_\varphi^{(m)} (vt, \rho, z) = 
\int_0^\infty \nu d \nu V_m^h (\nu | vt, \rho, z)
\left( J_{m-1} (\nu \rho) - J_{m+1} (\nu \rho) \right)
\end{aligned} \end{equation}

Для аналізу та порівняння з лінійним розв'язком зручно перейти до 
декартових проекцій вектору напруженості, тоді

\textcolor{blue} { \begin{equation*} \begin{aligned}
E'_x = E'_\rho \cos \varphi - E'_\varphi \sin \varphi
\end{aligned} \end{equation*} }
%
\textcolor{blue} { \begin{equation*} \begin{aligned}
E'_x = - \frac{\epsilon_0 \chi_e^{(3)} A_0^3}{2^7} \frac{v_{NL}}{v_{LN}}
\left( \frac{\mu_0 \mu}{\epsilon_0 \epsilon} \right)^2 \cdot \\ \cdot
\left(\hat{E}_\rho^{(1)} \cos^2 \varphi +
\hat{E}_\rho^{(3)} \cos \varphi \cos 3 \varphi - 
\hat{E}_\varphi^{(1)} \sin^2 \varphi -
\hat{E}_\varphi^{(3)} \sin \varphi \sin 3 \varphi \right)
\end{aligned} \end{equation*} }
%
\textcolor{blue} { \begin{equation*} \begin{aligned}
E'_x = - \frac{\epsilon_0 \chi_e^{(3)} A_0^3}{2^7} \frac{v_{NL}}{v_{LN}}
\left( \frac{\mu_0 \mu}{\epsilon_0 \epsilon} \right)^2 \cdot \\ \cdot
\int_0^\infty \nu d \nu V_1^h \left( 
\left( J_0 (\nu \rho) + J_2 (\nu \rho) \right) \cos^2 \varphi -
\left( J_0 (\nu \rho) - J_2 (\nu \rho) \right) \sin^2 \varphi \right) - \\
- \frac{\epsilon_0 \chi_e^{(3)} A_0^3}{2^7} \frac{v_{NL}}{v_{LN}}
\left( \frac{\mu_0 \mu}{\epsilon_0 \epsilon} \right)^2 \cdot \\ \cdot
\int_0^\infty \nu d \nu V_3^h \left( 
\left( J_2 (\nu \rho) + J_4 (\nu \rho) \right) \cos \varphi \cos 3 \varphi -
\left( J_2 (\nu \rho) - J_4 (\nu \rho) \right) \sin \varphi \sin 3 \varphi 
\right)
\end{aligned} \end{equation*} }

\textcolor{red} {TODO: ефект самофокусування за рахунок залежності від кута}

\begin{figure}[htbp] \begin{center}
\includegraphics[scale=0.99]{KerrTRz2}
\caption{Нелінійна поправка до поля в точці спостереження $ z = 2 $.} 
\label{fig:KerrTRz2}
\end{center} \end{figure}

\begin{figure}[htbp] \begin{center}
\includegraphics[scale=0.99]{KerrTRz3}
\caption{Нелінійна поправка до поля в точці спостереження $ z = 3 $.} 
\label{fig:KerrTRz3}
\end{center} \end{figure}

На рис.~\ref{fig:KerrTRz2}--\ref{fig:KerrTRz3} зображено поправку
до перехідної функції плаского диску, що враховує слабку нелінійну взаємодію 
поля з середовищем в двох точках на осі випромінювання $ z = 2R $ та $ z = 3R $.
З графіків видно, що дисперсія імпульсу зменшується, а крутизна його фронтів і
максимальна амплітуда зростають. Цей ефект відомий в літературі, як ефект 
нелінійного самофокусування імпульсу.

%%%%%%%%%%%%%%%%%%%%%%%%%%%%%%%%%%%%%%%%%%%%%%%%%%%%%%%%%%%%%%%%%%%%%%%%%%%%%%%%
\section{Поширення прямокутного імпульсу в нелінійному середовищі}

\textcolor{red} {TODO: Порушення закону збереження та принципу суперпозиції}

\textcolor{red} {TODO: Може вдається якось виділити залежність від 
тривалості імпульсу аналітично???}

%%%%%%%%%%%%%%%%%%%%%%%%%%%%%%%%%%%%%%%%%%%%%%%%%%%%%%%%%%%%%%%%%%%%%%%%%%%%%%%%
\section{Узагальнення для слабкої нелінійності}

Опираючись на геометрію джерела та на властивості модового базису, можна 
довести, що 

\textcolor{red} { \begin{equation} \begin{aligned} \label{eq:erho_norm}
\hat{E}_\rho^{(1)} = \int_0^\infty \nu d \nu 
\frac{J_1(\nu \rho)}{\nu \rho} \hat{V_1^h} \approx
\int_0^\infty \nu d \nu \frac{J_1(\nu \rho)}{\nu \rho} 
\frac{J_1(\nu R) J_0(\nu \sqrt{v^2t^2-z^2})}{\nu} = I_1,
\end{aligned} \end{equation} }
%
отже компонент з лінійною залежністю від кута, повторює за формою 
імпульс, отриманий у лінійному наближенні, та менший за амплітудою на 
декілька порядків, а отже, його внеском можна знехтувати.

\begin{equation*} \begin{aligned}
\lim_{n \to \infty} 
\sqrt{ \frac{\epsilon}{ \epsilon + \chi_e^{(2n+1)}} } = 1
\end{aligned} \end{equation*}

\textcolor{blue} { \begin{equation*} \begin{aligned}
E'_\rho = - \frac{\epsilon_0 \chi_e^{(3)} A_0^3}{2^7}
\frac{v_{NL}}{v_{LN}}
\left( \frac{\mu_0 \mu}{\epsilon_0 \epsilon} \right)^2
\left(\hat{E}_\rho^{(1)} \cos \varphi +
\hat{E}_\rho^{(3)} \cos 3 \varphi \right)
\end{aligned} \end{equation*} }

\textcolor{blue} { \begin{equation*} \begin{aligned}
E'_\rho = - \frac{\epsilon_0 \chi_e^{(3)} A_0^3}{2^7}
\frac{v_{NL}}{v_{LN}}
\left( \frac{\mu_0 \mu}{\epsilon_0 \epsilon} \right)^2 
\hat{E}_\rho^{(3)} \cos 3 \varphi
\end{aligned} \end{equation*} }

\begin{equation*} \begin{aligned}
E'_\rho = - \frac{1}{2} \sum_{n=1}^{\infty} 
\frac{\epsilon_0 \chi_e^{(2n+1)} A_0^{2n+1} }{ 4^{2n+1} }
\left( \frac{\mu_0 \mu}{\epsilon_0 \epsilon} \right)^{n+1}
\hat{E}_\rho^{(2n+1)} \cos (2n + 1) \varphi
\end{aligned} \end{equation*}

%%%%%%%%%%%%%%%%%%%%%%%%%%%%%%%%%%%%%%%%%%%%%%%%%%%%%%%%%%%%%%%%%%%%%%%%%%%%%%
\section*{Висновки до розділу \ref{ch:nonlinear}}

Описано підхід до отримання значень електромагнітного поля в слабонелінійному 
середовищі. Отримано та проаналізовано розв’язок задачі випромінювання 
нестаціонарного імпульсного поля, що породжено пласким диском з рівномірно 
розподіленим електричним струмом в наближенні слабкої нелінійності. Хоча сам 
розв’язок представлений тільки у вигляді кратного інтегралу від циліндричних 
функцій, проведено числовий розрахунок напруженості поля.

Застосований підхід дозволив зафіксувати ефекти механізму нелінійної керрівської 
взаємодії поля з середовища поширення, що пов'язані з відтоком енергії до 
мод вищих порядків.

Так як існуючі розв’язки задачі в лінійному наближені не задовольняли постанові 
задачі, для розв'язання в нелінійному наближенні методом теорії збурень
застосовано аналітичний розв'язок, що отримано в минулому розділі.

% Проаналізовано електричне поле, породжене пласким диском з різними часовими 
% залежностями збуджувального розподілу струму - у вигляді функції Хевісайда та 
% прямокутного імпульсу. Вигляд цього джерела свідчить про появу нових кутових 
% мод у випроміненому електромагнітному полі.

Результати цього розділу відображені в роботах автора 
\cite{my:Vesnik2015, my:Vesnik2017, my:Vesnik2017-2, my:MMET2014,
my:UWBUSIS2014, my:ICATT2015, my:UWBUSIS2016, my:KPI2016, my:DIPED2019}.

% \begin{tabular}{ | l | l | }
% \hline 
% Призначення змінної                          & Область визначенням       \\ 
% \hline
% Відстань, що проходить сигнал за час $t$     & $ 0 \le vt' \le vt $      \\ 
% \hline
% Відстань, від точки спостереження до джерела & $ 0 \le z' \le z $        \\  
% \hline
% Принцип причинності для проміжних подій      & $ 0 < vt - vt' - z + z' $ \\ 
% \hline
% Наслідок з 3 та 1                            & $ 0 < vt' < vt - z + z' $ \\ 
% \hline
% \end{tabular}
