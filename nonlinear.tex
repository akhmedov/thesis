\chapter{Розповсюдження випромінювання плаского диску в нелінійному середовищі}
\label{ch:nonlinear}

%%%%%%%%%%%%%%%%%%%%%%%%%%%%%%%%%%%%%%%%%%%%%%%%%%%%%%%%%%%%%%%%%%%%%%%%%%%%%%%%
\section{Матеріальні рівняння, як модель нелінійного середовища}

Взаємодію поля крізь середовище, завдяки теорії суперпозиції, можна
представити у вигляді додаткового стороннього джерела поля, що буде 
просторово розподілене в усій області розповсюдження породжувальної 
сильної хвилі. Для сильних хвиль, що мають імпульсну природу 
крім просторового розподілу доводиться розглядати, ще і причинний 
зв'язок. Назвемо таке джерело вторинним.

Розглянемо модель, де характер взаємодії електромагнітного поля і середовища 
задається в матеріальних рівняннях, де поляризація та намагніченість 
розглядаються, як \textcolor{red}{джерела} електромагнітної індукції.

\begin{equation*}
\vect{D} = \epsilon_0 \vect{E} + \vect{P} \left( \vect{E}, \vect{H} \right) =
\epsilon_0 \vect{E} + \epsilon_0 \chi_e \left( \vect{E}, \vect{H} \right)
\end{equation*}

\begin{equation*}
\vect{B} = \mu_0 \vect{H} + \mu_0 \vect{M} \left( \vect{E}, \vect{H} \right) =
\mu_0 \vect{H} + \mu_0 \chi_m \left( \vect{E}, \vect{H} \right)
\end{equation*}

Виключимо з розглядання гіротропні і не-хіральні середовища, тоді взаємний 
вплив магнітної та електричної індукції зникне і вектор поляризації стане 
функцією лише електричної напруженості, а намагніченість - лише магнітної
напруженості. Також, клас середовищ, що розглядається, обмежимо сталими, 
однорідними та ізотропними властивостями. Розглянемо середовище з лінійною 
магнітною індукцією

\begin{equation}
\vect{B} = \mu_0 \vect{H} + \mu_0 \chi_m \left( \vect{H} \right) =
\mu_0 \left( \chi_m + 1 \right) \vect{H} = \mu_0 \mu \vect{H}
\end{equation}
%
та нелінійною електричною індукцією

\begin{equation} \label{eq:d_voltera}
\vect{D} = \epsilon_0 \vect{E} + \epsilon_0 \chi_e \left( \vect{E} \right) = 
\epsilon_0 \vect{E} + \epsilon_0 \sum_{k=1}^{\infty} \int_0^t
\chi_e^{(k)} (\tau) \vect{E}^k (\tau) d \tau,
\end{equation}
%
де $\chi_e^{(k)} (t) $ коефіцієнти розкладу Вольтера нелінійної функції 
$ \chi_e $ по параметру $ \vect{E} $.

Розклад Вольтера ілюструє затримку у відгуках середовища на помірно сильні 
збудження. Розклад \eqref{eq:d_voltera} може застосовуватись у випадках,
коли породжуюче поле розповсюджуватись у середовищі не змінює його квантовий
стан. Такий тип нелінійності називається параметричним та випадку слабкої 
нелінійності. Тепер запропонуємо анзац, що нелінійні ефекти в середовищі,
що розглядається, не мають "ефекту пам'яті" та породжують індукційний відгук 
миттєво. Тоді, від розкладу в ряд Вольтера перейдемо до Тейлорівської 
моделі нелінійності:

\begin{equation} \label{eq:d_teilor}
\vect{D} = \epsilon_0 \vect{E} + 
\epsilon_0 \sum_{k=1}^{\infty} \chi_e^{(k)} \vect{E}^k (t).
\end{equation}

Розглянемо другий доданок в виразі \eqref{eq:d_teilor}: з точки зору 
матеріальних рівнянь він описує поляризаційні властивості середовища,
а згадуючи властивість симетрії вектору поляризації 
$ \vect{P} \left( \vect{E} \right) = - \vect{P} \left( - \vect{E} \right) $
лише непарні доданки ряду Тейлора можуть бути не нульовими. Також,
відокремивши лінійну поляризацію отримаємо вираз

\begin{equation} \label{eq:d_teilor_odd}
\vect{D} = \epsilon_0 \epsilon \vect{E} + 
\epsilon_0 \sum_{k=1}^{\infty} \chi_e^{(2k+1)} \vect{E}^{2k+1} (t).
\end{equation}

Для широкого класу прикладних задач \textcolor{red}{[UPSALA]} розглядається
лише перший нелінійний доданок розкладу. Таким чином отримаємо кубічну 
нелінійну складову вектору поляризації, що зустрічається в оптиці під 
назвою нелінійна поляризація Керра

\begin{equation} \label{eq:d_kerr}
\vect{D} = 
\epsilon_0 \epsilon \vect{E} + \epsilon_0 \chi_e^{(3)} \vect{E}^{3} = 
\epsilon_0 \epsilon \vect{E} + \vect{P}^\prime.
\end{equation}

Хоча розклад в ряд тейлора \eqref{eq:d_teilor_odd} не гарантує зменшення 
впливу кожного наступного доданку, тобто $ \chi_e^{(i)} < \chi_e^{(i+1)} $
на практиці врахування лише першого нелінійного доданку найчастіше дає
гарну точність до наближення слабкої нелінійності і врахуванням доданків
вищих порядків нехтують.

\textcolor{red}{ Які нелінійні ефекти спостерігаються в 
керрівському середовищі?}

Перша складова вектору електричної індукції \eqref{eq:d_kerr} відповідає 
полю при лінійному наближенні $ \vect{E} $, що породжене деяким 
струмом $ \vect{J} $, а другий доданок відповідає нелініному Керрівському 
відгуку середовища, який згідно з принципом суперпозиції, можна розглянути, 
як деяке поле $ \vect{E}^\prime $, породжене додатковим розподілом струму 
зміщення $ \vect{J}^\prime $ (далі вторинне джерело). Тоді, відповідно до
аналогії зі струмом зміщення, індукований нелінійний вторинний струм

\begin{equation} \label{eq:j_kerr}
\vect{J^\prime} = \partder{\vect{P}^\prime}{t} = 
\epsilon_0 \chi_e^{(3)} \partder{\vect{E}^3}{t}.
\end{equation}

Згідно описаної моделі, деяке стороннє джерело $ \vect{J} $ породжує 
імпульсне електромагнітне поле $ \vect{E} $ та $ \vect{H} $ розраховане
з припущенням лінійності електромагнітної індукції. Це лінійне поле, 
розповсюджуючись зі втратами крізь середовище, формує струм зміщення 
$ \vect{J}^\prime $, який в свою чергу є джерелом поля $ \vect{E}^\prime $ 
та $ \vect{H}^\prime $. Тоді, нелінійна самодія хвилі крізь середовище, 
згідно принципу суперпозиції $ \vect{E} + \vect{E}^\prime $ та 
$ \vect{H} + \vect{H}^\prime $.

Вторинне джерело $ \vect{J}^\prime $ забирає енергію породжуючої хвилі та 
формується частиною енергії втрат $ \vect{E} $, що характеризують середовище.

\textcolor{red}{ Розглянута модель, не пояснює нелінійні ефекти в вакуумі. 
Можливо, що природа нелінійних ефектів значно глибша і їх ефект впливає на 
характер енергетичної взаємодії та змінює простір. Відповідно довжина вектора 
в декартовому сенсі втрачає зміст. }

Вторинне джерело поля не є реальним джерелом, в прямому розумінні. 
Джерело $ \vect{J^\prime} $ наближено моделює нелінійну природу фізичних 
явищ розповсюдження сильних електромагнітних хвиль. Згадуючи всі обмеження,
які були велені при побудові моделі зазначимо властивості середовищ для 
яких цю модель можна застосовувати... \textcolor{red}{TODO...}

Розглянемо в якості породжувальної хвилі $ \vect{E} $ поле породжене 
пласким диском електричного струму з часовою залежністю у вигляді 
функції Хевісайда - моментальний стрибок амплітуди струму від нуля до 
значення $ A_0 $.

\textcolor{lightgray}{ \begin{equation*} 
\vect{J^\prime} = 
\vect{\rho_0}    \partder{}{t} P_\rho^\prime    \left( \vect{E} \right) + 
\vect{\varphi_0} \partder{}{t} P_\varphi^\prime \left( \vect{E} \right) + 
\vect{z_0}       \partder{}{t} P_z^\prime       \left( \vect{E} \right) 
\end{equation*} }
%
\textcolor{lightgray}{ \begin{equation*}
\vect{P^\prime} \left( \vect{E} \right) = \epsilon_0 \chi_e^{(3)} 
\dotprod{ \vect{E} }{ \vect{E} } \cdot \vect{E} 
\end{equation*} }
%
\textcolor{lightgray}{ \begin{equation*} \begin{aligned}
\vect{P^\prime} \left( \vect{E} \right) = 
\frac{ {A_0}^3 \epsilon_0 \chi_e^{(3)} }{ 8 } \left( \frac{\mu_0 \mu}
{\epsilon_0 \epsilon} \right)^{3/2} \left( {I_1}^2 \cos^2 \varphi + 
\left( I_2 - I_1 \right)^2 \sin^2 \varphi \right) \cdot \\ 
\cdot \Big( \vect{\rho_0} I_1 \cos \varphi - 
\vect{ \varphi_0 } \left( I_2 - I_1 \right) \sin \varphi \Big)
\end{aligned} \end{equation*} }
%
\textcolor{lightgray}{ \begin{equation*} \begin{aligned}
\vect{P^\prime} \left( \vect{E} \right) = 
\frac{ {A_0}^3 \epsilon_0 \chi_e^{(3)} }{ 8 } \left( \frac{\mu_0 \mu}
{\epsilon_0 \epsilon} \right)^{3/2} \left( {I_1}^2 \cos^2 \varphi + 
\left( I_2 - I_1 \right)^2 \sin^2 \varphi \right) \cdot \\ 
\cdot \Big( \vect{\rho_0} I_1 \cos \varphi - 
\vect{ \varphi_0 } \left( I_2 - I_1 \right) \sin \varphi \Big)
\end{aligned} \end{equation*} }
%
\textcolor{lightgray}{ \begin{equation*}
\vect{E} = \frac{A_0}{2} \sqrt{\frac{\mu_0 \mu}{\epsilon_0 \epsilon}}
\Big( \vect{\rho_0} I_1 \cos \varphi - 
\vect{ \varphi_0 } \left( I_2 - I_1 \right) \sin \varphi \Big)
\end{equation*} }
%
\textcolor{lightgray}{ \begin{equation*} \begin{aligned}
\vect{E}^2 = \frac{A_0^2}{4} \frac{\mu_0 \mu}{\epsilon_0 \epsilon}
\Big( I_1^2 \cos^2 \varphi + \left( I_2 - I_1 \right)^2 \sin^2 \varphi \Big)
\end{aligned} \end{equation*} }
%
\textcolor{lightgray}{ \begin{equation*} \begin{aligned}
\partder{ \vect{E}^2 }{t} = \frac{A_0^2}{4} 
\frac{\mu_0 \mu}{\epsilon_0 \epsilon}
\left( 2 I_1 \partder{I_1}{t} \cos^2 \varphi + 
2 ( I_2 - I_1 ) \left( \partder{I_2}{t} - \partder{I_1}{t} \right) 
\sin^2 \varphi \right)
\end{aligned} \end{equation*} }

Користуючись виразом для вторинного струму Керра \eqref{eq:j_kerr} та 
напруженістю електричного поля \eqref{eq:linear_e_cyl}, запишемо компоненти 
струму.

\textcolor{lightgray} { \begin{equation*} \begin{aligned}
\partder{P_\rho^\prime}{t}   = \frac{ {A_0}^3 \epsilon_0 \chi_e^{(3)} }{ 8 } 
\left( \frac{\mu_0 \mu} {\epsilon_0 \epsilon} \right)^{3/2} \left(
\left( {I_1}^2 \cos^2 \varphi + ( I_2 - I_1 )^2 \sin^2 \varphi \right)
\partder{I_1}{t} \cos \varphi + \right. \\
\left. + I_1 \cos \varphi \left( 2 I_1 \partder{I_1}{t} \cos^2 \varphi + 
2 ( I_2 - I_1 ) \left( \partder{I_2}{t} - \partder{I_1}{t} \right) 
\sin^2 \varphi \right) \right) = \\ 
= \frac{ {A_0}^3 \epsilon_0 \chi_e^{(3)} }{ 8 } 
\left( \frac{\mu_0 \mu} {\epsilon_0 \epsilon} \right)^{3/2} \left(
\partder{I_1}{t} {I_1}^2 \cos^3 \varphi + \partder{I_1}{t} ( I_2 - I_1 )^2 
\cos \varphi \sin^2 \varphi + \right. \\
\left. + 2 {I_1}^2 \partder{I_1}{t} \cos^3 \varphi + 
2 I_1 ( I_2 - I_1 ) \left( \partder{I_2}{t} - \partder{I_1}{t} \right) 
\cos \varphi \sin^2 \varphi \right)
\end{aligned} \end{equation*} }
%
\begin{equation*} \begin{aligned}
\partder{P_\rho^\prime}{t} = \frac{ {A_0}^3 \epsilon_0 \chi_e^{(3)} }{ 8 } 
\left( \frac{\mu_0 \mu} {\epsilon_0 \epsilon} \right)^{3/2} \left(
3 {I_1}^2 \partder{I_1}{t} \cos^3 \varphi + \right. \\
+ \left. ( I_2 - I_1 ) \cos \varphi \sin^2 \varphi \left( 
\partder{I_1}{t} ( I_2 - I_1 ) + 2 I_1 \left( \partder{I_2}{t} - 
\partder{I_1}{t} \right) \right) \right)
\end{aligned} \end{equation*}
%
\textcolor{lightgray} { \begin{equation*} \begin{aligned}
\partder{P_\varphi^\prime}{t}   = 
- \frac{ {A_0}^3 \epsilon_0 \chi_e^{(3)} }{ 8 } 
\left( \frac{\mu_0 \mu} {\epsilon_0 \epsilon} \right)^{3/2} \left(
\left( {I_1}^2 \cos^2 \varphi + ( I_2 - I_1 )^2 \sin^2 \varphi \right)
\left( \partder{I_2}{t} - \partder{I_1}{t} \right) \sin \varphi + \right. \\
\left. + (I_2 - I_1) \sin \varphi \left( 2 I_1 \partder{I_1}{t} \cos^2 \varphi + 
2 ( I_2 - I_1 ) \left( \partder{I_2}{t} - \partder{I_1}{t} \right) 
\sin^2 \varphi \right) \right) = \\ 
= - \frac{ {A_0}^3 \epsilon_0 \chi_e^{(3)} }{ 8 } 
\left( \frac{\mu_0 \mu} {\epsilon_0 \epsilon} \right)^{3/2} \left(
{I_1}^2 \left( \partder{I_2}{t} - \partder{I_1}{t} \right) 
\sin \varphi \cos^2 \varphi + \right. \\ \left. 
+ ( I_2 - I_1 )^2 \left( \partder{I_2}{t} - \partder{I_1}{t} \right) 
\sin^3 \varphi + 2 I_1 \partder{I_1}{t} (I_2 - I_1) 
\sin \varphi \cos^2 \varphi + \right. \\ 
+ \left. 2 ( I_2 - I_1 )^2 \left( \partder{I_2}{t} - \partder{I_1}{t} \right) 
\sin^3 \varphi \right)
\end{aligned} \end{equation*} }
%
\begin{equation*} \begin{aligned}
\partder{P_\varphi^\prime}{t} = 
- \frac{ {A_0}^3 \epsilon_0 \chi_e^{(3)} }{ 8 } 
\left( \frac{\mu_0 \mu} {\epsilon_0 \epsilon} \right)^{3/2} \left(
3 ( I_2 - I_1 )^2 \left( \partder{I_2}{t} - \partder{I_1}{t} \right)
\sin^3 \varphi \right. + \\
+ \left. I_1 \sin \varphi \cos^2 \varphi \left( 
I_1 \left( \partder{I_2}{t} - \partder{I_1}{t} \right) + 
2 \partder{I_1}{t} (I_2 - I_1) \right) \right)
\end{aligned} \end{equation*}

Так як поздовжня напруженість електричного поля відсутня 

\begin{equation*} \begin{aligned}
\partder{P_z^\prime}{t} = 0.
\end{aligned} \end{equation*}

Вирази для поперечних компонентів поля стають шматочно-визначеними,
через свої залежності від $ I_1 $ та $ I_2 $, а також від їх похідних в 
кожному з доданків, згрупованих по залежностях від азимутального кута.
Область визначення інтегралів $ I_1 $ та $ I_2 $ відома та має вигляд 
$ S_1 \cup S_2 \cup S_3 $, де кожна з під-областей 
\eqref{eq:s1zone}-\eqref{eq:s3zone} залежить від часу.
Тоді, строго виписана похідна міститиме дельта-функції в точках дотику
часово-просторових областей випромінювання $ S_1 $, $ S_2 $, $ S_3 $.
Користуючись неоднозначністю векторного потенціалу 
\cite[ст. 77]{imp:LandauII} звільнимося від дельта-функцій в виразі для 
похідних від $ I_1 $ та $ I_2 $, тоді

\begin{equation*} \begin{aligned}
\frac{1}{v} \partder{I_\alpha}{t} = 
\frac{1}{v} \partder{ I_\alpha \{ S_{2} \} }{t} 
\Big( H \left( vt^2 - z^2 - (\rho - R)^2 \right)  - 
H \left( vt^2 - z^2 - (\rho + R)^2 \right) \Big),
\end{aligned} \end{equation*}
%
де $ v = c/\sqrt{\epsilon \mu} $ - швидкість світла в середовищі при 
лінійному наближенні вектору поляризації. Таким чином область часу-простору,
де розподілений вторинний струм обмежена лише $ S_2 $, а у всіх точках 
спостереження, що відповідають співвідношенню

\begin{equation*} \begin{aligned}
S^\prime \in S_2 \subset (\rho-R)^2 < vt^2 - z^2 < (\rho+R)^2.
\end{aligned} \end{equation*}

Тоді явні вирази для похідних будуть:

\textcolor{lightgray}{ \begin{equation*} \begin{aligned}
I_1 \left\{ S_2 \right\} = \frac{\rho^2 + R^2}{4 \pi \rho^2} \arccos 
\frac{c^2 t^2 - z^2 - \rho^2 - R^2}{2 \rho R}  -
\frac{\sqrt{4 \rho^2 R^2 - (\rho^2 + R^2 - c^2t^2 + z^2)^2}}{4 \pi \rho^2} - \\
- \frac{ |\rho^2 - R^2| }{2 \pi \rho^2} 
\arctan \sqrt{ \frac{(\rho - R)^2}{(\rho + R)^2} \cdot
\frac{\left( \rho + R \right)^2 - \left( c^2t^2 - z^2 \right)} 
{\left( c^2t^2 - z^2 \right) - \left( \rho - R \right)^2} }
\end{aligned} \end{equation*} }
%
\textcolor{lightgray}{ \begin{equation*} \begin{aligned}
\partder{I_1 \left\{ S_2 \right\}}{t} = \frac{\rho^2 + R^2}{4 \pi \rho^2}
\partder{}{t} \arccos \frac{c^2 t^2 - z^2 - \rho^2 - R^2}{2 \rho R} - \\
- \partder{}{t} \frac{\sqrt{4 \rho^2 R^2 - (\rho^2 + R^2 - c^2t^2 + z^2)^2}}
{4 \pi \rho^2} - \\ - \frac{ |\rho^2 - R^2| }{2 \pi \rho^2} \partder{}{t} 
\arctan \sqrt{ \frac{(\rho - R)^2}{(\rho + R)^2} \cdot
\frac{\left( \rho + R \right)^2 - \left( c^2t^2 - z^2 \right)} 
{\left( c^2t^2 - z^2 \right) - \left( \rho - R \right)^2} }
\end{aligned} \end{equation*} }
%
\textcolor{lightgray}{ \begin{equation*} \begin{aligned}
\partder{}{t} \arccos \frac{c^2 t^2 - z^2 - \rho^2 - R^2}{2 \rho R} = 
- \frac{2 c^2 t}
{ \sqrt{4 \rho^2 R^2 - \left(c^2 t^2 - z^2 - \rho^2 - R^2 \right)^2} }
\end{aligned} \end{equation*} }
%
\textcolor{lightgray}{ \begin{equation*} \begin{aligned}
- \partder{}{t} \left( \rho^2 + R^2 - c^2t^2 + z^2 \right)^2 = 
- 2 (\rho^2 + R^2 -c^2t^2 + z^2) (-2 c^2 t)
\end{aligned} \end{equation*} }
%
\textcolor{lightgray}{ \begin{equation*} \begin{aligned}
\partder{}{t} \frac{\sqrt{4 \rho^2 R^2 - (\rho^2 + R^2 - c^2t^2 + z^2)^2}}
{4 \pi \rho^2} = \frac{1}{8 \pi \rho^2} 
\frac{ 4 c^2 t (\rho^2 + R^2 - c^2 t^2 + z^2) }
{ \sqrt{4 \rho^2 R^2 - (\rho^2 + R^2 - c^2t^2 + z^2)^2} } = \\
= \frac{c^2 t}{2 \pi \rho^2} \frac{\rho^2 + R^2 - c^2 t^2 + z^2}
{ \sqrt{4 \rho^2 R^2 - (\rho^2 + R^2 - c^2t^2 + z^2)^2} }
\end{aligned} \end{equation*} }
%
\textcolor{lightgray}{ \begin{equation*} \begin{aligned}
\partder{}{t} \arctan \sqrt{ \frac{x}{y} } = 
\frac{1}{1 + \frac{x}{y}} \frac{1}{2} 
\sqrt \frac{y}{x} \partder{}{t} \frac{x}{y}
\end{aligned} \end{equation*} }
%
\textcolor{lightgray}{ \begin{equation*} \begin{aligned}
- \frac{1}{(\rho + R)^2} + \frac{1}{(\rho - R)^2} = 
\frac{- (\rho-R)^2 + (\rho+R)^2 }{ (\rho^2 - R^2)^2 }
\end{aligned} \end{equation*} }
%
\textcolor{lightgray}{ \begin{equation*} \begin{aligned}
\partder{}{t} \arctan \sqrt{ \frac{(\rho - R)^2}{(\rho + R)^2}
\frac{\left( \rho + R \right)^2 - \left( c^2t^2 - z^2 \right)} 
{\left( c^2t^2 - z^2 \right) - \left( \rho - R \right)^2} } = 
\partder{}{t} \arctan \sqrt{ \frac
{1 - \frac{c^2t^2 - z^2}{\left( \rho + R \right)^2} } 
{ \frac{c^2t^2 - z^2}{ \left( \rho - R \right)^2 } - 1} } = \\
= \frac{1}{1 + \frac{1 - \frac{c^2t^2 - z^2}{\left( \rho + R \right)^2} } 
{ \frac{c^2t^2 - z^2}{ \left( \rho - R \right)^2 } - 1} } \frac{1}{2}
\sqrt{ \frac{ \frac{c^2t^2 - z^2}{ \left( \rho - R \right)^2 } - 1 }
{1 - \frac{c^2t^2 - z^2}{\left( \rho + R \right)^2} } } 
\frac{ - \frac{2 c^2 t}{\left( \rho + R \right)^2} 
\left( \frac{c^2t^2 - z^2}{ \left( \rho - R \right)^2} - 1 \right) - 
\frac{ 2 c^2 t }{ \left( \rho - R \right)^2 } 
\left( 1 - \frac{c^2t^2 - z^2}{\left( \rho + R \right)^2} \right) }
{\left( \frac{c^2t^2 - z^2}{ \left( \rho - R \right)^2 } - 1 \right)^2} = \\
= - c^2 t \frac{ \frac{c^2t^2-z^2}{(\rho-R)^2} - 1 }
{ \frac{c^2t^2-z^2}{(\rho-R)^2} - 1 + 1 - 
\frac{c^2t^2 - z^2}{\left( \rho + R \right)^2} }
\sqrt{ \frac{ \frac{c^2t^2 - z^2}{ \left( \rho - R \right)^2 } - 1}
{1 - \frac{c^2t^2 - z^2}{\left( \rho + R \right)^2} } } \frac
{ \frac{c^2t^2 - z^2}{ \left( \rho^2 - R^2 \right)^2 } - \frac{1}{(\rho+R)^2} + 
\frac{1}{(\rho-R)^2} - \frac{ c^2t^2 - z^2 }{ \left( \rho^2 - R^2 \right)^2 } }
{ \left( \frac{c^2t^2 - z^2}{ \left( \rho - R \right)^2} - 1 \right)^2 } = \\
= - \frac{4 \rho R c^2 t}{ \left( \rho^2 - R^2 \right)^2 } 
\frac{ 1 }{ \frac{c^2t^2-z^2}{(\rho-R)^2} - 
\frac{c^2t^2 - z^2}{\left( \rho + R \right)^2} }
\sqrt{ \frac{ \frac{c^2t^2 - z^2}{ \left( \rho - R \right)^2 } - 1}
{1 - \frac{c^2t^2 - z^2}{\left( \rho + R \right)^2} } } \frac
{ 1 }{ \frac{c^2t^2 - z^2}{ \left( \rho - R \right)^2} - 1 } = \\
= - \frac{c^2 t}{ c^2 t^2 - z^2 } \frac{1} { 
\sqrt{ 1 - \frac{c^2t^2 - z^2}{(\rho + R)^2 } } 
\sqrt{ \frac{c^2t^2 - z^2}{ (\rho - R)^2 } - 1} }
\end{aligned} \end{equation*} }
%
\textcolor{lightgray}{ \begin{equation*} \begin{aligned}
\partder{ I_1 \{ S_2 \} }{t} = - \frac{c^2 t}{2 \pi \rho^2}
\frac{\rho^2 + R^2}
{ \sqrt{4 \rho^2 R^2 - \left(c^2 t^2 - z^2 - \rho^2 - R^2 \right)^2} } - \\
- \frac{c^2 t}{2 \pi \rho^2} \frac{\rho^2 + R^2 - c^2 t^2 + z^2}
{ \sqrt{4 \rho^2 R^2 - (\rho^2 + R^2 - c^2t^2 + z^2)^2} } + \\ 
+ \frac{ c^2 t }{2 \pi \rho^2} \frac{|\rho^2 - R^2|}{ c^2 t^2 - z^2 } \frac{1} 
{ \sqrt{ 1 - \frac{c^2t^2 - z^2}{(\rho + R)^2 } } 
\sqrt{ \frac{c^2t^2 - z^2}{ (\rho - R)^2 } - 1} }
\end{aligned} \end{equation*} }
%
\begin{equation} \begin{aligned} \label{eq:i1_partder}
\frac{1}{v} \partder{ I_1 \{ S_2 \} }{t} = \frac{ vt }{2 \pi \rho^2} 
\frac{ (\rho^2 - R^2)^2  (v^2 t^2 - z^2)^{-1} } 
{ \sqrt{ (\rho + R)^2 - v^2t^2 + z^2 } 
\sqrt{ v^2t^2 - z^2 - (\rho - R)^2 } } - \\
- \frac{vt}{2 \pi \rho^2} \frac{2 (\rho^2 + R^2) - (v^2 t^2 - z^2)}
{ \sqrt{4 \rho^2 R^2 - (v^2t^2 - z^2 - \rho^2 - R^2)^2} };
\end{aligned} \end{equation}
%
\textcolor{lightgray}{ \begin{equation*} \begin{aligned}
\partder{ I_2 \{ S_2 \} }{t} = \frac{1}{\pi} \partder{}{t} \arccos 
\frac{c^2t^2 - z^2 + \rho^2 - R^2}{2 \rho \sqrt{c^2t^2 - z^2}} = \\
= - \frac{1}{\pi} \frac{1} { \sqrt{ 1 - \frac{ (c^2t^2 - z^2 + \rho^2 - R^2)^2 }
{4 \rho^2 (c^2t^2 - z^2)^2} } } \frac{1}{2 \rho} \partder{}{t} 
\frac{c^2t^2 - z^2 + \rho^2 - R^2} {\sqrt{c^2t^2 - z^2}} = \\
= - \frac{1}{2 \rho \pi} \frac{1} 
{ \sqrt{ 1 - \frac{ (c^2t^2 - z^2 + \rho^2 - R^2)^2 }
{4 \rho^2 (c^2t^2 - z^2)} } } \frac{2c^2t \sqrt{c^2t^2 - z^2} - 
\frac{c^2t}{\sqrt{c^2t^2 - z^2}} (c^2t^2 - z^2 + \rho^2 - R^2)
}{c^2t^2 - z^2} = \\ = - \frac{c^2 t}{2 \pi \rho} \frac{1} 
{ \sqrt{ 1 - \frac{ (c^2t^2 - z^2 + \rho^2 - R^2)^2 }
{4 \rho^2 (c^2t^2 - z^2)} } } \frac{2 \sqrt{c^2t^2 - z^2} - 
\frac{c^2t^2 - z^2 + \rho^2 - R^2}{\sqrt{c^2t^2 - z^2}}}{c^2t^2 - z^2} = \\
= - \frac{c^2 t}{2 \pi \rho (c^2t^2 - z^2)} \frac{ 2 \sqrt{c^2t^2 - z^2} - 
\frac{c^2t^2 - z^2 + \rho^2 - R^2}{\sqrt{c^2t^2 - z^2}} } 
{ \sqrt{ 1 - \frac{ (c^2t^2 - z^2 + \rho^2 - R^2)^2 }
{4 \rho^2 (c^2t^2 - z^2)} } } = \\
= - \frac{c^2 t}{\pi (c^2t^2 - z^2) } 
\frac{ 2 (c^2t^2 - z^2) - (c^2t^2 - z^2 + \rho^2 - R^2) } 
{ \sqrt{ 4 \rho^2 (c^2t^2 - z^2) - (c^2t^2 - z^2 + \rho^2 - R^2)^2 } } = \\
= - \frac{c^2 t}{\pi (c^2t^2 - z^2) } \frac{ c^2t^2 - z^2 -  \rho^2 + R^2 } 
{ \sqrt{ 4 \rho^2 (c^2t^2 - z^2) - (c^2t^2 - z^2 + \rho^2 - R^2)^2 } }
\end{aligned} \end{equation*} }
%
\begin{equation} \begin{aligned} \label{eq:i2_partder}
\frac{1}{v} \partder{ I_2 \{ S_2 \} }{t} = 
- \frac{vt}{\pi (v^2t^2 - z^2) } \frac{ v^2t^2 - z^2 - \rho^2 + R^2 } 
{ \sqrt{ 4 \rho^2 (v^2t^2 - z^2) - (v^2t^2 - z^2 + \rho^2 - R^2)^2 } }.
\end{aligned} \end{equation}

Графічно можемо визначити, що проекція вторинного струм обернена за знаком 
відносно тієї ж проекції компоненти напруженості електричного поля.

\begin{figure}[h] \begin{center}
\includegraphics[scale=0.5]{Jperp_A1}
\caption{Нелінійність амплітуди вторинного джерела}
\label{fig:jx_secondary}
\end{center} \end{figure}

З виразів \eqref{eq:i1_partder}, \eqref{eq:i2_partder} очевидна нелінійна 
залежність струму від $ A_0 $.

%%%%%%%%%%%%%%%%%%%%%%%%%%%%%%%%%%%%%%%%%%%%%%%%%%%%%%%%%%%%%%%%%%%%%%%%%%%%%%%%
\section{Енергетичний розподіл поля плаского диску в лінійному наближенні}

Вторинний електричний струм розподілений в усьому напівпросторі $ z > 0 $,
але за рахунок згасання енергії з відстанню в межах $ [1/R, 1/R^2] $ (в 
залежності від напрямку спостереження) можемо обмежити область де треба 
враховувати нелінійні ефекти за рахунок високої концентрації енергії. Для 
визначення параметричних меж застосування введеної моделі нелінійності та 
оцінки границі зони, де нелінійні ефекти треба враховувати, розглянемо 
енергетичні характеристики поля в ближній зоні

Тепер розглянемо енергетичний розподіл від поля плаского диску при різних
часових залежностей $ f(t) $ стороннього струму. Побудова класичної 
енергетичної діаграми спрямованості мало інформативне дослідження нелінійних
ефектів - в даній роботі важливим є енергетичний розподіл в ближній зоні.

Розглянемо густину енергії електромагнітного поля $ \vect{E} $ та 
$ \vect{H} $, збудженого пласким диском електричного струму з довільною 
часовою залежністю $ f(t) $

\begin{equation}
W \left( \vect{r} \right) = \int_0^\infty \dotprod{\vect{E}}{\vect{H}} dt.
\end{equation}

Значення напруженостей поля $ \vect{E} $ та $ \vect{H} $ можуть бути 
враховані чисельно за допомогою інтегралу Дюамеля \eqref{eq:duhamel},
використовуючи перехідну функцію знайдену в розділі \ref{sec:tranc_resp}.
Запишемо енергетичний розподіл поля, користуючись шматочним визначенням 
перехідної функції на області визначення:

\begin{equation}
W = \int_0^\infty 
\dotprod{\vect{E}\left\{S_1 \cup S_2\right\}}
{\vect{H}\left\{S_1 \cup S_2 \right\}} dt + 
\int_0^\infty \dotprod{\vect{E}\{S_3\}}{\vect{H}\{S_3\}} dt.
\end{equation}

Так як, $ \vect{E}\left\{S_1 \cup S_2\right\} $ та  
$ \vect{H}\left\{S_1 \cup S_2 \right\}$ пропорційні, 
а $ \vect{E}\{S_3\} = 0 $ можемо звільнитись від магнітних компонент 
поля, що спростить числові розрахунки:

\begin{equation}
W = \int_0^\infty \dotprod{\vect{E}}{\vect{H}} dt =
\sqrt{\frac{\epsilon_0 \epsilon}{\mu_0 \mu}} \int_0^\infty \vect{E}^2 dt.
\end{equation}

Також, користуючись властивостями перехідної функції можемо обмежити
область інтегрування за часом, а також спростити під-інтегральний вираз

\begin{equation} \label{eq:energy}
W = \sqrt{\frac{\mu_0 \mu}{\epsilon_0 \epsilon}}
\int_{ct_1}^{c\tau_0+ct_3} \left( E_\rho^2 + E_\varphi^2 \right) dt
\end{equation}

Також оцінено енергію перехідної функції, а тобто \ref{eq:energy} при 
часовій залежності сигналу у вигляді функції Гевісайда $ f(t) = H(t) $:

\textcolor{lightgray}{ \begin{equation*} \begin{aligned}
\vect{E}^2 = \frac{A_0^2}{4} \frac{\mu_0 \mu}{\epsilon_0 \epsilon}
\Big( I_1^2 \cos^2 \varphi + \left( I_2 - I_1 \right)^2 \sin^2 \varphi \Big)
\end{aligned} \end{equation*} }
%
\begin{equation} \label{eq:energy_tr}
W_{tr} = \frac{A_0^2}{4} \sqrt{\frac{\mu_0 \mu}{\epsilon_0 \epsilon}}
\int_{ct_1}^{ct_3}  \Big( I_1^2 \cos^2 \varphi + 
\left( I_2 - I_1 \right)^2 \sin^2 \varphi \Big) dt.
\end{equation}

Розглядаючи особливий випадок \ref{eq:energy_tr} для $ \rho = 0 $, помічаємо,
що енергія випромінювання плаского диску, завжди лежить у наступних межах для 
сигналів з часовою залежністю $ f(t) $ з областю значень 
$ \left[ -1, 1 \right] $ та тривалістю $ \tau_0 $:

\textcolor{lightgray}{ \begin{equation*}
\left. W \right|^{\rho=0} = 
\frac{A_0^2}{16} \sqrt{\frac{\mu_0 \mu}{\epsilon_0 \epsilon}}
\Big( \sqrt{R^2+z^2} - z \Big)
\end{equation*} }
%
\textcolor{lightgray}{ \begin{equation*}
\int_{ct_1}^{c\tau_0+ct_3} 
\left( \int_0^t f(\tau) d \tau < \tau_0 \right) dt < \tau_0 R
\end{equation*} }
%
\begin{equation}
0 \leq W_{max} \left( \tau_0, f(t), \vect{r} \right) < 
\frac{\tau_0 R A_0^2}{16} \sqrt{\frac{\mu_0 \mu}{\epsilon_0 \epsilon}},
\end{equation}
%
де $ W_{max} $ - густина енергії, $ \tau_0 $ - ефективна тривалість імпульсу 
за визначеною метрикою, $ A_0 $ - максимальна амплітуда сигналу, $ R $ - 
радіус апертури, а вираз під коренем - імпеданс у середовищі розповсюдження 
хвилі.

Будуватимемо поперечні зрізи значень енергій
для різних довжин імпульсів та для різних відстаней від джерела.
Для збереження кутового розміру зрізів візьмемо його $ z + 2R $.

\textcolor{red}{ TODO: можна строго порівняти діаграму спрямованості 
отриману в часовій області з діаграмою Баума, а також оцінити її відхилення 
в ближній зоні }

%%%%%%%%%%%%%%%%%%%%%%%%%%%%%%%%%%%%%%%%%%%%%%%%%%%%%%%%%%%%%%%%%%%%%%%%%%%%%%%%
\section{Модовий розподіл Керрівської поправки}

Застосуємо метод еволюційних рівнянь для розв'язання задачі випромінювання 
вторинним струмом \eqref{eq:j_kerr}. Для цього, спершу, запишемо модовий
розподіл джерела, який є правою частиною рівняння Клейна-Гордона. 

\textcolor{lightgray} { \begin{equation*}
j_m \left( r, t; \nu \right) = \frac{\sqrt{\mu_0}}{2\pi} 
\int \limits_{0}^{2\pi} d \varphi \int \limits_0^\infty \rho d \rho 
\vect{j_0} \crossprod{ \nabla_\perp \Psi_m^* }{ \vect{z_0} }
\end{equation*} }
%
\textcolor{lightgray} { \begin{equation*} 
\vect{J^\prime} = 
\vect{\rho_0}    \partder{}{t} P_\rho^\prime    \left( \vect{E} \right) + 
\vect{\varphi_0} \partder{}{t} P_\varphi^\prime \left( \vect{E} \right) + 
\vect{z_0}       \partder{}{t} P_z^\prime       \left( \vect{E} \right) 
\end{equation*} }
%
\textcolor{lightgray} { \begin{equation*} \begin{aligned}
\crossprod{ \nabla_\perp \Psi_m^* }{ \vect{z_0} } =
- \vect{\rho_0} i m e^{-im\varphi} \frac{J_m (\nu \rho)}{\rho \sqrt{\nu}}
- \vect{\varphi_0} \sqrt{\nu} e^{-im\varphi} 
\frac{J_{m-1} (\nu \rho) - J_{m+1} (\nu \rho)}{2}
\end{aligned} \end{equation*} }
%
\textcolor{lightgray} { \begin{equation*} \begin{aligned}
\vect{J^\prime} \crossprod{ \nabla_\perp \Psi_m^* }{ \vect{z_0} } = 
- i e^{-im\varphi} m \frac{J_m (\nu \rho)}{\rho \sqrt{\nu}}
\partder{}{t} P_\rho^\prime \left( \vect{E} \right) - \\
- \sqrt{\nu} e^{-im\varphi} \frac{J_{m-1} (\nu \rho) - J_{m+1} (\nu \rho)}{2}
\partder{}{t} P_\varphi^\prime \right)
\end{aligned} \end{equation*} }

\begin{equation*} \begin{aligned}
j_m = - \frac{\sqrt{\mu_0}}{2\pi} 
\int_{0}^{2\pi} d \varphi \int \limits_{0}^{\infty} \rho d \rho
e^{-im\varphi} \left( i  m \frac{J_m (\nu \rho)}{\rho \sqrt{\nu}}
\partder{j_\rho^\prime}{t} + \sqrt{\nu}
\frac{J_{m-1} (\nu \rho) - J_{m+1} (\nu \rho)}{2}
\partder{j_\varphi^\prime}{t} \right)
\end{aligned} \end{equation*}

Спершу, перейдемо від комплексної області визначення модового розподілу 
до уявної. Згрупувавши доданки за тригонометричними функціями,
знайдемо інтеграли за азимутальним кутом $ \varphi $, користуючись 
аналітичними інтегралами \eqref{eq:int_exp3}, \eqref{eq:int_exp4}, 
\eqref{eq:int_exp5}, \eqref{eq:int_exp6}. Отримаємо вирази для інтегралів
від компонентів вектору вторинного нелінійного струму з ядром інтегралу у 
вигляді комплексної експоненти.

\textcolor{lightgray} { \begin{equation*} \begin{aligned}
\int_{0}^{2 \pi} d \varphi e^{-im \varphi} \partder{P_\rho^\prime}{t} = 
\frac{ {A_0}^3 \xi_3 }{ 8 } \int_{0}^{2\pi} d \varphi
e^{-im\varphi} \left( \frac{\mu_0 \mu} {\epsilon_0 \epsilon} \right)^{3/2} 
\left( 3 {I_1}^2 \partder{I_1}{t} \cos^3 \varphi + \right. \\
+ \left. ( I_2 - I_1 ) \cos \varphi \sin^2 \varphi \left( 
\partder{I_1}{t} ( I_2 - I_1 ) + 2 I_1 \left( \partder{I_2}{t} - 
\partder{I_1}{t} \right) \right) \right) = \\
= \frac{ {A_0}^3 \xi_3 }{ 8 } 
\left( \frac{\mu_0 \mu} {\epsilon_0 \epsilon} \right)^{3/2}
\left( \frac{3 \pi}{4} {I_1}^2 \partder{I_1}{t} \left( \delta_{m,-3} - 
\delta_{m,3} + 3 \delta_{m,-1} + 3 \delta_{m,1} \right) + \right. \\
+ \frac{\pi }{4} \left. ( I_2 - I_1 ) \left( \delta_{m,1} - 
\delta_{m,-3} + \delta_{m,-1} - \delta_{m,3} \right) \left( 
\partder{I_1}{t} ( I_2 - I_1 ) + 2 I_1 \left( \partder{I_2}{t} - 
\partder{I_1}{t} \right) \right) \right)
\end{aligned} \end{equation*} }
%
\textcolor{lightgray} { \begin{equation*} \begin{aligned}
\int_{0}^{2\pi} e^{-i m \varphi} \cos^3 \varphi d \varphi = 
\frac{\pi}{4} \delta_{m,-3} + \frac{\pi}{4} \delta_{m,3} + 
\frac{3 \pi}{4} \delta_{m,-1} + \frac{3 \pi}{4} \delta_{m,1}
\end{aligned} \end{equation*} }
%
\textcolor{lightgray} { \begin{equation*} \begin{aligned}
\int_{0}^{2\pi} e^{-i m \varphi} \cos \varphi \sin^2 \varphi d \varphi = 
\frac{\pi \delta_{m,1} }{4} - \frac{\pi \delta_{m,-3} }{4} + 
\frac{\pi \delta_{m,-1} }{4} - \frac{\pi \delta_{m,3} }{4}
\end{aligned} \end{equation*} }
%
\begin{equation*} \begin{aligned}
\frac{\epsilon_0 \chi_e^{(3)}}{2 \pi} \int_{0}^{2\pi} d \varphi 
e^{-im \varphi} \partder{P_\rho^\prime}{t} = 
\frac{ {A_0}^3 \epsilon_0 \chi_e^{(3)}}{ 64 } 
\left( \frac{\mu_0 \mu} {\epsilon_0 \epsilon} \right)^{3/2} \cdot \\ 
\cdot \left( 3 {I_1}^2 \partder{I_1}{t} \left( \delta_{m,-3} - 
\delta_{m,3} + 3 \delta_{m,-1} + 3 \delta_{m,1} \right) + \right. \\
+ \left. ( I_2 - I_1 ) \left( \delta_{m,1} - 
\delta_{m,-3} + \delta_{m,-1} - \delta_{m,3} \right) \left( 
\partder{I_1}{t} ( I_2 - I_1 ) + 2 I_1 \left( \partder{I_2}{t} - 
\partder{I_1}{t} \right) \right) \right)
\end{aligned} \end{equation*}
%
\textcolor{lightgray} { \begin{equation*} \begin{aligned}
\int_{0}^{2\pi} e^{-i m \varphi} \sin^3 \varphi d \varphi = 
\frac{3 \pi i}{4} \delta_{m,-1} - \frac{3 \pi i}{4} \delta_{m,1} - 
\frac{\pi i}{4} \delta_{m,-3} - \frac{\pi i}{4} \delta_{m,3}
\end{aligned} \end{equation*} }
%
\textcolor{lightgray} { \begin{equation*} \begin{aligned}
\int_{0}^{2\pi} e^{-i m \varphi} \sin \varphi \cos^2 \varphi d \varphi = 
\frac{\pi i }{4} \delta_{m,-3} - \frac{\pi i }{4} \delta_{m,1} + 
\frac{\pi i }{4} \delta_{m,3} + \frac{\pi i }{4} \delta_{m,-1}
\end{aligned} \end{equation*} }
%
\textcolor{lightgray} { \begin{equation*} \begin{aligned}
\int_{0}^{2 \pi} d \varphi e^{-im \varphi} \partder{P_\varphi^\prime}{t} = \\
= - \frac{ {A_0}^3 \xi_3 }{ 8 } \int_{0}^{2 \pi} d \varphi e^{-im \varphi}
\left( \frac{\mu_0 \mu} {\epsilon_0 \epsilon} \right)^{3/2} \left(
3 ( I_2 - I_1 )^2 \left( \partder{I_2}{t} - \partder{I_1}{t} \right)
\sin^3 \varphi \right. + \\
+ \left. I_1 \sin \varphi \cos^2 \varphi \left( 
I_1 \left( \partder{I_2}{t} - \partder{I_1}{t} \right) + 
2 \partder{I_1}{t} (I_2 - I_1) \right) \right) = 
- \frac{ {A_0}^3 \xi_3 }{ 8 V } \cdot \\ 
\cdot \left( \frac{\mu_0 \mu} {\epsilon_0 \epsilon} \right)^{3/2} \left(
\frac{3 \pi i}{4} ( I_2 - I_1 )^2 \left( \partder{I_2}{t} - 
\partder{I_1}{t} \right) \left( 3 \delta_{m,-1} - 3 \delta_{m,1} - 
\delta_{m,-3} - \delta_{m,3} \right) \right. + \\
+ \left. \frac{\pi i}{4} I_1 \left(  \delta_{m,-3} - \delta_{m,1} + 
\delta_{m,3} + \delta_{m,-1} \right) \left( 
I_1 \left( \partder{I_2}{t} - \partder{I_1}{t} \right) + 
2 \partder{I_1}{t} (I_2 - I_1) \right) \right)
\end{aligned} \end{equation*} }
%
\begin{equation*} \begin{aligned}
\frac{\epsilon_0 \chi_e^{(3)}}{2 \pi} \int_{0}^{2 \pi} d \varphi 
e^{-im \varphi} \partder{P_\varphi^\prime}{t} = 
- \frac{ {A_0}^3 \epsilon_0 \chi_e^{(3)}  i}{ 64 }
\left( \frac{\mu_0 \mu} {\epsilon_0 \epsilon} \right)^{3/2} \cdot \\ 
\cdot \left( 3 ( I_2 - I_1 )^2 \left( \partder{I_2}{t} - 
\partder{I_1}{t} \right) \left( 3 \delta_{m,-1} - 3 \delta_{m,1} - 
\delta_{m,-3} - \delta_{m,3} \right) \right. + \\
+ \left. I_1 \left(  \delta_{m,-3} - \delta_{m,1} + 
\delta_{m,3} + \delta_{m,-1} \right) \left( 
I_1 \left( \partder{I_2}{t} - \partder{I_1}{t} \right) + 
2 \partder{I_1}{t} (I_2 - I_1) \right) \right)
\end{aligned} \end{equation*}

Як видно з останніх виразів, інтригування за кутом $ \varphi $ дає дискретний 
модовий розподіл вторинного струму.  Також помічаємо, що у розподілах присутні 
лише моді з номерами $ \pm 1 $ та $ \pm 3 $ - вклад вторинного струму в інші
моди відсутній. Випишемо окрему кожну з ненульових мод розподілу вторинного 
струму.

\textcolor{lightgray} { \begin{equation*} \begin{aligned}
j_m = - \frac{\sqrt{\mu_0}}{2\pi} 
\int_0^{2\pi} d \varphi \int \limits_0^\infty \rho d \rho
e^{-im\varphi} \left( i m \frac{J_m (\nu \rho)}{\rho \sqrt{\nu}}
\partder{j_\rho^\prime}{t} + \sqrt{\nu}
\frac{J_{m-1} (\nu \rho) - J_{m+1} (\nu \rho)}{2}
\partder{j_\varphi^\prime}{t} \right)
\end{aligned} \end{equation*} }

\begin{equation*} \begin{aligned}
j_1 = - \frac{i A_0^3 \sqrt{\mu_0} \epsilon_0 \chi_e^{(3)}}{64}
\left( \frac{\mu_0 \mu}{\epsilon_0 \epsilon} \right)^{3/2}
\int_0^\infty \rho d \rho
\end{aligned} \end{equation*}


%%%%%%%%%%%%%%%%%%%%%%%%%%%%%%%%%%%%%%%%%%%%%%%%%%%%%%%%%%%%%%%%%%%%%%%%%%%%%%%%
\section{Чсловий розрахунок нелінійної поправки}

Тепер коли визначено джерело можна застосувати метод еволюційних рівнянь.
Почнемо з розв'язку рівняння Клейна-Гордона.
%
\begin{equation*}
\frac{\epsilon \mu}{ \sqrt{\epsilon_0 \mu_0}} 
\frac{\partial^2 h_m}{\partial t^2} - \frac{\partial^2 h_m}{\partial z^2} + 
\nu^2 h_m = j_m (t',z'; \nu)
\end{equation*}
%
Коефіцієнт $ h_m $ доведеться розраховувати чисельно за формолою
%
\begin{equation*}
h_m (z, t; \nu) = \int_{0}^{\infty} \int_{0}^{\infty}
j_m (t',z'; \nu) G(t,t',z,z') dz' dt'
\end{equation*}
%
де 
%
\begin{equation*}
G(t,t',z,z') = \frac{\mathit{V}}{2} H \left( \mathit{V} (t-t') - (z-z') \right)
J_0 \left( \nu \sqrt{\mathit{V}^2 (t-t')^2 - (z-z')^2} \right)
\end{equation*}
%
З $ h_m $ можна отримати інші магнітні коефіцієнти
%
\begin{equation*}
I_m^h = \partial_z (h_m)
\end{equation*}
%
\begin{equation*}
V_m^h = - \mu \partial_{ct} (h_m)
\end{equation*}
%
Аналогічно для продольного коєфіціенту електричного поля
%
\textcolor{lightgray}{ \begin{equation*}
- \partial_{ct}(\mu I_n^e) - \partial_z V_n^e + \chi^2 e_n = 0
\end{equation*} }
%
\textcolor{lightgray}{ \begin{equation*}
\epsilon \mu \partial_{ct} \left( \partial_{ct} e_n + 
\frac{\sqrt{\mu_0}}{2 \pi} \int_0^{2\pi} d \varphi 
\int_0^{\infty} \rho d \rho \Phi_n^* (\chi) J_z \right) - 
\partial^2_z e_n + \chi^2 e_n = 0
\end{equation*} }
%
\begin{equation*}
\frac{\epsilon \mu}{ \sqrt{\epsilon_0 \mu_0}} 
\frac{\partial^2 e_n}{\partial t^2} - 
\frac{\partial^2 e_n}{\partial z^2} + \chi^2 e_n = 
- \frac{\sqrt{\mu_0}}{2 \pi c} 
\int_0^{2\pi} d \varphi 
\int_0^{\infty} \rho d \rho \Phi_n^* (\chi) \partder{J_z}{t}
\end{equation*}
%
Поздовжний вторинний електричний струм рівний нулю, тому рівняння 
Клейна-Гордона для $ e_n (z, t; \chi) $ є однорідним та має нульовий розвязок.
%
\begin{equation*}
e_n (z, t; \chi) = \iint_S j_n (t',z', \chi) G(t,t',z,z') dt' dz' = 0
\end{equation*}
%
Також чисельно знайдемо розвязки еволюційних рівнянь, що залишились.
%
\textcolor{lightgray}{ \begin{equation*}
\partial_{ct} (\epsilon e_n) = - I_n^e - 
\frac{\sqrt{\mu_0}}{2 \pi} \int_0^{2\pi} d \varphi 
\int_0^{\infty} \rho d \rho \Phi_n^* (\chi) J_z
\end{equation*} }
%
\begin{equation*}
I_n^e = - \partial_{ct} (\epsilon e_n) - 
\frac{\sqrt{\mu_0}}{2 \pi} \int_0^{2\pi} d \varphi 
\int_0^{\infty} \rho d \rho \Phi_n^* (\chi) J_z
\end{equation*}
%
\begin{equation*}
\partial_{z} e_n = V_n^e
\end{equation*}
%
\textcolor{lightgray} { \begin{equation*} \begin{aligned}
\vect{E_\perp} = \frac{1}{\sqrt{\epsilon_0}} \left( 
\sum \limits_{m=-\infty}^{\infty} \int \limits_{0}^{\infty} 
d \nu V_m^h \crossprod{ \nabla_\perp \Psi_m }{ \vect{z_0} } +
\sum \limits_{n=-\infty}^{\infty} \int \limits_{0}^{\infty}
d \chi V_n^e \nabla_\perp \Phi_n \right)
\end{aligned} \end{equation*} }
%
\textcolor{lightgray} { \begin{equation*} \begin{aligned}
\crossprod{ \nabla_\perp \Psi_m }{ \vect{z_0} } = 
- e^{im\varphi} \left( \vect{\varphi_0} \sqrt{\nu} 
\frac{J_{m-1} (\nu \rho) - J_{m+1} (\nu \rho)}{2} - 
i m \vect{\rho_0} \frac{J_m (\nu \rho)}{ \rho \sqrt{\nu}} \right)
\end{aligned} \end{equation*} }
%
Останнам кроком для отримання результату мала стати чисельна оцінка 
інтегралів по безперервному модовому спектру $ \nu $. Для кожної з п'яти
компонентів поля.
%
\begin{equation*} \begin{aligned} \label{eq:KerrAmendErhoInit}
E_\rho = \frac{1}{\sqrt{\epsilon_0}} \sum_{m=-\infty}^{\infty} 
i m e^{im\varphi} \int_{0}^{\infty} \frac{d \nu}{\sqrt{\nu}} 
V_m^h \frac{J_m(\nu \rho)}{\rho}
\end{aligned} \end{equation*}
%
\begin{equation*} \begin{aligned}
E_\varphi = - \frac{1}{2 \sqrt{\epsilon_0}} \sum_{m=-\infty}^{\infty} 
e^{im\varphi} \int_{0}^{\infty} \sqrt{\nu} d \nu 
V_m^h \left( J_{m-1} (\nu \rho) - J_{m+1} (\nu \rho) \right)
\end{aligned} \end{equation*}
%
\begin{equation*} 
E_z = \frac{1}{\sqrt{\epsilon_0}} \sum_{n=-\infty}^{\infty}
\int_0^\infty \chi^2 d \chi e_n \Phi_n = 0
\end{equation*}
%
\textcolor{lightgray} { \begin{equation*}
\vect{H_\perp} = \frac{1}{\sqrt{\mu_0}} \left( 
\sum \limits_{m=-\infty}^{\infty} \int \limits_{0}^{\infty} d \nu
I_m^h \nabla_\perp \Psi_m + \sum \limits_{n=-\infty}^{\infty}
\int \limits_{0}^{\infty} d \chi I_n^e 
\crossprod{\vect{z_0}}{\nabla_\perp \Phi_n} \right)
\end{equation*} }
%
\textcolor{lightgray} { \begin{equation*} \begin{aligned}
\nabla_\perp \Psi_m = e^{i m \varphi} \left( \vect{\rho_0} 
\sqrt{\nu} \frac{ J_{m-1}(\nu \rho) - J_{m+1}(\nu \rho) }{2} +
i m \vect{\varphi_0} \frac{J_m(\nu \rho)}{\sqrt{\nu} \rho} \right)
\end{aligned} \end{equation*} }
%
\begin{equation*}
H_\rho = \frac{1}{2 \sqrt{\mu_0}} \sum_{m=-\infty}^{\infty} 
e^{im\varphi} \int_{0}^{\infty} \sqrt{\nu} d \nu
I_m^h \left( J_{m-1}(\nu \rho) - J_{m+1}(\nu \rho) \right)
\end{equation*}
%
\begin{equation*}
H_\varphi = \frac{1}{\sqrt{\mu_0}} \sum_{m=-\infty}^{\infty} 
i m e^{im\varphi} \int_{0}^{\infty} \frac{d \nu}{\sqrt{\nu}}
I_m^h \frac{J_m(\nu \rho)}{\rho}
\end{equation*}
%
\textcolor{lightgray} { \begin{equation*} 
H_z = \frac{1}{\sqrt{\mu_0}} \sum_{m=-\infty}^{\infty}
\int_0^\infty \nu^2 d \nu h_m \Psi_m
\end{equation*} }
%
\textcolor{lightgray} { \begin{equation*} 
\Psi_m (\nu) = \frac{J_m(\nu \rho)}{\sqrt{\nu}} e^{i m \varphi}
\end{equation*} }
%
\begin{equation*} 
H_z = \frac{1}{\sqrt{\mu_0}} \sum_{m=-\infty}^{\infty}
e^{i m \varphi} \int_0^\infty \nu^{3/2} d \nu h_m 
J_m(\nu \rho)
\end{equation*}

Чисельний розрахунок нелінійних поправок в тому виді, що було отримано методом 
еволюційних рівнянь, представляє собою нездійсненну чисельну задачу. Наприклад 
розрахунок електрисної радиальної компоненти для однієї події з відностою 
похибкою $ 5\% $ займає 21 рік на процессорі Intel Core i5-4670 3.40GHz, 
при однопоточному виконнані. \textcolor{red} { Посилання на розділ про ПО. }

Причина розрахункових проблем пояснюється відразу декількома факторами.
Перш за все доводиться рахувати всі інтигралу двічі через комплексну область 
визнаення функцій. Сумування за кутовими модами також суттево впливає на 
швидкість розрахунків. Сам вираз містить 4 інтиграли і розрахунок всіх їх за 
формулами чисельної оцінки кратного інтегралу спростило б задачу. Скорочення 
кількості викликів функцій також позитивно вплине на розрахунковий час. 
Аналітичний розрахунок похідної зменшить похибку в порівнянні з чисельною 
оцінкою і також позитивно вплине на розрахунковий час. Ще помічаємо, що 
наявність електродинамиічих констант в підінтегральних виразах зменшує точність 
розрахунків зарахунок різниці порядків. Спробуємо максимально оптимізувати 
вирази для компонентів поля.

При дослідженні лінійного випромінювання плаского диску помічаємо ми 
використовували правило Лейбніца для звільнення від одного інтегралу.
Запишимо вираз $ V_m^h $ для нелінійного випадку та порівняємо.
%
\textcolor{lightgray} { \begin{equation*} \begin{aligned} 
V_m^h = - \mu \partial_{ct} \int_0^\infty \int_0^\infty j'_m (t',z') G dz' dt'
\end{aligned} \end{equation*} }
%
\textcolor{lightgray} { \begin{equation*} \begin{aligned} 
V_m^h = - \frac{v \mu}{2} \frac{\sqrt{\epsilon \mu}}{\sqrt{\epsilon \mu}} 
\partder{}{ct} \int_0^\infty dz' \int_0^\infty 
dt' H \left( v (t-t') - (z-z') \right) \cdot \\
\cdot J_0 \left( \nu \sqrt{v^2 (t-t')^2 - (z-z')^2} \right) j'_m (t',z')
\end{aligned} \end{equation*} }
%
\textcolor{lightgray} { \begin{equation*} \begin{aligned} 
V_m^h = - \frac{\mu}{2 \sqrt{\epsilon \mu}} \partder{}{vt} 
\int_0^\infty dz' \int_0^\infty dvt' 
H \left( v (t-t') - (z-z') \right) \cdot \\
\cdot J_0 \left( \nu \sqrt{v^2 (t-t')^2 - (z-z')^2} \right) j'_m (t',z')
\end{aligned} \end{equation*} }
%
\textcolor{lightgray} { \begin{equation*} \begin{aligned} 
V_m^h = - \frac{1}{2} \sqrt{\frac{\mu}{\epsilon}} \partder{}{vt} 
\int_0^\infty dz' \int_0^\infty dvt' 
H \left( v (t-t') - (z-z') \right) \cdot \\
\cdot J_0 \left( \nu \sqrt{v^2 (t-t')^2 - (z-z')^2} \right) j'_m (t',z')
\end{aligned} \end{equation*} }
%
Також, для скорочення запису, введемо нові зменні $ \tau = vt $ та 
$ \tau' = vt' $.
%
\begin{equation*} \begin{aligned} 
V_m^h = - \frac{1}{2} \sqrt{\frac{\mu}{\epsilon}} \partder{}{\tau} 
\int_0^\infty dz' \int_0^\infty d \tau'
H \left( \tau - \tau' - z + z' \right) \cdot \\
\cdot J_0 \left( \nu \sqrt{(\tau-\tau')^2 - (z-z')^2} \right) j'_m (\tau',z')
\end{aligned} \end{equation*}
%
Користуючись властивістю функції Хевісайда \textcolor{red} {[ПОСИЛАННЯ]} 
врахуємо її вплив змінивши межі інтегрування.
%
\textcolor{lightgray} { \begin{equation*} \begin{aligned} 
H \left( \tau - \tau' - z + z' \right) = 
H \left( - \tau' - ( z  - \tau - z' ) \right)
\end{aligned} \end{equation*} }
%
\begin{equation*} \begin{aligned} 
V_m^h = - \frac{1}{2} \sqrt{\frac{\mu}{\epsilon}} \partder{}{\tau} 
\int_0^\infty dz' \int_0^{\tau - z + z'} d \tau'
J_0 \left( \nu \sqrt{(\tau-\tau')^2 - (z-z')^2} \right) j'_m (\tau',z')
\end{aligned} \end{equation*}
%
Як і в лінійному випадку випромінювання похідну можна внести під інтеграл, але 
через складну залкжність $ j'_m $ аналітичне значення інтералу отримати не вдасться.
%
\begin{equation*} \begin{aligned} 
\int_0^{\tau - z + z'} \partder{J_0}{\tau} j'_m (\tau',z') d \tau' =
- \int_0^{\tau - z + z'} \partder{J_0}{\tau'} j'_m (\tau',z') d \tau'
\end{aligned} \end{equation*}
%
Спробуємо інший підхід: внесемо похідну під інтеграл та розглянемо похідну функції
Рімана, як складну функцію.
%
Радиальна компонента електричного поля може бути отримана підстановкою 
еволюційних коефіцієнтів $ V_m^h $ до \eqref{eq:KerrAmendErhoInit}.
%
\textcolor{lightgray} {  \begin{equation*} \begin{aligned} 
E'_\rho = \frac{1}{\sqrt{\epsilon_0}} \sum_{m=-\infty}^{\infty} 
i m e^{im\varphi} \int_{0}^{\infty} \frac{d \nu}{\sqrt{\nu}} 
V_m^h \frac{J_m(\nu \rho)}{\rho}
\end{aligned} \end{equation*} }
%
\begin{equation*} \begin{aligned}
E'_\rho = - \frac{1}{2}  \sqrt{\frac{\mu}{\epsilon_0 \epsilon}}
\sum_{m=-\infty}^\infty i m e^{im\varphi} \int_0^\infty 
\frac{d \nu}{\sqrt{\nu}} \frac{J_m(\nu \rho)}{\rho} \int_0^\infty dz' 
\int_0^\infty d \tau' j'_m (\tau',z') \cdot \\
\cdot \partder{}{\tau} \left( H(\Delta \tau - \Delta z) 
J_0 \left( \nu \sqrt{(\Delta \tau^2 - \Delta z^2} \right) \right)
\end{aligned} \end{equation*}
%
\textcolor{lightgray} {  \begin{equation*} \begin{aligned} 
j'_m (\tau',z') = - \frac{i \xi_3 A_0^3 \sqrt{\mu}}{64 \sqrt{\nu}}
\left( \frac{\mu_0 \mu} {\epsilon_0 \epsilon} \right)^{3/2}
\int_0^\infty S_m (\rho',t',z') d \rho'
\end{aligned} \end{equation*} }
%
\textcolor{lightgray} {  \begin{equation*} \begin{aligned}
E'_\rho = \frac{\xi_3 A_0^3}{128} 
\left( \frac{\mu_0 \mu}{\epsilon_0 \epsilon} \right)^2
\sum_{m=-\infty}^\infty i^2 m e^{im\varphi} \int_0^\infty d \nu 
\frac{J_m(\nu \rho)}{\nu \rho} \int_0^\infty dz' 
\int_0^\infty d \tau' \int_0^\infty d \rho' \cdot \\
\cdot S_m (\rho',t',z') \partder{}{\tau} \left( H(\Delta \tau - \Delta z) 
J_0 \left( \nu \sqrt{(\Delta \tau^2 - \Delta z^2} \right) \right)
\end{aligned} \end{equation*} }
%
Тепер маємо змогу звільнитись від комплексності та внести похідну під 
кратний інтеграл.
%
\begin{equation*} \begin{aligned}
E'_\rho = - \frac{\xi_3 A_0^3}{128} 
\left( \frac{\mu_0 \mu}{\epsilon_0 \epsilon} \right)^2
\int_0^\infty d \nu \int_0^\infty dz' 
\int_0^\infty d \tau' \int_0^\infty d \rho' \cdot \\
\cdot \left[ H(\Delta \tau - \Delta z) 
J_0 \left( \nu \sqrt{(\Delta \tau^2 - \Delta z^2} \right) \right]'_\tau
 \sum_{m=-\infty}^\infty m \frac{J_m(\nu \rho)}{\nu \rho} 
S_m (\rho',t',z') \cos m \varphi
\end{aligned} \end{equation*}
%
Функції Ріммана не залежить від номкру кутової моди $ m $, а отже її похідну 
можна винести зпід знаку суми. Таким чином, пропонуєеться ввести інтегральний
оператор згортки електричного поля по модовому базису в виді
%
\begin{equation*} \begin{aligned}
E_4 = \int_0^\infty d\nu \int_0^\infty d \tau' \int_0^\infty dz' 
\int_0^\infty d\rho' \left[ H(\Delta \tau - \Delta z) 
J_0 \left( \nu \sqrt{(\Delta \tau^2 - \Delta z^2} \right) \right]'_\tau
\end{aligned} \end{equation*}
%
Індекс $ 4 $ означає що інтегрування іде по чотиртом змінним, а залежність 
від кутової моди будо знайдено аналітично. Аналізуючи власитвості похідної 
функції Ріммана, яка має вигляд
%
\textcolor{lightgray} { \begin{equation*} \begin{aligned}
\partder{}{\tau} J_0 \left( \nu \sqrt{\Delta \tau^2 - \Delta z^2} \right) = 
- \nu \Delta \tau 
\frac{J_1 \left( \nu \sqrt{\Delta \tau^2 - \Delta z^2} \right)}
{\sqrt{\Delta \tau^2 - \Delta z^2}}
\end{aligned} \end{equation*} }
%
\textcolor{lightgray} { \begin{equation*} \begin{aligned}
\partder{}{\tau} \left( H \left( \Delta \tau - \Delta z \right)
J_0 \left( \nu \sqrt{ \Delta \tau^2 - \Delta z^2 } \right) \right) = \\
= \partder{H}{\tau} J_0 \left( \nu \sqrt{ \Delta \tau^2 - \Delta z^2 } \right) + 
H \left( \Delta \tau - \Delta z \right) \partder{J_0}{\tau} = \\
= \delta \left( \Delta \tau - \Delta z \right)
J_0 \left( \nu \sqrt{ \Delta \tau^2 - \Delta z^2 } \right) - 
\nu \Delta \tau H \left( \Delta \tau - \Delta z \right)
\frac{J_1 \left( \nu \sqrt{\Delta \tau^2 - \Delta z^2} \right)}
{\sqrt{\Delta \tau^2 - \Delta z^2}}
\end{aligned} \end{equation*} }
%
\begin{equation*} \begin{aligned}
\partder{}{\tau} \left( H \left( \Delta \tau - \Delta z \right)
J_0 \left( \nu \sqrt{ \Delta \tau^2 - \Delta z^2 } \right) \right) = \\
= \delta \left( \Delta \tau - \Delta z \right)
J_0 \left( \nu \sqrt{\Delta \tau^2 - \Delta z^2} \right) - 
\Delta \tau H(\Delta \tau - \Delta z) 
\frac{J_1 \left( \nu \sqrt{\Delta \tau^2 - \Delta z^2} \right)}
{\sqrt{\Delta \tau^2 - \Delta z^2}}
\end{aligned} \end{equation*}
%
Тепер можемо записати спрощений варіант 
компоненти $ E'_\rho $.
%
\begin{equation*} \begin{aligned}
E'_\rho = - \frac{\xi_3 A_0^3}{4^3}
\left( \frac{\mu_0 \mu} {\epsilon_0 \epsilon} \right)^2
E_4 \Big[ \sum_{m=-\infty}^\infty m \cos m \varphi 
S_m (r') \frac{J_m(\nu \rho)}{\rho \nu} \Big]
\end{aligned} \end{equation*}
%
В останьому множнику помічаємо сингулярність, розкриємо її користуючись
формулою \eqref{eq:bessel_order_change}.
%
\begin{equation*} \begin{aligned}
E'_\rho = - \frac{\xi_3 A_0^3}{4^3}
\left( \frac{\mu_0 \mu} {\epsilon_0 \epsilon} \right)^2
E_4 \Big[ \sum_{m=-\infty}^\infty \cos m \varphi 
S_m (r') \frac{J_{m-1}(\nu \rho) + J_{m+1}(\nu \rho)}{2} \Big]
\end{aligned} \end{equation*}
%
Аналогічно до компоненти $ E'_\rho $ запишимо формулу для $ E'_\varphi $
%
\textcolor{lightgray} { \begin{equation*} \begin{aligned}
E_\varphi = - \frac{1}{2 \sqrt{\epsilon_0}} \sum_{m=-\infty}^{\infty} 
e^{im\varphi} \int_{0}^{\infty} \sqrt{\nu} d \nu 
V_m^h \left( J_{m-1} (\nu \rho) - J_{m+1} (\nu \rho) \right)
\end{aligned} \end{equation*} }
%
\textcolor{lightgray} { \begin{equation*} \begin{aligned} 
V_m^h = - \frac{1}{2} \sqrt{\frac{\mu}{\epsilon}} \partder{}{\tau} 
\int_0^\infty dz' \int_0^\infty d \tau'
H \left( \tau - \tau' - z + z' \right) \cdot \\
\cdot J_0 \left( \nu \sqrt{(\tau-\tau')^2 - (z-z')^2} \right) j'_m (\tau',z')
\end{aligned} \end{equation*} }
%
\textcolor{lightgray} { \begin{equation*} \begin{aligned}
j'_m (t',z') =  - \frac{i \xi_3 A_0^3 \sqrt{\mu_0}}{64 \sqrt{\nu}}
\left( \frac{\mu_0 \mu} {\epsilon_0 \epsilon} \right)^{3/2}
\int_{0}^{\infty} S_m (\rho',t',z') d \rho'
\end{aligned} \end{equation*} }
%
\begin{equation*} \begin{aligned}
E'_\varphi = \frac{\xi_3 A_0^3}{4^3}
\left( \frac{\mu_0 \mu} {\epsilon_0 \epsilon} \right)^2
E_4 \Big[ \sum_{m=-\infty}^\infty 
\sin m \varphi S_m (r')
\frac{J_{m-1}(\nu \rho) - J_{m+1}(\nu \rho)}{2} \Big]
\end{aligned} \end{equation*}
%
Для чисельних розрахунків краще перевантажити функцію знаходження компоненти 
$ E'_x $, що скоротить розрахунки вдвоє.
%
\textcolor{lightgray} { \begin{equation*} \begin{aligned}
E'_x = E'_\rho \cos \varphi - E'_\varphi \sin \varphi
\end{aligned} \end{equation*} }
%
\begin{equation*} \begin{aligned}
E'_x = - \frac{\xi_3 A_0^3}{2 \cdot 4^3}
\left( \frac{\mu_0 \mu} {\epsilon_0 \epsilon} \right)^2
E_4 \left[ \sum_{m=-\infty}^\infty
\Big( \cos \varphi \cos m \varphi 
\big( J_{m-1}(\nu \rho) + J_{m+1}(\nu \rho) \big) \Big. \right. + \\
+ \left. \Big. \sin \varphi \sin m \varphi 
\big( J_{m-1}(\nu \rho) - J_{m+1}(\nu \rho) \big) \Big) 
S_m (\nu | r') \right]
\end{aligned} \end{equation*}
%
\begin{equation*} \begin{aligned}
E'_x = - \frac{\xi_3 A_0^3}{2 \cdot 4^3} \left(
\frac{\mu_0 \mu} {\epsilon_0 \epsilon} \right)^2
\left( E_4 \left[ \sum_{m=-\infty}^\infty 
e_m^x (\nu | r) S_m^* (\nu | r') \right] \right. + \\
+ \left. E_4 \left[ \sum_{m=-\infty}^\infty 
e_m^x (\nu | r) 2 \tau' \delta \left( {\tau'}^2 - {z'}^2 - (\rho'+R)^2 \right) 
S_m^\delta (\nu | r') \right] \right)
\end{aligned} \end{equation*}

%%%%%%%%%%%%%%%%%%%%%%%%%%%%%%%%%%%%%%%%%%%%%%%%%%%%%%%%%%%%%%%%%%%%%%%%%%%%%%%%
\section{Властивості оператора електричного поля}

помічаємо, що для уникненя сингкляроності в виді дельта-функції треба 
брати два кратних інтеграли окремо. Один за чотирьма змінними (доданок 
з функцією Хевісайда), а другий за трьома (доданок, що містить дельта-функцію).
Інтеграл по часу або по поздовжній кординаті можна взяти аналітично, для
другого доданку, користуючить наступною формулою, що може бути отримана з 
властивостей дельта-функції.
%
\textcolor{lightgray} { \begin{equation*} \begin{aligned}
\int_{-\infty}^{\infty} \delta(x-a) f(x) dx = f(a) 
\end{aligned} \end{equation*} }
%
\textcolor{lightgray} { \begin{equation*} \begin{aligned}
\int_0^\infty \int_0^\infty \delta(x-y-a) f(x,y) dx dy = 
\int_0^\infty f(y+a,y) dy
\end{aligned} \end{equation*} }
%
\textcolor{red} { \begin{equation*} \begin{aligned}
\int_0^\infty \int_0^\infty \delta(-x-y-a) f(x,y) dx dy = 
\int_0^\infty f(-y-a,y) dy
\end{aligned} \end{equation*} }
%
Для інтегралу за чотирма змінними застосуєм властивість функції Хевісайда.
%
\textcolor{lightgray} { \begin{equation*} \begin{aligned}
\int_{-\infty}^\infty H(-x+a) f(x) dx = 
\int_{-\infty}^{a} f(x) dx
\end{aligned} \end{equation*} }
%
\textcolor{red} { \begin{equation*} \begin{aligned}
\int_0^\infty \int_0^\infty H(-x+y-a) f(x,y) dx dy = 
\int_0^\infty \int_0^{y-a} f(x,y) dx dy
\end{aligned} \end{equation*} }
%
Таким чином ітеграл від похідної функції Ріманна матиме вигляд:
%
\textcolor{lightgray} { \begin{equation*} \begin{aligned}
\int_0^\infty \int_0^\infty \left[ H(\Delta \tau - \Delta z) 
J_0 \left( \nu \sqrt{(\Delta \tau^2 - \Delta z^2} \right) 
\right]'_\tau f(\tau',z') d \tau' dz' = \\
= \int_0^\infty \left( \int_0^\infty 
\delta \left( \Delta \tau - \Delta z \right)
J_0 \left( \nu \sqrt{\Delta \tau^2 - \Delta z^2} \right) 
f(\tau',z') d \tau' \right. - \\
- \left. \int_0^\infty \nu \Delta \tau H(\Delta \tau - \Delta z) 
\frac{J_1 \left( \nu \sqrt{\Delta \tau^2 - \Delta z^2} \right)}
{\sqrt{\Delta \tau^2 - \Delta z^2}} f(\tau',z') d \tau' \right) dz' = \\
= \int_0^\infty \left( f(\tau - \Delta z,z') - \int_0^{\tau - \Delta z} 
\nu \Delta \tau \frac{J_1 \left( \nu \sqrt{\Delta \tau^2 - \Delta z^2} \right)}
{\sqrt{\Delta \tau^2 - \Delta z^2}} f(\tau',z') d \tau' \right) dz'
\end{aligned} \end{equation*} }
%
\begin{equation*} \begin{aligned}
\int_0^\infty \int_0^\infty \left[ H(\Delta \tau - \Delta z) 
J_0 \left( \nu \sqrt{(\Delta \tau^2 - \Delta z^2} \right) 
\right]'_\tau f(\nu | \tau',z',\rho') d \tau' dz' = \\
= \int_0^\infty \left( f(\nu | \tau - \Delta z,z',\rho') - 
\int_0^{\tau - \Delta z} \nu \Delta \tau 
\frac{J_1 \left( \nu \sqrt{\Delta \tau^2 - \Delta z^2} \right)}
{\sqrt{\Delta \tau^2 - \Delta z^2}} 
f(\nu | \tau',z',\rho') d \tau' \right) dz'
\end{aligned} \end{equation*}
%
Разом з можливістю змінювати порядок інтегремання цю формулу можна 
використати для чисельної оцінки інтегралу. Спочатку викоричтаєто метод 
Сімпсона для оцінки значення інтегралу за $ \tau' $, а потім оцінемо 
кратний інтеграл медодом Сімпсона, що оптимізовано для кратних інтигралів.
%
\textcolor{red} { Всенаправленій излучатель єто тот что имеет только 4 
аргумента под оператором. }
%
\begin{equation*} \begin{aligned}
\int_0^\infty \int_0^\infty \int_0^\infty \int_0^\infty 
\sum_{m=-\infty}^\infty \left[ H(\Delta \tau - \Delta z) 
J_0 \left( \nu \sqrt{(\Delta \tau^2 - \Delta z^2} \right) 
\right]'_\tau f_m (\nu | \tau',z',\rho') d \tau' dz' d \rho' d \nu = \\
= \int_0^\infty \int_0^\infty \int_0^\infty
\left( \sum_{m=-\infty}^\infty f_m (\alpha) - 
\int_0^{\tau - z + z'} \nu^2 \Delta \tau 
\frac{J_1 \left( \nu \sqrt{\Delta \tau^2 - \Delta z^2} \right)}
{\nu \sqrt{\Delta \tau^2 - \Delta z^2}} 
\sum_{m=-\infty}^\infty f_m (\beta) d \tau' \right) dz' d \rho' d \nu
\end{aligned} \end{equation*}
%
де набір параметрив $ \alpha = \{ \nu | \tau-z+z', z', \rho' \} $ та 
$ \beta = \{ \nu | \tau', z', \rho' \} $
%
\begin{equation*} \begin{aligned}
\frac{J_1 \left( \nu \sqrt{\Delta \tau^2 - \Delta z^2} \right)}
{\nu \sqrt{\Delta \tau^2 - \Delta z^2}} =
\frac{J_0 \left( \nu \sqrt{\Delta \tau^2 - \Delta z^2} \right) +
J_2 \left( \nu \sqrt{\Delta \tau^2 - \Delta z^2} \right)}{2}
\end{aligned} \end{equation*}
%
\begin{equation*} \begin{aligned}
\nu J_m (\nu) J_m (\nu) \approx \frac{1}{2\pi} 
e^{\pm i \left( \nu - \frac{m\pi}{2} - \frac{\pi}{4} \right)}
\end{aligned} \end{equation*}
%
\begin{tabular}{ | l | l | l | }
\hline Номер & Призначення змінної               & Співідношення             \\ \hline
1 & Відстань, що проходить сигнал за час $t$     & $ 0 \le vt' \le vt $      \\ \hline
2 & Відстань, від точки спостереження до джерела & $ 0 \le z' \le z $        \\ \hline
3 & Принцип причинності                          & $ vt - z > 0 $            \\ \hline
4 & Принцип причинності для проміжних подій      & $ vt - vt' - z + z' > 0 $ \\ \hline
5 & Наслідок з 4 та 1                            & $ vt - z + z' > vt' > 0 $ \\ \hline
\end{tabular}

Дослідимо вплив інтегрльного опреатора $ E_4 $ на добуток  дельта-функції 
з агрументом $ {\tau'}^2 - {z'}^2 - (\rho'+R)^2 $ на довільну функцію 
відповідних аргументів. Користуючись властивістю дельта-функції звільнемось 
від одного інтегралу. Простіше за все вибрати інтеграл по $ \rho' $. 
Спочатку скористоємось властивістю симетрії дельта-функції.
%
\begin{equation*} \begin{aligned}
\delta \left( a - x \right) = \delta \left( - (x - a) \right) = 
\delta \left( x - a \right)
\end{aligned} \end{equation*}
%
Далі скористаємось методом заміни змінної, як показано далі.
%
\begin{equation*} \begin{aligned}
\int_0^\infty f(x) \delta \left( (x-a)^2 - b \right) dx = 
\left[ \begin{array}{l}
u = (x-a)^2 \\
du = 2 (x-a) dx \\
x = \pm \sqrt{u} + a
\end{array} \right] = \\  
= \int_0^\infty \frac{f(\pm \sqrt{u}+a)}{2 \sqrt{u}} 
\delta \left( u - b \right) du = 
\frac{f(\pm \sqrt{b}+a)}{2 \sqrt{b}}
\end{aligned} \end{equation*}
%
З огляду на фізику задачі $ \pm $ в останій формулі зникає, так як 
$ 0 \leq \pm \sqrt{b}+a $ та $ 0 < a $. Тоді, оператор $ E_4 $
спрощується 
%
\begin{equation*} \begin{aligned}
2 E_4 \left[ f(\nu,\tau',\rho',z') \delta 
\left( {\tau'}^2 - {z'}^2 - (\rho'+R)^2 \right) \right] = \\
= \int_0^\infty \int_0^\infty \int_0^\infty
\left( \frac{f (\gamma)}{\sqrt{{\tau'}^2 - {z'}^2}} - 
\int_0^{\tau - z + z'} \nu^2 \Delta \tau 
\frac{J_1 \left( \nu \sqrt{\Delta \tau^2 - \Delta z^2} \right)}
{\nu \sqrt{\Delta \tau^2 - \Delta z^2}} 
\frac{f (\lambda)}{\sqrt{{\tau'}^2 - {z'}^2}} 
d \tau' \right) dz' d \rho' d \nu
\end{aligned} \end{equation*}
%
де набір параматрів 
$ \gamma = \{\nu, \tau-z+z', \sqrt{(\tau-z+z')^2 - {z'}^2} + R, z'\} $ та
$ \lambda = \{\nu, \tau', \sqrt{{\tau'}^2 - {z'}^2} + R, z'\} $ 
%
Тепер запишимо остаточний вираз для чисельного розрахунку.
%
\textcolor{lightgray} { \begin{equation*} \begin{aligned}
E'_x = - \frac{\xi_3 A_0^3}{2 \cdot 4^3} \left(
\frac{\mu_0 \mu} {\epsilon_0 \epsilon} \right)^2
\left( E_4 \left[ \sum_{m=-\infty}^\infty 
e_m^x (\nu | r) S_m^* (\nu | r') \right] \right. + \\
+ \left. E_4 \left[ \sum_{m=-\infty}^\infty 
e_m^x (\nu | r) 2 \tau' \delta \left( {\tau'}^2 - {z'}^2 - (\rho'+R)^2 \right) 
S_m^\delta (\nu | r') \right] \right) = \\
= - \frac{\xi_3 A_0^3}{2 \cdot 4^3} \left(
\frac{\mu_0 \mu} {\epsilon_0 \epsilon} \right)^2
\int_0^\infty \int_0^\infty \left( \sum_{m=-\infty}^\infty f_m (\gamma) - 
\int_0^{\tau - z + z'} g(\Delta \tau, \Delta z)
\sum_{m=-\infty}^\infty f_m (\lambda) d \tau' + \right. \\ 
\left. + \int_0^\infty \left( \sum_{m=-\infty}^\infty f_m (\alpha) - 
\int_0^{\tau - z + z'} g(\Delta \tau, \Delta z)
\sum_{m=-\infty}^\infty f_m (\beta) d \tau' \right) d \rho' \right) dz' d \nu
\end{aligned} \end{equation*} }
%
де
%
\begin{equation*} \begin{aligned}
g(\tau, \tau', z, z') = \nu^2 (\tau - \tau') 
\frac{J_1 \left( \nu \sqrt{(\tau - \tau')^2 - (z - z')^2} \right)}
{\nu \sqrt{(\tau - \tau')^2 - (z - z')^2}}
\end{aligned} \end{equation*}
%
\begin{equation*} \begin{aligned}
f_m (\alpha) = e_m^x (\tau,\rho,\varphi,z) S_m^* (\tau-z+z',\rho',z')
\end{aligned} \end{equation*}
%
\begin{equation*} \begin{aligned}
f_m (\beta) = e_m^x (\tau,\rho,\varphi,z) S_m^* (\tau',\rho',z')
\end{aligned} \end{equation*}
%
\textcolor{lightgray} { \begin{equation*} \begin{aligned}
f_m (\gamma) = \frac{\tau'}{\sqrt{{\tau'}^2 - {z'}^2}} 
e_m^x (\tau,\rho,\varphi,z) S_m^\delta 
\left( \tau-z+z',\sqrt{(\tau-\Delta z)^2 - {z'}^2}+R,z' \right)
\end{aligned} \end{equation*} }
%
\textcolor{lightgray} { \begin{equation*} \begin{aligned}
f_m (\lambda) = \frac{\tau'}{\sqrt{{\tau'}^2 - {z'}^2}}
e_m^x (\tau,\rho,\varphi,z) S_m^\delta 
\left( \tau',\sqrt{(\tau-\Delta z)^2 - {z'}^2}+R,z' \right)
\end{aligned} \end{equation*} }

Для чисельного розрахунку краще перейти до власних інтегралів, обмежевши
вкрхню межу інтегруваня кінцевим значенням. Також необхідно визначитись з 
мінімально не обхідною кількістю вузлів для квадратурного чисельного методу.

Верхню межу для просторових інтегралів вибрати просто. На графічних 
зображеннях електричного поля видно, що на віддалені від диску амплітуда
імпульсу затухає швидше ніж в близу до нього, що впливає на збудження 
вториннго джерела нелінійного електричного струму. Користуючись цим фізичним
обмеженням обмежимо просторові інтеграли межою в $ 2R $. 

Кількість точнок для $ z $ та $ \rho $ необхідно вибрати відповідно до поведінки 
функції $ I_\alpha I_\beta {I_\alpha}' $.

З інтегралом по безперервному спектральному параметру $ \nu $ складніще.
Нема фізичних передумов, що дозволяють обмежити його верхньої межі, але сам 
інтеграл сходиться, що дозволяє обмежити його область визначення відповідно 
до необхідної тоності. Оцінемо осцелятивні властивості через аналіз 
асимптотики частини підінтегральної функції, що залежить від $ \nu $.
%
\textcolor{lightgray} { \begin{equation*} \begin{aligned}
f_m (\nu) = J_1 \left( \nu \sqrt{\Delta t^2 - \Delta z^2} \right) \cdot \\ 
\cdot \Big( J_{m-1} (\nu \rho) \pm J_{m+1} (\nu \rho) \Big) 
\Big( J_{m-1} (\nu \rho') \pm J_{m+1} (\nu \rho') \Big)
\end{aligned} \end{equation*} }
%
\textcolor{lightgray} { \begin{equation*} \begin{aligned}
f_1 (\nu) = J_1 \left( \nu \sqrt{\Delta t^2 - \Delta z^2} \right) \cdot \\ 
\cdot \Big( J_{0} (\nu \rho) \pm J_{2} (\nu \rho) \Big) 
\Big( J_{0} (\nu \rho') \pm J_{2} (\nu \rho') \Big)
\end{aligned} \end{equation*} }
%
\textcolor{lightgray} { \begin{equation*} \begin{aligned}
f_3 (\nu) = J_1 \left( \nu \sqrt{\Delta t^2 - \Delta z^2} \right) \cdot \\ 
\cdot \Big( J_{2} (\nu \rho) \pm J_{4} (\nu \rho) \Big) 
\Big( J_{2} (\nu \rho') \pm J_{4} (\nu \rho') \Big)
\end{aligned} \end{equation*} }
%
\textcolor{lightgray} { \begin{equation*} \begin{aligned}
\cos \left( \nu \rho - \frac{3 \pi}{4} \right)
\cos \left( \nu \rho' - \frac{3 \pi}{4} \right)
\cos \left( \nu \sqrt{\Delta t^2 - \Delta z^2} - \frac{3\pi}{4} \right)
\end{aligned} \end{equation*} }
%
\textcolor{lightgray} { \begin{equation*} \begin{aligned}
\cos \left( \nu \rho - \frac{3 \pi}{4} \right)
\cos \left( \nu \rho' - \frac{5 \pi}{4} \right)
\cos \left( \nu \sqrt{\Delta t^2 - \Delta z^2} - \frac{3\pi}{4} \right)
\end{aligned} \end{equation*} }
%
\textcolor{lightgray} { \begin{equation*} \begin{aligned}
\cos \left( \nu \rho - \frac{5 \pi}{4} \right)
\cos \left( \nu \rho' - \frac{5 \pi}{4} \right)
\cos \left( \nu \sqrt{\Delta t^2 - \Delta z^2} - \frac{3\pi}{4} \right)
\end{aligned} \end{equation*} }
%
\textcolor{lightgray} { \begin{equation*} \begin{aligned}
\cos \left( \nu \rho - \frac{5 \pi}{4} \right)
\cos \left( \nu \rho' - \frac{9 \pi}{4} \right)
\cos \left( \nu \sqrt{\Delta t^2 - \Delta z^2} - \frac{3\pi}{4} \right)
\end{aligned} \end{equation*} }
%
\textcolor{lightgray} { \begin{equation*} \begin{aligned}
\cos \left( \nu \rho - \frac{9 \pi}{4} \right)
\cos \left( \nu \rho' - \frac{9 \pi}{4} \right)
\cos \left( \nu \sqrt{\Delta t^2 - \Delta z^2} - \frac{3\pi}{4} \right)
\end{aligned} \end{equation*} }
%
\textcolor{lightgray} { \begin{equation*} \begin{aligned}
f (\nu) \sim \cos \left( \nu \rho \right) \cos \left( \nu \rho' \right)
\cos \left( \nu \sqrt{\Delta t^2 - \Delta z^2} \right)
\end{aligned} \end{equation*} }
%
\textcolor{lightgray} { \begin{equation*} \begin{aligned}
f (\nu) \sim \cos \left( \nu \sqrt{\Delta t^2 - \Delta z^2} \right) \Big( 
\cos \left( \nu \left( \rho - \rho' \right) \right) +
\cos \left( \nu \left( \rho + \rho' \right) \right) \Big) \sim \\
\sim \cos \left( \nu \left( \rho - \rho' - 
\sqrt{\Delta t^2 - \Delta z^2} \right) \right) + 
\cos \left( \nu \left( \rho - \rho' + 
\sqrt{\Delta t^2 - \Delta z^2} \right) \right) + \\
+ \cos \left( \nu \left( \rho + \rho' - 
\sqrt{\Delta t^2 - \Delta z^2} \right) \right) +
\cos \left( \nu \left( \rho - \rho' + 
\sqrt{\Delta t^2 - \Delta z^2} \right) \right)
\end{aligned} \end{equation*} }
%
\begin{equation*} \begin{aligned}
f (\nu) \sim \sum_i \alpha_i (\nu) 
\cos \big( \left| \omega_i \right| \nu \big)
\end{aligned} \end{equation*}
%
Підінтегральна функція $ f(\nu) $ пропорцяйна до деякої лінійної комбінації
парних тригонометричних функцій. Частоти осциляцій $ \omega_i $ залежать, 
як від змінних інтегрування, так і від кординат токи спостереження.
%
\begin{equation*} \begin{aligned}
\omega_1 = \rho - \rho' - \sqrt{(\tau-\tau')^2 - (z-z')^2}
\end{aligned} \end{equation*}
%
\begin{equation*} \begin{aligned}
\omega_2 = \rho - \rho' + \sqrt{(\tau-\tau')^2 - (z-z')^2}
\end{aligned} \end{equation*}
%
\begin{equation*} \begin{aligned}
\omega_3 = \rho + \rho' - \sqrt{(\tau-\tau')^2 - (z-z')^2}
\end{aligned} \end{equation*}
%
\begin{equation*} \begin{aligned}
\omega_4 = \rho + \rho' + \sqrt{(\tau-\tau')^2 - (z-z')^2}
\end{aligned} \end{equation*}
%
Найменше значення частоти визначає період зміни підінтегральної функції.
Для $ \rho = 0 $ найменше значення періоду є $ \omega_1 $ чи $ \omega_2 $.
Для чисельного інтегрування візьмемо деяку кількісь таких періодів, що 
забеспечить фіксовану точність. Найбільше значення частоти $ \omega_4 $ вкаже 
на необхідну кількість вузлів на період при необхідній точності.
%
\begin{equation*} \begin{aligned}
E'_x = - \frac{\xi_3 A_0^3}{2 \cdot 4^3} \left(
\frac{\mu_0 \mu} {\epsilon_0 \epsilon} \right)^2
\int_0^{2R} \int_0^{2R}
\left( \int_0^{N_\alpha} \sum_{m=-\infty}^\infty f_m (\alpha) d \nu \right. - \\ 
\left. - \int_0^{\tau - z + z'} \int_0^{N_\beta} g(\Delta \tau, \Delta z)
\sum_{m=-\infty}^\infty f_m (\beta) d \nu d \tau' \right) d \rho' dz'
\end{aligned} \end{equation*}
%
\begin{equation*} \begin{aligned}
N_\alpha = \left| \rho - \rho' \right|
\end{aligned} \end{equation*}
%
\begin{equation*} \begin{aligned}
N_\beta = \left| \rho - \rho' + \sqrt{(\tau-\tau')^2 - (z-z')^2} \right|
\end{aligned} \end{equation*}

%%%%%%%%%%%%%%%%%%%%%%%%%%%%%%%%%%%%%%%%%%%%%%%%%%%%%%%%%%%%%%%%%%%%%%%%%%%%%%
\section{Врахування провідності середовища}

Для нестаціонарного випромінювання

%%%%%%%%%%%%%%%%%%%%%%%%%%%%%%%%%%%%%%%%%%%%%%%%%%%%%%%%%%%%%%%%%%%%%%%%%%%%%%
\textcolor{red} {\section{Лінійне поле від прямокутного збуджуючого імпульсу}}
%
\begin{equation}
\vect{j} \left( r, t \right) = \vect{x_0} A_0 \delta(z) 
\left(  H(\rho) - H(\rho - R) \right) \left( H(t) - H(t-\tau) \right)
\end{equation}
%
\begin{equation}
\vect{j} \left( r, t \right) = \vect{j_0} \left( r, t \right) -
\vect{j_0} \left( r, t - \tau \right)
\end{equation}
%
\textcolor{lightgray} { \begin{equation*}
\vect{j_0} \left( r, t \right) = \vect{x_0} A_0 H(t) \delta(z) 
\left(  H(\rho) - H(\rho - R) \right)
\end{equation*} }
%
\begin{equation}
\vect{E} \left( r, t \right) = \vect{E_0} \left( r, t \right) -
\vect{E_0} \left( r, t - \tau \right)
\end{equation}
%
\begin{equation}
\vect{E} \left( r, t, A_0, \tau \right) = \frac{A_0}{2} 
\sqrt{\frac{\mu_0 \mu}{\epsilon_0 \epsilon}}
\Big( \vect{\rho_0} I_1 \cos \varphi - 
\vect{\varphi_0} \left( I_2 - I_1 \right) \sin \varphi \Big)
\end{equation}
%
\textcolor{red} { Переходячи до термінології теорії довгих ліній помічаємо, 
що область $ S_{11} $ відповідає коефіцієнту відбиття чотириполюсника.}
%
\begin{equation}
S = \left\{ S_{11}, S_{12}, S_{22}, S_{23}, S_{33} \right\}
\end{equation}
%
\begin{equation}
I'_m \{S_{ij}\} = I_m (t) \{S_i\} - I_m (t-\tau) \{S_j\}
\end{equation}
%
\textcolor{lightgray} { \begin{equation*}
S_1: 0 < v^2 t^2 - z^2 < \left( \rho - R \right)^2
\end{equation*}
%
\begin{equation*}
S_2: \left( \rho - R \right)^2 < v^2 t^2 - z^2 < \left( \rho + R \right)^2
\end{equation*}
%
\begin{equation*}
S_3: v^2 t^2 - z^2 > \left( \rho + R \right)^2
\end{equation*} }
%
Для лінійного розв'язку кінець сигналу не може упередити його початок, 
тому будуть присутні тільки наступні області, а при нелінійному розвязку 
можлива поява областей де початок сигналу опиняється після його кінця.
%
\begin{equation*}
S_{11}: 0 < v^2 (t-\tau)^2 - z^2 < v^2 t^2 - z^2 < 
\left( \rho - R \right)^2 < \left( \rho + R \right)^2
\end{equation*}
%
\begin{equation*}
S_{12}: 0 < v^2 (t-\tau)^2 - z^2 < \left( \rho - R \right)^2 < 
v^2 t^2 - z^2 < \left( \rho + R \right)^2
\end{equation*}
%
\begin{equation*}
S_{22}: 0 < \left( \rho - R \right)^2 < v^2 (t-\tau)^2 - z^2 < 
v^2 t^2 - z^2 < \left( \rho + R \right)^2
\end{equation*}
%
\begin{equation*}
S_{23}: 0 < \left( \rho - R \right)^2 < v^2 (t-\tau)^2 - z^2 < 
\left( \rho + R \right)^2 < v^2 t^2 - z^2 
\end{equation*}
%
\begin{equation*}
S_{33}: 0 < \left( \rho - R \right)^2 < \left( \rho + R \right)^2 <
v^2 (t-\tau)^2 - z^2 < v^2 t^2 - z^2
\end{equation*}
%
\begin{equation*}
S_{13}: 0 < v^2 (t-\tau)^2 - z^2 < \left( \rho - R \right)^2 < 
\left( \rho + R \right)^2 < v^2 t^2 - z^2
\end{equation*}

%%%%%%%%%%%%%%%%%%%%%%%%%%%%%%%%%%%%%%%%%%%%%%%%%%%%%%%%%%%%%%%%%%%%%%%%%%%%%%
\section{Вторинне поле від прямокутного збуджуючого імпульсу}


