\chapter{Метод еволюційних рівнянь}
\label{ch:evolution}

%%%%%%%%%%%%%%%%%%%%%%%%%%%%%%%%%%%%%%%%%%%%%%%%%%%%%%%%%%%%%%%%%%%%%%%%%%%%%%
\section{Матеріальні рівняння середовища}

Електромагнітні властивості середовища можна математично описати шляхом
визначення векторів електричної $ \vect{D} $ та магнітної $ \vect{B} $ 
індукції за допомогою математичних рівнянь.
%
\begin{equation} \label{eq:MInduct}
\vect{D} = \epsilon_0 \vect{E} + \func{\vect{P}}{\vect{E},\vect{H}}
\end{equation}
%
\begin{equation} \label{eq:EInduct} 
\vect{B} = \mu_0 \vect{H} + \mu_0 \func{\vect{M}}{\vect{E},\vect{H}}
\end{equation}

Нелінійне середовище характеризується нелінійною залежністю поляризації
$ \vect{P} $ і намагніченості $ \vect{M} $ від векторів напруження 
електромагнітного поля. 

\textcolor{red}{Коли Р залежить від Н? Чи можна її не враховувати 
надалі. Приклади.}

\textcolor{red}{Яка фізична остова лежить у відмінності розмірностей доданих?}

В загальному випадку вектор поляризації має довільній вид та залежать від 
магнітної та електричної складової поля, а у лінійному випадку має вид
$ \epsilon \vect{E} $. Відносна діелектрична проникність середовища 
$ \epsilon $ взагалі є матриця, кожний з елементів якої, залежить від 
повного переліку незалежних координат та часу. Розглянемо шарувате середовище, 
як середу для розповсюдження і припустимо що фронт хвильового пакету проходить 
через шари під прямим кутом, тоді $ \epsilon $ є скалярна функція, що залежить 
лише від поздовжньої координати та часу. Аналогічні міркування можна провести 
і для вектора намагніченості.

\textcolor{red}{Уточнити чи не є $ \epsilon $ тензором для нелінійних 
компонент.}

Для задач слабкої нелінійності оптичної фізики використовується ряд Тейлора, 
так як при \textcolor{red}{не надто сильних полях} вклад більших степенів,
дійсно, мінімізується за рахунок невеликого відхилення від лінійної
функції \textcolor{red}{[джерело]}.
%
\begin{equation} \label{eq:polar}
\vect{P} = \epsilon_0 \left( \epsilon - 1 \right) \vect{E} + 
\vect{P^\prime}  = \epsilon_0 \left( \epsilon - 1 \right)
\vect{E} + \sum\limits_{i=2}^\infty  {\chi^e}_i \vect{E}^i 
\end{equation}
%
\textcolor{lightgray}{ \begin{equation*} \begin{aligned}
\vect{D} = \epsilon_0 \epsilon \vect{E} + \vect{P^\prime}
\end{aligned} \end{equation*} }
%
\begin{equation} \label{eq:magnit}
\vect{M} = \left( \mu - 1 \right) \vect{H} + 
\vect{M^\prime} = \left( \mu - 1 \right)
\vect{H} + \sum\limits_{i=2}^\infty  {\chi^m}_i \vect{H}^i 
\end{equation}
%
\textcolor{lightgray}{ \begin{equation*} \begin{aligned}
\vect{B} = \mu_0 \mu  \vect{H} + \mu_0 \vect{M^\prime}
\end{aligned} \end{equation*} }

Тут перший додаток має особливий фізичний смисл -- це лінійна складова поля та 
складова з найбільшим абсолютнім значенням коефіцієнту при векторі 
напруженості. Фізично, коефіцієнт є відносною проникністю середовища для 
відповідного степеню поля. Всі додатки крім першого це нелінійні складові поля,
кожен з яких має свій фізичний смисл.  Позначмо суму нелінійних складових 
векторів поляризації та намагніченості $ \vect{P^\prime} $ та 
$ \vect{M^\prime} $ відповідно.

\textcolor{red}{Смисл перших 5и додатків (таблиця).}

%%%%%%%%%%%%%%%%%%%%%%%%%%%%%%%%%%%%%%%%%%%%%%%%%%%%%%%%%%%%%%%%%%%%%%%%%%%%%%
\section{Рівняння Максвела}

\textcolor{red}{Закон Ампера}
\begin{equation} \label{eq:AmpereLow}
\crossprod{\nabla}{\vect{H}} = 
\frac{\partial \vect{D}}{\partial t} + \vect{J^\sigma} + \vect{J^e}
\end{equation}
%
\textcolor{red}{Закон индукції Фарадея}
\begin{equation} \label{eq:FaradayInduction}
-\crossprod{\nabla}{\vect{E}} =
\frac{\partial \vect{B}}{\partial t} + \vect{J^{h}}
\end{equation}
%
\textcolor{red}{Теорема Гаусса}
\begin{equation} \label{eq:GaussTheorem}
\dotprod{\nabla}{\vect{D}} = \rho^\sigma + \rho^e
\end{equation}
%
\textcolor{red}{Теорема Гаусса для магнітного поля}
\begin{equation} \label{eq:GaussMagnetic}
\dotprod{\nabla}{\vect{B}} = \rho^h
\end{equation}

%%%%%%%%%%%%%%%%%%%%%%%%%%%%%%%%%%%%%%%%%%%%%%%%%%%%%%%%%%%%%%%%%%%%%%%%%%%%%%
\subsection{Узагальнене джерело поля для задач випромінювання}

Додатки з нелінійними складовими векторів поляризації та намагніченості мають
розмірність густин струму, відповідно. Введемо узагальнений електричний 
$ \vect{J} $ та $ \vect{I} $ магнітній струми таким чином, щоб ці додатки 
не заважали майбутнім міркуванням. Ця дія відповідає фізичному змісту цих 
додатків та не порушує математичної консеквентності, що буде обумовлено далі.
%
\begin{equation*}
\vect{J} = \partder{\vect{P^\prime}}{t} + 
\vect{J^\sigma} + \vect{J^e}
\end{equation*}
%
\begin{equation*}
\vect{I} = \mu_0 \partder{\vect{M^\prime}}{t} + \vect{J^h}
\end{equation*}

Підставляючи поляризацію \eqref{eq:polar} і намагніченість 
\eqref{eq:magnit} до матеріальних рівнянь \eqref{eq:EInduct} и 
\eqref{eq:MInduct} з наступною підставковою в роторні рівняння Максвелла
\eqref{eq:AmpereLow} и \eqref{eq:FaradayInduction} отримаємо наступне: 
%
\textcolor{lightgray}{ \begin{equation*} \begin{aligned}
\crossprod{\nabla}{\vect{H}} = \epsilon_0 \partder{}{t} \left[ 
\vect{E} + \left( \epsilon - 1 \right) \vect{E} \right] + 
\partder{\vect{P^\prime}}{t} + \vect{J^\sigma} + \vect{J^e}= \\
= \epsilon_0 \partder{}{t} \left( \epsilon \vect{E} \right) +
\partder{\vect{P^\prime}}{t} + \vect{J^\sigma} + \vect{J^e} = 
\epsilon_0 \left( \partder{\epsilon}{t} 
\vect{E} + \epsilon \partder{\vect{E}}{t} \right) + 
\partder{\vect{P^\prime}}{t} + \vect{J^\sigma} + \vect{J^e}
\end{aligned} \end{equation*} }
%
\begin{equation} \label{eq:rotHfromE}
\crossprod{\nabla}{\vect{H}} = 
\epsilon_0 \partder{}{t} \left( \epsilon \vect{E} \right) + \vect{J}
\end{equation}
%
\begin{equation} \label{eq:rotEfromH} 
- \crossprod{\nabla}{\vect{E}} = 
\mu_0 \partder{}{t} \left( \mu \vect{H} \right) + \vect{I}
\end{equation}

Схожа ситуація і для джерел що представлена зарядами. Нехай наступні вирази
опишуть узагальнену електричну $ \varrho $ (ро) та магнітну $ g $ густини 
заряду.
%
\begin{equation*}
\varrho = \rho^\sigma + \rho^e - \dotprod{\nabla}{\vect{P^\prime}}
\end{equation*}
%
\begin{equation*}
g = \rho^h - \mu_0 \dotprod{\nabla}{\vect{M^\prime}}
\end{equation*}

Підставляючи поляризацію та намагніченість до відповідних формулювань теореми 
Гаусса отримаємо її вигляд для задачі слабкої нелінійності в анізотропному 
середовищі.
%
\textcolor{lightgray}{ \begin{equation*} \begin{aligned}
\dotprod{\nabla}{ \left( \epsilon_0 \epsilon \vect{E} + 
\vect{P^\prime} \right) } = \rho^\sigma + \rho^e \\
\dotprod{\nabla}{ \epsilon_0 \epsilon \vect{E} } = \rho^\sigma + \rho^e -
\dotprod{\nabla}{ \vect{P^\prime} }
\end{aligned} \end{equation*} }
%
\begin{equation} \label{eq:divE} 
\epsilon_0 \dotprod{\nabla}{ \epsilon \vect{E} } = \varrho
\end{equation}
%
\begin{equation} \label{eq:divH}
\mu_0 \dotprod{\nabla}{ \mu \vect{H} } = g
\end{equation}

%%%%%%%%%%%%%%%%%%%%%%%%%%%%%%%%%%%%%%%%%%%%%%%%%%%%%%%%%%%%%%%%%%%%%%%%%%%%%%
\subsection{Виключення поздовжних компонент поля}

Диференціальні рівняння першого порядку \eqref{eq:divE}, \eqref{eq:divH} та 
векторні другого \eqref{eq:rotHfromE}, \eqref{eq:rotEfromH} формують систему 
рівнянь Максвелла відносно невідомих векторних величин $ \vect{E} $ і 
$ \vect{H} $.

Для спрощення цієї системи пропонується використати метод розділення змінних
Фур'є. Аналогічно до методу функції Гріна з класичної електродинаміки, 
спрощення відбувається шляхом зменшення кількості невідомих 
\textcolor{red}{Джерело}, вилучаючи їх з рівняння. 

Метод Функції Гріна як і будь-який метод частотної області, вибирає саме час, 
як змінну для виключення, обмежуючи себе розгляданням квазі-стаціонарних 
процесів. Метод еволюційних рівнянь, в свою чергу, пропонує виключення 
просторової змінної. З трьох просторових координат можна виділити одну -- вісь 
розповсюдження поля. 

Виключення саме цієї просторової залежності зумовлено тісним зв'язком 
координати розповсюдження з координатою часу через принцип причинності. Його 
сутність в термінології спеціальної теорії відносності полягає в тому, що дві 
події можуть бути причинно зв'язані одна з одної тоді, і тільки коли, інтервал 
між ними часоподібний, що напряму слідує з того, що ніяка взаємодія не може 
розповсюджуватись швидше за світло. \cite[ст. 22]{imp:LandauII}. В 
електродинамічному сенсі це означає, що поле не може розповсюдитись далі у 
вільному просторі, ніж може пройти світло за той самий час та по тій самій осі 
випромінювання. Математично це можна записати, як $ ct - z > 0 $, де $ z $
поздовжна просторова координата розповсюдження, а $ c = 2,998 \cdot 10^8 $ м/с 
-- фундаментальна константа, швидкість світла в вакуумі.

З рівнянь Максвела відокремимо векторну компоненту 
$ \vect{z_0} $ для всіх величин. Як відомо, буль-який вектор можна розписати, 
як суму добутків ортів та відповідних проекцій: так оператор $ \nabla $ можна 
записати як $ \nabla_\perp + \vect{z_0} \partder{}{z} $, а довільний вектор
$ \vect{A} $, як $ \vect{A_\perp} + \vect{z_0} A_z $. Користуючись визначенням 
векторного добутку лінійної комбінації векторів, з \eqref{eq:rotHfromE} 
отримаємо два незалежні рівняння.
%
\textcolor{lightgray}{ \begin{equation*} \begin{aligned}
\rot{\vect{A}} = \crossprod{\nabla}{\vect{A}} = \crossprod
{\left( \nabla_\perp + \vect{z_0} \partder{}{z} \right)}
{\left( \vect{A_\perp} + \vect{z_0} A_z \right)} = \\
= \crossprod{\nabla_\perp}{\vect{A_\perp}} + 
\crossprod{\nabla_\perp}{\vect{z_0} A_z} +
\crossprod{\vect{z_0} \partder{}{z}}{\vect{A_\perp}} +
\crossprod{ \vect{z_0} \partder{}{z} }{ \vect{z_0} A_z } = \\
= \crossprod{\nabla_\perp}{\vect{A_\perp}} + 
\crossprod{\nabla_\perp}{\vect{z_0}} A_z +
\partder{}{z} \crossprod{\vect{z_0}}{\vect{A_\perp}}
\end{aligned} \end{equation*} }
%
\textcolor{lightgray}{ \begin{equation*} \begin{aligned}
\crossprod{\nabla}{\vect{H}} = 
\crossprod{\nabla_\perp}{\vect{H_\perp}} + 
\crossprod{\nabla_\perp}{\vect{z_0}} H_z +
\partder{}{z} \crossprod{\vect{z_0}}{\vect{H_\perp}} = \\
= \epsilon_0 \partder{}{t} \left( \epsilon  \vect{E_\perp} + 
\epsilon \vect{z_0} E_z \right) + \vect{J_\perp} + \vect{z_0} J_z
\end{aligned} \end{equation*} }
%
\begin{equation} \label{eq:rotHt} 
\crossprod{\nabla_\perp}{\vect{z_0}} H_z +
\partder{}{z} \crossprod{\vect{z_0}}{\vect{H_\perp}} =
\epsilon_0 \partder{}{t} \left( \epsilon  \vect{E_\perp} \right) + 
\vect{J_\perp}
\end{equation}
%
\textcolor{lightgray}{ \begin{equation*} \begin{aligned}
\dotprod{\vect{z_0}}{\crossprod{\nabla_\perp}{\vect{H_\perp}}} =
\triple{\vect{z_0}}{\nabla_\perp}{\vect{H_\perp}}
\end{aligned} \end{equation*} }
%
\begin{equation} \label{eq:rotHz}
\triple{\vect{z_0}}{\nabla_\perp}{\vect{H_\perp}} = 
\epsilon_0 \partder{}{t} \left( \epsilon  E_z \right) + J_z 
\end{equation}

Аналогічні міркування можна провести по відношенню до закону індукції Фарадея 
записаного по формі \eqref{eq:rotEfromH}.
%
\textcolor{lightgray}{ \begin{equation*} \begin{aligned}
- \crossprod{\nabla}{\vect{E}} = 
- \crossprod{\nabla_\perp}{\vect{E_\perp}} - 
\crossprod{\nabla_\perp}{\vect{z_0}} E_z -
\partder{}{z} \crossprod{\vect{z_0}}{\vect{E_\perp}} = \\
= \mu_0 \partder{}{t} \left( \mu  \vect{H_\perp} + \mu \vect{z_0} H_z \right) + 
\vect{I_\perp} + \vect{z_0} I_z
\end{aligned} \end{equation*} }
%
\begin{equation} \label{eq:rotEt} 
- \crossprod{\nabla_\perp}{\vect{z_0}} E_z -
\partder{}{z} \crossprod{\vect{z_0}}{\vect{E_\perp}} = 
\mu_0 \partder{}{t} \left( \mu  \vect{H_\perp} \right) + \vect{I_\perp}
\end{equation}
%
\textcolor{lightgray}{ \begin{equation*} \begin{aligned}
- \dotprod{\vect{z_0}}{\crossprod{\nabla_\perp}{\vect{E_\perp}}} = 
- \triple{\vect{z_0}}{\nabla_\perp}{\vect{E_\perp}}
\end{aligned} \end{equation*} }
%
\begin{equation} \label{eq:rotEz}
- \triple{\vect{z_0}}{\nabla_\perp}{\vect{E_\perp}} =
\mu_0 \partder{}{t} \left(\mu H_z \right) + I_z
\end{equation}

З теореми Гауса \eqref{eq:divE} та її інтерпретації до магнітного поля 
\eqref{eq:divH} також виключимо $ \vect{z_0} $ компоненту, тепер, користуючись 
комутативними та асоціативними властивостями скалярного добутку векторів.
%
\textcolor{lightgray}{ \begin{equation*} \begin{aligned}
\dotprod{\nabla}{\vect{A}} = \dotprod
{\left( \nabla_\perp + \vect{z_0} \partder{}{z} \right)}
{\left( \vect{A_\perp} + \vect{z_0} A_z \right)} = \\
= \dotprod{\nabla_\perp}{\vect{A_\perp}} + 
\dotprod{\nabla_\perp}{\vect{z_0} A_z}  +
\dotprod{\vect{z_0} \partder{}{z}}{\vect{A_\perp}} +
\dotprod{\vect{z_0} \partder{}{z}}{\vect{z_0} A_z} = \\
= \dotprod{\nabla_\perp}{\vect{A_\perp}} +
\dotprod{\nabla_\perp}{\vect{z_0}} A_z +
\dotprod{\vect{z_0}}{\partder{\vect{A_\perp}}{z}} +
\dotprod{\vect{z_0}}{\vect{z_0}} \partder{A_z}{z} = \\
= \dotprod{\nabla_\perp}{\vect{A_\perp}} + \partder{A_z}{z}
\end{aligned} \end{equation*} }
%
\begin{equation} \label{eq:divEt} 
\epsilon_0 \partder{}{z} \left( \epsilon E_z \right) = 
\varrho - \epsilon_0 \epsilon \dotprod{\nabla_\perp}{\vect{E_\perp}}
\end{equation}
%
\begin{equation} \label{eq:divHt}
\mu_0 \partder{}{z} \left( \mu H_z \right) = 
g - \mu_0 \mu \dotprod{\nabla_\perp}{\vect{H_\perp}}
\end{equation}

Як зазначалось раніше, в данні роботі розглядається пошарово неоднорідне 
середовище, а отже $ \epsilon = \epsilon(z,t) $ и $ \mu = \mu(z,t) $. Тому
маємо змогу винести показники проникності середовища з під оператора у 
від'ємнику.

Як видно з рівнянь \eqref{eq:rotHt} -- \eqref{eq:divHt}, поздовжні компоненти 
поля однозначно розділились, але саме поле залишилось електромагнітним, що 
типово для нестаціонарних задач. Для стаціонарних задач, в просторовому та 
електродинамічному планах, поле розділилось б на чисто електричне та чисто
магнітне за рахунок нульових похідних від часу.

Випишемо в окрему систему тільки ті рівняння, що містять $ H_z $ та виразимо з 
них саму цю компоненту. Тепер, діючи на рівняння \eqref{eq:rotHt} операторами 
$ \mu_0 \partder{}{t} \mu $ і $ \mu_0 \partder{}{z} \mu $ виключимо поздовжну 
магнітну компоненту з рівнянь Максвелла, підставивши, відповідно, 
\eqref{eq:divHt} та \eqref{eq:rotEz}. В результаті маємо систему не з трьох 
рівнянь, а вже з двох відносно поперечних компонент поля. Результат зашипимо не 
в вигляді системи, а як чотиривимірне векторне рівняння.
%
\textcolor{lightgray}{ \begin{equation*} \begin{aligned}
\begin{cases} 
\crossprod{\nabla_\perp}{\vect{z_0}} H_z =
\epsilon_0 \partder{}{t} \left( \epsilon \vect{E_\perp} \right) -
\partder{}{z} \crossprod{\vect{z_0}}{\vect{H_\perp}} + \vect{J_\perp} \\
- \triple{\vect{z_0}}{\nabla_\perp}{\vect{E_\perp}} =
\mu_0 \partder{}{t} \left(\mu H_z \right) + I_z \\ 
\mu_0 \partder{}{z} \left( \mu H_z \right) = 
g - \mu_0 \mu \dotprod{\nabla_\perp}{\vect{H_\perp}}
\end{cases}
\end{aligned} \end{equation*} }
%
\textcolor{lightgray}{ \begin{equation*} \begin{aligned}
\begin{cases} 
\left. \crossprod{\nabla_\perp}{\vect{z_0}} H_z = \vect{F_H} 
\right| \cdot \mu_0 \partder{}{z} \mu \\
\left. \crossprod{\nabla_\perp}{\vect{z_0}} H_z = \vect{F_H} 
\right| \cdot \mu_0 \partder{}{t} \mu \\
\mu_0 \partder{}{z} \left( \mu H_z \right) = 
g - \mu_0 \mu \dotprod{\nabla_\perp}{\vect{H_\perp}} \\
\mu_0 \partder{}{t} \left(\mu H_z \right) =
\triple{\nabla_\perp}{\vect{z_0}}{\vect{E_\perp}} - I_z
\end{cases}
\end{aligned} \end{equation*} }
%
\textcolor{lightgray}{ \begin{equation*} \begin{aligned}
\begin{cases} 
\crossprod{\nabla_\perp}{\vect{z_0}} \left(
g - \mu_0 \mu \dotprod{\nabla_\perp}{\vect{H_\perp}} \right) =
\mu_0 \partder{}{z} \left( \mu \vect{F_H} \right) \\
\crossprod{\nabla_\perp}{\vect{z_0}} \left(
\triple{\nabla_\perp}{\vect{z_0}}{\vect{E_\perp}} - I_z \right) = 
\mu_0 \partder{}{t} \left( \mu \vect{F_H} \right)
\end{cases}
\end{aligned} \end{equation*} }
%
\textcolor{lightgray}{ \begin{equation*} \begin{aligned}
\begin{cases} 
- \mu_0 \mu \dotprod{\crossprod{\nabla_\perp}{\vect{z_0}} \nabla_\perp}
{\vect{H_\perp}} = \mu_0 \partder{}{z} \left( \mu \vect{F_H} \right) -
\crossprod{\nabla_\perp}{\vect{z_0}} g \\
\left. \crossprod{\nabla_\perp 
\triple{\nabla_\perp}{\vect{z_0}}{\vect{E_\perp}}
}{\vect{z_0}} = \mu_0 \partder{}{t} \left( \mu \vect{F_H} \right) +
\crossprod{\nabla_\perp}{\vect{z_0}} I_z \right| \times \vect{z_0}
\end{cases}
\end{aligned} \end{equation*} }
%
\textcolor{lightgray}{ \begin{equation*} \begin{aligned}
\crossprod{ \crossprod
{\nabla_\perp \triple{\nabla_\perp}{\vect{z_0}}{\vect{E_\perp}}}
{\vect{z_0}} }{ \vect{z_0} } = \crossprod{ \crossprod{\nabla_\perp \phi}
{\vect{z_0}} }{\vect{z_0}} = \\ = - \crossprod{ \vect{z_0} }{ 
\crossprod{\nabla_\perp \phi}{\vect{z_0}} } = - \dotprod{\nabla_\perp \phi}
{ \dotprod{\vect{z_0}}{\vect{z_0}} } + \dotprod{\vect{z_0}}
{ \dotprod{\vect{z_0}}{\nabla_\perp \phi} } = \\ = - \nabla_\perp \phi = 
- \nabla_\perp \dotprod{\crossprod{\nabla_\perp}{\vect{z_0}}}{\vect{E_\perp}} = 
\dotprod{\nabla_\perp \crossprod{\vect{z_0}}{\nabla_\perp}}{\vect{E_\perp}}
\end{aligned} \end{equation*} }
%
\textcolor{lightgray}{ \begin{equation*} \begin{aligned}
\crossprod {\crossprod{\nabla_\perp}{\vect{z_0}} I_z}{\vect{z_0}} = 
- \crossprod {\vect{z_0}}{\crossprod{\nabla_\perp}{\vect{z_0}} I_z} = \\
= - \dotprod{{\nabla_\perp}}{\dotprod{\vect{z_0}}{\vect{z_0}}} I_z + 
\dotprod{\vect{z_0}}{\dotprod{\vect{z_0}}{{\nabla_\perp}}} I_z = 
- \nabla_\perp I_z 
\end{aligned} \end{equation*} }
%
\textcolor{lightgray}{ \begin{equation*} \begin{aligned}
\begin{cases} 
\dotprod{\crossprod{\vect{z_0}}{\nabla_\perp} \nabla_\perp} {\vect{H_\perp}} = 
\mu^{-1} \partder{}{z} \left( \mu \vect{F_H} \right) +
\left( \mu_0 \mu \right)^{-1} \crossprod{\vect{z_0}}{\nabla_\perp} g \\
\dotprod{\nabla_\perp \crossprod{\vect{z_0}}{\nabla_\perp}}{\vect{E_\perp}}
= - \mu_0 \partder{}{t} \left( \mu \crossprod{\vect{z_0}}{\vect{F_H}} \right) -
\nabla_\perp I_z 
\end{cases}
\end{aligned} \end{equation*} }
%
\begin{equation}
\left( \begin{array}{c} 
\dotprod{\crossprod{\vect{z_0}}{\nabla_\perp} \nabla_\perp} {\vect{H_\perp}} \\
\dotprod{\nabla_\perp \crossprod{\vect{z_0}}{\nabla_\perp}}{\vect{E_\perp}} \\
\end{array} \right) = \left( \begin{array}{c} 
\frac{1}{\mu} \partder{}{z} \left( \mu \vect{F_H} \right) +
\frac{1}{\mu_0 \mu} \crossprod{\vect{z_0}}{\nabla_\perp} g \\
- \mu_0 \partder{}{t} \left( \mu \crossprod{\vect{z_0}}{\vect{F_H}} \right) -
\nabla_\perp I_z 
\end{array} \right)
\end{equation}
%
\begin{equation*}
\vect{F_H} = \epsilon_0 \partder{}{t} \left( \epsilon \vect{E_\perp} \right) - 
\partder{}{z} \crossprod{\vect{z_0}}{\vect{H_\perp}} + \vect{J_\perp}
\end{equation*}

Переозначення $ \vect{F_H} $ не несе фізичного змісту, а введено лише для 
спрощення виду формул. Аналогічним чином виключимо компоненту $ E_z $. Отримане 
векторне рівняння матиме наступний вид.
%
\begin{equation}
\left( \begin{array}{c} 
\dotprod{\nabla_\perp \crossprod{\vect{z_0}}{\nabla_\perp}} {\vect{H_\perp}} \\
\dotprod{\crossprod{\vect{z_0}}{\nabla_\perp} \nabla_\perp}{\vect{E_\perp}} \\
\end{array} \right) = \left( \begin{array}{c} 
- \epsilon_0 \partder{}{t} \left( \epsilon \crossprod{\vect{F_E}}{\vect{z_0}} 
\right) - \nabla_\perp J_z \\
\frac{1}{\epsilon} \partder{}{z} \left( \epsilon \vect{F_E} \right) +
\frac{1}{\epsilon_0 \epsilon} \crossprod{\nabla_\perp \varrho}{\vect{z_0}}
\end{array} \right)
\end{equation}
%
\begin{equation*}
\vect{F_E} = \mu_0 \partder{}{t} \left( \mu  \vect{H_\perp} \right) +
\partder{}{z} \crossprod{\vect{z_0}}{\vect{E_\perp}} + \vect{I_\perp}
\end{equation*}

Ліва частина матричних рівнянь може бути представлена в виді оператора, 
що діє на чотиривимірній вектор. Таким чином для  

%%%%%%%%%%%%%%%%%%%%%%%%%%%%%%%%%%%%%%%%%%%%%%%%%%%%%%%%%%%%%%%%%%%%%%%%%%%%%%
% \section{Побудова модового базису}

%%%%%%%%%%%%%%%%%%%%%%%%%%%%%%%%%%%%%%%%%%%%%%%%%%%%%%%%%%%%%%%%%%%%%%%%%%%%%%
\section{Згортка електромагнітного поля по базису}

\textcolor{red} { \begin{equation}
\vect{H_\perp} = \frac{1}{\sqrt{\mu_0}} \left( 
\sum \limits_{m=-\infty}^{\infty} \int \limits_{0}^{\infty} d \nu
I_m^h \nabla_\perp \Psi_m + \sum \limits_{n=-\infty}^{\infty}
\int \limits_{0}^{\infty} d \chi I_n^e 
\crossprod{\vect{z_0}}{\nabla_\perp \Phi_n} \right)
\end{equation} }
%
\textcolor{red} { \begin{equation} 
\vect{E_\perp} = \frac{1}{\sqrt{\epsilon_0}} \left( 
\sum \limits_{m=-\infty}^{\infty} \int \limits_{0}^{\infty} 
d \nu V_m^h \crossprod{ \nabla_\perp \Psi_m }{ \vect{z_0} } +
\sum \limits_{n=-\infty}^{\infty} \int \limits_{0}^{\infty}
d \chi V_n^e \nabla_\perp \Phi_n \right)
\end{equation} }
%
\textcolor{red} { \begin{equation} 
H_z (r,t) = \frac{1}{\sqrt{\mu_0}} \sum \limits_{m=-\infty}^{\infty}
\int \limits_0^\infty \nu^2 d \nu h_m \Psi_m
\end{equation} }
%
\textcolor{red} { \begin{equation} 
E_z (r,t) = \frac{1}{\sqrt{\epsilon_0}} \sum \limits_{n=-\infty}^{\infty}
\int \limits_0^\infty \chi^2 d \chi e_n \Phi_n
\end{equation} }

%%%%%%%%%%%%%%%%%%%%%%%%%%%%%%%%%%%%%%%%%%%%%%%%%%%%%%%%%%%%%%%%%%%%%%%%%%%%%%
\section{Еволюційні рівняння}

\begin{equation}
\partial_z (\mu h_m) = \mu I_m^h + \frac{\sqrt[-2]{\mu_0}}{2 \pi}
\int_0^{2\pi} d \varphi \int_0^{\infty} \rho d \rho
\Psi_m^* (\nu) g;
\end{equation}
%
\begin{equation}
\partial_{ct} (\mu h_m) = - V_m^h - \frac{\sqrt{\epsilon_0}}{2 \pi}
\int_0^{2\pi} d \varphi \int_0^{\infty} \rho d \rho
\Psi_m^* (\nu) I_z;
\end{equation}
%
\begin{equation}
- \partial_{ct} (\epsilon V_m^h) - \partial_z I_m^h + \nu^2 h_m = 
\frac{\sqrt{\mu_0}}{2 \pi} \int_0^{2\pi} d \varphi 
\int_0^{\infty} \rho d \rho \crossprod{\vect{z_0}}{\vect{J_\perp}}
\nabla_\perp \Psi_m^* (\nu)
\end{equation}
%
\textcolor{lightgray} { \begin{equation*}
\partial_{ct} (\epsilon V_n^h) + \partial_z I_n^h = 
- \frac{\sqrt{\mu_0}}{2 \pi} \int_0^{2\pi} d \varphi 
\int_0^{\infty} \rho d \rho 
\dotprod {\vect{J_\perp}} {\nabla_\perp \Phi_n^* (\chi)}
\end{equation*} }
%
\begin{equation}
\partial_{ct} (\epsilon e_n) = - I_n^e - 
\frac{\sqrt{\mu_0}}{2 \pi} \int_0^{2\pi} d \varphi 
\int_0^{\infty} \rho d \rho \Phi_n^* (\chi) J_z
\end{equation}
%
\begin{equation}
\partial_{z} (\epsilon e_n) = \epsilon V_n^e + 
\frac{\sqrt[-2]{\epsilon_0}}{2 \pi} \int_0^{2\pi} d \varphi 
\int_0^{\infty} \rho d \rho \Phi_n^* (\chi) \varrho
\end{equation}
%
\begin{equation}
- \partial_{ct}(\mu I_n^e) - \partial_z V_n^e + \chi^2 e_n = 
\frac{\sqrt{\epsilon_0}}{2 \pi} \int_0^{2\pi} d \varphi 
\int_0^{\infty} \rho d \rho \crossprod{\vect{I_\perp}}{\vect{z_0}}
\nabla_\perp \Phi_n^* (\chi)
\end{equation}
%
\textcolor{lightgray} { \begin{equation*}
\partial_{ct}(\mu I_m^e) + \partial_z V_m^e = - 
\frac{\sqrt{\epsilon_0}}{2 \pi} \int_0^{2\pi} d \varphi 
\int_0^{\infty} \rho d \rho 
% \dotprod {\vect{I_\perp}} {\nabla_\perp \Phi_m^* (\nu)}
\dotprod {\crossprod {\vect{I_\perp}} {\vect{z_0}} } 
{ \nabla_\perp \Phi_m^* (\nu) }
\end{equation*} }


%%%%%%%%%%%%%%%%%%%%%%%%%%%%%%%%%%%%%%%%%%%%%%%%%%%%%%%%%%%%%%%%%%%%%%%%%%%%%%
% \section{Ітеративний підхід до врахування нелінійності}
