\chapter{Імпульсне поле випромінювача з круговою апертурою}
\label{ch:linear}

%%%%%%%%%%%%%%%%%%%%%%%%%%%%%%%%%%%%%%%%%%%%%%%%%%%%%%%%%%%%%%%%%%%%%%%%%%%%%%%
\section{Кругова апертура як модель антен імпульсного випромінювання}

На початку 60-х років інтерес до імпульсної радіофізики був збуджений
військовим застосуванням переваг надширокосмугових радарних та 
телекомунікаційних систем. Дослідження велись, як в Україні 
\cite{imp:Dumin1996} так і за кордоном \cite{imp:BaumIN0105}. Ці дослідження, сьогодні,
знайшли своє застосування у системах інтернету речей \cite{imp:Hartmann2015}, 
автомобільної індустрії \cite{imp:Yarovoy2017}, а також медицині 
\cite{imp:Cho2016}.

Широким класом технічних рішень для формування напрямленого випромінювання 
надширокосмугового електричного струму є антени імпульсного випромінювання.
Такі антени можна класифікувати за способом вирівнювання фронту хвиль:

\begin{enumerate}
	\item без вирівнювання сферичного фронту;
	\item лінзові сповільнювачі;
	\item Рефлекторні антени;
	\item Комбіновані архітектури \cite{imp:BaumSSN0379}.
\end{enumerate}

Живлення для таких антен зазвичай виконується ТЕМ рупором, що під'єднується 
до коаксіального кабелю через балун \cite{imp:BaumSSN0357}. Даний розділ 
присвячується дослідженню саме лінзових антен імпульсного випромінювання (LIRA). 

Антени типу LIRA мають численні переваги над рефлекторними. Перш за все, це
більш високий коефіцієнт підсилення антени \cite{imp:BaumUWBSP1}. По-друге,
лінзові антени не мають області тіні від опромінювача та краще узгоджуються
на практиці \cite{imp:BaumSSN0377}. Також експериментальне порівняння LIRA 
з рефлекторними антенами показує, що імпульсні характеристики перших мають 
меншу тривалість при тих самих електричного розмірах \cite{imp:BaumSSN0377}.
З недоліків варто відзначити важкість виготовлення лінз точної форми та 
вагу антени.

Розглянемо задачу збудження такої антени нестаціонарним імпульсним струмом
з деякою часовою залежністю $ f(t) $ з умовою існування першої та другої 
похідної для $ f(t) $. Ефективна тривалість перехідного процесу $ f(t) $ 
прийнято визначати за повною шириною на рівні половинної амплітуди 
(full width at half maximum або FWHM), що зручно на практиці. 
В роботі діапазон значень ефективної тривалості розглядається в межах 
від десятків пікосекунд до декількох наносекунд, що є найбільш цікавим 
діапазоном тривалостей для сучасної надширокосмугової електроніки.

Сферична хвиля проходить крізь систему діелектричних лінз, розташованих у 
розкриві, формуючи квазі-одномоментне збудження плаского фронту у розкриві. 
Формою розкриву, зазвичай, вибирають кругову апертуру. Таким чином, у першому 
наближенні, у розкриві формується рівномірно розподілений сторонній плаский 
електричний струм, напрямлений від одного плеча рупора до іншого. Вперше така
апроксимація була запропонована 1985 р. \cite{imp:Wu1985} та 
емпірично перевірена через декілька років \cite{imp:Wu1991}.

Серед лінзового класу антен, що формують розподіл стороннього струму у 
вигляді плаского диску варто відзначити антену Рис.~\ref{fig:lira_baum}, що 
спершу представив Ву \cite{imp:Wu1987}, а згодом і, незалежно, Карл Баум 
\cite{imp:BaumSSN0377}. Лінза антени виконана у формі витягнутого сфероїда, 
а розкрив ТЕМ-рупора цілком заповнено діелектриком. Повне заповнення 
розкриву рупора мінімізує відбиття та покращує стійкість антени до механічних 
пошкоджень \cite{imp:BaumSSN0377}. Скруглений рупор починається в одному 
фокусі еліпсоїда, а закінчується в другому, таким чином радіус розкриву є 
фокальним параметром еліпсоїда. В якості матеріалу для лінзи пропонується 
використовувати поліетилен високої густини (HDPE) з низькою діелектричною 
проникністю $\epsilon = 2.3 $.

\begin{figure}[htbp] \begin{center}
\includegraphics[scale=0.5]{Baum_LIRA}
\caption{Геометрія лінзевої антени Баума та Ву} \label{fig:lira_baum}
\end{center} \end{figure}

При використанні апроксимації плаского диску з електричним струмом для 
розв'язання задачі випромінювання антени зі сфероїдоподібною лінзою 
помічаємо, що уявний диск зі струмом і поверхня рівних фаз будуть розташовані 
поза межами розкриву рупора. Як покажемо далі, це не впливає на точність 
моделі, як в ближній, так і в дальній зоні, а відхилення тримаються в межах
систематичної похибки через внутрішній опір антени, який модель не враховує.

В 1991 р. Ву представив антену, для якої уявний диск зі струмом буде 
розташовуватись у розкриві рупора \cite{imp:Wu1991}. Гіперболічна лінза в 
цій антені забезпечує положення поверхні рівних фаз в самому розкриві 
рис. \ref{fig:lira_wu}. Одним з способів покращити цю антену є 
заміна діелектричного наповнення $ \epsilon_1 $ та лінзи $ \epsilon_2 $ на 
матеріал, діелектрична характеристика якого є функцією координат 
$ \epsilon(\rho, z) $. Таким чином, відбиття від внутрішньої поверхні лінзи 
знижується і характеристики антени покращуються. Важливим мінусом такої 
конструкції стає важкість виготовлення лінзи.

\begin{figure}[htbp] \begin{center}
\includegraphics[scale=0.5]{Wu_LIRA}
\caption{Геометрія лінзової антени Ву} \label{fig:lira_wu}
\end{center} \end{figure}

Призначенням LIRA є телекомунікація, радіолокація і лабораторні вимірювання. 
Проте, цікавість таких антен пояснюється ще і аномально повільним згасанням 
енергії імпульсного поля з відстанню, що було теоретично передбачено 
\cite{imp:Wu1987}. Цей ефект відомий у вітчизняній та закордонній літературі 
за назвою електромагнітний снаряд (electromagnetic missile).

Фізична модель плаского диску описує поле LIRA лише у першому наближенні.
Така модель не враховує вихровий магнітний сторонній струм, що існує на ряду 
з пласким електричним, а також не враховує струми, що течуть назад в 
генератор відбившись від краю рупора - поле такого струму залишає ``хвіст'' 
після основного імпульсу. Проте, емпіричні дослідження 
\cite{imp:BaumSSN0396,imp:BaumSSN0401} показують, що при належному 
узгодженні, паразитний вплив відбиття фактично відсутній, а перехідна 
функція отримана експериментальним шляхом і відповідає моделі плаского диску.

Для отримання розв'язку задачі плаского диску застосовували широкий спектр 
методів. Першими були отримані наближені розв'язки в частотній області 
\cite{imp:Wu1985,imp:Sodin1992-10}. Також, розв'язок для цієї задачі частково
знайдено у часовій області \cite{imp:Dumin1996}. Недоліком наявних розв'язків 
є те, що вони не надають часову залежність напруженості поля в довільних точках 
спостереження в явному виді, а отже не можуть використовуватись в широкому 
спектрі практичних задач, як, наприклад, врахування ефектів самодії у 
нелінійному середовищі. При самоїді поля крізь середовище, на значення 
напруженості поля в кожній точці спостереження та в будь-який момент часу 
впливають всі причинно пов'язані зі спостереженням події. Таким чином, 
наявність розв'язання в будь-якій точці спостереження - необхідна умова для 
врахування нелінійних ефектів методом теорії збурень.

\begin{figure}[htbp] \begin{center}
\includegraphics[scale=1.35]{lira_cst}
\caption{Модель антени в симуляторі CST Studio} \label{fig:lira_cst}
\end{center} \end{figure}

%%%%%%%%%%%%%%%%%%%%%%%%%%%%%%%%%%%%%%%%%%%%%%%%%%%%%%%%%%%%%%%%%%%%%%%%%%%%%%%
\section{Розв'язання методом еволюційних рівнянь} \label{sec:tranc_resp}

Розглянемо сторонній електричний нестаціонарний струм $ \vect{j_0} (r,t) $ 
в якості єдиного джерела електромагнітного поля. Нехай струм 
однонапрямлений, рівномірнорозподілений та має форму плаского диску 
нульової товщини. Для розв'язання прямої задачі електродинаміки для 
довільної часової залежності $ f(t) $ нестаціонарного струму 
$ \vect{j_0} (r,t) $ достатньо отримати розв'язок для  $ f(t) = H(t) $, 
де $ H(t) $ - функція Хевісайда, а далі, користуючись принципом суперпозиції
будувати розв'язок для довільної часової залежності $ f(t) $.
Тоді, джерело, що розглядається, математично можна описати в циліндричних 
координатах $ \rho, \varphi, z $, як

\begin{equation}
\vect{j_0} \left( r, t \right) = \vect{J} = \vect{x_0} A_0 H(t) \delta(z) 
\left(  H(\rho) - H(\rho - R) \right),
\end{equation}
%
де $ A_0 $ - максимальна амплітуда струму, що вимірюється в В/м, $ R $ - 
радіус диску, що вимірюється в метрах, $ \delta(z) $ - символ Кронекера, а 
$ \vect{x_0} = \vect{\rho_0} \cos \varphi - \vect{\varphi_0} \sin \varphi $ 
- декартовий орт OX.

\begin{figure}[htbp] \begin{center}
\includegraphics[scale=0.55]{PlaneDisk}
\caption{Геометрія випромінювача} \label{fig:pdisk}
\end{center} \end{figure}
%
%\textcolor{blue} { \begin{equation*} \begin{aligned}
%\begin{cases}
%\vect{\rho_0} = \vect{x_0} \cos \varphi + \vect{y_0} \sin \varphi \\
%\vect{\varphi_0} = - \vect{x_0} \sin \varphi + \vect{y_0} \cos \varphi
%\end{cases} \Rightarrow \mathbf{A} = \left( \begin{array}{cc}
%\cos \varphi & \sin \varphi \\
%- \sin \varphi & \cos \varphi
%\end{array} \right)
%\end{aligned} \end{equation*} }
%
%\textcolor{blue} { \begin{equation*} \begin{aligned}
%\vect{j_0} \left( \vect{\rho_0}, \vect{\varphi_0} \right) = 
%\mathbf{A} \vect{j_0} \left( \vect{x_0}, \vect{y_0} \right) = \\
%= H(t) \delta(z) (  H(\rho) - H(\rho - R) ) 
%( \vect{\rho_0} \cos \varphi - \vect{\varphi_0} \sin \varphi )
%\end{aligned} \end{equation*} }
%
Для застосування методу еволюційних рівнянь, спершу, знайдемо модовий 
розклад струму, застосувавши наступне перетворення

\begin{equation} \label{eq:jm_base}
j_m \left( r, t; \nu \right) = \frac{\sqrt{\mu_0}}{2\pi} 
\int \limits_{0}^{2\pi} d \varphi \int \limits_{0}^{\infty} \rho d \rho 
\vect{j_0} \crossprod{ \nabla_\perp \Psi_m^* }{ \vect{z_0} },
\end{equation}
%
де $ \Psi_m^* $ - комплексно спряжена базисна функція \cite{imp:Dumin2010}.
%
%\textcolor{blue} { \begin{equation*} \begin{aligned}
%\crossprod{ \nabla_\perp \Psi_m^* }{ \vect{z_0} } = 
%- \sqrt{\nu} e^{-im\varphi} \left( 
%\vect{\varphi_0} \frac{J_{m-1} (\nu \rho) - J_{m+1} (\nu \rho)}{2} + 
%\right. \\ + \left. i m \vect{\rho_0} \frac{J_m (\nu \rho)}
%{\rho \nu} \right) = - \sqrt{\nu} e^{-im\varphi} \left( 
%\vect{\varphi_0} \frac{J_{m-1} (\nu \rho) - J_{m+1} (\nu \rho)}{2} + 
%\right. \\ + \left. i \vect{\rho_0} \frac{J_{m-1} (\nu \rho) + 
%J_{m+1} (\nu \rho)}{2} \right)
%\end{aligned} \end{equation*} }
%
%\textcolor{blue} { \begin{equation*} \begin{aligned}
%\vect{j_0} \crossprod{ \nabla_\perp \Psi_m^* }{ \vect{z_0} } = 
%- \sqrt{\nu} ( \cos m \varphi - i \sin m \varphi ) 
%H(t) \delta(z) ( H(\rho) - H(\rho - R) ) \cdot \\ \cdot \left( 
%i \frac{J_{m-1} (\nu \rho) + J_{m+1} (\nu \rho)}{2} \cos \varphi
%- \frac{J_{m-1} (\nu \rho) - J_{m+1} (\nu \rho)}{2} \sin \varphi
%\right)
%\end{aligned} \end{equation*} }
%
%\textcolor{blue} { \begin{equation*} \begin{aligned}
%j_m = \frac{\sqrt{\mu_0}}{2\pi} \sqrt{\nu} \delta(z) H(t) \cdot \\
%\cdot \Big( \int \limits_{0}^{2\pi} d \varphi \sin \varphi 
%( \cos m \varphi - i \sin m \varphi) \int \limits_{0}^{R} 
%\frac{J_{m-1} (\nu \rho) - J_{m+1} (\nu \rho)}{2} \rho d \rho - \\
%- i \int \limits_{0}^{2\pi} d \varphi \cos \varphi 
%( \cos m \varphi - i \sin m \varphi) \int \limits_{0}^{R} 
%\frac{J_{m-1} (\nu \rho) + J_{m+1} (\nu \rho)}{2} \rho d \rho \Big)
%\end{aligned} \end{equation*} }
%
%\textcolor{blue} { \begin{equation*} \begin{aligned}
%j_m = \frac{\sqrt{\mu_0}}{2\pi} \sqrt{\nu} \delta(z) H(t) 
%i\pi ( \delta_{m,-1} - \delta_{m,1} ) \int \limits_{0}^{R} 
%\frac{J_{m-1} (\nu \rho) - J_{m+1} (\nu \rho)}{2} \rho d \rho - \\
%- \frac{\sqrt{\mu_0}}{2\pi} \sqrt{\nu} \delta(z) H(t) 
%i\pi ( \delta_{m,-1} + \delta_{m,1} ) \int \limits_{0}^{R} 
%\frac{J_{m-1} (\nu \rho) + J_{m+1} (\nu \rho)}{2} \rho d \rho =
%\end{aligned} \end{equation*} }
%
%\textcolor{blue} { \begin{equation*} \begin{aligned}
%= i \frac{\sqrt{\mu_0 \nu}}{4} \delta(z) H(t)
%\delta_{m,-1} \int \limits_{0}^{R} \left( J_{-2} (\nu \rho) - 
%J_0 (\nu \rho) \right) \rho d \rho - \\
%- i \frac{\sqrt{\mu_0 \nu}}{4} \delta(z) H(t)
%\delta_{m,1} \int \limits_{0}^{R} \left( J_{0} (\nu \rho) - 
%J_2 (\nu \rho) \right) \rho d \rho - \\
%- i \frac{\sqrt{\mu_0 \nu}}{4} \delta(z) H(t)
%\delta_{m,-1} \int \limits_{0}^{R} \left( J_{-2} (\nu \rho) +  
%J_0 (\nu \rho) \right) \rho d \rho - \\
%- i \frac{\sqrt{\mu_0 \nu}}{4} \delta(z) H(t)
%\delta_{m,1} \int \limits_{0}^{R} \left( J_{0} (\nu \rho) +
%J_2 (\nu \rho) \right) \rho d \rho =
%\end{aligned} \end{equation*} }
%
%\textcolor{blue} { \begin{equation*} \begin{aligned}
%= - i \frac{\sqrt{\mu_0 \nu}}{2} \delta(z) H(t) 
%(\delta_{m,1} + \delta_{m,-1}) 
%\int \limits_{0}^{R} \left( J_{0} (\nu \rho) + 
%J_2 (\nu \rho) \right) \rho d \rho - \\
%- i \frac{\sqrt{\mu_0 \nu}}{2} \delta(z) H(t) 
%(\delta_{m,1} + \delta_{m,-1}) 
%\int \limits_{0}^{R} \left( J_{0} (\nu \rho) -
%J_2 (\nu \rho) \right) \rho d \rho = \\
%= - i \frac{\sqrt{\mu_0 \nu}}{2} \delta(z) H(t) 
%(\delta_{m,1} + \delta_{m,-1}) 
%\int \limits_{0}^{R} J_{0} (\nu \rho) \rho d \rho
%\end{aligned} \end{equation*} }
%
%\textcolor{blue} { \begin{equation*} \begin{aligned}
%\int \limits_{0}^{R} J_{0} (\nu \rho) \rho d \rho = 
%\frac{1}{\nu^2} \int \limits_{0}^{R} J_{0} (\nu \rho) \nu \rho d \nu \rho =
%\left. \frac{\rho J_1 (\nu \rho) }{\nu} \right|_{0}^{R} = 
%\frac{R J_1 (\nu R)}{\nu}
%\end{aligned} \end{equation*} }
%
Після інтегрування $ \eqref{eq:jm_base} $ за кутом $ \varphi $ отримаємо 
тільки дві не нульові рівні між собою моди, які зручно записати одним 
виразом, використовуючи символи Кронекера $ \delta_{m,\pm1} $:

\begin{equation} 
j_m (z, t; \nu) = - i R A_0 \frac{\sqrt{\mu_0}}{2} \delta(z) H(t) 
\frac{\delta_{m,1} + \delta_{m,-1}}{\sqrt{\nu}} J_1 (\nu R).
\end{equation}

У методі еволюційних рівнянь електромагнітне поле є розкладом за 
деякими базисними функціями. Для ТЕ задач випромінювання, як ця, еволюційні 
рівняння значно спрощуються, а самі коефіцієнти стають пропорційними.
Фактично, пошук еволюційних коефіцієнтів зводиться до розв'язання одного
рівняння Клейна-Гордона відносно $ h_1 $ та $ h_{-1} $
%
%\textcolor{blue} { \begin{equation*} \begin{aligned}
%- \epsilon \partial_{ct} (V_m^h) - \partial_z I_m^h + \nu^2 h_m = 
%\frac{\sqrt{\mu_0}}{2 \pi} \int_0^{2\pi} d \varphi 
%\int_0^{\infty} \rho d \rho \crossprod{\vect{z_0}}{\vect{J_\perp}}
%\nabla_\perp \Psi_m^* (\nu) 
%\end{aligned} \end{equation*} }
%
%\textcolor{blue} { \begin{equation*} \begin{aligned}
%\crossprod{\vect{z_0}}{\vect{J_\perp}} \nabla_\perp \Psi_m^* (\nu) =
%\vect{J_\perp} \crossprod{\nabla_\perp \Psi_m^* (\nu)}{\vect{z_0}}
%\end{aligned} \end{equation*} }
%
%\textcolor{blue} { \begin{equation*} \begin{aligned}
%\epsilon \partial_{ct} \left( \mu \partial_{ct} h_m \right) -
%\mu^{-1} \partial_z \left( \mu  \partial_z h_m \right) + 
%\nu^2 h_m = j_m (z,t,\nu)
%\end{aligned} \end{equation*} }
%
\begin{equation} \begin{aligned} \label{eq:klein_gordon}
\frac{\epsilon \mu}{c^2} \frac{\partial^2 h_m}{\partial t^2} - 
\frac{\partial^2 h_m}{\partial z^2} + \nu^2 h_m = j_m (z,t,\nu).
\end{aligned} \end{equation}
%
Рівняння \eqref{eq:klein_gordon} було отримано з припущенням, що середовище 
в якому поширюється поле однорідне, стаціонарне та характеризується 
відносною діелектричною $ \epsilon $ та магнітною $ \mu $ проникненнями.
Буде зручно позначити швидкість світла в цьому середовищі за 
$ v = \frac{c}{\sqrt{\epsilon \mu}} $. Рівняння Клейна-Гордона
має відомий розв'язок через функцію Рімана:

\begin{equation} \label{eq:klein_gordon_sol}
h_m (z, t; \nu) = \iint_S j_m (t',z') G(t,t',z,z') dt' dz',
\end{equation}
%
де $ G(t,t',z,z') $ функція Рімана 
%
%\begin{equation*}
%G = \frac{\mathit{v}}{2} H \left( \mathit{v} (t-t') - (z-z') \right)
%J_0 \left( \nu \sqrt{\mathit{v}^2 (t-t')^2 - (z-z')^2} \right).
%\end{equation*}
%
З вигляду розв'язку \eqref{eq:klein_gordon_sol} можна зробити висновок, що
функція Рімана $ G(t,t',z,z') $ - це аналог функції Гріна в часовому просторі,
а розв'язок рівняння Клейна-Гордона є еквівалентом принципу суперпозиції
для сферичних нестаціонарних хвиль, що випромінюються кожною з точок джерела
(випромінювачами Гюгенца) у деякій точці спостереження у визначений час.
%
%\textcolor{blue} { \begin{equation*} \begin{aligned}
%h_m (z, t; \nu) = - i \mathit{V} R \frac{\sqrt{\mu_0}}{4} 
%\frac{\delta_{m,1} + \delta_{m,-1}}{\sqrt{\nu}} J_1 (\nu R)
%\int \limits_{0}^{\infty} \delta(z) \cdot \\ \cdot
%\int \limits_{t - \frac{z}{\mathit{V}}}^{0} 
%J_0 \left( \nu \sqrt{\mathit{V}^2 (t-t')^2 - (z-z')^2} \right) dt' dz' = 
%i \mathit{V} R \frac{\sqrt{\mu_0}}{4} 
%\frac{\delta_{m,1} + \delta_{m,-1}}{\sqrt{\nu}} J_1 (\nu R)
%\cdot \\ \cdot \int \limits_{0}^{\infty} \delta(z)
%\int \limits_{0}^{t - \frac{z}{\mathit{V}}} 
%J_0 \left( \nu \sqrt{\mathit{V}^2 (t-t')^2 - (z-z')^2} \right) dt' dz
%\end{aligned} \end{equation*} }
%
%\textcolor{blue} { \begin{equation*} \begin{aligned}
%h_m (z, t; \nu) = i \mathit{V} R \frac{\sqrt{\mu_0}}{4} 
%\frac{\delta_{m,1} + \delta_{m,-1}}{\sqrt{\nu}} J_1 (\nu R)
%\int \limits_{0}^{t - \frac{z}{\mathit{V}}} 
%J_0 \left( \nu \sqrt{\mathit{V}^2 (t-t')^2 - z^2} \right) dt'
%\end{aligned} \end{equation*} }
%
Користуючись властивостями дельта-функції Дірака та функції Хевісайда 
запишемо поздовжні модові коефіцієнти $ h_1 $ та $ h_{-1} $ в наступному 
виді:

\begin{equation} \label{eq:hm_int}
h_m = \frac{i R A_0}{4} \frac{\delta_{m,1} + \delta_{m,-1}}
{\sqrt{\nu} \sqrt{\epsilon_0 \epsilon \mu}} J_1 (\nu R) 
\int \limits_{0}^{t - \frac{z}{v}} 
J_0 \left( \nu \sqrt{v^2 (t-t')^2 - z^2} \right) dt'.
\end{equation}

Знайшовши поздовжній магнітний модовий коефіцієнт, просто поперечні морові 
коефіцієнти визначити через нього. Для отримання виразу для 
$ V_m^h = - \frac{\mu}{c} \partder{h_m}{t} $ необов'язково брати інтеграл в 
$ h_m $. Спробуємо спростити вираз, скориставшись залежністю через похідну по 
часу, тобто застосуємо правило інтегрування Лейбніца \cite{imp:Flanders1973}, 
помітивши, що
%
%\begin{equation*} \begin{aligned}
%\partder{}{t'} J_0 \left( \nu \sqrt{v^2 (t-t')^2 - z} \right) =
%- \partder{}{t} J_0 \left( \nu \sqrt{v^2 (t-t')^2 - z} \right);
%\end{aligned} \end{equation*}
%
тоді отримаємо
%
%\textcolor{blue} { \begin{equation*} \begin{aligned}
%\partder{}{\theta} \int_{a(\theta)}^{b(\theta)} f(x,\theta) dx = 
%\int_{a(\theta)}^{b(\theta)} \partder{f}{\theta} dx + 
%f\big( b(\theta), \theta \big) \partder{b}{\theta} -
%f\big( a(\theta), \theta \big) \partder{a}{\theta}
%\end{aligned} \end{equation*} }
%
%\textcolor{blue} { \begin{equation*} \begin{aligned}
%\partder{}{t} J_0 \left( \nu \sqrt{v^2 (t-t')^2 - z} \right) = 
%- \nu J_1 \left( \nu \sqrt{v^2 (t-t')^2 - z} \right) 
%\partder{}{t} \sqrt{v^2 (t-t')^2 - z} = \\
%-  J_1 \left( \nu \sqrt{v^2 (t-t')^2 - z} \right)
%\frac{2 \nu v^2 (t-t')}{2 \sqrt{v^2 (t-t')^2 - z}} = - \nu v^2 (t-t') 
%\frac{J_1 \left( \nu \sqrt{v^2 (t-t')^2 - z} \right)}
%     {\sqrt{v^2 (t-t')^2 - z}}
%\end{aligned} \end{equation*} }
%
%\textcolor{blue} { \begin{equation*} \begin{aligned}
%\partder{}{t} \int \limits_{0}^{t - \frac{z}{v}} 
%J_0 \left( \nu \sqrt{v^2 (t-t')^2 - z^2} \right) dt' = \\
%= \int \limits_{0}^{t - \frac{z}{v}} 
%\partder{}{t} J_0 \left( \nu \sqrt{v^2 (t-t')^2 - z^2} \right) dt' +
%J_0 (0) - 0 \cdot \left( \nu \sqrt{v^2 (t-t')^2 - z^2} \right) = \\
%= - \int \limits_{0}^{t - \frac{z}{v}} 
%\partder{}{t'} J_0 \left( \nu \sqrt{v^2 (t-t')^2 - z^2} \right) dt' + 1 =
%- \Big. J_0 \left( \nu \sqrt{v^2 (t-t')^2 - z^2} \right) \Big|_{0}^{t - \frac{z}{v}} + 1 = \\
%- J_0 \left( \nu \sqrt{z^2 - z^2} \right) + J_0 \left( \nu \sqrt{v^2 t^2 - z^2} \right) + 1 = 
%J_0 \left( \nu \sqrt{v^2 t^2 - z^2} \right)
%\end{aligned} \end{equation*} }
%
%\begin{equation*} \begin{aligned}
%\partder{}{t} \int \limits_{0}^{t - \frac{z}{v}} 
%J_0 \left( \nu \sqrt{v^2 (t-t')^2 - z^2} \right) dt' =
%J_0 \left( \nu \sqrt{v^2 t^2 - z^2} \right),
%\end{aligned} \end{equation*}
%
%\textcolor{blue} { \begin{equation*} \begin{aligned}
%V_m^h = - \frac{\mu}{c} \partder{h_m}{t} = 
%\sqrt{\mu_0} \sqrt{\frac{\mu}{\epsilon}} \frac{iR A_0}{4} 
%\frac{\delta_{m,1} + \delta_{m,-1}}{\sqrt{\nu}} J_1 (\nu R)
%J_0 \left( \nu \sqrt{\mathit{v}^2 t^2 - z^2} \right).
%\end{aligned} \end{equation*} }

Трохи спростивши вираз, можемо записати формулу для коефіцієнтів $ V_m^h $
у наступному вигляді:
%
\begin{equation} \label{eq:vmh}
V_m^h (z, t; \nu) = - \frac{iR A_0}{4} \sqrt{\frac{\mu_0 \mu}{\epsilon}} 
\frac{\delta_{m,1} + \delta_{m,-1}}{\sqrt{\nu}} J_1 (\nu R)
J_0 \left( \nu \sqrt{\mathit{v}^2 t^2 - z^2} \right).
\end{equation}
%
Далі отримаємо модовий коефіцієнт $ I_m^h $, що знадобиться для визначення
магнітних компонентів поля. Для цього запишемо поздовжній магнітний модовий
коефіцієнт \eqref{eq:hm_int} через спеціальну функцію Ломмеля для двох 
змінних (дійсної та уявної) \cite{imp:Boersma1961}:
%
%\textcolor{blue} { \begin{equation*} \begin{aligned}
%\int \limits_{0}^{t - \frac{z}{\mathit{v}}} 
%J_0 \left( \nu \sqrt{\mathit{v}^2 (t-t')^2 - z^2} 
%\right) dt' = \left[ \begin{array}{cc} 
%\nu \mathit{v} (t-t') = s & t' = t - \frac{ds}{\nu \mathit{v}} \\
%dt' = -\frac{ds}{\nu \mathit{v}} & \\
%s(0) = \nu \mathit{v} t & s \left( t - \frac{z}{\mathit{v}} \right) = \nu z
%\end{array} \right] = \\ = - \frac{1}{\nu \mathit{v}} 
%\int_{\nu \mathit{v} t}^{\nu z} ds 
%J_0 (\sqrt{s^2 - \nu^2 z^2}) = \frac{1}{\nu \mathit{v}} 
%\int_{\nu z}^{\nu \mathit{v} t} ds
%J_0 (\sqrt{s^2 - \nu^2 z^2})
%\end{aligned} \end{equation*} }
%
%\textcolor{blue} { \begin{equation*} \begin{aligned}
%\int_{\nu z}^{\nu \mathit{v} t} ds e^{-i0s} J_0 (\sqrt{s^2 - \nu^2 z^2}) = \\ 
%= \frac{1}{i} (U_1[W_+,Z] + i U_2[W_+,Z] - U_1[W_-,Z] - i U_2[W_-,Z]) = \\
%= \frac{1}{i} (-U_1[W_-,Z] + i U_2[W_+,Z] - U_1[W_-,Z] - i U_2[W_+,Z]) = \\
%= \left[ \begin{array}{c} W_\pm = \pm i (\nu \mathit{v} t - \nu z) \\
%Z = \sqrt{\nu^2 \mathit{v}^2 t^2 - \nu^2 z^2} \end{array} \right] = 
%2i U_1 \left[ -i \nu (\mathit{v}t-z), \nu \sqrt{\mathit{v}^2 t^2-z^2} \right]
%\end{aligned} \end{equation*} }
%
%\textcolor{blue} { \begin{equation*} \begin{aligned}
%\int \limits_{0}^{t - \frac{z}{\mathit{v}}} 
%J_0 \left( \nu \sqrt{\mathit{v}^2 (t-t')^2 - z^2} 
%\right) dt' = \frac{2i}{\nu \mathit{v}} U_1 
%\left[ -i \nu (\mathit{v}t-z), \nu \sqrt{\mathit{v}^2t^2-z^2} \right]
%\end{aligned} \end{equation*} }
%
%\textcolor{blue} { \begin{equation*} \begin{aligned}
%h_m (z, t; \nu) = \mathit{v} \sqrt{\mu_0} \frac{iR A_0}{4} 
%\frac{\delta_{m,1} + \delta_{m,-1}} {\sqrt{\nu}} J_1 (\nu R) 
%\frac{2i}{\nu \mathit{v}} U_1 \left[ W_-, Z \right]
%\end{aligned} \end{equation*} }
%
\begin{equation} \label{eq:hm_lommel}
h_m (z, t; \nu) = - \sqrt{\mu_0} \frac{R A_0}{2} 
\frac{\delta_{m,1} + \delta_{m,-1}}
{\nu^{3/2}} J_1 (\nu R) U_1 \left[ W_-, Z \right].
\end{equation}
%
Тепер підставивши \eqref{eq:hm_lommel} в вираз для коефіцієнту
$ I_{m}^{h} = \partder{h_m}{z} $ отримаємо:
%
%\textcolor{blue} { \begin{equation*} \begin{aligned}
%I_{m}^{h} = \partder{h_m}{z} = 
%- \sqrt{\mu_0} \frac{R A_0}{2} 
%\frac{\delta_{m,1} + \delta_{m,-1}}
%{\nu^{3/2}} J_1 (\nu R) \partder{}{z} U_1 [ W_-, Z ]
%\end{aligned} \end{equation*} }
%
%\textcolor{blue} { \begin{equation*} \begin{aligned}
%\begin{array}{lcr}
%\derivat{W_-}{z} = i \nu & &
%\derivat{Z}{z} = \frac{\nu}{2 \sqrt{\mathit{V}^2 t^2 - z^2}} (-2z) = 
%- \frac{\nu z}{\sqrt{\mathit{V}^2 t^2 - z^2}} \\
%\end{array}
%\end{aligned} \end{equation*} }
%
%\textcolor{blue} { \begin{equation*} \begin{aligned}
%\left( \frac{Z}{W} \right)^2 = 
%\left( - \frac{ \sqrt{\mathit{V}^2 t^2-z^2}}{i(\mathit{V} t-z)} \right)^2 =
%\left( \frac{ i \sqrt{\mathit{V}^2 t^2-z^2}}{\mathit{V}t-z} \right)^2 =
%- \frac{\mathit{V}^2 t^2-z^2}{(\mathit{V} t-z)^2} = 
%- \frac{\mathit{V}t+z}{\mathit{V}t-z}
%\end{aligned} \end{equation*} }
%
%\textcolor{blue} { \begin{equation*} 
%\partder{}{Z} U_n (W,Z) = - \frac{Z}{W} U_{n+1} (W,Z)
%\end{equation*} }
%
%\textcolor{blue} { \begin{equation*}
%2 \partder{}{W} U_n (W,Z) = U_{n-1} (W,Z) + 
%\left( \frac{Z}{W} \right)^2 U_{n+1} (W,Z)
%\end{equation*} }
%
%\textcolor{blue} { \begin{equation*} \begin{aligned}
%\partder{}{z} U_1 \left[ -i \nu (ct-z), \nu \sqrt{c^2t^2-z^2} \right] =
%\partder{}{z} U_1[W,Z] = \partder{U_1}{W} \derivat{W}{z} + 
%\partder{U_1}{Z} \derivat{Z}{z} = \\
%= \frac{i \nu}{2} \left( U_0 - \frac{ct+z}{ct-z} U_2 \right) -
%\frac{\nu z}{\sqrt{c^2t^2 - z^2}} 
%\left( - \frac{i \sqrt{c^2t^2-z^2}}{ct-z} \right) U_2 = \\
%= \frac{i \nu}{2} U_0 - \frac{i \nu}{2} \frac{ct+z}{ct-z} U_2 +
%\frac{i \nu z}{ct-z} U_2 = \\ = \frac{i \nu}{2} U_0 - \frac{i \nu}{2} U_2
%\left( \frac{ct}{ct-z} + \frac{z}{ct-z} - \frac{2z}{ct-z} \right) = 
%\frac{i \nu}{2} (U_0[W_-,Z] - U_2[W_-,Z])
%\end{aligned} \end{equation*} }
%
\begin{equation} \label{eq:imh}
I_{m}^{h} = - \sqrt{\mu_0} \frac{iR A_0}{4} 
\frac{\delta_{m,1} + \delta_{m,-1}}{\sqrt{\nu}} 
J_1 (\nu R) \left( U_0 [ W_-, Z ] - U_2 [ W_-, Z ] \right)
\end{equation}
%
Електричні модові коефіцієнти $ e_n $, $ I_n^e $, $ V_n^e $ для всіх $ n $
рівні нулю. Математично, це є наслідком того, що розв'язок однорідного рівняння 
Клейна-Гордона відносно $ e_n $ має тільки тривіальний розв'язок. Такі модові
розклади характерні саме для TE хвиль.
%
Таким чином отримано аналітично всі еволюційні коефіцієнти. Підставимо
\eqref{eq:vmh} в розклад вектору напруженості електричного поля по
базисним функціям \cite{imp:Dumin2010}. Таким чином отримаємо електричне 
поле плаского диску в циліндричних компонентах 
$ \vect{\rho_0}, \vect{\varphi_0}, \vect{z_0} $, як функцію циліндричних 
координат $ \rho, \varphi, z $ та часу $ t $.
%
%\textcolor{blue} { \begin{equation*} \begin{aligned}
%\vect{E_\perp} = \frac{1}{\sqrt{\epsilon_0}} \left( 
%\sum \limits_{m=-\infty}^{\infty} \int \limits_{0}^{\infty} 
%d \nu V_m^h \crossprod{ \nabla_\perp \Psi_m }{ \vect{z_0} } +
%\sum \limits_{n=-\infty}^{\infty} \int \limits_{0}^{\infty}
%d \chi V_n^e \nabla_\perp \Phi_n \right)
%\end{aligned} \end{equation*} }
%
%\textcolor{blue} { \begin{equation*} \begin{aligned}
%\crossprod{ \nabla_\perp \Psi_m }{ \vect{z_0} } = 
%- e^{im\varphi} \left( \vect{\varphi_0} \sqrt{\nu} 
%\frac{J_{m-1} (\nu \rho) - J_{m+1} (\nu \rho)}{2} - 
%i m \vect{\rho_0} \frac{J_m (\nu \rho)}{ \rho \sqrt{\nu}} \right)
%\end{aligned} \end{equation*} }
%
%\textcolor{blue} { \begin{equation*} \begin{aligned}
%\vect{E_\perp} = \frac{1}{\sqrt{\epsilon_0}} \int_{0}^{\infty} 
%V_{-1}^h \crossprod{ \nabla_\perp \Psi_{-1}  }{ \vect{z_0} } +
%\frac{1}{\sqrt{\epsilon_0}} \int \limits_{0}^{\infty} 
%V_{1}^h \crossprod{ \nabla_\perp \Psi_{1} }{ \vect{z_0} } = \\
%= \frac{i R A_0}{4} \sqrt{\frac{\mu_0 \mu}{\epsilon_0 \epsilon}} 
%e^{- i \varphi} \int_{0}^{\infty} \frac{J_1 (\nu R)}{\sqrt{\nu}} 
%J_0 \left( \nu \sqrt{c^2 t^2 - z^2} \right) \cdot \\
%\cdot \left( \vect{\varphi_0} \sqrt{\nu} 
%\frac{J_2 (\nu \rho) - J_0 (\nu \rho)}{2} +
%i \vect{\rho_0} \frac{J_1 (\nu \rho)}{ \rho \sqrt{\nu}} \right) - \\
%+ \frac{i R A_0}{4} \sqrt{\frac{\mu_0 \mu}{\epsilon_0 \epsilon}}
%e^{i \varphi} \int \limits_{0}^{\infty} \frac{J_1 (\nu R)}{ \sqrt{\nu}}
%J_0 \left( \nu \sqrt{c^2 t^2 - z^2} \right) \cdot \\
%\cdot \left( \vect{\varphi_0} \sqrt{\nu}
%\frac{J_0 (\nu \rho) - J_2 (\nu \rho)}{2} - 
%i \vect{\rho_0} \frac{J_1 (\nu \rho)}{ \rho \sqrt{\nu}} \right)
%\end{aligned} \end{equation*} }
%
%\textcolor{blue} { \begin{equation*} \begin{aligned}
%E_\varphi = \frac{i R A_0}{8} \sqrt{\frac{\mu_0 \mu}{\epsilon_0 \epsilon}} 
%e^{-i \varphi} \int \limits_{0}^{\infty} J_1 (\nu R)
%J_0 \left( \nu \sqrt{c^2 t^2 - z^2} \right)
%\left( J_2 (\nu \rho) - J_0 (\nu \rho) \right) + \\
%+ \frac{i R A_0}{8} \sqrt{\frac{\mu_0 \mu}{\epsilon_0 \epsilon}} 
%e^{i \varphi} \int \limits_{0}^{\infty} J_1 (\nu R)
%J_0 \left( \nu \sqrt{c^2 t^2 - z^2} \right)
%\left( J_0 (\nu \rho) - J_2 (\nu \rho) \right) = \\
%= \frac{i R A_0}{4} \sqrt{\frac{\mu_0 \mu}{\epsilon_0 \epsilon}}
%\frac{e^{i \varphi} - e^{-i \varphi} }{2} \int \limits_{0}^{\infty} 
%J_1 (\nu R) J_0 \left( \nu \sqrt{c^2 t^2 - z^2} \right) 
%\left( J_0 (\nu \rho) - J_2 (\nu \rho) \right) =
%\end{aligned} \end{equation*} }
%
%\textcolor{blue} { \begin{equation*} \begin{aligned}
%= \frac{R A_0}{4} \sqrt{\frac{\mu_0 \mu}{\epsilon_0 \epsilon}} 
%\frac{e^{i \varphi} - e^{-i \varphi} }{2i} \int \limits_{0}^{\infty} 
%J_1 (\nu R) J_0 \left( \nu \sqrt{c^2 t^2 - z^2} \right) 
%\left( J_2 (\nu \rho) - J_0 (\nu \rho) \right) = \\
%= \frac{R A_0}{4} \sqrt{\frac{\mu_0 \mu}{\epsilon_0 \epsilon}} \sin \varphi 
%\int \limits_{0}^{\infty} J_1 (\nu R) 
%J_0 \left( \nu \sqrt{c^2 t^2 - z^2} \right) 
%\left( J_2 (\nu \rho) - J_0 (\nu \rho) \right)
%\end{aligned} \end{equation*} }
%
%\textcolor{blue} { \begin{equation*} \begin{aligned}
%J_2 (\nu \rho) - J_0 (\nu \rho) = \frac{2}{\nu \rho} J_1 (\nu \rho) - 
%2 J_0 (\nu \rho)
%\end{aligned} \end{equation*} }
%
%\textcolor{blue} { \begin{equation*} \begin{aligned}
%E_\varphi = \frac{R A_0}{2} \sqrt{\frac{\mu_0 \mu}{\epsilon_0 \epsilon}}
%\sin \varphi \int \limits_{0}^{\infty} J_1 (\nu R) 
%J_0 \left( \nu \sqrt{c^2 t^2 - z^2} \right) 
%\left( \frac{J_1 (\nu \rho)}{\nu \rho} - J_0 (\nu \rho) \right)
%\end{aligned} \end{equation*} }
%
%\textcolor{blue} { \begin{equation*} \begin{aligned}
%E_\rho = \frac{i R A_0}{4} \sqrt{\frac{\mu_0 \mu}{\epsilon_0 \epsilon}}  
%e^{- i \varphi} \int \limits_{0}^{\infty} \frac{J_1 (\nu R)}{\sqrt{\nu}} 
%J_0 \left( \nu \sqrt{c^2 t^2 - z^2} \right) 
%\left( - i \frac{J_1 (\nu \rho)}{\rho \sqrt{\nu}} \right) + \\
%+ \mu \frac{i R A_0}{4} \sqrt{\frac{\mu_0}{\epsilon_0}}  e^{i \varphi}
%\int \limits_{0}^{\infty} \frac{J_1 (\nu R)}{\sqrt{\nu}}
%J_0 \left( \nu \sqrt{c^2 t^2 - z^2} \right) 
%\left( - i \frac{J_1 (\nu \rho)}{ \rho \sqrt{\nu}} \right) = \\
%= \mu \frac{R A_0}{2} \sqrt{\frac{\mu_0 \mu}{\epsilon_0 \epsilon}} 
%\frac{e^{i \varphi} + e^{-i \varphi}}{2}
%\int \limits_{0}^{\infty} \frac{J_1 (\nu R)}{\sqrt{\nu}}
%J_0 \left( \nu \sqrt{c^2 t^2 - z^2} \right) 
%\frac{J_1 (\nu \rho)}{ \rho \sqrt{\nu}} = \\
%= \mu \frac{R A_0}{2} \sqrt{\frac{\mu_0 \mu}{\epsilon_0 \epsilon}} 
%\cos \varphi \int \limits_{0}^{\infty} \frac{d \rho}{\nu \rho} 
%J_1 (\nu \rho) J_1 (\nu R) J_0 \left( \nu \sqrt{c^2 t^2 - z^2} \right)
%\end{aligned} \end{equation*} }
%
\begin{equation} \label{eq:linear_e_cyl}
\vect{E} \left( r, t \right) = \frac{A_0}{2} 
\sqrt{\frac{\mu_0 \mu}{\epsilon_0 \epsilon}}
\Big( \vect{\rho_0} I_1 \cos \varphi - 
\vect{ \varphi_0 } \left( I_2 - I_1 \right) \sin \varphi \Big),
\end{equation}
%
де
%
%\begin{equation*}
%I_1 = R \int \limits_{0}^{\infty} \frac{d \nu}{\nu \rho} J_1 (\nu \rho) 
%J_1 (\nu R) J_0 \left( \nu \sqrt{\frac{c^2 t^2}{\epsilon \mu} - z^2} \right)
%\end{equation*}
%
%\begin{equation*}
%I_2 = R \int_{0}^{\infty} d \nu J_1 (\nu R) J_0 (\nu \rho) 
%J_0 \left( \nu \sqrt{\frac{c^2 t^2}{\epsilon \mu} - z^2} \right),
%\end{equation*}
%
а їх аналітичні розв'язки, що представлено в додатку \ref{sec:i1anal} і 
\ref{sec:i2anal}.

Розглянемо вектор напруженості електричного поля в базису Декартової 
системи координат, тоді: 
%
%\textcolor{blue} { \begin{equation*} \begin{aligned}
%\mathbf{A} = \left( \begin{array}{cc}
%\cos \varphi & \sin \varphi \\
%- \sin \varphi & \cos \varphi
%\end{array} \right) \begin{array}{ccc}
%	& \det A = 1 		&	\\
%	& A^{-1} = A^{T}	&
%\end{array} 
%\mathbf{A^{-1}} = \left( \begin{array}{cc}
%\cos \varphi & - \sin \varphi \\
%\sin \varphi & \cos \varphi
%\end{array} \right) 
%\end{aligned} \end{equation*} }
%
%\textcolor{blue} { \begin{equation*} \begin{aligned}
%\vect{E} = 
%\mathbf{A^{-1}} \vect{E} \left( \vect{\rho_0}, \vect{\varphi_0} \right) = 
%\frac{A_0}{2} \sqrt{\frac{\mu_0 \mu}{\epsilon_0 \epsilon}}
%\left( \begin{array}{cc} \cos \varphi & - \sin \varphi \\
%\sin \varphi & \cos \varphi \end{array} \right)
%\left( \begin{array}{c} I_1 \cos \varphi \\
%- (I_2 - I_1) \sin \varphi \end{array} \right) = \\
%= \frac{A_0}{2} \sqrt{\frac{\mu_0 \mu}{\epsilon_0 \epsilon}}
%\left( \begin{array}{c} I_1 \cos^2 \varphi + (I_2 - I_1) \sin^2 \varphi \\
%I_1 \sin \varphi \cos \varphi - (I_2 - I_1) 
%\sin \varphi \cos \varphi \end{array} \right)
%\end{aligned} \end{equation*} }
%
\begin{equation} \begin{aligned} \label{eq:Exyz}
\left( \begin{array}{c} E_x \\ E_y \\ E_z \end{array} \right) = 
\frac{A_0}{2}  \sqrt{\frac{\mu_0 \mu}{\epsilon_0 \epsilon}} 
\left( \begin{array}{c} 
I_1 \cos^2 \varphi + (I_2 - I_1) \sin^2 \varphi \\
- I_2 \sin \varphi \cos \varphi \\
0
\end{array} \right)
\end{aligned} \end{equation}

З аналітичних розв'язків для інтегралів $ I_1 $ та $ I_2 $ бачимо, що 
компоненти поля - шматочно-визначені функції з областю визначення 
$ S = S_1 \cup S_2 \cup S_3 $, де
%
\begin{equation} \begin{aligned} \label{eq:s1zone}
S_1 \subset 0 \leq \frac{c^2t^2}{\epsilon \mu} - z^2 < (\rho - R)^2 
\cup \rho < R
\end{aligned} \end{equation}
%
\begin{equation} \begin{aligned} \label{eq:s2zone}
S_2 \subset (\rho - R)^2 < \frac{c^2t^2}{\epsilon \mu} - z^2 < (\rho + R)^2,
\end{aligned} \end{equation}
%
\begin{equation} \begin{aligned} \label{eq:s3zone}
S_3 \subset (\rho + R)^2 \leq \frac{c^2t^2}{\epsilon \mu} - z^2.
\end{aligned} \end{equation}

Звертаючись до схематичного зображення причинного зв'язку спостерігача та 
джерела (Рис.~\ref{fig:part_rad}), помічаємо, що область $ S_1 $ об'єднує 
просторово-часові події, які причинно не пов'язані з жодним із крайніх точок 
джерела. Саме тут спостерігається ефект електромагнітного снаряду: 
спостерігач в цій області простору-часу завжди причинно пов'язаний з 
частиною джерела, яка має круглу форму.

\begin{figure}[h] \begin{center}
\includegraphics[scale=0.45]{PartialRadiation}
\caption{Фізичний зміст областей випромінювання} \label{fig:part_rad}
\end{center} \end{figure}

Область $ S_2 $ відповідає за події, коли частина крайніх точок джерела вже 
причинно пов'язана зі спостерігачем. Тобто, частина джерела про яку вже відомо
спостерігачу (часоподібна), вже не має круглої форми та ще не охоплює 
всього джерела. В цій області спостерігається деякий перехідний процес, в
якому значення перехідної функції поступово згасає на нуль.

Спостерігачі в області $ S_3 $ вже отримали всю інформацію про форму джерела.
Для них перехідний процес скінчено і зміни напруженості поля не 
спостерігається, а отже, напруженість електричного поля тут відсутня.

Також, аналізуючи геометрію процесу випромінювання на 
Рис.~\ref{fig:part_rad}, бачимо, що сигнал спостерігається з моменту 

\begin{equation} \begin{aligned} \label{eq:time1}
vt_1 = \begin{cases}
z, \rho < R \\
\sqrt{(\rho-R)^2+z^2}, \rho > R
\end{cases}.
\end{aligned} \end{equation}

Далі наступає момент початку області $ S_2 $, коли поле від найближчого 
краю диска досягає спостерігача:

\begin{equation} \begin{aligned} \label{eq:time2}
vt_2 = \sqrt{(\rho-R)^2+z^2} + \rho - \rho / R,
\end{aligned} \end{equation}
%
а закінчується область $ S_2 $ моментом, коли поле від найвіддаленішого 
краю диску досягає спостерігача:

\begin{equation} \begin{aligned} \label{eq:time3}
vt_3 = \sqrt{(\rho+R)^2+z^2}.
\end{aligned} \end{equation}

Перейдемо до магнітних складових перехідної функції плаского диску. Для 
отримання магнітних компонент поля скористаємось еволюційним коефіцієнтом 
\eqref{eq:imh}.

%\textcolor{blue} { \begin{equation*} \begin{aligned}
%\vect{H_\perp} = \frac{1}{\sqrt{\mu_0}} \left( 
%\sum \limits_{m=-\infty}^{\infty} \int \limits_{0}^{\infty} d \nu
%I_m^h \nabla_\perp \Psi_m + \sum \limits_{n=-\infty}^{\infty}
%\int \limits_{0}^{\infty} d \chi I_n^e 
%\crossprod{\vect{z_0}}{\nabla_\perp \Phi_n} \right)
%\end{aligned} \end{equation*} }
%
%\textcolor{blue} { \begin{equation*} \begin{aligned}
%\nabla_\perp \Psi_m = e^{i m \varphi} \left( \vect{\rho_0} 
%\sqrt{\nu} \frac{ J_{m-1}(\nu \rho) - J_{m+1}(\nu \rho) }{2} +
%i m \vect{\varphi_0} \frac{J_m(\nu \rho)}{\sqrt{\nu} \rho} \right)
%\end{aligned} \end{equation*} }
%
%\textcolor{blue} { \begin{equation*} \begin{aligned}
%\vect{H_\perp} = \frac{1}{\sqrt{\mu_0}} \left( 
%\int \limits_{0}^{\infty} d \nu I_{-1}^h \nabla_\perp \Psi_{-1} +
%\int \limits_{0}^{\infty} d \nu I_1^h \nabla_\perp \Psi_1 \right) = \\
%= - \frac{A_0}{\sqrt{\mu_0}} \int \limits_{0}^{\infty} d \nu
%\sqrt{\mu_0} \frac{iR}{4} J_1 (\nu R)
%\frac{ U_0 [ W_-, Z ] - U_2 [ W_-, Z ] }{\sqrt{\nu}}  
%e^{- i \varphi} \cdot \\ \cdot \left( \vect{\rho_0} 
%\sqrt{\nu} \frac{ J_{2}(\nu \rho) - J_{0}(\nu \rho) }{2} +
%i \vect{\varphi_0} \frac{J_1(\nu \rho)}{\sqrt{\nu} \rho} \right) -
%\frac{A_0}{\sqrt{\mu_0}} \int \limits_{0}^{\infty} d \nu 
%\sqrt{\mu_0} \frac{iR}{4} J_1 (\nu R) \cdot \\
%\cdot \frac{ U_0 [ W_-, Z ] - U_2 [ W_-, Z ] }{\sqrt{\nu}} 
%e^{i \varphi} \left( \vect{\rho_0} 
%\sqrt{\nu} \frac{ J_{0}(\nu \rho) - J_{2}(\nu \rho) }{2} +
%i \vect{\varphi_0} \frac{J_1(\nu \rho)}{\sqrt{\nu} \rho} \right)
%\end{aligned} \end{equation*} }
%
%\textcolor{blue} { \begin{equation*} \begin{aligned}
%H_\varphi = \frac{R A_0}{4} 
%\frac{e^{i \varphi} + e^{- i \varphi}}{\rho} \int \limits_{0}^{\infty} 
%\frac{d\nu}{\nu} (U_0[ W_-, Z ] - U_2[ W_-, Z ]) J_1(\nu R) J_1(\nu \rho) = \\
%= \frac{R}{2} \cos \varphi \int \limits_{0}^{\infty}
%\frac{d\nu}{\nu \rho} (U_0[ W_-, Z ] - U_2[ W_-, Z ]) 
%J_1(\nu R) J_1(\nu \rho)
%\end{aligned} \end{equation*} }
%
%\textcolor{blue} { \begin{equation*} \begin{aligned}
%H_\rho = \frac{R A_0}{4} \frac{e^{i \varphi} - e^{- i \varphi}}{2i}
%\int \limits_{0}^{\infty} d \nu (J_{0}(\nu \rho) - J_{2}(\nu \rho))
%J_1(\nu R) (U_0[ W_-, Z ] - U_2[ W_-, Z ]) = \\
%= \frac{R}{2} \sin \varphi \int \limits_{0}^{\infty} d \nu 
%(J_0(\nu \rho) - \frac{J_1(\nu \rho)}{\nu \rho})
%J_1(\nu R) (U_0[ W_-, Z ] - U_2[ W_-, Z ]) = \\
%\end{aligned} \end{equation*} }
%
%\textcolor{blue} { \begin{equation*} \begin{aligned}
%\vect{H_\perp} \left( r, t \right) = \frac{A_0}{2} \left( 
%\vect{\rho_0} \left( I_4 - I_3 \right) \sin \varphi +
%\vect{\varphi_0} I_3 \cos \varphi  \right)
%\end{aligned} \end{equation*} }
%
%\textcolor{blue} { \begin{equation*} \begin{aligned}
%H_z (r,t) = \frac{1}{\sqrt{\mu_0}} \sum \limits_{m=-\infty}^{\infty}
%\int \limits_0^\infty \nu^2 d \nu h_m \Psi_m
%\end{aligned} \end{equation*} }
%
%\textcolor{blue} { \begin{equation*} \begin{aligned}
%\Psi_m (\nu) = \frac{J_m(\nu \rho)}{\sqrt{\nu}} e^{im \varphi} 
%\end{aligned} \end{equation*} }
%
%\textcolor{blue} { \begin{equation*} \begin{aligned}
%H_z (r,t) = 
%\frac{1}{\sqrt{\mu_0}} \int \limits_0^\infty \nu^2 d \nu h_{1} \Psi_{1} +
%\frac{1}{\sqrt{\mu_0}} \int \limits_0^\infty \nu^2 d \nu h_{-1} \Psi_{-1}
%\end{aligned} \end{equation*} }
%
%\textcolor{blue} { \begin{equation*} \begin{aligned}
%H_z (r,t) = R A_0 \frac{e^{im \varphi}-e^{-im \varphi}}{2} \int_0^\infty 
%d \nu J_1(\nu \rho) J_1 (\nu R)
%U_1 \left[ -i \nu (ct-z), \nu \sqrt{c^2t^2-z^2} \right]
%\end{aligned} \end{equation*} }
%
%\textcolor{blue} { \begin{equation*} \begin{aligned}
%H_z (r,t) = - R A_0 \sin \varphi \int_0^\infty 
%d \nu J_1(\nu \rho) J_1 (\nu R) U_1 [ W_-, Z ] = \\
%= - i R A_0 \sin \varphi \int_{0}^{\infty} J_1 \left( \nu R \right)
%J_1 \left( \nu \rho \right) U_1 [ W_-, Z ]
%\end{aligned} \end{equation*} }
%
%\textcolor{blue} { \begin{equation*} \begin{aligned}
%H_z \left( r, t \right) = - A_0 I_5 \sin \varphi
%\end{aligned} \end{equation*} }
%
\begin{equation} \label{eq:linear_h_cyl}
\vect{H} (r, t) = \frac{A_0}{2} \Big( 
\vect{\rho_0} \left( I_4 - I_3 \right) \sin \varphi +
\vect{\varphi_0} I_3 \cos \varphi -
\vect{z_0} I_5 \sin \varphi \Big),
\end{equation}
%
де 
%
%\begin{equation*}
%I_3 = R \int \limits_{0}^{\infty}
%\frac{d\nu}{\nu \rho} J_1(\nu R) J_1(\nu \rho)
%\Big( U_0[ W, Z ] - U_2[ W, Z ] \Big),
%\end{equation*}
%
%\begin{equation*}
%I_4 = R \int \limits_{0}^{\infty} d\nu J_1(\nu R) J_0(\nu \rho)
%\Big( U_0[ W, Z ] - U_2[ W, Z ] \Big),
%\end{equation*}
%
%\begin{equation*}
%I_5 = i R \int \limits_0^\infty 
%d \nu J_1(\nu \rho) J_1 (\nu R)
%U_1 \left[ -i \nu \left( \frac{ct}{\sqrt{\epsilon \mu}} - z \right), 
%\nu \sqrt{\frac{c^2t^2}{\epsilon \mu}-z^2} \right].
%\end{equation*}

В Декартовому базисі вектор напруженості магнітного поля матиме вигляд

\begin{equation} \begin{aligned} \label{eq:Hxyz}
\left( \begin{array}{c} H_x \\ H_y \\ H_z \end{array} \right) = 
\frac{A_0}{2} \left( \begin{array}{c}
- I_4 \sin \varphi \cos \varphi \\
I_3 \cos^2 \varphi + (I_4 - I_3) \sin^2 \varphi \\
- I_5 sin \varphi
\end{array} \right).
\end{aligned} \end{equation}

Для інтегралів $ I_3, I_4, I_5 $ аналітичні розв'язки, які представлено в 
додатку \ref{ch:lommel}, вдалось знайти лише на осі випромінювання 
($ \rho = 0 $). Відзначимо, що всі інтеграли мають дійсні значення, що витікає
з властивостей функції Ломмеля. Застосовуючи визначення функції Ломеля для  
виразу \eqref{eq:linear_h_cyl} на великій відстані від джерела, де плаский 
диск можна розглядати, як матеріальну точку, тобто $ z \gg R $ помічаємо, що 

%\begin{equation*}
%U_0[ W, Z ] - U_2[ W, Z ] = 
%J_0 \left( \nu \sqrt{\frac{c^2t^2}{\epsilon \mu}  - z^2} \right),
%\end{equation*}
%
тоді ортогональні поперечні компоненти електромагнітного поля попарно 
пропорційні через значення імпедансу вільного простору, в якому 
поширюється хвиля

\begin{equation} \label{eq:e2h}
\frac{E_x}{H_y} = \frac{E_y}{H_x} = 
\sqrt{\frac{\mu_0 \mu}{\epsilon_0 \epsilon}},
\end{equation}
%
що відповідає властивостям пласкої хвилі, а також підтверджує можливість 
апроксимації антен імпульсного випромінювання фізичною моделлю плаского 
стороннього електричного струму. Такого висновку можна дійти з того, що
саме пласку хвилю (сферичну хвилю нескінченного радіусу) очікується побачити 
на великій відстані від антени.

Звертаючись до аналітики з додатку \ref{ch:lommel}, також помічаємо, що 
рівність \eqref{eq:e2h} також строго виконується і поблизу апертури для 
всієї тривалості перехідного процесу. Рівність не виконується лише для 
області $ S_3 $, де $ \vect{E} = 0 $, а $ \vect{H}_\perp = const $.

Помічаємо, що кутова залежність $ \varphi $ електричної 
\eqref{eq:linear_e_cyl} та магнітної \eqref{eq:linear_h_cyl} перехідної 
функції антени відокремлена від інших змінних $ \rho $ та $ z $, що тісно 
зв'язані між собою. Таке групування змінних $ \rho $ та $ z $ та відділення 
$ \varphi $ є наслідком того, що кутова залежність напруженості поля, 
породженого апертурними антенами, змінюється лише від напрямку вектору 
струму, коли поведінка за іншими змінними залежить ще і від відстані.
Так як в дальній зоні джерело можна вважати точковим, поле, породжене ним, 
буде симетричним відносно кута спостереження. Так само і у ближній зоні -
симетричність розподілу струму відносно $ \varphi $ зберігається за рахунок
симетрії круглого джерела. З іншого боку, вплив відстані від джерела на 
форму імпульсу, що воно породжує, сильно залежить від відстані, що 
визначається змінними $ \rho $ та $ z $.

%%%%%%%%%%%%%%%%%%%%%%%%%%%%%%%%%%%%%%%%%%%%%%%%%%%%%%%%%%%%%%%%%%%%%%%%%%%%%%%
\section{Властивості перехідної функції плаского диску}

Перехідною функцією антени називається електромагнітне випромінювання, що 
збуджується нею, при часовій залежності сигналу у вигляді функції Хевісайда
\cite{imp:Kharkevich1950}. Перехідна функція дозволяє отримати 
випромінювання антени при збуджені сигналом довільної форми, не 
розв'язуючи задачу випромінювання, а користуючись принципом суперпозиції.

Знаючи момент приходу сигналу \eqref{eq:time1} та момент \eqref{eq:time3},
коли перехідна функція обертається в нуль, можна визначити тривалість
сигналу, збудженого струмом довільної часової залежності $ f(t) $ та 
ефективною тривалістю $ \tau_0 $:

\begin{equation} \label{eq:e2h}
\frac{c \tau}{\sqrt{\epsilon \mu}} = \frac{c \tau_0}{\sqrt{\epsilon \mu}} + 
\sqrt{(\rho+R)^2 + z^2} - \begin{cases} z, \rho < R \\ 
\sqrt{(\rho-R)^2 + z^2}, \rho > R \end{cases},
\end{equation}
%
а отже, тривалість електромагнітного імпульсу $ \tau $ пропорційна до 
$ \sqrt{\epsilon \mu} $.

Спираючись на принцип суперпозиції, оцінити вплив ефектів ближньої зони на 
імпульсне поле зі збудженням довільної форми можна, проілюструвавши поведінку 
перехідної функції в залежності від напрямку спостереження. Для цього 
побудуємо форму породжених електромагнітних імпульсів для деяких значень 
$ \rho $ та при фіксованих значеннях $ z $ і $ \varphi $.
 
\begin{figure}[h] \begin{center}
\includegraphics[scale=1.6]{MissileEffect}
\caption{Ефект електромагнітного снаряду ($ z = 2 $ м)} \label{fig:emp_rho}
\end{center} \end{figure}

На Рис.~\ref{fig:emp_rho} область $ S_1 $ зі сталим не нульовим значенням 
$ E_x $ компоненти поля спостерігаються лише для $ \rho < R $. За нею у 
часі наступає область $ S_2 $, де напруженість поля $ E_x $ поступово спадає 
на нуль. В області $ S_3 $ спостерігається $ E_x = 0 $.

Амплітуди всіх компонентів поля в області $ S_1 $ сталі, а значення 
цих амплітуд для $ E_x $ та $ H_y $ компонентів можна отримати з 
\eqref{eq:Exyz} та \eqref{eq:Hxyz} при $ \rho = 0 $:

\begin{equation*} \begin{aligned}
\vect{E} \{ S_1 \} = 
\vect{x_0} \frac{A_0}{4} 
\sqrt{\frac{\mu_0 \mu}{\epsilon_0 \epsilon}};
\end{aligned} \end{equation*}

\begin{equation*} \begin{aligned} 
\vect{H} \{ S_1 \} = \vect{y_0} \frac{A_0}{4}.
\end{aligned} \end{equation*}

Просторово-часова область з  постійними амплітудами напруженостей поля є 
одним з проявів ефекту електромагнітного снаряду і
зустрічається в багатьох роботах Содіна \cite{imp:Sodin1991, 
imp:Sodin1992-5, imp:Sodin1992-10, imp:Sodin1997}, роботах Ву 
\cite{imp:Wu1985, imp:Wu1987, imp:Wu1991}, в роботах Думіна
\cite{imp:Dumin1996} та в роботі Самсонова \cite{imp:Samsonov1986} але 
саме аналітичне значення амплітуди записано вперше, що є важливим для 
дослідження сильних полів.

Перехідна функція отримана без спрощень геометричної оптики, а отже 
справедлива для всіх точок спостереження ближньої зони. Отриманий результат 
дає змогу побачити форму електромагнітного імпульсу в залежності від 
азимутального кута при деякому $ z $ та $ \rho $, що лежать в ближній зоні.

З виразів для інтенсивності поля \eqref{eq:Exyz} та \eqref{eq:Hxyz} бачимо
центральну симетрію випромінювання відносно осі $ oZ $ для поперечних 
компонентів поля, отже достатньо дослідити лише залежність першій чверті.

\begin{figure}[h] \begin{center}
\includegraphics[scale=0.7]{LinearPulsShape}
\caption{Кутова залежність форми імпульсу ($ \rho = R/2 .. 2R $ м)} 
\label{fig:emp_shape}
\end{center} \end{figure}

На Рис.~\ref{fig:emp_shape} зображено залежність форми випроміненого 
імпульсу в залежності від кута спостереження в безпосередній близькості 
до джерела. Тут спостерігається вплив ефектів ближньої зони 
\cite{imp:Schantz2005}, при яких форма імпульсу залежить від напрямку 
поширення і від відстані. З рисунку §бачимо, що форма імпульсу в 
кожному напрямку в межах одного періоду симетрії унікальна, що можна 
використовувати в радарних та телекомунікаційних задачах 
(Дивись розділ~\ref{ch:neuron}).

За визначенням функції Ломмеля, інтеграли $ I_3 $ та $ I_4 $ є 
нескінченною сумою інтегралів по трьом функціям Бесселя 
(\eqref{eq:i3_pol_int}, \eqref{eq:i4_pol_int}). При аналізі магнітного 
поля на осі випромінювання помічаємо, що поперечне магнітне поле для 
області $ S_1 $ пропорційне по значенням з компонентами 
електричного. Для області $ S_3 $ поперечне магнітне поле описується 
рядом з поліномів Лягерра \eqref{eq:i3onaxis}.

\begin{figure}[h] \begin{center}
\includegraphics[scale=0.5]{SingulatiyFactorization}
\caption{Розклад сингулярності джерела} \label{fig:singulatiy_factorization}
\end{center} \end{figure}

На Рис.~\ref{fig:singulatiy_factorization} відображено процес сходження 
магнітної компоненти поля $ H_y $ при врахуванні різної кількості доданків
нескінченного поліному \eqref{eq:i3onaxis}. Бачимо, що з закінченням 
перехідного процесу наступає стаціонарне магнітне поле. Аналітично отриманий 
ряд сходиться дуже повільно. Можемо зробити висновок, що модовий базис 
погано підходить для опису стаціонарних процесів. Доданки вищих порядків 
мають значний вклад лише при пропорційно великому часі спостереження,
тобто, доданок $ m $ має внесок у значення функції $ H_z $ при $ ctm $.
На Рис.~\ref{fig:emp_h_z}, можемо оцінити згасання амплітуди 
магнітостатики з відстанню від джерела. На Рис.~\ref{fig:emp_h_z} 
спостерігається майже квадратичне згасання статичної 
компоненти з відстанню по $ z $. 

\begin{figure}[h] \begin{center}
\includegraphics[scale=0.6]{StaticOnAxis}
\caption{Магнітностатичне поле ($ \rho = 0 $ м)} \label{fig:emp_h_z}
\end{center} \end{figure}

Розрахунок магнітних компонентів - розрахунок нескінченного 
поліному від невласних інтегралів з повільною збіжністю,
хоча і наближено, але дає змогу оцінити розподіл магнітного поля в 
залежності від віддаленням від джерела. З Рис.~\ref{fig:emp_h_z} також 
помічаємо погане сходження поліному Лягерра близько до сингулярності, тому 
результати числового розрахунку там не аналізуємо. Схожий результат було 
отримано методами частотної області для усталеного процесу випромінювання 
статичного магнітного поля пласким диском електричного струму в перших, 
роботах присвячених випромінюванню плаского диску \cite{imp:BaumIN0009}, 
де вирази магнітного поля також містять поліноми Лягерра, що свідчить про 
правильність отриманих результатів.

\begin{figure}[h] \begin{center}
\includegraphics[scale=0.7]{LinearMagnetic}
\caption{Магнітностатичне поле ($ z = 2 $ м)} \label{fig:emp_h_rho}
\end{center} \end{figure}

На Рис.~\ref{fig:emp_h_rho} зображено зміну $ H_y $ за часом для 
різних значень $ 0 < \rho < 2R $ при $ z = 2R $. Знову спостерігаємо 
поступове згасання амплітуди магнітоcтатичного поля з відстанню.

З рис.~\ref{fig:part_rad} та з аналітичних розв'язків для 
$ I_1, I_2, I_3, I_4 $ видно, що, знаходячись в області $ S_1 $, 
спостерігач не отримує ніякої інформації про джерело окрім його наявності.
Відсутність нової інформації дозволяє зробити висновок, що у всій області 
$ S_1 $ всі компоненти векторів напруженості поля мають стале значення.
Наприклад, $ H_x = 0 $, а $ H_y = A_0/4 $. Отже, користуючись відсутністю 
змін у джерелі в області спостереження $ S_1 $, визначивши значення для 
компоненти поля в одній точці, отримаємо розв'язок для всієї області $ S_1 $.  

Розглянемо поведінку поздовжньої магнітної компоненти $ H_z $. Користуючись
визначенням функції Ломмеля, поздовжню компоненту вектору напруженості 
магнітного поля можна записати у вигляді нескінченного ряду з невласних 
інтегралів \eqref{eq:i5series}. Для першого доданку значення інтегралу 
вдається знайти аналітично \eqref{eq:i50}. Також помічаємо, що при 
$ \rho = 0 $ поздовжнє магнітне поле теж відсутнє. Користуючись висновком 
про те, що в області $ S_1 $ всі компоненти поля мають постійні значення 
напруженості та тим, що в усіх точках, де $ 0 < ct - z \cup \rho = 0 $ 
поздовжнє магнітне поле відсутнє, робимо висновок, що $ H_z \{ S_1 \} = 0 $.
Таке твердження додатково підтверджується числовими розрахунками, а також
аналітичним розв'язком для першого доданку, який не протирічить отриманому
результату.

\begin{figure}[h] \begin{center}
\includegraphics[scale=0.9]{Hz_terms}
\caption{$H_z$ в ($\rho = R/2$, $\varphi = -\pi/2$, $z = R$) при 
різній кількості доданків $ N $} \label{fig:hz_terms}
\end{center} \end{figure}

На рис.~\ref{fig:hz_terms} зображено поздовжню напруженість магнітного поля,
розрахованою чисельно за формулою \eqref{eq:i5series} для різної кількості 
врахованих доданків. Відносні похибка в розрахунку кожного невласного 
інтегралу не перевищує одного відсотку та отримано з застосуванням формули
Рунге \cite{imp:NumRecipes2007} та квадратурного правила Сімпсона 
\cite{imp:NumRecipes2007}. Помічаємо, що після двох врахованих 
доданків відхилення непомітне оку. Малий вплив доданків вищих порядків 
свідчить про те, що базис методу еволюційних рівнянь гарно підходить для 
опису нестаціонарних процесів.

Розглянемо співвідношення продільної магнітної компоненти $ H_z $ та 
поперечної електричної компоненти $ E_x $. На Рис.~\ref{fig:ex_vs_hz}
бачимо, що компонента $ H_z $ починається з затримкою відносно компоненти 
$ E_x $, а тривалість затримки співпадає з тривалістю ефекту 
електромагнітного снаряду. 

Плаский диск струму є апроксимацією ТЕМ джерела, а у вільному просторі 
поздовжня компонента повинна існувати, таким чином, затримку поздовжньої 
компоненти можна розуміти, як трансформацію ТЕМ хвилі в ТЕ з плином 
часу. Таким чином, ефект електромагнітного снаряду можна трактувати, як 
наслідок процесу формування хвилі у вільному просторі з TEM хвилі у рупорі.

\begin{figure}[h] \begin{center}
\includegraphics[scale=0.9]{Ex_vs_Hz}
\caption{$E_x$ і $H_z$ в ($\rho = R/2$, $\varphi = -\pi/2$, $z = R$)} 
\label{fig:ex_vs_hz}
\end{center} \end{figure}

В роботах Герца \cite{imp:Hertz1938} зустрічається твердження, що 
випромінює не антена, а простір довкола неї. Саме це і спостерігається в 
області електромагнітного снаряду: через відсутність причинного зв'язку з 
краєм рупора, хвиля знаходячись у вільному просторі, має властивості хвилі у 
ТЕМ хвилеводі. Тобто, протягом тривалості електромагнітного сняряду ми 
спостерігаємо перетворення хвилі у хвилеводі на хвилю у вільному просторі, 
яка не може існувати без поздовжньої компоненти ($ H_z \neq 0 $) 
\cite{imp:Borisov1991, imp:Harmuth1985}. Також відмітимо, що форма часової 
залежності збуджувального імпульсу не впливає на затримку у появі $ H_z $
компоненти поля. Вона залежить лише від наявності в апертурі антени
поверхні рівних фаз та спостерігатиметься в усіх точках простору, що 
знаходиться на нормалі до цієї "пласкої" частини фронту в апертурі.

Таким чином, можемо узагальнити твердження Герца для нестаціонарного 
процесу: поширення хвилі у вільному просторі почнеться з моменту, 
коли спостерігач дізнається про те, що розподіл струму в апертурі просторово
змінюється, а до цього, спостерігається той самий процес, що і у 
внутрішньому просторі антени. Звісно, такий ефект досягається за рахунок 
апертури рівних фаз і за рахунок часової залежності збуджувального струму у 
вигляді ступеневої функції Хевісайда, що неможливо для реальних джерел через 
похибки у виготовлені лінз і наявність передімпульсних коливань генератора.

Також, узагальнимо фізичний смисл розділу областей випромінювання: 
$ S_1 $ - область формування хвилі у вільному просторі, $ S_2 $ - область 
несталого (перехідного) процесу випромінювання, $ S_3 $ - область
усталеного процесу.

%%%%%%%%%%%%%%%%%%%%%%%%%%%%%%%%%%%%%%%%%%%%%%%%%%%%%%%%%%%%%%%%%%%%%%%%%%%%%%
\section{Збудження імпульсом довільної форми}

Оцінимо вплив ефектів ближньої зони на поле, що породжене пласким диском 
електричного струму на прямокутний збуджувальний імпульс тривалістю 
$ \tau_0 $, тобто вмикання струму до амплітуди $ A_0 $ Вольт і послідовне 
його вимкнення з затримкою $ \tau_0 $ секунд:

\begin{equation}
\vect{j} (r,t) = \vect{j_0} (r,t) - \vect{j_0} (r,t-\tau_0),
\end{equation}
%
де $ \vect{j_0} $ - струм, що породжує електричну $ \vect{E_0} $ і магнітну 
$ \vect{H_0} $ перехідні функції. Таким чином, користуючись принципом 
суперпозиції маємо:

\begin{equation}
\vect{E} (r,t) = \vect{E_0} (r,t) - \vect{E_0} (r,t-\tau_0).
\end{equation}

\begin{table}[ht]
\caption{Імпульсне електричне поле породжене LIRA, яка збуджується 
прямокутним імпульсом тривалістю R} \label{tab:meander_shape}
\centering
\begin{tabular}{cccc}

& $ \varphi = 0 $ & $ \varphi = \pi/4 $ & $ \varphi = \pi/2 $ \\

$ \rho = 0 $ &
\includegraphics[scale=2]{meander/00_00} & 
\includegraphics[scale=2]{meander/00_45} & 
\includegraphics[scale=2]{meander/00_90} \\

$ \rho < R $ &
\includegraphics[scale=2]{meander/05_00} & 
\includegraphics[scale=2]{meander/05_45} & 
\includegraphics[scale=2]{meander/05_90} \\

$ \rho > R $ &
\includegraphics[scale=2]{meander/20_00} & 
\includegraphics[scale=2]{meander/20_45} & 
\includegraphics[scale=2]{meander/20_90} \\

\end{tabular}
\end{table}

В таб.~\ref{tab:meander_shape} наведено залежність форми випроміненого 
імпульсу антени типу LIRA при прямокутному збуджені від напрямку 
випромінювання.

Тривалість перехідної функції $ \vect{E_0} $ на осі випромінювання 
$ \sqrt{z^2+R^2} - z $, а її максимальне значення досягається при малих 
відстанях до апертури та наближається до $ R $. Тому, якщо затримка перед 
вимкненням струму $ c \tau_0 $ триваліша за $ R $, накладання відгуків 
на вмикання струму та на його вимикання не спостерігається. Як відомо, саме 
це накладання є причиною мінливості форми імпульсу у ближній зоні. Якщо, 
також, врахувати $ \rho \neq 0 $, то можна отримати, що тривалість імпульсу 
$ \tau_0 > 2R/c $, коли накладання буде відсутнє для довільної 
точки спостереження.

Користуючись методикою інтегрування Дюамеля, \cite[ст. 40]{imp:Kharkevich1950} 
знайдемо поле від антени з відомою перехідною функцією $ \vect{E_0} (r,t) $, 
що збуджено сигналом з плавною часовою залежністю $ f(t) $ при 
максимальній амплітуді збуджувального струму $ A_0 $ В:

\begin{equation} \label{eq:duhamel}
\vect{E} = \int_0^t \derivat{f}{\tau} \vect{E_0} (t - \tau) d \tau.
\end{equation}

Інтеграл \eqref{eq:duhamel} є узагальненням принципу суперпозиції, а отже
працюватиме лише для лінійної залежності електричної індукції від 
напруженості поля. Проте, його використання доцільне для вивчення 
напруженості імпульсного поля, яка при накладанні може викликати 
необхідність врахування нелінійних ефектів, на кшталт, пробою в сильних полях.

Розглянемо в якості форми збудження плавне наростання амплітуди струму до 
значення $ A_0 $ протягом $ \tau_0 $ за метрикою 1-99 risetime у вигляді 
сигмоїдальної функції.

\begin{figure}[h] \begin{center}
\includegraphics[scale=0.4]{Sigmoidal}
\caption{$ E_x $ компонента поля від сигмоїдального збудження}
\label{fig:ex_sigmoidal}
\end{center} \end{figure}

Результати, представлені на Рис.~\ref{fig:ex_sigmoidal}, сходяться з 
експериментальними дослідженнями отримані незалежно Баумом та Ву.

Тепер розглянемо збуджувальний сигнал у вигляді $ f(t) = \sinc t $.
Така часова залежність збудження для плаского диску породжує імпульсне
швидко-осцилююче поле. Розглянемо інтерференцію цього поля в ближній зоні.
На Рис.~\ref{fig:ex_sinc} зображено залежність напруженості електричного 
поля $ E_x $ від часу для двох точок спостереження.

\begin{figure}[h] \begin{center}
\includegraphics[scale=0.4]{Sinc}
\caption{$ E_x $ компонента поля від $ \sinc t $ збудження}
\label{fig:ex_sinc}
\end{center} \end{figure}

З Рис.~\ref{fig:ex_sinc} бачимо, що на більшому віддалені від джерела, 
максимальна напруженість поля більша за максимальну напруженість, що була 
досягнута при менших відстанях. При накладанні перехідних функцій початку 
збудження та дзеркального його продовження спостерігається протифазне 
накладання, що з відстанню переходить у синфазне і амплітуда сигналу 
збільшується в межах $ 20\% $. Таким чином, при роботі з сильними 
швидко-осцилюючими імпульсними полями цей ефект треба враховувати, щоб 
уникнути електричного пробою в межах короткотривалих піках напруженості.
Звісно, перехідну функцію LIRA було отримано з припущенням відсутності 
втрат при поширенні, отже реальний вплив цього накладання виявиться 
меншим.

Відомо, що в дальній зоні форма сигналу, що випромінюється, схожа на похідну
від часової залежності збуджувального струму. Розглянемо 

\begin{figure}
\subfloat[Збудження у вигляді похідної від гаусіана]{\includegraphics[width = 3in]{gauss1}} 
\subfloat[Збудження у вигляді гаусіана]{\includegraphics[width = 3in]{gauss2}}
\caption{Електричне поле антени типу LIRA збуджене струмом з 
різними часовими залежностями}
\label{fig:gauss_shape}
\end{figure}

На рис.~\ref{fig:gauss_shape} зображено напруженість електричного поля, 
що породжено LIRA із часовою залежністю збудження у формі функції Гауса (а), 
та у формі її похідної (б). Рисунок ілюструє, що в ближній зоні можна знайти
такі точки спостереження, де різне збудження породить поле майже однакового 
вигляду. Як вже зазначалось, в дальній зоні, а також у більшості випадків у 
ближній зоні, форма випроміненого імпульсу є похідною від форми збуджувального 
імпульсу, але якщо спостерігати поле збуджене, струмом з часовою залежністю у 
вигляді похідної гаусінана в точці $ \rho = R/2, \varphi = \pi/4, z = R $
електромагнітний імпульс матиме такий вигляд, наче ми спостерігаємо імпульс, 
збуджений іншим струмом. Невелика різниця між формами сигналів може легко 
губитись в шумах та зробить процес розрізнення імпульсів стандартними 
алгоритмами FPGA неможливим.

%%%%%%%%%%%%%%%%%%%%%%%%%%%%%%%%%%%%%%%%%%%%%%%%%%%%%%%%%%%%%%%%%%%%%%%%%%%%%%
\section*{Висновки до розділу \ref{ch:linear}}

Побудовано аналітичне розв'язання у вигляді кусково визначеної функції для 
задачі випромінювання круглої апертури при нестаціонарному збуджені у 
вигляді прямокутної функції. Розв'язок отримано без наближення дальньої 
зони та визначено для всіх точок спостереження в кожен момент часу. 
Використання моделі круглої апертури, як моделі антен типу LIRA перевірено 
на експериментальних даних в окремих точках та на даних отриманих методом 
FDTD з комерційного електромагнітного симулятора CST Studio.

Отримане розв'язання задачі випромінювання плаского диску при збуджені у 
вигляді функції Хевісайда в лінійному наближенні має чітку просторово-часову 
зональність та ілюструє твердження Фарадея, що випромінює не антена, а 
простір довкола неї. Отримані області випромінювання наступають послідовно 
для довільної точки спостереження. Остання за часом настання область $ S_3 $ 
відповідає стаціонарному (усталеному) процесу випромінювання, коли всі точки 
апертури поєднані зі спостерігачем за принципом причинності. Настанню 
усталеного процесу передує область деякого транзитивного процесу $ S_2 $, 
поки поле від всієї апертури не досягне спостерігача. Найпершою для 
спостерігача просторово-часовою областю випромінювання в прожекторній зоні 
круглої апертури настає область електромагнітного снаряду $ S_1 $, де з 
хвилі у ТЕМ рупора формується ТЕ хвиля у вільному просторі.

%\textcolor{red}{TODO: можна строго порiвняти дiаграму спрямованостi отриману в
%	часовiй областi з дiаграмою Баума, а також оцiнити її вiдхилення в ближнiй
%	зонi}


% \textcolor{red}{ТОDО: Врахування краєвих ефектів випромінювання. 
% Шматько Александр Александрович на защите Лены Овсянниковой 
% замечал что, для того чтоб задачу можно было назвать апертурной, 
% нужно учеть в модели краевое излечение}
% ANS: тема майбутніх досліджень, усклавднить, уникли тому що резистори

Результати цього розділу відображені в роботах автора 
\cite{my:Telecom2018, my:UKRCON2017, my:UWBUSIS2018, my:UKRCON2019}.
