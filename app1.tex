\chapter{Свойства функции Бесселя первого рода}
\label{ch:bessel}

\section{Определение и линейные свойства}

\begin{equation}
J_{-n} \left( z \right) = \left( -1 \right)^n J_n \left( z \right)
\end{equation}

\begin{equation}
J_{n+1} \left( z \right) + J_{n-1} \left( z \right) = 
\frac{2n}{z} J_n \left( z \right)
\end{equation}

\section{Интегродифференциальные свойства}

\begin{equation}
2 \derivat{}{z} J_n \left( z \right) = 
J_{n-1} \left( z \right) - J_{n+1} \left( z \right) 
\end{equation}

\begin{equation}
\derivat{}{z} J_n \left( z \right) = 
J_{n-1} \left( z \right) - \frac{n}{z} J_{n} \left( z \right) 
\end{equation}

\begin{equation}
\derivat{}{z} J_n \left( z \right) = 
\frac{n}{z} J_{n} \left( z \right) - J_{n+1} \left( z \right) 
\end{equation}

\begin{equation}
\derivat{}{z} \frac{ J_n \left( z \right) }{ z^n }  = 
- \frac{ J_{n+1} \left( z \right) }{ z^n }
\end{equation}

\begin{equation}
\derivat{}{z} \left( z^n J_n \left( z \right) \right)  = 
z^n J_{n-1} \left( z \right)
\end{equation}

\section{Интеграл 1}

\begin{equation}
I_1 = \int\limits_{0}^{\infty} \frac{d\nu}{\nu} 
J_1 \left( \nu R \right) J_1 \left( \nu \rho \right) 
J_0 \left( \nu \sqrt{c^2 t^2 - z^2} \right)
\end{equation}

Интегралы такого вида встречаются в \cite[ст. 398]{Watson1922}.
\begin{equation} \begin{aligned}
\int\limits_{0}^{\infty} \frac{d t}{t^{\lambda + \nu}} 
J_\mu \left( at \right) J_\nu \left( bt \right) J_\nu \left( ct \right) =
\frac{ \left( bc/2 \right) ^\nu }
{ \Gamma \left( \nu + 1/2 \right) \Gamma \left( 1/2 \right) } \cdot \\
\cdot \int\limits_{0}^{\infty} \int\limits_{0}^{\pi}
\frac{J_\mu \left( at \right) J_\nu \left( \omega t \right)}
{\omega^\nu t^\lambda} \sin^{2\nu}{\phi} d\phi dt, \\
\omega = \sqrt{b^2 + c^2 - 2bc \cos \phi}
\end{aligned} \end{equation}

\textcolor{gray}{ \begin{equation*} \begin{aligned}
a = \sqrt{c^2 t^2 - z^2}; b = R; c = \rho; \lambda = 0 \\
\nu = 1; \mu = 0; \omega = \sqrt{R^2 + \rho^2 - 2 \rho R \cos \phi} \\
\int\limits_{0}^{\infty} \frac{d\nu}{\nu} 
J_1 \left( \nu R \right) J_1 \left( \nu \rho \right) 
J_0 \left( \nu \sqrt{c^2 t^2 - z^2} \right) = 
\frac{\rho R}{ 2 \Gamma \left( 3/2 \right) \Gamma \left( 1/2 \right) } \cdot \\
\int\limits_{0}^{\pi} 
\frac{\sin^2{\phi}}{\sqrt{R^2 + \rho^2 - 2 \rho R \cos \phi}}
\int\limits_{0}^{\infty} d \nu J_1 \left( \nu \omega \right) 
J_0 \left( \nu \sqrt{c^2 t^2 - z^2} \right) d \phi = \\
\end{aligned} \end{equation*} }

Далее для вычислений пригодится \cite[ст. 209]{SpecFunc1983}.
\begin{equation} \begin{aligned}
\int\limits_{0}^{\infty} d \nu
J_n \left( \nu t \right) J_{n-1} \left( \nu t \right) =
\end{aligned} \end{equation}

\section{Интеграл 2}

\cite[ст. 10]{SpecFunc1983}
