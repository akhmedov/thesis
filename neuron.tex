\chapter{Задача синтизу протоколу передачі інформації}
\label{ch:neuron}

%%%%%%%%%%%%%%%%%%%%%%%%%%%%%%%%%%%%%%%%%%%%%%%%%%%%%%%%%%%%%%%%%%%%%%%%%%%%%%%
\section{Сучасні системи імпульсного зв'язку}

Використання імпульсного випромінювання дозволяє вести безпечний швидкісний 
радіозв'язок в середовищах з великими втратами. Сучасна стандартизація 
використання імпульсного випромінювання [ieee] не передбачає його використання 
на відстанях більше 10 метрів. Це пояснюесться неможлівістю визначити наявнісь 
сигналу в умовах шуму методами сучасної радіоелектроніки.

В переважній більшості сисмем надширокосмугового імпульсного зв'язку 
використовують детектування наявності імпульсу за амплітудними показником.
Для цього передають серію імпульсів та отримують їх через каскад ліній затримок 
для визначення бінарного коду сигналу, що передається. Згідно сучасних
стандартаїв зв'язку такий підхд дозволяє перевищіти швидкість передачі даних 
гармонйних систем в 10 разів.

Подану схему можна вдосконалити використовуючи апаратно виконаний каскад 
нейронних мереж: автоенкодери для виділення сигналу над рівнем шуму та мережі 
довготривалої та короткострокової памяті для класифікації сигналів. Це дозволить 
використовувати кодування канального рівня більшої бітності, що підвищить 
емність каналу та збільшити відстань стабільного зв'язку.

Пошук оптимальних параметрів імпульсного зв'язку назвемо задачею синтезу каналу 
даних. Змінюючи параметри довжини імпульсу, скважності, кількості повторів, 
бітності кодування та форми імпульсів можна отримати різну емність каналу.

Розв'язання задачі синтезу каналу звя'зку передбачає, що просторово-часова 
імпульсна характеристика антенни відома. Використовуючи інтеграл Дюамеля 
отримаємо значення поля в будь-якій точці простору у довільний час. 
Електродинамічні параметри зв'язку виступають в якості глобальних параметрів 
мережі таким чином електродинаміну задачу можна звести до задачі оптимізації. 
Накладаючи адитивний білий гаусовий шум на значення амплітуд поля отриаємо данні 
для тренування нейронної мережі на класифікацію сигналів. Простіші методи 
показують себе гірше через залежність форми імпульсу від точки спостереження.

Використовуючи такий підхід оцінемо емність каралу зв'язку вода-земля та 
вода-вода в термінології критеріїв якасті классифікації та теорії інформації. 
На прикладі імпульсних характеристик для плаского диску рівномірно розподіленого 
електричного струму.

Використання систем класифікації дозволить будувати адаптивн локальні мережі, що 
не мають чітко встановлених вікон звязку. Це можливо завдяки використанню адитивних 
властивотей імпульсного малоінтерферуючого випромінювааня. Тобто два або більше 
сигналів при наклвдвнні сумуються за амплітудами та розпізнаються нейронною мережею,
як набір сігналів з піком вірогітності приналежності у відповідних сигналах.

%%%%%%%%%%%%%%%%%%%%%%%%%%%%%%%%%%%%%%%%%%%%%%%%%%%%%%%%%%%%%%%%%%%%%%%%%%%%%%%
\section{Задача визначення кута за формую імпульсу}

Станом на сьогодні зокалізація за допомогою надширокосмугових імпульсних систем
відбувається через вимірювання часу надходження відбитого випромінювання.

Використання нейроної мережі дозволить підвищіти точність такого вимірювання
через визначення не тільки часу, а і азимутального кута прийому.

Для цього навчемо нейронну мережу розпізнавати кут за формою імпульсу, яка залежить 
від точки спостереження.

%%%%%%%%%%%%%%%%%%%%%%%%%%%%%%%%%%%%%%%%%%%%%%%%%%%%%%%%%%%%%%%%%%%%%%%%%%%%%%%
\section{Задача классифікації імпульсу в лінійному просторі}

%%%%%%%%%%%%%%%%%%%%%%%%%%%%%%%%%%%%%%%%%%%%%%%%%%%%%%%%%%%%%%%%%%%%%%%%%%%%%%%
\section{Задача классифікації імпульсу в нелінійному просторі}

%%%%%%%%%%%%%%%%%%%%%%%%%%%%%%%%%%%%%%%%%%%%%%%%%%%%%%%%%%%%%%%%%%%%%%%%%%%%%%%
\section{Задача клвсифікації в умовах детермінованого шуму}

%%%%%%%%%%%%%%%%%%%%%%%%%%%%%%%%%%%%%%%%%%%%%%%%%%%%%%%%%%%%%%%%%%%%%%%%%%%%%%%
\section{Нілінійність, як засіб підвищення кількості переданої інформації}

%\section{Методологія написання коду та вимоги до програмного продукту}
%\section{Програмний дизаін та архітектура коду}
%\section{Тестування програмного забеспечення та вілідація чисельних розв'язків}
%\section{Чисельні методи розрахунку імпульсних полів}
%\section{Імпульсна антенна, як копмонент мереживного обряднання компьютера}
