\chapter*{Вступ}

%%%%%%%%%%%%%%%%%%%%%%%%%%%%%%%%%%%%%%%%%%%%%%%%%%%%%%%%%%%%%%%%%%%%%%%%%%%%%%%
\paragraph{Актуальність теми}

Час та частота - це абстракції, що описують одне явище природи - зміна
енергетичних взаємодій в системі. Для макроскопічної електродинаміки, зокрема,
існують підходи, що застосовують обидві абстракції. Вибір абстракції, тобто 
підходу, визначаються характером задачі, що вирішується. Історично склалось, що
більш широке розповсюдження отримала частотно-орієнтована методологія, яка в 
повній мірі виправдала себе. Однак новітні технології все частіше потребують 
розв'язків, які методи частотної області надати не зможуть. Наприклад, 
розв'язання задач розповсюдження та випромінювання в нестаціонарних 
неоднорідних середовищах, що характеризуються нелінійними та анізотропними 
ефектами. Це відновило інтерес до методів часової області.

%%%%%%%%%%%%%%%%%%%%%%%%%%%%%%%%%%%%%%%%%%%%%%%%%%%%%%%%%%%%%%%%%%%%%%%%%%%%%%%
\paragraph{Зв'язок роботи з науковими програмами}

Erasmus+

%%%%%%%%%%%%%%%%%%%%%%%%%%%%%%%%%%%%%%%%%%%%%%%%%%%%%%%%%%%%%%%%%%%%%%%%%%%%%%%
\paragraph{Мета та задача дослідження}

%%%%%%%%%%%%%%%%%%%%%%%%%%%%%%%%%%%%%%%%%%%%%%%%%%%%%%%%%%%%%%%%%%%%%%%%%%%%%%%
\paragraph{Методи дослідження}

\textcolor{red}{
Мотивуючись цим, вибираємо часовий метод для аналізу імпульсного випромінювання,
а саме метод еволюційних рівнянь. В якості основи для метода є вилучення 
поперечних компонент поля методом Рімана-Вольтера, також відомого у вітчизняній 
літературі як метод еволюційних рівнянь. Це дозволяє звести задачу розв'язання 
системи рівнянь Максвела до розв'язання диференціального рівняння другого 
порядку в часних похідних. У випадку вакуумного середовища це рівняння 
зводиться до рівняння Клейна-Гордона.}

В данній роботі використано метод теорії збуджень для побудови рекурентного
ітеративного методу врахування нелінійності.

Напрям пошуку солітоноподібних розвя'зків задачі випромінювання спрямовано 
згідно досліджень, що передбачають їх появу в нелінійний оптиці. Той факт,
що за основні ідеї та положення нелінійної оптики близькі до положень 
нелінійної радіофізики в теоретичномц плані дозволяє говорити про правильний 
напрям досліджень.

\textcolor{red}{Рекурентні нейронні мережі}

\textcolor{red}{Метод перехідної функції}

%%%%%%%%%%%%%%%%%%%%%%%%%%%%%%%%%%%%%%%%%%%%%%%%%%%%%%%%%%%%%%%%%%%%%%%%%%%%%%%
\paragraph{Наукова новизна отриманих результатів}

\textcolor{red}{Розв'язок декількох задач випромінювання в часовій області для 
лінійного та нелінійного простору}

\textcolor{red}{Адаптація методу еволюційних рівнянь для чисельного розв'язку
та побудова відповідного програмного комплексу, що розповсюджується під 
ліцензією GPL, як один з проектів GNU спів-товариства}

\textcolor{red}{Авторський метод синтезу протоколів передачі інформації}

%%%%%%%%%%%%%%%%%%%%%%%%%%%%%%%%%%%%%%%%%%%%%%%%%%%%%%%%%%%%%%%%%%%%%%%%%%%%%%%
\paragraph{Практичне значення отриманих результатів}

\begin{enumerate} 
	\item Шведський проект по аналізу властивостей надпровідникових матеріалів
	\item Передача інформації імпульсами на велику відстань
	\item Анаітичний розв'язок для лінзевих антен збуджених TEM рупором
	\item Оцінка нелінійних явищ що супроводжують випромінювання LIRA-и
	\item Новий підхід до прийому надширокосмугового сигналу без АЦП
\end{enumerate} 

%%%%%%%%%%%%%%%%%%%%%%%%%%%%%%%%%%%%%%%%%%%%%%%%%%%%%%%%%%%%%%%%%%%%%%%%%%%%%%%
\paragraph{Особистий вклад дисертанта}

Аналітичний розв'язок задачі плаского диску у всіх точках
Розв'язок задачі плаского диску слабкого нелінійного простору
Авторський метод синтезу протоколів передачі інформації

%%%%%%%%%%%%%%%%%%%%%%%%%%%%%%%%%%%%%%%%%%%%%%%%%%%%%%%%%%%%%%%%%%%%%%%%%%%%%%%
\paragraph{Апробація результатів дослідження}

\begin{enumerate} 
	\item Erasmus+
	\item Upsala Univ.
	\item Young scientist award UWBUSIS 2018
\end{enumerate} 

%%%%%%%%%%%%%%%%%%%%%%%%%%%%%%%%%%%%%%%%%%%%%%%%%%%%%%%%%%%%%%%%%%%%%%%%%%%%%%%
\paragraph{Публікації}

%%%%%%%%%%%%%%%%%%%%%%%%%%%%%%%%%%%%%%%%%%%%%%%%%%%%%%%%%%%%%%%%%%%%%%%%%%%%%%%
\paragraph{Зміст роботи}

%%%%%%%%%%%%%%%%%%%%%%%%%%%%%%%%%%%%%%%%%%%%%%%%%%%%%%%%%%%%%%%%%%%%%%%%%%%%%%%
\paragraph{Подяка}
