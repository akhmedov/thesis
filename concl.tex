\chapter*{Висновки}

\begin{enumerate}

\item Побудовано аналітичне розв'язання у вигляді кусково визначеної функції для 
задачі випромінювання круглої апертури при нестаціонарному збуджені у вигляді 
прямокутної функції. Розв'язок отримано без наближення дальньої зони та визначено 
для всіх точок спостереження в кожен момент часу. Використання моделі круглої 
апертури, як моделі антен типу LIRA перевірено на експериментальних даних в 
окремих точках та на даних отриманих методом FDTD з комерційного електромагнітного 
симулятора CST Studio.

\item Отримане розв'язання задачі випромінювання плаского диску при збуджені у 
вигляді функції Хевісайда в лінійному наближенні має чітку просторово-часову 
зональність та ілюструє твердження Фарадея, що випромінює не антена, а простір 
довкола неї. Отримані області випромінювання наступають послідовно для довільної 
точки спостереження. Остання за часом настання область $ S_3 $ відповідає 
стаціонарному (усталеному) процесу випромінювання, коли всі точки апертури 
поєднані зі спостерігачем за принципом причинності. Настанню усталеного процесу 
передує область деякого транзитивного процесу $ S_2 $, поки поле від всієї 
апертури не досягне спостерігача. Найпершою для спостерігача просторово-часовою 
областю випромінювання в прожекторній зоні круглої апертури настає область 
електромагнітного снаряду $ S_1 $, де з хвилі у ТЕМ рупора формується ТЕ хвиля 
у вільному просторі.

\item При урахуванні нелінійних ефектів самодії у керрівському середовищі, 
квазі-плаский фронт хвилі, що формується пласким диском електричного струму, 
за своєю формою наближається до сферичного. При цьому, тип хвилі зберігається і
хвиля з урахуванням нелінійних ефектів залишається поперечною електричною (ТЕ). 

\item Нейронне радіо дозволяє на практиці реалізувати максимальний теоретичний
потенціал імпульсних надширокосмугових радіосистем у всіх областях застосування:  
радіолокації, телекомунікації, зондування і тд. Головними перевагами таких систем 
в порівнянні з класичними є енергоефективність, а також якість розв'язання задач 
sequence-to-label і sequence-to-sequence за рахунок гнучкості системи.
Даний винахід розширює область застосування імпульсного радіо за 
рахунок покращених робочих характеристик. Підвищена стійкість до 
шуму дозволяє вирішувати радарні та телекомунікаційні задачі на 
більших відстанях. Можливість розпізнавати імпульси різної форми 
збудження уможливлює кодування корисного сигналу імпульсами різної форми, 
що підвищує швидкість передачі даних. 

\end{enumerate}
