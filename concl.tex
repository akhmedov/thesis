\chapter*{Висновки}

\begin{enumerate}
%
\item Побудовано аналітичне розв'язання у вигляді кусково визначеної функції для 
задачі випромінювання круглої апертури при нестаціонарному збуджені у вигляді 
прямокутної функції. Розв'язок отримано без наближення дальної зони та визначено 
для всіх точок спостереження в кожен момент часу. Використання моделі круглої 
апертури, як моделі антен типу LIRA перевірено на експерементальних даних в 
окремих точках та на даних отриманих методом FDTD з комерційного електромагнітного 
симулятора CST Studio.
%
\item Отримане розв'язання задачі випромінювання плаского диску при збуджені у 
вигляді функції Хевісайда в лінійному наближенні має чітку просторово-часову 
зональність та ілюструє твердження Фарадея, що випромінює не антена, а простір 
довколі неї. Отримані області випромінювання наступають послідовно для довільної 
точки спостереженя. Остання за часом настання область $ S_3 $ відповідає 
стаціонарному (усталеному) процесу випромінювання, коли всі точки апертури 
поєднані зі спостерегічем за принципом причинності. Настанню усталеного процесу 
передує область деякого транзитивного процесу $ S_2 $, поки поле від всієї 
апертури не досягне спостерігача. Найпершою для спостерігача просторово-часовою 
областю випромінювання в прожекторній зоні круглої апертури настає область 
електромагнітного снаряду $ S_1 $, де з хвилі у ТЕМ рупора формується ТЕ хвиля 
у вільному просторі.
%
\item 
%
\end{enumerate}
%

