\chapter*{Вступ}

%%%%%%%%%%%%%%%%%%%%%%%%%%%%%%%%%%%%%%%%%%%%%%%%%%%%%%%%%%%%%%%%%%%%%%%%%%%%%%%
\paragraph{Обґрунтування вибору теми дослідження}

Дисертаційну роботу присвячено дослідженню процесів випромінювання, поширення 
і приймання нестаціонарних електромагнітних хвиль. Поширення хвиль у 
середовищі досліджується з урахуванням їх нелінійної взаємодії. 

В якості джерела поля розглядається лінзова антена імпульсного випромінювання 
з круговою апертурою. Хоча, антена і знайшла широке застосування в задачах 
метрології і зондування, її характеристики, особливо у ближній зоні, 
ще недостатньо вивчені.

Відомо, що напрямленість лінзових антен у купі з ефектом електромагнітного 
снаряду концентрує енергію в ближній зоні випромінювача. Це явище викликає 
необхідність враховувати нелінійну взаємодію поля з середовищем розповсюдження
при значній крутизні фронту імпульсу.

Отже, побудова моделі випромінювання імпульсів довільної геометричної форми
лінзовою антеною з круговою апертурою без спрощень на лінійність поляризації
дозволить використовувати імпульси з більш крутим фронтами. З точки зору 
технічного результату це уточнить визначення ефективної площини розсіювання 
в задачах радарних задачах та знизить коефіцієнт бітових помилок в задачах 
комунікації.

Інший фактор, що впливає на ефективність застосування імпульсної 
радіоелектроніки, -- це залежність імпульсних характеристик надширокосмугових 
антен від напрямку спостереження, особливо у ближній зоні.
Виділення корисної інформації з електромагнітного імпульсу ускладнене 
залежністю його геометричної форми від точки спостереження. В той же час, 
розвиток методик виділення кількісних та якісних характеристик, що несуть 
корисну інформацію, з нестаціонарного електромагнітного поля дозволить 
покращити робочі характеристики імпульсної радіоелектроніки.

%%%%%%%%%%%%%%%%%%%%%%%%%%%%%%%%%%%%%%%%%%%%%%%%%%%%%%%%%%%%%%%%%%%%%%%%%%%%%%%
\paragraph{Зв'язок роботи з науковими програмами, планами, темами}

Дисертаційну роботу виконано на кафедрі прикладної електродинаміки факультету 
радіофізики, біомедичної електроніки та комп’ютерних систем Харківського 
національного університету імені~В.~Н.~Каразіна відповідно до планів 
науково-дослідних робіт: ``Моделювання та дослідження 
нелінійних нанорозмірних систем із нестаціонарними та гармонійними 
збудженнями для перетворення полів та створення елементів спінтроники'' 
(номер державної реєстрації: 0114U002585, здобувач -- виконавець), 
``Імпульсні та синусоїдальні поля у нелінійних і шаруватих електродинамічних 
структурах та наносистемах як перетворювачах полів і моделей елементів 
спінтроніки'' (номер державної реєстрації: 0117U004851, здобувач -- виконавець).

Здобувачем, в рамках даного дослідження, було пройдено стажування в 
Университеті Мурсії на факультеті математики по програмі академічної 
мобільності для здобувачів "Erasmus+" в 2017 році.

%%%%%%%%%%%%%%%%%%%%%%%%%%%%%%%%%%%%%%%%%%%%%%%%%%%%%%%%%%%%%%%%%%%%%%%%%%%%%%%
\paragraph{Мета і завдання дослідження}

Метою роботи є дослідження поведінки поля у ближній зоні, у тому числі, з
урахуванням нелінійної взаємодії із середовищем, а також вдосконалення 
методик обробки імпульсних хвиль за рахунок використання особливостей поля 
ближньої зони.

Задачі дослідження:

\begin{enumerate}

\item Встановлення перехідної функції лінзової антени імпульсного 
випромінювання, як явної функції від просторових координат та часу, що 
справедлива для довільної точки спостереження;

\item Аналіз енергетичних та нестаціонарних властивостей поля лінзової антени 
імпульсного випромінювання у ближній зоні для імпульсів різної форми;

\item Уточнення перехідної функції лінзової антени імпульсного 
випромінювання за рахунок врахування поліноміальної нелінійності 
поляризаційних характеристик середовища;

\item Розробка універсального методу врахування нестабільності геометричної 
форми імпульсів у ближній зоні при прийманні нестаціонарних електромагнітних 
хвиль;

\item Розвиток методики виділення кількісних та якісних характеристик, що 
несуть корисну інформацію, з нестаціонарного електромагнітного поля за 
рахунок застосування новітніх методів науки аналізу даних

\end{enumerate}

Об’єкт дослідження -- електромагнітне поле, що випромінюється лінзовою антеною 
імпульсного випромінювання

Предмет дослідження -- вплив ефектів ближньої зони випромінювання і нелініних 
ефектів взаємодії поля із середовищем на геометричну форму часової залежності
електромагнітних імпульсів та його інформаційну ємність

%%%%%%%%%%%%%%%%%%%%%%%%%%%%%%%%%%%%%%%%%%%%%%%%%%%%%%%%%%%%%%%%%%%%%%%%%%%%%%%
\paragraph{Методи дослідження}

Теоретичну основу дисертації становлять наукові праці вітчизняних та 
зарубіжних дослідників. Методологічну основу дисертації складають 
загальнонаукові та спеціально наукові методи пізнання, серед яких:

\begin{enumerate}

\item Метод еволюційних рівнянь у часовій області. Метод теоретичної 
радіофізики, що застосовано для зведення системи рівнянь Максвела до 
системи рівнянь відносно скалярних функцій, шляхом вилучення поперечних 
компонент поля за методикою Рімана-Вольтера.

\item Метод функції Рімана. Метод розв'язку диференціальних рівнянь, що
застосовано для розв'язку системи еволюційних рівнянь відносно коефіцієнтів 
розкладу електромагнітного поля по модовому базису.

\item Метод теорії збурень. Метод розв'язку математичних задач, що 
застосовано для врахування слабкої нелінійності середовища, що представлена 
у вигляді поліноміального розкладу вектору поляризації за ступенями 
напруженості електричного поля.

\item Метод зворотнього поширення помилки. Метод машинного навчання,
що застосовано для розв'язання задачі оптимізації параметрів запропонованої 
моделі виділення корисної інформації з надширокосмугового радіосигналу.

\end{enumerate}

%%%%%%%%%%%%%%%%%%%%%%%%%%%%%%%%%%%%%%%%%%%%%%%%%%%%%%%%%%%%%%%%%%%%%%%%%%%%%%%
\paragraph{Наукова новизна отриманих результатів}

Автором побудовано модель випромінювання поля пласким диском однонапрямного 
нестаціонарного електричного струму. Модель побудовано для довільної точки 
спостереження без наближення дальньої зони у наближенні лінійності 
поляризаційних властивостей середовища. Отримана модель -- явна функція 
просторових координат та часу.

Отримано поправку до лінійної моделі випромінювання плаского диску, що 
дозволяє враховувати кубічну нелінійність матеріальних рівнянь середовища.

В роботі запропоновано авторську методику виділення корисної інформації з 
радіосигналу, що має ряд переваг над існуючую: врахування залежності форми 
електромагнітного імпульсу від напрямку спостереження, можливість 
переналаштування на роботу з імпульсами іншої форми, можливість роботи при 
амплітудній $ SNR < 1 $.

%%%%%%%%%%%%%%%%%%%%%%%%%%%%%%%%%%%%%%%%%%%%%%%%%%%%%%%%%%%%%%%%%%%%%%%%%%%%%%%
\paragraph{Практичне значення отриманих результатів}

Дисертаційна робота відноситься до основних наукових напрямів сучасної 
радіофізики та визначає тенденції її подальшого розвитку. Напрямок
досліджень -- теорія випромінювання та приймання нестаціонарних 
нелінійних електромагнітних хвиль.

\textcolor{red}{TODO: незрозуміло, що тут писати. робота теоретична.
У першому наближені отримано перехідну функцію лінзової антени імпульсного 
випромінювання з урахуванням ефектів ближньої зони і нелінійних 
явищ, що дозволяє якісно моделювати задачі з її застосуванням.
Подано патентну заяву НОМЕР-ДЕРЖ-РЕЕСТРАЦІЇ на захист нової 
методики виділення корисної інформації з надширокосмугових сигналів. 
Патентним повіреним України НОМЕР-ДЕРЖ-РЕЕСТРАЦІЇ проведено аналіз 
патентоздатності, новизни та технічного рівня об'єкту захисту 
інтелектуальної власності.}

%%%%%%%%%%%%%%%%%%%%%%%%%%%%%%%%%%%%%%%%%%%%%%%%%%%%%%%%%%%%%%%%%%%%%%%%%%%%%%%
\paragraph{Особистий внесок здобувача}

В роботах \cite{my:Telecom2018, my:UKRCON2017, my:UKRCON2019} здобувач провів
аналіз існуючих моделей поля лінзової антени імпульсного випромінювання 
при довільному нестаціонарному збуджені і створив модель, що не має недоліків,
які спостерігались в аналогічних моделях. Отриманий розв'язок описує 
поведінку поля в усіх точках простору. Здобувачем проведено порівняння з 
існуючими моделями і не виявлено розбіжностей, що свідчать про помилковість.

В роботі \cite{my:Vesnik2017-2} побудовано поперечні розподіли енергії, а
в роботі \cite{imp:Vesnik2018} теоретично обґрунтовано енергетичні згустки, 
що спостерігаються.

В роботах \cite{my:Vesnik2015, my:Vesnik2017, my:Vesnik2017-2, my:MMET2014, 
my:UWBUSIS2014, my:ICATT2015, my:UWBUSIS2016, my:KPI2016, my:DIPED2019} 
здобувач провів аналітичну роботу по врахуванню нелінійної взаємодії 
нестаціонарного електромагнітного поля з тривимірним середовищем, 
подібним за нелінійними властивостями до атмосфери землі. Також здобувач
провів числове моделювання процесу нелінійної взаємодії та проаналізував 
його результати.

Здобувачем запропоновано \cite{my:UWBUSIS2018} і теоретично обґрунтовано
\textcolor{red}{[ПАТЕНТ]} авторську методику виділення корисної інформації 
з нестаціонарної електромагнітної хвилі. Здобувачем проведено числове 
моделювання \textcolor{red}{[СТАТТЯ]}, що демонструє працездатність та 
переваги нової концепції.

%%%%%%%%%%%%%%%%%%%%%%%%%%%%%%%%%%%%%%%%%%%%%%%%%%%%%%%%%%%%%%%%%%%%%%%%%%%%%%%
\paragraph{Апробація матеріалів дисертації}

Основні результати дисертації були представлені на конференціях і 
семінарах міжнародного рівня:

\begin{enumerate}

	\item 15th International Conference on Mathematical Methods in 
	Electromagnetic Theory (MMET) (Dnipropetrovsk, 2014)

	\item Proc. 7th International Conference on Ultrawideband and 
	Ultrashort Impulse Signals (UWBUSIS) (Kharkiv, 2015) (автор отримав 
	винагороду за кращу доповідь секції)

	\item International Conference on Antenna Theory and Techniques
	(ICATT) (Kharkiv, 2015)

	\item Ultrawideband and Ultrashort Impulse Signals 
	(UWBUSIS) (Odessa, 2016)

	\item Radio Electronics and Info Communications (Kiev, 2016)

	\item 2017 IEEE First Ukraine Conference on Electrical and Computer 
	Engineering (UKRCON) (Kiev, 2017)

	\item Ultrawideband and Ultrashort Impulse Signals (UWBUSIS) 
	(Odessa, 2018) (робота зайняла 1е місце на конкурсі робіт молодих вчених)

	\item 2017 IEEE 2nd Ukraine Conference on Electrical and Computer 
	Engineering (UKRCON) (Lviv, 2019)

	\item 2019 XXIVth International Seminar/Workshop on Direct and Inverse
	Problems of Electromagnetic and Acoustic Wave Theory (DIPED) (Lviv, 2019)
\end{enumerate} 

\paragraph{Публікації}

Матеріали дисертації опубліковано у ... наукових працях, серед яких ... 
статей у наукових фахових виданнях (зокрема з них ... статті, що входять до 
наукометричної бази даних Scopus), і ... у матеріалах та тезах доповідей на 
конференціях. Також на основі даних, отриманих в процесі виконання 
дисертаційної роботи, автором було отримано 1 патент України на винахід.

%%%%%%%%%%%%%%%%%%%%%%%%%%%%%%%%%%%%%%%%%%%%%%%%%%%%%%%%%%%%%%%%%%%%%%%%%%%%%%%
\paragraph{Обсяг і структура дисертації}

Дисертація складається зі вступу, 4 розділів, висновків, списку використаної 
літератури і 4 додатків. Загальний обсяг дисертації становить ... сторінок, з 
яких ... сторінок основного тексту. Список використаної літератури на ... 
сторінках включає в себе ... найменування. Всього у дисертації ... рисунка, 
... таблиць.

%%%%%%%%%%%%%%%%%%%%%%%%%%%%%%%%%%%%%%%%%%%%%%%%%%%%%%%%%%%%%%%%%%%%%%%%%%%%%%%
% \paragraph{Подяка}

% Проведення наукового дослідження, а також написання цієї кваліфікаційної роботи
% було б неможливе без заохочення і фінансової підтримки, що було мені надано.

% Я з гордістю висловлюю подяку колись моїм викладачам, а тепер і колегам

% На різних етапах мого дослідження долучались...

% заохочення і неоціненний інтелектуальний внесок

% Подяка контреб'юторам проекту Maxwell

% Моїй сім'ї -- найдорожчим людям, без яких я -- ніхто.

