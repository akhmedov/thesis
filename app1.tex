\chapter{Властивості функції Бесселя першого роду}
\label{ch:bessel}

\section{Визначення на лінійні властивості}

\begin{equation}
J_{-n} \left( z \right) = \left( -1 \right)^n J_n \left( z \right)
\end{equation}

\begin{equation}
J_{n+1} \left( z \right) + J_{n-1} \left( z \right) = 
\frac{2n}{z} J_n \left( z \right)
\end{equation}

\section{Інтегропохідні властивості}

\begin{equation}
2 \derivat{}{z} J_n \left( z \right) = 
J_{n-1} \left( z \right) - J_{n+1} \left( z \right) 
\end{equation}

\begin{equation}
\derivat{}{z} J_n \left( z \right) = 
J_{n-1} \left( z \right) - \frac{n}{z} J_{n} \left( z \right) 
\end{equation}

\begin{equation}
\derivat{}{z} J_n \left( z \right) = 
\frac{n}{z} J_{n} \left( z \right) - J_{n+1} \left( z \right) 
\end{equation}

\begin{equation}
\derivat{}{z} \frac{ J_n \left( z \right) }{ z^n }  = 
- \frac{ J_{n+1} \left( z \right) }{ z^n }
\end{equation}

\begin{equation}
\derivat{}{z} \left( z^n J_n \left( z \right) \right)  = 
z^n J_{n-1} \left( z \right)
\end{equation}

\section{Інтеграл 1}

\begin{equation*}
I_1 = \int\limits_{0}^{\infty} \frac{d\nu}{\nu} 
J_1 \left( \nu R \right) J_1 \left( \nu \rho \right) 
J_0 \left( \nu \sqrt{c^2 t^2 - z^2} \right)
\end{equation*}

Інтеграли такого виду зустрічаються в \cite[ст. 398]{Watson1922}.
\begin{equation} \begin{aligned} \label{eq:intJJJtable}
\int\limits_{0}^{\infty} \frac{d t}{t^{\lambda + \nu}} 
J_\mu \left( at \right) J_\nu \left( bt \right) J_\nu \left( ct \right) =
\frac{ \left( bc/2 \right) ^\nu }
{ \Gamma \left( \nu + 1/2 \right) \Gamma \left( 1/2 \right) } \cdot \\
\cdot \int\limits_{0}^{\infty} \int\limits_{0}^{\pi}
\frac{J_\mu \left( at \right) J_\nu \left( \omega t \right)}
{\omega^\nu t^\lambda} \sin^{2\nu}{\phi} d\phi dt, \\
\omega = \sqrt{b^2 + c^2 - 2bc \cos \phi} \\
\Re \left( \nu \right) > - \frac{1}{2};
\Re \left( \mu + \nu + 2 \right) > \Re \left( \lambda + 1 \right) > 0
\end{aligned} \end{equation}
%
\textcolor{lightgray}{ \begin{equation*} \begin{aligned}
a = \sqrt{c^2 t^2 - z^2}; b = R; c = \rho; \lambda = 0 \\
\nu = 1; \mu = 0; \omega = \sqrt{R^2 + \rho^2 - 2 \rho R \cos \phi} \\
\int\limits_{0}^{\infty} \frac{d\nu}{\nu} 
J_1 \left( \nu R \right) J_1 \left( \nu \rho \right) 
J_0 \left( \nu \sqrt{c^2 t^2 - z^2} \right) = 
\frac{\rho R}{ 2 \Gamma \left( 3/2 \right) \Gamma \left( 1/2 \right) } \cdot \\
\int\limits_{0}^{\pi} 
\frac{\sin^2{\phi}}{\sqrt{R^2 + \rho^2 - 2 \rho R \cos \phi}}
\int\limits_{0}^{\infty} d \nu J_1 \left( \nu \omega \right) 
J_0 \left( \nu \sqrt{c^2 t^2 - z^2} \right) d \phi
\end{aligned} \end{equation*} }
%
\textcolor{lightgray}{ \begin{equation*} \begin{aligned}
\Gamma \left( 3/2 \right) \Gamma \left( 1/2 \right) = 
\frac{\sqrt{\pi}}{2} \cdot \sqrt{\pi} = \frac{\pi}{2} 
\end{aligned} \end{equation*} }
%
\textcolor{lightgray}{ \begin{equation*} \begin{aligned}
I_1 = \frac{\rho R}{\pi} \int\limits_{0}^{\pi} 
\frac{\sin^2{\phi}}{\sqrt{R^2 + \rho^2 - 2 \rho R \cos \phi}}
\int\limits_{0}^{\infty} d \nu J_1 \left( \nu \omega \right) 
J_0 \left( \nu \sqrt{c^2 t^2 - z^2} \right) d \phi
\end{aligned} \end{equation*} }

Використання формули \eqref{eq:intJJJtable} дозволяє спростити $ I_1 $ до 
інтегралу по двом функціям Бесселя в ядрі замість трьох. Використаємо наступну 
формулу з \cite{Golubovic2013} для пошуку рішення нового інтегралу. 
%
\begin{equation} \begin{aligned} \label{eq:intJJtable}
\int\limits_{0}^{\infty} d \nu
J_n \left( a \nu \right) J_{n-1} \left( b \nu \right) = \begin{cases} 
b^{n-1} / a^n , 0 < b < a \\
1 / 2 b , 0 < a = b \\
0 , 0 < a < b
\end{cases} 
\end{aligned} \end{equation}
%
\textcolor{lightgray}{ \begin{equation*} \begin{aligned}
\int\limits_{0}^{\infty} d \nu J_1 \left( \nu \omega \right) 
J_0 \left( \nu \sqrt{c^2 t^2 - z^2} \right) = \begin{cases}
\left( R^2 + \rho^2 - 2 \rho R \cos \phi \right)^{-1/2}, 0 < b < a \\
\frac{1}{2} \left( c^2 t^2 - z^2 \right)^{-1/2}, 0 < a = b \\
0 , 0 < a < b
\end{cases} 
\end{aligned} \end{equation*} }

Згідно з умовами інтегрування з \eqref{eq:intJJtable}, повинно виконуватись 
співвідношення $ 0 < b \leq a $ для того, щоб значення інтегралу $ I_1 $ 
існувало та було б відмінне від нуля. Застосовуючи цю умову в рамках поставленої 
задачі бачимо її фізичність: відсутність протиріч з принципом причинності. 
Доцільно виписати діапазон значень $ \phi $ користуючись областю значень 
інтегралу з \eqref{eq:intJJJtable} та умови інтегрування \eqref{eq:intJJtable}.
%
\textcolor{lightgray}{ \begin{equation*} \begin{aligned}
\sqrt{R^2 + \rho^2 - 2 \rho R \cos \phi} \geq \sqrt{c^2 t^2 - z^2} \\
R^2 + \rho^2 - 2 \rho R \cos \phi \geq c^2 t^2 - z^2 \\
\cos \phi \leq \frac{R^2 + \rho^2}{2 \rho R} - \frac{c^2 t^2 - z^2}{2 \rho R} \\
\phi \leq \arccos \left( \frac{\rho^2 + R^2}{2 \rho R} - 
\frac{c^2 t^2 - z^2}{2 \rho R} \right), 0 \leq \phi \leq \pi
\end{aligned} \end{equation*} }
%
\textcolor{red}{ \begin{equation*}
0 \leq \phi \leq \arccos \left( \frac{\rho^2 + R^2}{2 \rho R} - 
\frac{c^2 t^2 - z^2}{2 \rho R} \right)
\end{equation*} }

Зазначимо, що тригонометрична функція $ \arccos $ завжди менша за $ \pi $,
тому остання нерівність не протирічить тому, що $ \phi \leq \pi $.

Очевидно що застосування \eqref{eq:intJJtable} накладає умови на співвідношення
просторових координат а часу. Користуючись областю значень 
$ -1 \leq \cos \phi \leq 1 $, запишемо систему що визначає світловий конус 
\cite[ст. 22]{LandauII} для компоненти поля, що мітить даний інтеграл. 
\textcolor{red}{ Так як значення інтегралу при від'ємних значеннях $ \cos \phi $ 
рівні нулю. Обмежимо область значень знизу: $ 0 \leq \cos \phi \leq 1 $ }. 
%
\textcolor{lightgray}{ \begin{equation*} \begin{aligned}
\begin{cases}
\frac{\rho^2 + R^2}{2 \rho R} - \frac{c^2 t^2 - z^2}{2 \rho R} \leq 1 \\
\frac{\rho^2 + R^2}{2 \rho R} - \frac{c^2 t^2 - z^2}{2 \rho R} \geq - 1
\end{cases}
\begin{cases}
\rho^2 + R^2 - c^2 t^2 + z^2 \leq 2 \rho R \\
\rho^2 + R^2 - c^2 t^2 + z^2 \geq - 2 \rho R
\end{cases} 
\end{aligned} \end{equation*} }
%
\textcolor{lightgray}{ \begin{equation*} \begin{aligned}
\begin{cases}
0 \leq \left( \rho - R \right)^2 \leq c^2 t^2 - z^2 \\ 
\left( \rho + R \right)^2 \geq c^2 t^2 - z^2 \geq 0
\end{cases} 
\end{aligned} \end{equation*} }
%
\textcolor{red}{ \begin{equation*} \begin{aligned}
\rho^2 + R^2 \geq c^2 t^2 - z^2 \geq \left( \rho - R \right)^2
\end{aligned}  \end{equation*} }
%
\begin{equation*}
\left( \rho + R \right)^2 \geq c^2 t^2 - z^2 \geq \left( \rho - R \right)^2
\end{equation*}
%
\textcolor{lightgray}{ \begin{equation*} \begin{aligned}
I_1 = \frac{\rho R}{\pi} \int\limits_{0}^{\pi} 
\frac{\sin^2{\phi}}{\sqrt{R^2 + \rho^2 - 2 \rho R \cos \phi}}
\int\limits_{0}^{\infty} d \nu J_1 \left( \nu \omega \right) 
J_0 \left( \nu \sqrt{c^2 t^2 - z^2} \right) d \phi = \\
= \frac{\rho R}{\pi} \int\limits_{0}^{\psi} 
\frac{\sin^2{\phi}}{\sqrt{R^2 + \rho^2 - 2 \rho R \cos \phi}}
\frac{1}{\sqrt{R^2 + \rho^2 - 2 \rho R \cos \phi}} d \phi = \\
= \frac{\rho R}{\pi} \int\limits_{0}^{\psi}
\frac{\sin^2{\phi}}{R^2 + \rho^2 - 2 \rho R \cos \phi} d \phi = 
\frac{\rho}{\pi R} \int\limits_{0}^{\psi}
\frac{\sin^2{\phi}}{1 + \frac{\rho^2}{R^2} - \frac{2 \rho}{R} \cos \phi} d \phi
\end{aligned} \end{equation*} }

Застосовуючи формули \eqref{eq:intJJtable} і \eqref{eq:intJJJtable} приведемо 
початкову форму $ I_1 $ до наступного виду.
%
\begin{equation*} \begin{aligned}
I_1 = \frac{\rho}{\pi R} \int\limits_{0}^{\psi}
\frac{\sin^2{\phi}}{1 + \frac{\rho^2}{R^2} - 
\frac{2 \rho}{R} \cos \phi} d \phi \\
\psi = \arccos \left( \frac{\rho^2 + R^2}{2 \rho R} - 
\frac{c^2 t^2 - z^2}{2 \rho R} \right)
\end{aligned} \end{equation*}

\textcolor{red} {Зауважимо, що для $ \psi = 0 $ підінтегральна функція не 
відповідає \eqref{eq:intJJtable} та і значення інтегралу в цій точці виходить не 
визначене. Але, згідно властивостей інтегралів Рімана, значення інтегралу в одній 
точці не впливає на значення означеного інтегралу}
%
\textcolor{lightgray}{ \begin{equation*} \begin{aligned}
\int \frac{\sin^2{\phi}}{a + b \cos \phi} d \phi = 
\int \frac{1 - \cos^2{\phi}}{a + b \cos \phi} d \phi = 
\int\frac{d \phi}{a + b \cos \phi}  -
\int \frac{\cos^2{\phi}}{a + b \cos \phi} d \phi = \\
= \int \frac{d \phi}{a + b \cos \phi}  - 
\int \frac{\cos^2{\phi}}{a + b \cos \phi} d \phi -
\frac{a}{b} \int \frac{\cos \phi}{a + b \cos \phi} d \phi + \\
+ \frac{a}{b} \int \frac{\cos \phi}{a + b \cos \phi} d \phi = 
\int \frac{d \phi}{a - b \cos \phi} +
\frac{a}{b} \int \frac{\cos \phi}{a + b \cos \phi} d \phi - \\
- \int \frac{\cos^2{\phi} + \frac{a}{b} \cos \phi} {a + b \cos \phi} d \phi =
\int \frac{d \phi}{a + b \cos \phi} + 
\frac{a}{b} \int\limits_{0}^{\psi} \frac{\cos \phi}{a + b \cos \phi} d \phi -
\end{aligned} \end{equation*} }
%
\textcolor{lightgray}{ \begin{equation*} \begin{aligned}
- \frac{1}{b} \int \frac{\cos \phi + a/b} {a/b +  \cos \phi} \cos \phi d \phi = 
\int \frac{d \phi}{a + b \cos \phi} + 
\frac{a}{b} \int \frac{\cos \phi}{a + b \cos \phi} d \phi - \\
- \frac{1}{b} \int \cos \phi d \phi = \int \frac{d \phi}{a + b \cos \phi} - 
\frac{1}{b} \int \cos \phi d \phi + \frac{a}{b^2} \int
\frac{a - a + b \cos \phi}{a + b \cos \phi} d \phi = \\ 
= \int\frac{d \phi}{a + b \cos \phi} - \frac{1}{b} \int \cos \phi d \phi +
\frac{a}{b^2} \int \frac{a + b \cos \phi}{a + b \cos \phi} d \phi - \\ 
- \frac{a^2}{b^2} \int \frac{d \phi}{a + b \cos \phi} = 
\left( 1 - \frac{a^2}{b^2} \right) \int\frac{d \phi}{a + b \cos \phi} - 
\frac{1}{b} \int \cos \phi d \phi + \frac{a}{b^2} \int \phi =
\end{aligned} \end{equation*} }
%
\textcolor{lightgray}{ \begin{equation*} \begin{aligned}
= \left( 1 - \frac{a^2}{b^2} \right)
\int\limits_{0}^{\psi} \frac{d \phi}{a + b \cos \phi} -
\frac{\sin \psi - \sin 0}{b} + a \frac{\psi}{b^2} = \\
= \left( 1 - \frac{a^2}{b^2} \right)
\int\limits_{0}^{\psi} \frac{d \phi}{a + b \cos \phi} +
\frac{\sin \psi}{b} + a \frac{\psi}{b^2}
\end{aligned} \end{equation*} }
%
\textcolor{lightgray}{ \begin{equation*} \begin{aligned}
a = 1 + \frac{\rho^2}{R^2}; b = - \frac{2 \rho}{ R } \\
\frac{\pi R}{\rho} I_1 = \left( 1 - \frac{a^2}{b^2} \right)
\int\limits_{0}^{\psi} \frac{d \phi}{a + b \cos \phi} +
\frac{\sin \psi}{b} + a \frac{\psi}{b^2} = \\
= \left( 1 - \left( \frac{1 + \frac{\rho^2}{R^2}} 
{ \frac{2 \rho}{R} } \right)^2 \right) 
\int \limits_{0}^{\psi} \frac{d \phi}{1 + \frac{\rho^2}{R^2} -  
\frac{2 \rho}{R} \cos \phi} -
\frac{\sin \psi}{\frac{2 \rho}{ R }} + \left( 1 + \frac{\rho^2}{R^2} \right) 
\frac{\psi}{\left( \frac{\rho^2}{R^2} \right)^2} = \\
= \left( R^2 - \left( \frac{R^2 + \rho^2} 
{2 \rho} \right)^2 \right) 
\int \limits_{0}^{\psi} \frac{d \phi}{R^2 + \rho^2 - 2 \rho R \cos \phi} -
\frac{R}{2 \rho} \sin \psi + \frac{R^2}{\rho^2} 
\left( \frac{R^2}{\rho^2} + 1 \right) \psi  
\end{aligned} \end{equation*} }
%
\textcolor{lightgray}{ \begin{equation*} \begin{aligned}
\frac{4 \rho^2}{4 \rho^2} R^2 - \left( \frac{R^2 + \rho^2}{2 \rho} \right)^2 =
\frac{4 \rho^2 R^2 - R^4 - 2 \rho^2 R^2 - \rho^4}{4 \rho^2} =
- \frac{\left( \rho^2 - R^2 \right)^2}{4 \rho^2} 
\end{aligned} \end{equation*} }
%
\textcolor{lightgray}{ \begin{equation*} \begin{aligned}
\frac{\pi R}{\rho} I_1 = 
- \frac{\left( \rho^2 - R^2 \right)^2}{4 \rho^2} 
\int \limits_{0}^{\psi} \frac{d \phi}{R^2 + \rho^2 - 2 \rho R \cos \phi} - \\
- \frac{R}{2 \rho} \sin \psi + \frac{R^2}{\rho^2} 
\left( \frac{R^2}{\rho^2} + 1 \right) \psi 
\end{aligned} \end{equation*} }

Тригонометричними перетвореннями зведемо поточний вид $ I_1 $ до табличного 
інтегралу.
%
\begin{equation*} \begin{aligned}
I_1 = - \frac{\left( \rho^2 - R^2 \right)^2}{4 \pi \rho R} 
\int \limits_{0}^{\psi} \frac{d \phi}{R^2 + \rho^2 - 2 \rho R \cos \phi} - 
\frac{\sin \psi}{2 \pi} + \frac{R}{\rho} 
\left( \frac{R^2}{\rho^2} + 1 \right) \frac{\psi}{\pi}
\end{aligned} \end{equation*}

Таблична формула для неозначеного випадку інтегралу може буде знайдена в 
\cite[ст. 181]{ElementFunc1983}.
%
\begin{equation} \label{eq:caseTableIntegral}
\int \frac{d x}{a + b \cos{x}} = \begin{cases}
\frac{2}{\sqrt{a^2-b^2}} \arctan \frac{\sqrt{a^2-b^2} \tan \frac{x}{2}}
{a + b}, a^2 > b^2 \\
\frac{1}{\sqrt{b^2-a^2}} \ln 
\frac{\sqrt{b^2-a^2} \tan \frac{x}{2} + a + b}
{\sqrt{b^2-a^2} \tan \frac{x}{2} - a - b}, a^2 < b^2
\end{cases}
\end{equation}

Помітимо, що застосування формули змушує нас розглядати поле прожекторної зони
($ \rho < R $) окремо від поля в іншому простору ($ \rho > R $), через умову 
співвідношення $ a $ та $ b $. Отримаємо значення інтегралу $ I_1 $ для 
\textcolor{red}{області Френеля} в явному вигляді.
%
\textcolor{lightgray}{ \begin{equation*} \begin{aligned}
a^2 > b^2  \Rightarrow  
\left( R^2 + \rho^2 \right)^2 > 4 \rho^2 R^2 \\
R^4 + 2 \rho^2 R^2 + \rho^4 - 4 \rho^2 R^2 > 0 \Rightarrow 
\left( \rho^2 - R^2 \right)^2 > 0 \\
\rho > R
\end{aligned} \end{equation*} }
%
\textcolor{lightgray}{ Далі знадобиться: }
%
\textcolor{lightgray}{ \begin{equation*} \begin{aligned}
a^2 - b^2 = - \left( b^2 - a^2 \right) = 
R^4 + 2 \rho^2 R^2 + \rho^4 - 4 \rho^2 R^2 = \left( \rho^2 - R^2 \right)^2 \\
\lim_{\alpha \to 0} \tan{\alpha} = 0 \Rightarrow
\lim_{\alpha \to 0} \arctan \left( a \tan{\alpha} \right) = 0
\end{aligned} \end{equation*} }
%
\textcolor{lightgray}{ \begin{equation*} \begin{aligned}
\int \limits_{0}^{\psi} \frac{d \phi}{R^2 + \rho^2 - 2 \rho R \cos \phi} =
\left. \begin{cases}
\frac{2}{\rho^2 - R^2} \arctan \left( \frac{\rho^2 - R^2}
{\left( \rho - R \right)^2} \tan \frac{\phi}{2} \right), \rho > R \\
- \frac{1}{\rho^2 - R^2} \ln
\frac{- \left( \rho^2 - R^2 \right) \tan \frac{\phi}{2} + \left( \rho - R \right)^2}
{- \left( \rho^2 - R^2 \right) \tan \frac{\phi}{2} - \left( \rho - R \right)^2}
, \rho < R
\end{cases} \right|_{0}^{\psi} = \\
= \frac{1}{\rho^2 - R^2} \left. \begin{cases} 
2 \arctan \left( \frac{\rho + R}{\rho - R} \tan \frac{\phi}{2} \right) \\
- \ln \left| \frac{\tan \frac{\phi}{2} - \frac{\rho - R}{\rho + R}} 
{\tan \frac{\phi}{2} + \frac{\rho - R}{\rho + R}} \right|
\end{cases} \right|_{0}^{\psi} = 
\frac{1}{\rho^2 - R^2} \begin{cases} 
2 \arctan \left( \frac{\rho + R}{\rho - R} \tan \frac{\psi}{2} \right) \\
- 0 + \ln \left| \frac{\tan \frac{\psi}{2} - \frac{\rho - R}{\rho + R}}
{\tan \frac{\psi}{2} + \frac{\rho - R}{\rho + R}} \right|
\end{cases}
\end{aligned} \end{equation*} }
%
\begin{equation*} \begin{aligned}
I_1 \left( \rho > R \right) = 
\frac{R^2}{\rho^2} \frac{\rho^2 + R^2}{\pi \rho R} \psi - 
\frac{\sin \psi}{2 \pi} - \frac{\rho^2 - R^2}{2 \pi \rho R} 
\arctan \left( \frac{\rho + R}{\rho - R} \tan \frac{\psi}{2} \right)
\end{aligned} \end{equation*}

\textcolor{red}{ Застосуємо \eqref{eq:caseTableIntegral} для наявних лімітів 
користуючись формулою Ньютона та побачимо, що при аргументі $ \phi = 0 $ інтеграл 
стає неозначеним. Це наслідок неправильного використання формули 
\eqref{eq:intJJtable}. Прирівнявши $ \cos \phi $ до нуля побачимо, що рівняння 
світлового конуса починає описувати область просторово подібних інтервалів. Це 
пояснює чому інтеграл дає ірраціональні значення: це просторово подібний інтервал в 
термінології СТВ. Для того щоб нейтралізувати помилку внесену застосуванням 
інтеграла \eqref{eq:intJJtable} розглянемо $ \ln |f(r,t)| $ замість 
$ \ln f(r,t) $.} Тепер випишемо функціональну залежність $ I_1 $ для прожекторної 
зони.
%
\begin{equation*} \begin{aligned}
I_1 \left( 0 < \rho < R \right) = 
\frac{R^2}{\rho^2} \frac{\rho^2 + R^2}{\pi \rho R} \psi - 
\frac{\sin \psi}{2 \pi} - \frac{\rho^2 - R^2}{4 \pi \rho R} 
\ln \left| \frac{\tan \frac{\psi}{2} - \frac{\rho - R}{\rho + R}}
{\tan \frac{\psi}{2} + \frac{\rho - R}{\rho + R}} \right|
\end{aligned} \end{equation*}

Що стосовно поля при $ \rho = R $, тобто $ a^2 = b^2 $, то очевидно, що обидва 
випадки однаково підходять.
%
\begin{equation*} \begin{aligned}
\left. I_1 \right|_{\rho = R} = 
\left. I_1 \left(0 < \rho < R \right) \right|_{\rho = R} =
\left. I_1 \left( \rho > R \right) \right|_{\rho = R}
\end{aligned} \end{equation*}

На останок, спростимо тригонометричні вирази, що містять $ \psi $. Розглянемо 
$ \psi = \arccos f(r,t) $, де $ f(r,t) $ задовільна функція координат. 
Тоді $ f(r,t) = \cos \psi $. Зазначимо, що з означення відомо, що 
$ \psi \in \left[ 0, \pi \right] $, тому $ \sin \psi \geq 0 $. Таким чином:
%
\begin{equation*} \begin{aligned}
\sin \psi = \sqrt{1 - \cos^2{\psi}} = \sqrt{1 - f^2(r,t)}
\end{aligned} \end{equation*}

Згадуючи введене означення для $ \psi $ зашипимо, що
%
\textcolor{lightgray}{ \begin{equation*} \begin{aligned}
\psi = \arccos \left( \frac{\rho^2 + R^2}{2 \rho R} - 
\frac{c^2 t^2 - z^2}{2 \rho R} \right)
\end{aligned} \end{equation*} }
%
\textcolor{lightgray}{ \begin{equation*} \begin{aligned}
\sin \psi = \sqrt{1 - \left( \frac{\rho^2 + R^2}{2 \rho R} - 
\frac{c^2 t^2 - z^2}{2 \rho R} \right)^2} = 
\sqrt{1 - \frac{\left( \rho^2 + R^2 - c^2 t^2 + z^2 \right)^2}{4 \rho^2 R^2} } = \\
= \sqrt{\frac{4 \rho^2 R^2}{4 \rho^2 R^2} - 
\frac{\left( \rho^2 + R^2 - c^2 t^2 + z^2 \right)^2}{4 \rho^2 R^2} } =
\sqrt{\frac{4 \rho^2 R^2 - \left( \rho^2 + R^2 - c^2 t^2 + z^2 \right)^2}
{4 \rho^2 R^2}} = \\
= \frac{1}{2 \rho R} \sqrt{4 \rho^2 R^2 - \left( \rho^2 + R^2 \right)^2 +
2 \left( \rho^2 + R^2 \right) \left( c^2 t^2 - z^2 \right) - 
\left( c^2 t^2 - z^2 \right)^2} = \\
= \frac{1}{2 \rho R} \sqrt{- \left( \rho^2 - R^2 \right)^2 +
2 \left( \rho^2 + R^2 \right) \left( c^2 t^2 - z^2 \right) - 
\left( c^2 t^2 - z^2 \right)^2} = \\
= \frac{c^2 t^2 - z^2}{2 \rho R} \sqrt{2 \frac{\rho^2 + R^2 }{c^2 t^2 - z^2} - 
\left( \frac{\rho^2 - R^2 }{c^2 t^2 - z^2} \right)^2 - 1}
\end{aligned} \end{equation*} }
%
\textcolor{lightgray}{ \begin{equation*} \begin{aligned}
\tan \frac{\psi}{2} = \pm \sqrt{ \frac{1 - \cos \psi}{1 + \cos \psi} } = 
\sqrt{ \frac{1- \frac{\rho^2 + R^2}{2 \rho R} + \frac{c^2 t^2 - z^2}{2 \rho R}}
{1 + \frac{\rho^2 + R^2}{2 \rho R} - \frac{c^2 t^2 - z^2}{2 \rho R}} } =
\sqrt{ \frac{c^2t^2 - z^2 - \left( \rho - R \right)^2}
{\left( \rho + R \right)^2 - \left( c^2t^2 - z^2 \right)} }
\end{aligned} \end{equation*} }

\section{Інтеграл 2}
