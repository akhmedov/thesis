%% mon2017dev-abs.tex  Приклад анотації (для mon2017dev.tex)

\begin{abstract}[language=ukrainian, header=true]

У дисертації розглядається вплив ефектів ближньої та проміжної зон
випромінювання на практичне використання надширокосмугових 
електромагнітних імпульсів в якості носія інформації. Зокрема, розглянуто 
вплив слабкої нелінійності Керра на форму електромагнітного імпульсу 
великої амплітуди. Також в роботі розглядається можливість врахування 
фізики процесу формування хвилі для збільшення точності виділення корисної
інформації з електромагнітного імпульсу.

В огляді літератури проаналізовано спеціально наукові методи, актуальні 
для предмету дослідження. Огляд містить результати сучасних досліджень 
як закордонних, так і відчинених авторів.

Автором вперше отримано аналітичний розв'язок задачі випромінювання 
поодинокого наносекундного електромагнітного імпульсу в нелінійне 
середовище з урахуванням процесу формування нестаціонарних електромагнітних 
хвиль у ближній зоні джерела. Нелінійний розв'язок отримано з лінійного 
наближення з використанням елементів теорії збурень та методу еволюційних 
рівнянь. Зовнішню задачу електродинаміки розв'язано для лінзової антени 
імпульсного випромінювання у наближенні плаского кругового рівнофазного 
однонапрямленого розподілу електричного струму. В роботі вперше представлено 
перехідну функцію для такої антени з явною залежністю від часу та 
просторових координат, яка справедлива в довільній точці спостереження.

Отримане аналітичне розв'язання задачі випромінювання у часовій області 
використано для моделювання процесу прийму електромагнітного імпульсу 
та виділення з нього корисної інформації. В роботі представлено авторську 
методику виділення корисної інформації з надширокостугового імпульсу з 
урахуванням залежності його форми від точки спостереження і 
неврахованих слабонелінійних ефектів при поширенні поля. В основі 
запропонованої методики -- топологічно розділена на енодер та декорер 
фізична нейронна мережа з тривалою короткочасною пам'яттю у якості 
структурного елементу.

В роботі використано лише спеціально наукові та загальнонаукові методи.
Текст дисертації складають опубліковані та апробовані матеріали наукових 
досліджень. Матеріал, що викладено у роботі пов'язаний з декількома 
науково-дослідницькими проектами та програмами. Нові наукові результати,
що отримано в роботі не суперечать сучасній науковій доктрині. На 
запропоновану автором методику виділення корисної інформації отримано 
винахідницький патент.

Практично значним результатом кваліфікаційної роботи є методика обробки 
прийнятого надширокосмугового сигналу. Вирази для нелінійного поля, 
доповнені енергетичними діаграми, дозволяють теоретично оцінювати 
необхідність врахування нелінійних ефектів в практичних задачах, де 
застосовуються антени імпульсного випромінювання. Інші результати роботи 
несуть лише фундаментально-наукову цінність -- отримана модель 
випромінювання лінзової антени імпульсного випромінювання може бути 
застосована для моделювання процесу поширення імпульсної 
надширокосмугової електромагнітної хвилі з урахуванням нелінійності 
середовища та ефектів ближньої зони.

\keywords{часова область, електромагнітний імпульс, надширокосмугова 
електродинаміка, Керрівська нелінійність, слабка нелінійність, 
метод еволюційних рівнянь, машинне нявчання, рекурентні нейронні мережі,
тривала короткочасна пам'ять}

\end{abstract}

% \begin{abstract}[language=english, header=true]
% In the thesis, we consider a problem of ...
% \keywords{ultrawidband electrodynamics in time domain, Kerr nonlinerity}
% \end{abstract}

\textcolor{red}{TODO: додати англійський переклад анотації і 
ключових слів}

\textcolor{red}{TODO: згідно нових правил сюди треба скопіювати Список 
публікацій здобувача з останнього додатку.}

% \nocite{Bar98fasp1,Bar98fasp2,PrB01umc}

% \begin{bibset}% [a]
%   {Список публікацій здобувача за~темою~дисертації}
%   % {Список публікацій здобувача}
%   \bibliographystyle{gost2008}
%   %
%   % Якщо не треба нумерація з крапкою, можна закоментувати наступні три рядки.
%   \makeatletter
%   \renewcommand\@biblabel[1]{#1.}
%   \makeatother
%   \bibliography{../my}
% \end{bibset}