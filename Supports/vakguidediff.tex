\documentclass[a4paper]{article}

\usepackage[T2A]{fontenc}
\usepackage[utf8]{inputenc}
\usepackage[ukrainian]{babel}

\usepackage{url}

\newcommand{\vakthesis}{\textsf{vakthesis}}

\begin{document}

\title{Порівняння рекомендацій ВАК України до~оформлення дисертацій 2006 і 2008~років\thanks{Файл доступний за адресою \protect\url{http://www.imath.kiev.ua/~baranovskyi/tex/vakthesis/support/vakguidediff.tex}.}}
\author{Олександр Барановський}
\date{Чернетка від 2009/05/17}

\maketitle

% \tableofcontents

\section{Вступ}

У цьому документі описано відмінності в опублікованих у різні роки
рекомендаціях ВАК України до оформлення дисертацій та авторефератів:
% FIXME: Порівняти доступні довідники інших років: 2001, 2004...
%        Це може бути корисно для програмування інтерфейсу vakthesis.
\begin{description}
\item[2006] Довідник здобувача наукового ступеня: Зб. нормат. док. та
  інформ.  матеріалів з питань атестації наук. кадрів вищої
  кваліфікації~/ Упоряд. Ю.~І.~Цеков; За ред. Р.~В.~Бойка. "--- 3-є
  вид., випр. і допов. "--- К.: Ред. <<Бюл. Вищої
  атестац. коміс. України>>; Вид-во <<Толока>>, 2006. "--- 70~с.

\item[2008] Файл \texttt{res.djvu}, який надіслав Сергій Лисовенко
  2009/04/27.
% FIXME: Додати бібліографічний опис книги.
\end{description}

\section{Помічені відмінності}

\subsection{Оформлення відповідно до державного стандарту}

\begin{description}
\item[2006, с.~14] <<\ldots{}дисертації\ldots{} оформлювати відповідно
  до державного стандарту України. Таким стандартом є ДСТУ~3008-95
  ,,Документація. Звіти у сфері науки і техніки. Структура і правила
  оформлення``>>.

\item[2008, с.~46] Не вказано конкретно, якого стандарту необхідно
  дотримуватися: <<\ldots{}дисертації\ldots{} оформлювати відповідно
  до державного стандарту>>.

\item[Що робити?] Взяти до уваги.
\end{description}

\subsection{Перелік умовних позначень}

\begin{description}
\item[2006, с.~15] Називається <<Перелік умовних позначень, символів,
  скорочень і термінів>>.

\item[2008, с.~46] Називається <<Перелік умовних позначень, символів,
  одиниць, скорочень і термінів>>.

\item[Що робити?] Наразі \vakthesis{} не має ніякої спеціальної
  підтримки для переліку умовних позначень, тому це залишається на
  розсуд користувача.
\end{description}

\subsection{Характеристика дисертації}

\begin{description}
\item[2006, с.~16] Рубрика <<Методи дослідження>> оформлена курсивом,
  тобто є рубрикою нижчого рівня, ніж <<Мета і завдання дослідження>>,
  яка оформлена жирним.

\item[2008, с.~47] Рубрики <<Методи дослідження>> і <<Мета і завдання
  дослідження>> оформлені однаково, тобто є рубриками одного рівня.

\item[2006] Відсутні рубрики <<Обгрунтованість і достовірність
  наукових положень, висновків і рекомендацій>> і <<Наукове значення
  роботи>>.

\item[2008, с.~48] За рубрикою <<Наукова новизна одержаних
  результатів>> ідуть слідом рубрики <<Обгрунтованість і достовірність
  наукових положень, висновків і рекомендацій>> і <<Наукове значення
  роботи>>.

\item[2006, с.~16] Рубрика називається <<Практичне значення одержаних
  результатів>>.

\item[2008, с.~48] Рубрика називається <<Практичне значення отриманих
  результатів>>.

\item[2006] Відсутня рубрика <<Структура дисертації>>.

\item[2008, с.~49] За рубрикою <<Публікації>> іде слідом рубрика
  <<Структура дисертації>>.

\item[Що робити?] ВАК рекомендує, файли-приклади \vakthesis{}
  підказують, але лише автор дисертації вирішує, які рубрики йому
  використовувати. Я~не маю потреби вносити зміни у \vakthesis{}, хіба
  що у файли-приклади. Може, вилучити взагалі відповідний текст з
  прикладів, щоб уникнути змінювати їх, як тільки зміняться вимоги
  ВАК?
\end{description}

\subsection{Список використаних джерел}

\begin{description}
\item[2006, с.~18 (також с.~25)] Рекомендується укладати
  <<\ldots{}одним із таких способів: у порядку появи посилань у тексті
  (найбільш зручний для користування і рекомендований при написанні
  дисертацій), в алфавітному порядку прізвищ перших авторів або
  заголовків, у хронологічному порядку>>. Бібліографічний опис
  складати відповідно до ГОСТ~7.1-84 <<СИБИД. Библиографическое
  описание документа. Общие требования и правила составления>> та
  інших стандартів.

\item[2008, с.~50 (також с.~62)] Рекомендується укладати <<\ldots{}в
  порядку згадування їх у тексті за наскрізною нумерацією>>. Про
  стандарти тут не згадується.

  Але на с.~62 стверджується те саме, що й у 2006, с.~25:
  упорядковувати можна різними способами і для оформлення
  користуватися стандартами ГОСТ~7.1-84 та іншими.

\item[Що робити?] Взяти до уваги, що 2008 суперечить сам собі. У
  будь-якому разі, оформлення списку використаних джерел майже
  повністю покладається на інше програмне забезпечення (Bib\TeX{} чи
  щось подібне) і/або на автора дисертації. Тому я не потребую
  змінювати \vakthesis{}.

\item[Примітка] Дивно, що 2008 рекомендує при оформленні
  бібліографічного опису дотримуватися ГОСТ~7.1-84. Бо файл
  \url{http://www.vak.org.ua/docs//documents/perelik_forms.pdf}
  рекомендує дотримуватися ДСТУ ГОСТ 7.1:2006 <<Система стандартів з
  інформації, бібліотечної та видавничої справи. Бібліографічний
  запис. Бібліографічний опис. Загальні вимоги та правила
  складання>>. Ще одна самосуперечність?
\end{description}

\subsection{Загальні вимоги до оформлення дисертації}

\begin{description}
\item[2006, с.~18] <<Дисертацію друкують машинописним способом або за
  допомогою принтера\ldots{} через два міжрядкових інтервали до
  тридцяти рядків на сторінці>>.

\item[2008, с.~50] <<Дисертацію друкують машинописним способом через
  два міжрядкових інтервали або за допомогою принтера\ldots{}>> Про
  тридцять рядків не згадується.

\item[2006, с.~19] <<Обсяг основного тексту дисертації на здобуття
  наукового ступеня доктора наук має становити 11--13~авторських
  аркушів\ldots{}>>

\item[2008, с.~50] <<Обсяг дисертації на здобуття наукового ступеня
  доктора наук становить 250--300~сторінок або 11--13~авторських
  аркушів\ldots{}>>

\item[2006, с.~19] <<Текст дисертації друкують, залишаючи поля таких
  розмірів: ліве "--- не менше 20~мм, праве "--- не менше 10~мм,
  верхнє "--- не менше 20~мм, нижнє "--- не менше 20~мм>>.

\item[2008, с.~51] <<Текст дисертації друкують, залишаючи, зазвичай,
  береги таких розмірів: лівий "--- 30~мм, правий "--- 10~мм, верхній
  "--- 20~мм, нижній "--- 20~мм>>.

\item[Що робити?] Вибір параметрів сторінки (береги, кількість рядків
  на сторінці тощо) покладається на автора дисертації (це можна
  робити, наприклад, за допомогою пакета \verb|geometry|). Тому я не
  потребую змінювати \vakthesis{}.

  Ймовірно, під <<обсягом основного тексту дисертації>> і <<обсягом
  дисертації>> розуміють одне й те саме. Бо далі говориться про
  <<загальний обсяг дисертації>>, до якого не входять додатки, список
  використаних джерел і~т.~д. Складається враження, що це те саме. У
  будь-якому разі, це не вимагає ніяких змін у \vakthesis{}.

  Зазначення граничного обсягу дисертації у сторінках може бути
  корисним для автора дисертації. Бо хто знає, що таке авторський
  аркуш?
\end{description}

\subsection{Ілюстрації}

\begin{description}
\item[2006, с.~20, 21] <<Ілюстрації позначають словом ,,Рис.``,
  ,,Мал.``\ldots{}>>

\item[2008, с.~52, 55] <<Ілюстрації позначають словом
  ,,Рис.``\ldots{}>>

\item[Що робити?] \vakthesis{} використовує слово <<Рис.>>. Щоб
  змінити, користувач може переозначити \verb|\figurename|:
\begin{verbatim}
\addto\captionsukrainian{\def\figurename{Мал.}}
\end{verbatim}
% FIXME: Зробити відповідне зауваження в документації.
\end{description}

\subsection{Таблиці}

\begin{description}
\item[2006, с.~20, 22] \textit{Таблиця} (курсив),
  \textit{Продовж. табл.} (курсив).

\item[2008, с.~52, 58] Таблиця (прямий), Продовження табл. (прямий).

\item[2006, с.~22] У прикладі побудови таблиці <<\textit{Таблиця}
  (номер)>> (без крапки).

\item[2008, с.~58] У прикладі побудови таблиці <<Таблиця (номер).>> (з
  крапкою).

\item[2006, с.~22] <<Назву наводять жирним шрифтом>>.

\item[2008, с.~58] <<Назву не підкреслюють>>.

\item[Що робити?] Переозначити команди \verb|\tablenamefont| і
  \verb|\tablecaptionfont|, щоб змінити оформлення заголовка
  таблиці. Для 2006 (\vakthesis{} v0.08 робить так):
\begin{verbatim}
\let\tablenamefont\itshape
\let\tablecaptionfont\bfseries
\end{verbatim}
Для 2008:
\begin{verbatim}
\let\tablenamefont\upshape
\let\tablecaptionfont\mdseries
\end{verbatim}
% FIXME: Поправити відповідне місце в документації, там невдало написано.

Відмінності у написанні словосполучення <<Продовження таблиці>> не
спонукають змінювати \vakthesis{}. Користувач мусить вказувати
відповідне словосполучення у тексті, якщо використовує, наприклад,
оточення \verb|longtable|. Неможливо зробити це автоматично.

Переозначити команди \verb|\@maketablecaption| і
\verb|\LT@makecaption|, щоб додати крапку після номера таблиці:
\begin{verbatim}
\makeatletter
\long\def\@maketablecaption#1#2{%
  \vskip\belowcaptionskip
  \raggedleft#1\TableCaptionSeparator\par% <--- змінено
  \sbox\@tempboxa{\tablecaptionfont#2}%
  \ifdim \wd\@tempboxa >\hsize
    \centering\tablecaptionfont#2\par
  \else
    \global \@minipagefalse
    \hb@xt@\hsize{\hfil\box\@tempboxa\hfil}%
  \fi
  \vskip\abovecaptionskip}
\AtBeginDocument{%
  \@ifpackageloaded{longtable}{%
    \def\LT@makecaption#1#2#3{%
      \LT@mcol\LT@cols c{\hbox to\z@{\hss\parbox[t]\LTcapwidth{\normalsize
        \raggedleft#1{#2}\TableCaptionSeparator\par% <--- змінено
        \sbox\@tempboxa{\tablecaptionfont#3}%
        \ifdim\wd\@tempboxa>\hsize
          \centering\tablecaptionfont#3%
        \else
          \hbox to\hsize{\hfil\box\@tempboxa\hfil}%
        \fi
        \endgraf\vskip.5\baselineskip}%
      \hss}}}%
  }{}
}
\makeatother
\end{verbatim}
і означити відокремлювач:
\begin{verbatim}
\newcommand{\TableCaptionSeparator}{.}
\end{verbatim}

Але мені це не подобається. Родо"=нумераційний заголовок і тематичний
заголовок таблиці розміщені на різних рядках, тому відокремлювати їх
ще й крапкою здається зайвим і некрасивим. Тим більше, що рекомендації
ВАК явно не вказують ставити крапку, це може бути просто друкарська
хиба у прикладі. Тому я схиляюся до того, щоб у наступній версії
додати команду \verb|\TableCaptionSeparator|, але залишити на розсуд
користувача означувати (чи не означувати) її як крапку.
\end{description}

\subsection{Примітки}

\begin{description}
\item[2006, с.~21] Примітки, примітка (в розбивку). Примітки мають
  додатковий горизонтальний відступ порівняно зі словом <<Примітки>>.

\item[2008, с.~52] Примітки, примітка (без розбивки). Примітки на
  одному рівні зі словом <<Примітки>>.

\item[Що робити?] Наразі \vakthesis{} не має ніякої спеціальної
  підтримки для приміток, тому це залишається на розсуд користувача.
\end{description}

\subsection{Додатки}

\begin{description}
\item[2006, с.~26] <<Додатки слід позначати послідовно великими
  літерами української абетки, за винятком літер Г, Є, І, Ї, Й, О, Ч,
  Ь\ldots{}>>

\item[2008, с.~63] <<Додатки слід позначати послідовно великими
  літерами української абетки, за винятком літер Ґ (чи Г? "--- О.Б.),
  Є, І, Ї, Й, О, Ч, Ь\ldots{}>>

\item[Примітка] На скані 2008 незрозуміло, яка літера написана: Ґ чи
  Г.
% FIXME: Запитати Сергія Лисовенка і виправити.
\end{description}

\subsection{Автореферат}

Порівняння ще не зроблено.

\section{Висновки}

Відмінності дрібні, інколи нелогічні. Зрозуміти, чому саме такі зміни
зроблені, і передбачити, які зміни можуть бути зроблені у майбутньому,
здається, неможливо. Тому налаштовувати нову версію \vakthesis{} на
нові рекомендації ВАК вважаю неефективним. Необхідно створити
інтерфейс, що дозволить перемикати \vakthesis{} на використання
рекомендацій різних років і легко додавати нові зміни у
майбутньому. Наприклад, через опції класу вигляду
\verb|vakguide=2008|.

Необхідно порівняти також з доступними довідниками ВАК інших років,
щоб мати більше інформації про можливі відмінності, підтримку яких має
забезпечувати \vakthesis{}.

\end{document}
