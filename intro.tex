\chapter*{Вступ}

%%%%%%%%%%%%%%%%%%%%%%%%%%%%%%%%%%%%%%%%%%%%%%%%%%%%%%%%%%%%%%%%%%%%%%%%%%%%%%%
\paragraph{Обґрунтування вибору теми дослідження}

Дисертаційну роботу присвячено дослідженню процесів випромінювання, поширення 
й приймання нестаціонарних електромагнітних хвиль. Поширення хвиль досліджено 
з урахуванням їхньої нелінійної взаємодії із середовищем. 

У ролі джерела поля розглянуто лінзову антену імпульсного випромінювання 
з круговою апертурою. Хоча антена й знайшла широке застосування в задачах 
метрології та зондування, її характеристики, особливо в ближній зоні, 
ще недостатньо вивчено.

Відомо, що напрямленість лінзових антен разом з ефектом електромагнітного 
снаряда призводить до концентрації енергії в ближній зоні випромінювача. Це 
явище викликає необхідність враховувати нелінійну взаємодію поля із середовищем 
поширення при значній амплітуді та крутизні фронту збуджувального імпульсу.

Отже, створення моделі випромінювання імпульсів довільної 
форми лінзовою антеною з круговою апертурою без спрощень на лінійність 
середовища дозволить точніше моделювати випромінювання імпульсів з 
крутими фронтами та великою амплітудою. Уточнена модель процесу 
випромінювання дозволить поліпшити застосування антени в прикладних 
задачах. Серед таких задач -- визначення ефективної площини 
розсіювання та передання інформації з цифровим кодуванням.

Інший фактор, що впливає на ефективність застосування короткоімпульсних
систем, -- це врахування залежності імпульсних характеристик 
надширокосмугових антен від напрямку спостереження, особливо в ближній зоні. 
Виокремлення корисної інформації з електромагнітного імпульсу ускладнено 
залежністю його форми від точки спостереження. Отже, 
розвиток методик виокремлення кількісних та якісних характеристик, що несуть 
корисну інформацію, з нестаціонарного електромагнітного поля дозволить 
покращити робочі характеристики імпульсної радіоелектроніки.

%%%%%%%%%%%%%%%%%%%%%%%%%%%%%%%%%%%%%%%%%%%%%%%%%%%%%%%%%%%%%%%%%%%%%%%%%%%%%%%
\paragraph{Зв'язок роботи з науковими програмами, планами, темами}

Дисертаційну роботу виконано на кафедрі прикладної електродинаміки факультету 
радіофізики, біомедичної електроніки та комп’ютерних систем Харківського 
національного університету імені~В.~Н.~Каразіна відповідно до планів 
науково-дослідних робіт: ``Моделювання та дослідження 
нелінійних нанорозмірних систем із нестаціонарними та гармонійними 
збудженнями для перетворення полів та створення елементів спінтроники'' 
(номер державної реєстрації: 0114U002585, здобувач -- виконавець), 
``Імпульсні та синусоїдальні поля у нелінійних і шаруватих електродинамічних 
структурах та наносистемах як перетворювачах полів і моделей елементів 
спінтроніки'' (номер державної реєстрації: 0117U004851, здобувач -- виконавець),
``Електромагнітні поля імпульсних джерел та наноосциляторів в однорідних, 
шаруватих та нелінійних середовищах'' (номер державної реєстрації: 0120U102309, 
здобувач -- виконавець).

Здобувач у межах цього дослідження пройшов стажування в 
Університеті Мурсії (Іспанія) на факультеті математики за програмою
академічної мобільності для здобувачів ``Erasmus+'' у 2017 році.

%%%%%%%%%%%%%%%%%%%%%%%%%%%%%%%%%%%%%%%%%%%%%%%%%%%%%%%%%%%%%%%%%%%%%%%%%%%%%%%
\paragraph{Мета та завдання дослідження}

Метою роботи є дослідження поведінки нестаціонарного поля в ближній зоні антен, 
зокрема, з урахуванням нелінійної взаємодії із середовищем, а також 
вдосконалення методик обробки прийнятих імпульсних надширокосмугових хвиль за 
рахунок використання особливостей поля ближньої зони. Задачі дослідження:

\begin{enumerate}

\item отримання перехідної функції лінзової антени імпульсного 
випромінювання, як явної функції від просторових координат та часу, що 
справедлива для довільної точки спостереження;

\item аналіз енергетичних та нестаціонарних властивостей поля лінзової антени 
імпульсного випромінювання в ближній зоні для імпульсів різної форми;

\item уточнення перехідної функції лінзової антени імпульсного 
випромінювання для випадку поліноміальної нелінійності середовища;

\item розвиток методики виокремлення корисної інформації з нестаціонарного 
електромагнітного поля шляхом застосування новітніх методів науки аналізу 
даних, що дозволить враховувати ефекти ближньої зони в реальному часі.

\end{enumerate}

Об’єкт дослідження -- електромагнітне поле, що випромінюється 
лінзовою антеною імпульсного випромінювання.

Предмет дослідження -- вплив ефектів ближньої зони випромінювання 
й нелінійних ефектів взаємодії поля із середовищем на часову залежність
електромагнітних імпульсів та їхню інформаційну місткість.

%%%%%%%%%%%%%%%%%%%%%%%%%%%%%%%%%%%%%%%%%%%%%%%%%%%%%%%%%%%%%%%%%%%%%%%%%%%%%%%
\paragraph{Методи дослідження}

Теоретичну основу дисертації становлять наукові праці вітчизняних та 
закордонних дослідників. Методологічну основу дисертації -- 
загальнонаукові та спеціальні наукові методи пізнання, серед яких:

\begin{enumerate}

\item Метод еволюційних рівнянь у часовій області. Метод теоретичної 
радіофізики, що застосовано для зведення системи рівнянь Максвела до 
системи рівнянь відносно скалярних функцій, шляхом неповного розділення 
змінних за методикою Рімана-Вольтера.

\item Метод функції Рімана. Метод розв'язання неоднорідних 
диференціальних рівнянь, що застосовано для розв'язання системи еволюційних рівнянь відносно коефіцієнтів розкладу електромагнітного 
поля за модовим базисом.

\item Метод теорії збурень. Метод лінеаризації математичних задач, що 
застосовано для врахування слабкої нелінійності середовища, яку представлено 
у вигляді поліноміального розкладу вектора поляризації за ступенями 
напруженості електричного поля.

\item Метод зворотного поширення помилки. Метод машинного навчання,
що застосовано для розв'язання задачі тренування, а саме: для 
оптимізації параметрів запропонованої моделі виокремлення корисної 
інформації з надширокосмугового радіосигналу.

\end{enumerate}

%%%%%%%%%%%%%%%%%%%%%%%%%%%%%%%%%%%%%%%%%%%%%%%%%%%%%%%%%%%%%%%%%%%%%%%%%%%%%%%
\paragraph{Наукова новизна отриманих результатів}

Вперше отримано властивості поширення нестаціонарних електромагнітних 
хвиль, породжених круговою апертурою в нелінійному керрівському 
середовищі, що дозволяє оцінити вплив нелінійних ефектів на поодинокий 
імпульс. Проаналізовано вплив перетворення мод у нелінійному середовищі 
при самодії поодинокого надширокосмугового електромагнітного імпульсу та 
встановлено укручення фронту імпульсу на осі випромінювання. 
Врахування нелінійних ефектів проведено на основі аналізу енергетичних 
властивостей поля в ближній зоні, де густина енергії найбільша.

У ході побудови нелінійної моделі випромінювання автором у першому 
наближені побудовано перехідну функцію антен імпульсного випромінювання. 
Тобто, отримано закон випромінювання нестаціонарних хвиль у вигляді 
явної функції всіх просторових координат та часу без наближення 
дальньої зони, що дозволяє розв’язувати задачі моделювання поля
лінзових та рефлекторних антен імпульсного випромінювання  у 
режимі реального часу.

Запропоновано авторську методику виокремлення корисної інформації з 
нестаціонарної імпульсної хвилі, що полягає в застосуванні аналогових 
рекурентних штучних нейронних мереж задля досягнення підвищеної точності 
в ближній зоні. Продемонстровано, що запропонована методика дозволяє 
досягти покращення якості радіоканалу в ближній, проміжній та дальній 
зонах імпульсних антен. Проведено моделювання задачі цифрової 
комунікації за запропонованою методикою обробки прийнятного 
сигналу, де кодування виконано надширокосмуговими наносекундними 
імпульсами різної форми.

%%%%%%%%%%%%%%%%%%%%%%%%%%%%%%%%%%%%%%%%%%%%%%%%%%%%%%%%%%%%%%%%%%%%%%%%%%%%%%%
\paragraph{Практичне значення отриманих результатів}

Дисертаційна робота належить до основних наукових напрямів сучасної 
радіофізики та визначає тенденції її подальшого розвитку. Напрям
досліджень -- теорія випромінювання та приймання нестаціонарних 
нелінійних електромагнітних хвиль.

Створено модель антен імпульсного випромінювання в ближній зоні, що 
покращує результати попередніх досліджень і дозволяє якісніше моделювати 
процес випромінювання та приймання електромагнітних хвиль із застосуванням 
такої антени. Врахування нелінійної природи електромагнітного поля дозволяє
моделювати поширення високоамплітудних надкоротких імпульсів у різноманітних
задачах радіофізики. Описано нові нелінійні ефекти самодії нестаціонарного 
імпульсного поля крізь середовище.

До державного підприємства ``Український інститут інтелектуальної 
власності'' подано заявку на отримання патенту на  винахід 
``Спосіб виокремлення корисної інформації з надширокосмугових (НШС) 
електромагнітних хвиль'' з номером a202004038. Патентним 
повіреним України з реєстраційним номером 464 проведено аналіз
патентоздатності, новизни та технічного рівня об'єкта захисту 
інтелектуальної власності. Матеріали патерну знаходяться на 
стадії кваліфікаційної експертизи. Використання розробленої 
методики дозволяє застосовувати математичні процеси високої 
складності в реальному часі при незначному споживанні енергії. 
Гнучкість методики  сприяє запровадженню моделей економіки 
замкненого циклу при виробництві.

Основні отримані в дисертації результати включено до навчального 
курсу ``Випромінювання надширокосмугових сигналів'', що викладають 
на факультеті радіофізики, біомедичної електроніки та комп’ютерних 
систем студентам IV--V курсів кафедри прикладної електродинаміки
Харківського національного Університету імені~В.~Н.~Каразіна 
Міністерства освіти і науки України.


%%%%%%%%%%%%%%%%%%%%%%%%%%%%%%%%%%%%%%%%%%%%%%%%%%%%%%%%%%%%%%%%%%%%%%%%%%%%%%%
\paragraph{Особистий внесок здобувача}

У працях \cite{my:Telecom2018, my:UKRCON2017, my:UKRCON2019} здобувач провів
аналіз наявних моделей поля лінзової антени імпульсного випромінювання 
при довільному нестаціонарному збудженні та створив модель, що не має недоліків,
які спостерігались в аналогічних дослідженнях. Отриманий розв'язок описує 
поведінку поля в усіх точках простору. Здобувач провів порівняння з 
наявними моделями і не виявив розбіжностей, що свідчать про помилковість.

У працях \cite{my:Vesnik2017-2} побудовано поперечні розподіли енергії, а
в працях] \cite{my:Vesnik2018, my:Vesnik2018-2} теоретично обґрунтовано енергетичні 
згустки, що спостерігаються.

У працях \cite{my:Vesnik2015, my:Vesnik2017, my:Vesnik2017-2, my:MMET2014, 
my:UWBUSIS2014, my:ICATT2015, my:UWBUSIS2016, my:KPI2016, my:DIPED2019} 
здобувач провів аналітичну роботу щодо врахування нелінійної взаємодії 
нестаціонарного електромагнітного поля з тривимірним середовищем, 
подібним за нелінійними властивостями до атмосфери землі. Також здобувач
провів числове моделювання процесу нелінійної взаємодії та проаналізував 
його результати.

Здобувач запропонував \cite{my:UWBUSIS2018} і теоретично обґрунтував
\cite{my:UKRCON2020} авторську методику виокремлення корисної інформації 
з нестаціонарної електромагнітної хвилі. Здобувач провів числове 
моделювання \cite{my:TKEA2020}, що демонструє працездатність 
та переваги нової концепції.

%%%%%%%%%%%%%%%%%%%%%%%%%%%%%%%%%%%%%%%%%%%%%%%%%%%%%%%%%%%%%%%%%%%%%%%%%%%%%%%
\paragraph{Апробація матеріалів дисертації}

Основні результати, що складають дисертацію, були представлено 
на конференціях і семінарах міжнародного рівня:

\begin{enumerate}

	\item 15th International Conference on Mathematical Methods in 
	Electromagnetic Theory (MMET) (Dnipropetrovsk, 2014);

	\item Proc. 7th International Conference on Ultrawideband and 
	Ultrashort Impulse Signals (UWBUSIS), Kharkiv, Ukraine, 2015 (автор отримав 
	винагороду за кращу доповідь секції);

	\item International Conference on Antenna Theory and Techniques
	(ICATT), Kharkiv, Ukraine, 2015;

	\item Ultrawideband and Ultrashort Impulse Signals 
	(UWBUSIS), Odessa, Ukraine, 2016;

	\item Radio Electronics and Info Communications, Kiev, Ukraine, 2016;

	\item 2017 IEEE First Ukraine Conference on Electrical and Computer 
	Engineering (UKRCON), Kiev, Ukraine, 2017;

	\item Ultrawideband and Ultrashort Impulse Signals (UWBUSIS), 
	Odessa, Ukraine, 2018 (робота зайняла 1-ше місце на конкурсі 
	робіт молодих вчених);

	\item 2017 IEEE 2nd Ukraine Conference on Electrical and Computer 
	Engineering (UKRCON), Lviv, Ukraine, 2019;

	\item 2019 XXIVth International Seminar/Workshop on Direct and Inverse
	Problems of Electromagnetic and Acoustic Wave Theory (DIPED) Lviv, 
	Ukraine, 2019;
	
	\item 2020 IEEE Ukrainian Microwave Week (UkrMW) Kharkiv, Ukraine, 2020.
	
\end{enumerate} 

\paragraph{Публікації}

Матеріали дисертації опубліковано у 17 наукових працях, серед яких -- 7
статей у наукових фахових виданнях (зокрема з них 1 входить, що входять 
до наукометричної бази даних Scopus), і 9 тез доповідей на конференціях 
міжнародного рівня.

%%%%%%%%%%%%%%%%%%%%%%%%%%%%%%%%%%%%%%%%%%%%%%%%%%%%%%%%%%%%%%%%%%%%%%%%%%%%%%%
\paragraph{Обсяг і структура дисертації}

Дисертація складається зі вступу, 4 розділів, висновків, списку використаної 
літератури та 4 додатків. Загальний обсяг дисертації становить 165 сторінок, 
з яких 118 сторінок основного тексту. Список використаної літератури на 13 
сторінках містить 122 найменування. Усього в дисертації 42 рисунки та 
2 таблиці.

%%%%%%%%%%%%%%%%%%%%%%%%%%%%%%%%%%%%%%%%%%%%%%%%%%%%%%%%%%%%%%%%%%%%%%%%%%%%%%%
% \paragraph{Подяка}

% Проведення наукового дослідження, а також написання цієї кваліфікаційної роботи
% було б неможливе без заохочення і фінансової підтримки, що було мені надано.

% Я з гордістю висловлюю подяку колись моїм викладачам, а тепер і колегам

% На різних етапах мого дослідження долучались...

% заохочення і неоціненний інтелектуальний внесок

% Подяка контреб'юторам проекту Maxwell

% Моїй сім'ї -- найдорожчим людям, без яких я -- ніхто.

