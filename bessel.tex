\chapter{Циліндрична функція Бесселя першого роду}
\label{ch:bessel}

%%%%%%%%%%%%%%%%%%%%%%%%%%%%%%%%%%%%%%%%%%%%%%%%%%%%%%%%%%%%%%%%%%%%%%%%%%%%%%%
\section{Визначення на лінійні властивості}
%
\begin{equation}
J_{-n} \left( z \right) = \left( -1 \right)^n J_n \left( z \right)
\end{equation}
%
\begin{equation} \label{eq:bessel_order_change}
J_{n+1} \left( z \right) + J_{n-1} \left( z \right) = 
\frac{2n}{z} J_n \left( z \right)
\end{equation}

%%%%%%%%%%%%%%%%%%%%%%%%%%%%%%%%%%%%%%%%%%%%%%%%%%%%%%%%%%%%%%%%%%%%%%%%%%%%%%%
\section{Асимптотичні властивості}
%
\begin{equation} \label{eq:limJ1toZ}
\lim_{z \to 0} \left. \frac{J_1 \left( z \right)}{z} \right. = \frac{1}{2}
\end{equation}

%%%%%%%%%%%%%%%%%%%%%%%%%%%%%%%%%%%%%%%%%%%%%%%%%%%%%%%%%%%%%%%%%%%%%%%%%%%%%%%
\section{Інтегродиференціальні властивості}
%
\begin{equation}
2 \derivat{}{z} J_n \left( z \right) = 
J_{n-1} \left( z \right) - J_{n+1} \left( z \right) 
\end{equation}
%
\begin{equation}
\derivat{}{z} J_n \left( z \right) = 
J_{n-1} \left( z \right) - \frac{n}{z} J_{n} \left( z \right) 
\end{equation}
%
\begin{equation}
\derivat{}{z} J_n \left( z \right) = 
\frac{n}{z} J_{n} \left( z \right) - J_{n+1} \left( z \right) 
\end{equation}
%
\begin{equation}
\derivat{}{z} \frac{ J_n \left( z \right) }{ z^n }  = 
- \frac{ J_{n+1} \left( z \right) }{ z^n }
\end{equation}
%
\begin{equation}
\derivat{}{z} \left( z^n J_n \left( z \right) \right)  = 
z^n J_{n-1} \left( z \right)
\end{equation}

%%%%%%%%%%%%%%%%%%%%%%%%%%%%%%%%%%%%%%%%%%%%%%%%%%%%%%%%%%%%%%%%%%%%%%%%%%%%%%%
\section{Інтеграл 1} \label{sec:i1anal}
%
\begin{equation} \label{eq:int1start}
I_1 = R \int\limits_{0}^{\infty} \frac{d\nu}{\rho \nu} 
J_1 \left( \nu R \right) J_1 \left( \nu \rho \right) 
J_0 \left( \nu \sqrt{c^2 t^2 - z^2} \right)
\end{equation}

Інтеграли такого виду зустрічаються в \cite[ст. 398]{imp:Watson1922}.
\begin{equation} \begin{aligned} \label{eq:intJJJtable}
\int\limits_{0}^{\infty} \frac{d t}{t^{\lambda + \nu}} 
J_\mu \left( at \right) J_\nu \left( bt \right) J_\nu \left( ct \right) =
\frac{ \left( bc/2 \right) ^\nu }
{ \Gamma \left( \nu + 1/2 \right) \Gamma \left( 1/2 \right) } \cdot \\
\cdot \int\limits_{0}^{\infty} \int\limits_{0}^{\pi}
\frac{J_\mu \left( at \right) J_\nu \left( \omega t \right)}
{\omega^\nu t^\lambda} \sin^{2\nu}{\phi} d\phi dt, \\
\omega = \sqrt{b^2 + c^2 - 2bc \cos \phi} \\
\Re \left( \nu \right) > - \frac{1}{2};
\Re \left( \mu + \nu + 2 \right) > \Re \left( \lambda + 1 \right) > 0
\end{aligned} \end{equation}
%
\textcolor{blue}{ \begin{equation*} \begin{aligned}
a = \sqrt{c^2 t^2 - z^2}; b = R; c = \rho; \lambda = 0 \\
\nu = 1; \mu = 0; \omega = \sqrt{R^2 + \rho^2 - 2 \rho R \cos \phi} \\
\int\limits_{0}^{\infty} \frac{d\nu}{\nu} 
J_1 \left( \nu R \right) J_1 \left( \nu \rho \right) 
J_0 \left( \nu \sqrt{c^2 t^2 - z^2} \right) = 
\frac{R^2}{ 2 \Gamma \left( 3/2 \right) \Gamma \left( 1/2 \right) } \cdot \\
\int\limits_{0}^{\pi} 
\frac{\sin^2{\phi}}{\sqrt{R^2 + \rho^2 - 2 \rho R \cos \phi}}
\int\limits_{0}^{\infty} d \nu J_1 \left( \nu \omega \right) 
J_0 \left( \nu \sqrt{c^2 t^2 - z^2} \right) d \phi
\end{aligned} \end{equation*} }
%
\textcolor{blue}{ \begin{equation*} \begin{aligned}
\Gamma \left( 3/2 \right) \Gamma \left( 1/2 \right) = 
\frac{\sqrt{\pi}}{2} \cdot \sqrt{\pi} = \frac{\pi}{2} 
\end{aligned} \end{equation*} }
%
\textcolor{blue}{ \begin{equation*} \begin{aligned}
I_1 = \frac{R^2}{\pi} \int\limits_{0}^{\pi} 
\frac{\sin^2{\phi}}{\sqrt{R^2 + \rho^2 - 2 \rho R \cos \phi}}
\int\limits_{0}^{\infty} d \nu J_1 \left( \nu \omega \right) 
J_0 \left( \nu \sqrt{c^2 t^2 - z^2} \right) d \phi
\end{aligned} \end{equation*} }

Використання формули \eqref{eq:intJJJtable} дозволяє спростити $ I_1 $ до 
інтегралу по двом функціям Бесселя в ядрі замість трьох. Використаємо наступну 
формулу з \cite{imp:Golubovic2013} для пошуку рішення нового інтегралу. 
%
\begin{equation} \begin{aligned} \label{eq:intJJtable}
\int\limits_{0}^{\infty} d \nu
J_n \left( a \nu \right) J_{n-1} \left( b \nu \right) = \begin{cases} 
b^{n-1} / a^n , 0 < b < a \\
1 / 2 b , 0 < a = b \\
0 , 0 < a < b
\end{cases} 
\end{aligned} \end{equation}
%
\textcolor{blue}{ \begin{equation*} \begin{aligned}
\int\limits_{0}^{\infty} d \nu J_1 \left( \nu \omega \right) 
J_0 \left( \nu \sqrt{c^2 t^2 - z^2} \right) = \begin{cases}
\left( R^2 + \rho^2 - 2 \rho R \cos \phi \right)^{-1/2}, 0 < b < a \\
\frac{1}{2} \left( c^2 t^2 - z^2 \right)^{-1/2}, 0 < a = b \\
0 , 0 < a < b
\end{cases} 
\end{aligned} \end{equation*} }
%
\textcolor{blue}{ \begin{equation*} \begin{aligned}
\sqrt{R^2 + \rho^2 - 2 \rho R \cos \phi} > \sqrt{c^2 t^2 - z^2} \\
R^2 + \rho^2 - 2 \rho R \cos \phi > c^2 t^2 - z^2 \\
\cos \phi < \frac{R^2 + \rho^2}{2 \rho R} - \frac{c^2 t^2 - z^2}{2 \rho R} \\
\phi > \arccos \left( \frac{\rho^2 + R^2}{2 \rho R} - 
\frac{c^2 t^2 - z^2}{2 \rho R} \right), 0 \leq \phi \leq \pi
\end{aligned} \end{equation*} }
%
\textcolor{blue}{ \begin{equation*}
\phi > \arccos \left( \frac{\rho^2 + R^2}{2 \rho R} - 
\frac{c^2 t^2 - z^2}{2 \rho R} \right)
\end{equation*} }
%
\textcolor{blue}{ \begin{equation*} \begin{aligned}
\begin{cases}
\frac{\rho^2 + R^2}{2 \rho R} - \frac{c^2 t^2 - z^2}{2 \rho R} \leq 1 \\
\frac{\rho^2 + R^2}{2 \rho R} - \frac{c^2 t^2 - z^2}{2 \rho R} \geq - 1
\end{cases}
\begin{cases}
\rho^2 + R^2 - c^2 t^2 + z^2 \leq 2 \rho R \\
\rho^2 + R^2 - c^2 t^2 + z^2 \geq - 2 \rho R
\end{cases}
\end{aligned} \end{equation*} }
%
\textcolor{blue}{ \begin{equation*} \begin{aligned}
\begin{cases}
0 \leq \left( R - \rho \right)^2 \leq c^2 t^2 - z^2 \\ 
\left( \rho + R \right)^2 \geq c^2 t^2 - z^2 \geq 0
\end{cases}
\begin{cases}
0 \leq R \leq \rho + \sqrt{c^2 t^2 - z^2} \\
R \geq \left| \rho - \sqrt{c^2 t^2 - z^2} \right| \geq 0
\end{cases}
\begin{cases}
R \leq f_+(r,t) \\
R \geq \left| f_-(r,t) \right|
\end{cases} 
\end{aligned} \end{equation*} }
%
\begin{equation*} \begin{aligned}
I_1 \in \begin{cases}
S_1: \{ 0 \leq \phi \leq \psi \}, 0 < R < 
\left| \rho - \sqrt{c^2 t^2 - z^2} \right| \\
S_2: \{ \psi \leq \phi \leq \pi \}, \left| \rho - \sqrt{c^2 t^2 - z^2} \right| \leq 
R \leq \rho + \sqrt{c^2 t^2 - z^2} \\
S_3: \{ 0 \leq \phi \leq \pi \}, R > \rho + \sqrt{c^2 t^2 - z^2}
\end{cases} 
\end{aligned} \end{equation*}
%
\begin{equation*} \begin{aligned}
I_1 \{ S_1 \} = 0
\end{aligned} \end{equation*}
%
\textcolor{blue}{ \begin{equation*} \begin{aligned}
I_1 = \frac{R^2}{\pi} \int\limits_{0}^{\pi} 
\frac{\sin^2{\phi}}{\sqrt{R^2 + \rho^2 - 2 \rho R \cos \phi}}
\int\limits_{0}^{\infty} d \nu J_1 \left( \nu \omega \right) 
J_0 \left( \nu \sqrt{c^2 t^2 - z^2} \right) d \phi = \\
= \frac{R^2}{\pi} \int_{\psi}^{\pi}
\frac{\sin^2{\phi}}{\sqrt{R^2 + \rho^2 - 2 \rho R \cos \phi}}
\frac{1}{\sqrt{R^2 + \rho^2 - 2 \rho R \cos \phi}} d \phi = \\
= \frac{R^2}{\pi} \int_{\psi}^{\pi}
\frac{\sin^2{\phi}}{R^2 + \rho^2 - 2 \rho R \cos \phi} d \phi = 
\frac{1}{\pi} \int_{\psi}^{\pi}
\frac{\sin^2{\phi}}{1 + \frac{\rho^2}{R^2} - \frac{2 \rho}{R} \cos \phi} d \phi
\end{aligned} \end{equation*} }

\textcolor{red} {Але, згідно властивостей інтегралів Рімана, значення інтегралу 
в одній точці не впливає на значення означеного інтегралу}
%
\begin{equation*} \begin{aligned}
I_1 = \frac{1}{\pi} \int_{\psi}^{\pi}
\frac{\sin^2{\phi}}{1 + \frac{\rho^2}{R^2} - 
\frac{2 \rho}{R} \cos \phi} d \phi \\
\psi = \arccos \left( \frac{\rho^2 + R^2}{2 \rho R} - 
\frac{c^2 t^2 - z^2}{2 \rho R} \right)
\end{aligned} \end{equation*}
%
\textcolor{blue}{ \begin{equation*} \begin{aligned}
\int \frac{\sin^2{\phi}}{a + b \cos \phi} d \phi = 
\int \frac{1 - \cos^2{\phi}}{a + b \cos \phi} d \phi = 
\int\frac{d \phi}{a + b \cos \phi}  -
\int \frac{\cos^2{\phi}}{a + b \cos \phi} d \phi = \\
= \int \frac{d \phi}{a + b \cos \phi}  - 
\int \frac{\cos^2{\phi}}{a + b \cos \phi} d \phi -
\frac{a}{b} \int \frac{\cos \phi}{a + b \cos \phi} d \phi + \\
+ \frac{a}{b} \int \frac{\cos \phi}{a + b \cos \phi} d \phi = 
\int \frac{d \phi}{a + b \cos \phi} +
\frac{a}{b} \int \frac{\cos \phi}{a + b \cos \phi} d \phi - \\
- \int \frac{\cos^2{\phi} + \frac{a}{b} \cos \phi} {a + b \cos \phi} d \phi =
\int \frac{d \phi}{a + b \cos \phi} + 
\frac{a}{b} \int\limits_{0}^{\psi} \frac{\cos \phi}{a + b \cos \phi} d \phi -
\end{aligned} \end{equation*} }
%
\textcolor{blue}{ \begin{equation*} \begin{aligned}
- \frac{1}{b} \int \frac{\cos \phi + a/b} {a/b +  \cos \phi} \cos \phi d \phi = 
\int \frac{d \phi}{a + b \cos \phi} + 
\frac{a}{b} \int \frac{\cos \phi}{a + b \cos \phi} d \phi - \\
- \frac{1}{b} \int \cos \phi d \phi = \int \frac{d \phi}{a + b \cos \phi} - 
\frac{1}{b} \int \cos \phi d \phi + \frac{a}{b^2} \int
\frac{a - a + b \cos \phi}{a + b \cos \phi} d \phi = \\ 
= \int\frac{d \phi}{a + b \cos \phi} - \frac{1}{b} \int \cos \phi d \phi +
\frac{a}{b^2} \int \frac{a + b \cos \phi}{a + b \cos \phi} d \phi - \\ 
- \frac{a^2}{b^2} \int \frac{d \phi}{a + b \cos \phi} = 
\left( 1 - \frac{a^2}{b^2} \right) \int\frac{d \phi}{a + b \cos \phi} - 
\frac{1}{b} \int \cos \phi d \phi + \frac{a}{b^2} \int d \phi
\end{aligned} \end{equation*} }
%
\textcolor{blue}{ \begin{equation*} \begin{aligned}
\int_{\psi}^{\pi} \frac{\sin^2{\phi}}{a + b \cos \phi} d \phi =  
\left( 1 - \frac{a^2}{b^2} \right)
\int_{\psi}^{\pi} \frac{d \phi}{a + b \cos \phi} -
\frac{\sin \pi - \sin \psi}{b} + \frac{a}{b^2} (\pi - \psi) = \\
= \left( 1 - \frac{a^2}{b^2} \right)
\int_{\psi}^{\pi} \frac{d \phi}{a + b \cos \phi} +
\frac{\sin \psi}{b} + \frac{a}{b^2} (\pi - \psi)
\end{aligned} \end{equation*} }
%
\textcolor{blue}{ \begin{equation*} \begin{aligned}
a = 1 + \frac{\rho^2}{R^2}; b = - \frac{2 \rho}{ R } \\
\frac{\pi R}{\rho} I_1 = \left( 1 - \frac{a^2}{b^2} \right)
\int_{\psi}^{\pi} \frac{d \phi}{a + b \cos \phi} +
\frac{\sin \psi}{b} + \frac{a}{b^2} (\pi - \psi) = \\
= \left( 1 - \left( \frac{1 + \frac{\rho^2}{R^2}} 
{ \frac{2 \rho}{R} } \right)^2 \right) 
\int_{\psi}^{\pi} \frac{d \phi}{1 + \frac{\rho^2}{R^2} -  
\frac{2 \rho}{R} \cos \phi} -
\frac{\sin \psi}{\frac{2 \rho}{ R }} + \left( 1 + \frac{\rho^2}{R^2} \right) 
\frac{\pi - \psi}{\frac{4 \rho^2}{R^2}} = \\
\left( R^2 - \left( \frac{R^2 + \rho^2}{2 \rho} \right)^2 \right) 
\int_{\psi}^{\pi} \frac{d \phi}{R^2 + \rho^2 - 2 \rho R \cos \phi} -
\frac{R}{2 \rho} \sin \psi + \frac{\rho^2 + R^2}{4 \rho^2} (\pi - \psi)  
\end{aligned} \end{equation*} }
%
\textcolor{blue}{ \begin{equation*} \begin{aligned}
\frac{4 \rho^2}{4 \rho^2} R^2 - \left( \frac{R^2 + \rho^2}{2 \rho} \right)^2 =
\frac{4 \rho^2 R^2 - R^4 - 2 \rho^2 R^2 - \rho^4}{4 \rho^2} =
- \frac{\left( \rho^2 - R^2 \right)^2}{4 \rho^2} 
\end{aligned} \end{equation*} }
%
\textcolor{blue}{ \begin{equation*} \begin{aligned}
\pi I_1 (S_2) = - \frac{\left( \rho^2 - R^2 \right)^2}{4 \rho^2} 
\int_{\psi}^{\pi} \frac{d \phi}{R^2 + \rho^2 - 2 \rho R \cos \phi} - \\
- \frac{R}{2 \rho} \sin \psi + \frac{\rho^2 + R^2}{4 \rho^2}  (\pi - \psi)
\end{aligned} \end{equation*} }

Тригонометричними перетвореннями зведемо поточний вид $ I_1 $ до табличного 
інтегралу.
%
\begin{equation*} \begin{aligned}
I_{1} \{ S_2 \} = - \frac{\left( \rho^2 - R^2 \right)^2}{4 \pi \rho^2} 
\int_{\psi}^{\pi} \frac{d \phi}{R^2 + \rho^2 - 2 \rho R \cos \phi} - 
\frac{R}{\rho} \frac{\sin \psi}{2 \pi} +  
\frac{\rho^2 + R^2}{4 \rho^2} \frac{\pi - \psi}{\pi}
\end{aligned} \end{equation*}
%
\textcolor{blue}{ \begin{equation*} \begin{aligned}
\int_{0}^{\pi} \frac{\sin^2{\phi}}{a + b \cos \phi} d \phi =  
\left( 1 - \frac{a^2}{b^2} \right)
\int_{0}^{\pi} \frac{d \phi}{a + b \cos \phi} -
\frac{\sin \pi - \sin 0}{b} + \frac{a}{b^2} (\pi - 0) = \\
= \left( 1 - \frac{a^2}{b^2} \right)
\int_{0}^{\pi} \frac{d \phi}{a + b \cos \phi} +
\frac{a}{b^2} \pi
\end{aligned} \end{equation*} }
%
\begin{equation*} \begin{aligned}
I_{1} \{ S_3 \} = - \frac{\left( \rho^2 - R^2 \right)^2}{4 \pi \rho^2} 
\int_{0}^{\pi} \frac{d \phi}{R^2 + \rho^2 - 2 \rho R \cos \phi} + 
\frac{\rho^2 + R^2}{4 \rho^2}
\end{aligned} \end{equation*}

Таблична формула для неозначеного випадку інтегралу може буде знайдена в 
\cite[ст. 181]{imp:ElementFunc1983}.
%
\begin{equation} \label{eq:caseTableIntegral}
\int \frac{d x}{a + b \cos{x}} = \begin{cases}
\frac{2}{\sqrt{a^2-b^2}} \arctan \frac{\sqrt{a^2-b^2} \tan \frac{x}{2}}
{a + b}, a^2 > b^2 \\
\frac{1}{\sqrt{b^2-a^2}} \ln 
\frac{\sqrt{b^2-a^2} \tan \frac{x}{2} + a + b}
{\sqrt{b^2-a^2} \tan \frac{x}{2} - a - b}, a^2 < b^2
\end{cases}
\end{equation}

Помітимо, що випадок $ a^2 > b^2 $ відповідає області $ \rho > R $, a 
$ a^2 < b^2 $, навпаки, для прожекторної зони випромінювання.
%
\textcolor{blue}{ \begin{equation*} \begin{aligned}
a^2 > b^2  \Rightarrow  
\left( R^2 + \rho^2 \right)^2 > 4 \rho^2 R^2 \\
R^4 + 2 \rho^2 R^2 + \rho^4 - 4 \rho^2 R^2 > 0 \Rightarrow 
\left( \rho^2 - R^2 \right)^2 > 0
\end{aligned} \end{equation*} }
%
\textcolor{blue}{ Далі знадобиться: }
%
\textcolor{blue}{ \begin{equation*} \begin{aligned}
a^2 - b^2 = - \left( b^2 - a^2 \right) = 
R^4 + 2 \rho^2 R^2 + \rho^4 - 4 \rho^2 R^2 = \left( \rho^2 - R^2 \right)^2 \\
\lim_{\alpha \to 0} \tan{\alpha} = 0 \Rightarrow
\lim_{\alpha \to 0} \arctan \left( a \tan{\alpha} \right) = 0 \\ 
\lim_{\alpha \to \pi/2} \tan{\alpha} = \infty \Rightarrow
\lim_{\alpha \to \pi/2} \arctan \left( a \tan{\alpha} \right) = \frac{\pi}{2}
\end{aligned} \end{equation*} }
%
\textcolor{blue}{ \begin{equation*} \begin{aligned}
\int_{\psi}^{\pi} \frac{d \phi}{R^2 + \rho^2 - 2 \rho R \cos \phi} =
\left. \frac{2}{ |\rho^2 - R^2| } \arctan \left( \frac{ |\rho^2 - R^2| }
{\left( \rho - R \right)^2} \tan \frac{\phi}{2} \right)
\right|_{\psi}^{\pi} = \\ = \frac{2}{ |\rho^2 - R^2| } \left.
\arctan \left( \frac{\rho + R}{ |\rho - R| } \tan \frac{\phi}{2} \right)
\right|_{\psi}^{\pi} = \\ = \frac{2}{ |\rho^2 - R^2| } \left( \frac{\pi}{2} -
\arctan \left( \frac{\rho + R}{ |\rho - R| } \tan \frac{\psi}{2} \right) \right)
\end{aligned} \end{equation*} }
%
\begin{equation*} \begin{aligned}
I_1 \{ S_2 \} = \frac{ | \rho^2 - R^2 | }{2 \pi \rho^2} \left(
\arctan \left( \frac{\rho + R}{ | \rho - R | } \tan \frac{\psi}{2} \right) -  
\frac{\pi}{2} \right) - \frac{R}{\rho} \frac{\sin \psi}{2 \pi} + 
\frac{\rho^2 + R^2}{4 \rho^2} \frac{\pi - \psi}{\pi}
\end{aligned} \end{equation*}
%
\textcolor{blue}{ \begin{equation*} \begin{aligned}
\int_{0}^{\pi} \frac{d \phi}{R^2 + \rho^2 - 2 \rho R \cos \phi} =
\left. \frac{2}{ | \rho^2 - R^2 | } \arctan \left( \frac{ | \rho^2 - R^2 | }
{\left( \rho - R \right)^2} \tan \frac{\phi}{2} \right)
\right|_{0}^{\pi} = \\ = \frac{2}{ | \rho^2 - R^2 | } \left.
\arctan \left( \frac{\rho + R}{ | \rho - R | } \tan \frac{\phi}{2} \right)
\right|_{0}^{\pi} = \frac{2}{ | \rho^2 - R^2 | } \frac{\pi}{2}
\end{aligned} \end{equation*} }
%
\begin{equation*} \begin{aligned}
I_1 \{ S_3 \} = \frac{\rho^2 + R^2}{4 \rho^2} - 
\frac{ |\rho^2 - R^2| }{4 \rho^2} = \begin{cases}
1/2 , \rho < R \\
R^2 / 2 \rho^2, \rho > R
\end{cases}
\end{aligned} \end{equation*}
%
Для того щоб отримати значення інтегралу на осі аплікат повернемось до 
початкового виду $ I_1 $ з \eqref{eq:int1start}. Користуючись асимптотичною 
властивістю функції Бесселя \eqref{eq:limJ1toZ} побачимо що інтеграл 
зведеться до випадку \eqref{eq:intJJtable}.
%
\textcolor{blue} {\begin{equation*} \begin{aligned}
\left. I_1 \right|_{\rho = 0} = R \int\limits_{0}^{\infty} d \nu
J_1 \left( \nu R \right) \frac{J_1 \left( \nu \rho \right) }{\nu \rho}
J_0 \left( \nu \sqrt{c^2 t^2 - z^2} \right) = \\
= \frac{R}{2} \int\limits_{0}^{\infty} d \nu
J_1 \left( \nu R \right) J_0 \left( \nu \sqrt{c^2 t^2 - z^2} \right) = 
\left. \frac{I_2}{2} \right|_{\rho = 0}
\end{aligned} \end{equation*} }
%
\textcolor{blue} {\begin{equation*}
\left. I_1 \right|_{\rho = 0} = \frac{1}{2} \begin{cases}
0, 0 < R < \sqrt{c^2t^2 - z^2} \\
\frac{R}{2} \left( c^2t^2 - z^2 \right)^{-1/2}, 0 < R = \sqrt{c^2t^2 - z^2} \\ 
1, 0 < \sqrt{c^2t^2 - z^2} < R 
\end{cases}
\end{equation*} }
%
\begin{equation}
\left. I_1 \right|_{\rho = 0} = \frac{1}{2} \begin{cases}
0, 0 < R < \sqrt{c^2t^2 - z^2} \\
1/2, 0 < R = \sqrt{c^2t^2 - z^2} \\ 
1, 0 < \sqrt{c^2t^2 - z^2} < R 
\end{cases}
\end{equation}

На останок, спростимо тригонометричні вирази, що містять $ \psi $. Розглянемо 
$ \psi = \arccos f(r,t) $, де $ f(r,t) $ задовільна функція координат. 
Тоді $ f(r,t) = \cos \psi $. Зазначимо, що з означення відомо, що 
$ \psi \in \left[ 0, \pi \right] $, тому $ \sin \psi \geq 0 $. Таким чином:
%
\begin{equation*} \begin{aligned}
\sin \psi = \sqrt{1 - \cos^2 \psi } = \sqrt{1 - f^2(r,t)}
\end{aligned} \end{equation*}

Згадуючи введене означення для $ \psi $ зашипимо, що
%
\textcolor{blue}{ \begin{equation*} \begin{aligned}
\psi = \arccos \left( \frac{\rho^2 + R^2}{2 \rho R} - 
\frac{c^2 t^2 - z^2}{2 \rho R} \right)
\end{aligned} \end{equation*} }
%
\textcolor{blue}{ \begin{equation*} \begin{aligned}
\sin \psi = \sqrt{1 - \left( \frac{\rho^2 + R^2}{2 \rho R} - 
\frac{c^2 t^2 - z^2}{2 \rho R} \right)^2} = 
\sqrt{1 - \frac{\left( \rho^2 + R^2 - c^2 t^2 + z^2 \right)^2}
{4 \rho^2 R^2} } = \\ = \sqrt{\frac{4 \rho^2 R^2}{4 \rho^2 R^2} - 
\frac{\left( \rho^2 + R^2 - c^2 t^2 + z^2 \right)^2}{4 \rho^2 R^2} } =
\sqrt{\frac{4 \rho^2 R^2 - \left( \rho^2 + R^2 - c^2 t^2 + z^2 \right)^2}
{4 \rho^2 R^2}} = \\
= \frac{1}{2 \rho R} \sqrt{4 \rho^2 R^2 - \left( \rho^2 + R^2 \right)^2 +
2 \left( \rho^2 + R^2 \right) \left( c^2 t^2 - z^2 \right) - 
\left( c^2 t^2 - z^2 \right)^2} = \\
= \frac{1}{2 \rho R} \sqrt{- \left( \rho^2 - R^2 \right)^2 +
2 \left( \rho^2 + R^2 \right) \left( c^2 t^2 - z^2 \right) - 
\left( c^2 t^2 - z^2 \right)^2} = \\
= \frac{c^2 t^2 - z^2}{2 \rho R} \sqrt{2 \frac{\rho^2 + R^2 }{c^2 t^2 - z^2} - 
\left( \frac{\rho^2 - R^2 }{c^2 t^2 - z^2} \right)^2 - 1}
\end{aligned} \end{equation*} }
%
\textcolor{red}{ \begin{equation*} \begin{aligned}
4 \rho^2 R^2 - (\rho^2 + R^2 - c^2 t^2 + z^2)^2 = \\
= 4 \rho^2 R^2 - (\rho^2 + R^2)^2 - (c^2 t^2 - z^2)^2 + 
2 (\rho^2 + R^2) (c^2 t^2 - z^2) = \\
= - (\rho^2 - R^2)^2 - (c^2 t^2 - z^2)^2 \pm 
2 R^2 (c^2 t^2 - z^2) + 2 (\rho^2 + R^2) (c^2 t^2 - z^2) = \\
= 4 R^2 (c^2 t^2 - z^2) - (\rho^2 - R^2)^2 - (c^2 t^2 - z^2)^2 +
2 (\rho^2 - R^2) (c^2 t^2 - z^2) = ?
\end{aligned} \end{equation*} }
%
\begin{equation*} \begin{aligned}
\frac{R}{\rho} \frac{\sin \psi}{2 \pi} = 
\frac{\sqrt{4 \rho^2 R^2 - (\rho^2 + R^2 - c^2t^2 + z^2)^2}}{4 \pi \rho^2}
\end{aligned} \end{equation*}
%
\textcolor{blue}{ \begin{equation*} \begin{aligned}
\tan \frac{\psi}{2} = \pm \sqrt{ \frac{1 - \cos \psi}{1 + \cos \psi} } = 
\sqrt{ \frac{1- \frac{\rho^2 + R^2}{2 \rho R} + \frac{c^2 t^2 - z^2}{2 \rho R}}
{1 + \frac{\rho^2 + R^2}{2 \rho R} - \frac{c^2 t^2 - z^2}{2 \rho R}} } =
\sqrt{ \frac{c^2t^2 - z^2 - \left( \rho - R \right)^2}
{\left( \rho + R \right)^2 - \left( c^2t^2 - z^2 \right)} }
\end{aligned} \end{equation*} }
%
\textcolor{blue}{ \begin{equation*} \begin{aligned}
\frac{\rho + R}{ |\rho - R| } \tan \frac{\psi}{2} = 
\sqrt{ \frac{ \frac{c^2t^2 - z^2}{\left( \rho - R \right)^2} - 1}
{ 1 - \frac{c^2t^2 - z^2}{\left( \rho + R \right)^2} } } = 
\sqrt{ \left( \frac{\rho + R}{\rho - R} \right)^2
\frac{c^2t^2 - z^2 - \left( \rho - R \right)^2}
{\left( \rho + R \right)^2 - \left( c^2t^2 - z^2 \right)} }
\end{aligned} \end{equation*} }
%
\begin{equation*} \begin{aligned}
\arctan \left( \frac{\rho + R}{ |\rho - R| } \tan \frac{\psi}{2} \right) - 
\frac{\pi}{2} = - \arctan \sqrt{ \left( \frac{\rho - R}{\rho + R} \right)^2
\frac{\left( \rho + R \right)^2 - \left( c^2t^2 - z^2 \right)} 
{\left( c^2t^2 - z^2 \right) - \left( \rho - R \right)^2} }
\end{aligned} \end{equation*}
%
\begin{equation*} \begin{aligned}
\pi - \psi = \arccos \left( \frac{c^2 t^2 - z^2 - \rho^2 - R^2}{2 \rho R} \right)
\end{aligned} \end{equation*}
%
Користуючись такими спрощеннями, можемо записати вираз в явному вигляді для 
області $ S_2 $
%
\begin{equation*} \begin{aligned}
I_1 \{ S_2 \} = \frac{\rho^2 + R^2}{4 \pi \rho^2} \arccos 
\left( \frac{c^2 t^2 - z^2 - \rho^2 - R^2}{2 \rho R} \right) - \\
- \frac{\sqrt{4 \rho^2 R^2 - (\rho^2 + R^2 - c^2t^2 + z^2)^2}}{4 \pi \rho^2} - \\
- \frac{ |\rho^2 - R^2| }{2 \pi \rho^2} 
\arctan \sqrt{ \frac{(\rho - R)^2}{(\rho + R)^2} \cdot
\frac{\left( \rho + R \right)^2 - \left( c^2t^2 - z^2 \right)} 
{\left( c^2t^2 - z^2 \right) - \left( \rho - R \right)^2} }
\end{aligned} \end{equation*}

%%%%%%%%%%%%%%%%%%%%%%%%%%%%%%%%%%%%%%%%%%%%%%%%%%%%%%%%%%%%%%%%%%%%%%%%%%%%%%%
\section{Інтеграл 2} \label{sec:i2anal}
%
\begin{equation} \label{eq:int2start}
I_2 = R \int \limits_{0}^{\infty} d \nu J_1 \left( \nu R \right) 
J_0 \left( \nu \rho \right) J_0 \left( \nu \sqrt{c^2t^2 - z^2} \right)
\end{equation}

Це табличний інтеграл, що може бути знайдений в 
\cite[ст. 228]{imp:SpecFunc1983}.
%
\begin{equation} \begin{aligned} \label{eq:intJ0J0J1tabel}
\int \limits_{0}^{\infty} d x J_0 \left( ax \right) 
J_0 \left( bx \right) J_1 \left( cx \right) = \begin{cases}
0, 0 < c < | a - b | \\ 
1/c, c > a + b
\end{cases} a, b > 0 \\
= \frac{1}{\pi c} \arccos \frac{a^2 + b^2 - c^2}{2ab},
| a - b | < c < a + b; a,b > 0
\end{aligned} \end{equation}

Фізичні властивості змінних в \eqref{eq:int2start} відповідають умові 
$ a,b,c > 0 $. Запишемо значення інтегралу відносно інших умов.
%
\begin{equation}
I_2 = \begin{cases}
0, 0 < R < | f_{-} \left( r, t \right) | \\
\frac{1}{\pi} \arccos \frac{c^2t^2 - z^2 + \rho^2 - R^2}
{2 \rho \sqrt{c^2t^2 - z^2}}, | f_{-} \left( r, t \right) | < R < 
f_{+} \left( r, t \right) \\ 1, f_{+} \left( r, t \right) < R \\
\end{cases}
\end{equation}

Тут для спрощення введено наступні переозначення:
%
\begin{equation*} \begin{aligned}
f_{-} \left( r, t \right) = \rho - \sqrt{c^2t^2 - z^2} \\
f_{+} \left( r, t \right) = \rho + \sqrt{c^2t^2 - z^2}
\end{aligned} \end{equation*}

Якщо $ \rho = 0 $, то область визначення інтегралу $ I_2 $ по формулі 
\eqref{eq:int2start} схропується і інтеграл стає невизначеним. У цьому випадку 
розглянемо інтеграл за допомогою формули \eqref{eq:intJJtable}.
%
\textcolor{blue}{ \begin{equation*}
I_2 \left( \rho = 0 \right) = \begin{cases}
0, 0 < R < \sqrt{c^2t^2 - z^2} \\
\frac{R}{2} \left( c^2t^2 - z^2 \right)^{-1/2}, 0 < R = \sqrt{c^2t^2 - z^2} \\ 
1, 0 < \sqrt{c^2t^2 - z^2} < R 
\end{cases}
\end{equation*} }
%
\begin{equation}
I_2 \left( \rho = 0 \right) = \begin{cases}
0, 0 < R < \sqrt{c^2t^2 - z^2} \\
1/2, 0 < R = \sqrt{c^2t^2 - z^2} \\ 
1, 0 < \sqrt{c^2t^2 - z^2} < R 
\end{cases}
\end{equation}
