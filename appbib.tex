\chapter{Список публікацій здобувача за темою дисертації}

\begin{center} 
\textit{\textbf{Наукові праці у наукових фахових виданнях України:}}
\end{center}

\newcounter{ItemsInMyWriting}

\begin{enumerate}

\item Dumin O.M., Tretyakov O.A., \textbf{Akhmedov R.D.}, Dumina O.O. 
Evolutionary Approach for the Problem of Electromagnetic Fiead 
Propagation Through Nonlinear Medium // Вісник Харківського національного 
університету імені В.Н. Каразіна, Серія ``Радіофізика та електроніка''. 
2015. випуск 24. С. 23--28.

\textit{Внесок здобувача: Аналітична робота по доказу та виведенню 
математичних співвідношень. Аналіз отриманих результатів.}

\item Думін О. М., \textbf{Ахмедов Р. Д.}, Міжмодове перетворення нестаціонарного 
електромагнітного поля в нелінійному необмеженому середовищі // Вісник 
Харківського національного університету імені В.Н. Каразіна, Серія 
``Радіофізика та електроніка''. 2017, випуск 26. С. 42--47.

\textit{Внесок здобувача: Отримання модового розкладу поля у відкритому 
просторі методом еволюційних рівнянь. Статистична обробка отриманих результатів. }

\item Думін О. М., \textbf{Ахмедов Р. Д.}, Випромінювання та розповсюдження 
електромагнітного снаряду в нелінійному середовищі // Вісник 
Харківського національного університету імені В.Н. Каразіна, Серія 
``Радіофізика та електроніка''. 2017, випуск 27. С. 37--42.

\textit{Внесок здобувача: Застосування теорії збурень для врахування 
нелінійних складових поляризації}

\item Думін О., \textbf{Ахмедов Р.}, Черкасов Д., Імпульсне випромінювання 
антени з круговою апертурою в ближній зоні // Вісник Харківського 
національного університету імені В. Н. Каразіна. Серія ``Радіофізика та 
електроніка''. 2018. випуск 28. C. 30--33.

\textit{Внесок здобувача: Підготовка графічних матеріалів до публікації. 
Аналітична робота над математичним апаратом методу еволюційних рівнянь.}

\item Думін О.М., \textbf{Ахмедов Р.Д.}, Черкасов Д.В., Поширення імпульсної 
електромагнітної хвилі в керрівському середовищі // Вісник Харківського 
національного університету імені В.Н. Каразіна. ``Радіофізика та 
електроніка''. 2018. Вип. 29. С.11--16.

\textit{Внесок здобувача: Розв'язання системи рівнянь Максвела з урахуванням
нелінійних властивостей середовища для неоднорідності у вигляді плаского 
диску з електричним струмом.}

\setcounter{ItemsInMyWriting}{\value{enumi}}
\end{enumerate}

\begin{center}
\textit{\textbf{Патенти:}}

\begin{enumerate}
\setcounter{enumi}{\value{ItemsInMyWriting}}

\item \textbf{Ахмедов Р. Д.}, Спосіб виділення корисної інформації з 
надширокосмугових (НШС) електромагнітних хвиль // Український інститут 
інтелектуальної власності. Київ. 2020.

\textit{Внесок здобувача: Розроблено методику виділення корисної інформації 
з нестаціонарних імпульсних електромагнітних хвиль. Проведено порівняльну 
характеристику різних.}

\setcounter{ItemsInMyWriting}{\value{enumi}}
\end{enumerate}

\begin{center} 
\textit{\textbf{Наукові праці у фахових виданнях, що входять до 
міжнародних наукометричних баз:}}
\end{center}

\begin{enumerate}
\setcounter{enumi}{\value{ItemsInMyWriting}}

\item \textbf{Akhmedov R.}, Dumin O., Katrich V., Impulse radiation of antenna 
with circular aperture // Telecommunications and Radio Engineering. Kharkiv. 
2018. Vol. 77. P. 1767--1784.

\textit{Внесок здобувача: Розв'язання задачі випромінювання імпульсу довільної
геометричної форми лінзовою імпульсною антеною з круговою апертурою. Аналітична
робота по отриманню перехідної функції для ближньої зони, як явної функції 
від просторових координат та часу.}

\setcounter{ItemsInMyWriting}{\value{enumi}}
\end{enumerate}

% \begin{center} 
% \textit{\textbf{Наукові праці у фахових закордонних виданнях:}}
% \end{center}

\begin{center} 
\textit{\textbf{Наукові праці апробаційного характеру (тези доповідей на 
наукових конференціях) за темою дисертації:}}
\end{center}

\begin{enumerate}
\setcounter{enumi}{\value{ItemsInMyWriting}}

\item Dumin O.M., Katrich V.A., \textbf{Akhmedov R.D.}, Tretyakov O.A., 
Dumina O.O., Evolutionary Approach for the Problems of Transient 
Electromagnetic Field Propagation in Nonlinear Medium // 15th International 
Conference on Mathematical Methods in Electromagnetic Theory (MMET).
Dnipropetrovsk. 2014.

\item Dumin O.M., Tretyakov O.A., \textbf{Akhmedov R.D.}, Stadnik Yu.B., 
Katrich V.A., Dumina, O.O., Modal Basis Method for Propagation of 
Transient Electromagnetic Fields in Nonlinear Medium // Proc. 7th 
International Conference on Ultrawideband and Ultrashort Impulse Signals 
(UWBUSIS). Kharkiv. 2014.

\item Dumin O.M., Tretyakov O.A., \textbf{Akhmedov R.D.}, and Dumina O.O., 
Transient Electromagnetic Field Propagation through Nonlinear Medium in 
time domain // International Conference on Antenna Theory and Techniques, 
21 -- 24 April, 2015. Kharkiv. 2015.

\item Dumin O.M., \textbf{Akhmedov R.D.}, Dumina O.O., Propagation of 
Transient Field Radiated from Plane Disk in Nonlinear Medium // 
Ultrawideband and Ultrashort Impulse Signals, 5--11 September 2016. 
Odessa. 2016.

\item Dumin O., \textbf{Akhmedov R.}, Dumina O., Transient Field 
Radiation of Plane Disk into Nonlinear Medium // Radio Electronics and 
Info Communications, 11--16 September 2016. Kiev. 2016.

\item Dumin O., \textbf{Akhmedov R.}, Katrich V., Dumina O., Transient 
Radiation of Circle with Uniform Current Distribution // 2017 IEEE First 
Ukraine Conference on Electrical and Computer Engineering (UKRCON), 
May 29 -- June 2 2017. Kiev. 2017.

\item \textbf{Akhmedov R.}, Dumin O., Ultrashort Impulse Radiation from 
Plane Disk with Uniform Current Distribution // Ultrawideband and 
Ultrashort Impulse Signals, 4--7 September 2018. Odessa. 2018.

\item Dumin O., \textbf{Akhmedov R.}, Dumina O., Cherkasov D., Near Zone 
of Plane Disk with Uniform Transient Current Distribution // 2017 IEEE 2nd 
Ukraine Conference on Electrical and Computer Engineering (UKRCON), 
June 2 -- June 9 2019. Lviv. 2019.

\item Dumin O., \textbf{Akhmedov R.}, Katrich V., Cherkasov D., 
Impulse Electromagnetic Wave Propagation in Kerr Medium // 2019 XXIVth 
International Seminar/Workshop on Direct and Inverse Problems of 
Electromagnetic and Acoustic Wave Theory (DIPED). Lviv. 2019.

\setcounter{ItemsInMyWriting}{\value{enumi}}
\end{enumerate}

\end{center}
