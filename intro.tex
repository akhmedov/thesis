\chapter*{Вступ}

\paragraph{Актіальність теми}

Зміна форми імпульсу від нелінійних факторів впливає на якість розв'язку 
електродинамычних задач. Визначення впливу слабкої нелінійності на 
розповсюдження нестаціонарних сигналів.

Теорія розповсюдженя імпульсів може бути використана для визначення 
стану середовща в нелінійних, нестаціонарних та неоднорідних випадках
випромінювання. Мова іде про такі параметри оточойочого середовища, як 
проникність, температура, густина, іонозованість, тощо.

% Здешевлення пристроїв детекції гравітаційних хвиль. Станом на 2017 рік 
% для дослідження равітаційних ефектів використовують лазерну технологію в 
% ізольовіній системі. Використання частотних радіопристроїв унеможливлене
% поганою точністю за рахунок вплмву оточуйочого фонового шуму на аненни 
% прийому передачі. Використання же коротких ті сильних імпульсів, вірішує 
% проблему точності за рахунок складної форми імпульсу.

Наступне покоління бездротових локальних систем передачі данних для 
квантових компьютерів. Велика множина форм імпульсів для радіозвязку 
дозволяє побудувати не бінарни протокол передачі данних канального рівня,
що підвищіть кількість інформації що передається, навіть у зашумленому або 
нелінійному просторі. Особливий приріст швиткості варто очікувати на 
квантових обчислюваних системах.

\paragraph{Звязок роботи з науковими програмами}

Erasmus+

\paragraph{Мета та задача дослідження}

\paragraph{Методи дослідження}

\begin{enumerate} 
	\item Рекурентні нейронні мережі
	\item Метод еволюційних рівнянь
	\item Метод перехідної функції
\end{enumerate} 

\paragraph{Наукова новизна отриманих результатів}

Розвязок задачі плаского диску для лінійного та слабкого нелінійного простору
Авторський метод синтезу протоколів передачі інформації

\paragraph{Практичне значення отриманних результатів}

\begin{enumerate} 
	\item Шведський проект по аналізу властивостей надпровідникових матеріалів
	\item Передача інформації імпульсами на велику відстань
\end{enumerate} 

\paragraph{Особистий вклад дисертанта}

Розвязок задачі плаского диску слабкого нелінійного простору
Авторський метод синтезу протоколів передачі інформації

\paragraph{Апробація результатів дослідження}

\begin{enumerate} 
	\item Erasmus+
	\item Upsala Univ.
\end{enumerate} 

\paragraph{Публікації}

\paragraph{Зміст роботи}

\paragraph{Подяка}
