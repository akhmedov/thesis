\chapter{Комплексі функції Ломмеля двох змінних}
\label{ch:lommel}

%%%%%%%%%%%%%%%%%%%%%%%%%%%%%%%%%%%%%%%%%%%%%%%%%%%%%%%%%%%%%%%%%%%%%%%%%%%%%%%
\section{Визначення та лінійні властивості}

В \cite{Boersma1961} приводиться визначення через функцію Бесселя.
%
\begin{equation}
U_n \left[ W, Z \right] = \sum \limits_{m = 0}^{\infty} (-1)^m
\left( \frac{W}{Z} \right)^{n + 2m} J_{n + 2m} (Z)
\end{equation}

В \cite{Boersma1961} також можна знайти наступну властивість.
%
\begin{equation}
U_n \left[ W, Z \right] + U_{n+2} \left[ W, Z \right] = 
\left( \frac{W}{Z} \right)^n J_n (Z)
\end{equation}
%
\textcolor{lightgray} { \begin{equation*} \begin{aligned}
W_\pm = \pm i (\nu ct - \nu z) \\
Z = \sqrt{\nu^2 c^2t^2 - \nu^2 z^2}
\end{aligned} \end{equation*} }
%
\textcolor{lightgray}{ \begin{equation*}
U_0 \left[ W, Z \right] = \sum \limits_{m = 0}^{\infty} (-1)^m
\left( \frac{W}{Z} \right)^{2m} J_{2m} (Z) = J_0 (Z) - \frac{W^2}{Z^2} J_2 (Z) +
\frac{W^4}{Z^4} J_4 (Z) - ...
\end{equation*} }
%
\textcolor{lightgray}{ \begin{equation*}
U_2 \left[ W, Z \right] = \sum \limits_{m = 0}^{\infty} (-1)^m
\left( \frac{W}{Z} \right)^{2 + 2m} J_{2 + 2m} (Z) = 
\frac{W^2}{Z^2} J_2 (Z) - \frac{W^4}{Z^4} J_4 (Z) + \frac{W^6}{Z^6} J_6 (Z) - ...
\end{equation*} }
%
\textcolor{lightgray}{ \begin{equation*} \begin{aligned}
U_0 [W, Z] - U_2 [W, Z] = J_0(Z) - 2 \left( \frac{W^2}{Z^2} J_2 (Z) - 
\frac{W^4}{Z^4} J_4 (Z) + \frac{W^6}{Z^6} J_6 (Z) - ... \right)
\end{aligned} \end{equation*} }
%
\textcolor{lightgray}{ \begin{equation*} \begin{aligned}
U_0 [W, Z] - U_2 [W, Z] = J_0(Z) - 2 U_2(W,Z)
\end{aligned} \end{equation*} }
%
\textcolor{lightgray}{ \begin{equation*} \begin{aligned}
\left( \frac{W}{Z} \right)^{2n} = \left( 
\frac{- i \nu (\mathit{V}t - z)}
{\nu \sqrt{\mathit{V}^2t^2 - z^2}} \right)^{2n} = 
(-i)^{2n} \nu^{2n} \frac{(\mathit{V}t - z)^{2n}}
{(\mathit{V}t - z)^n (\mathit{V}t + z)^n} = \\
= (-i)^{2n} \left( \frac{\mathit{V}t - z}{\mathit{V}t + z} \right)^n = 
(-1)^{n} \left( \frac{\mathit{V}t - z}{\mathit{V}t + z} \right)^n = 
\left( - \frac{\mathit{V}t - z}{\mathit{V}t + z} \right)^n
\end{aligned} \end{equation*} }
%
\textcolor{lightgray}{ \begin{equation*} \begin{aligned}
U_0 [W, Z] - U_2 [W, Z] = J_0(Z) + 2 \left[ 
\frac{\mathit{V}t - z}{\mathit{V}t + z} J_2(Z) + \left( 
\frac{\mathit{V}t - z}{\mathit{V}t + z} \right)^2 J_4(Z) + ... \right]
\end{aligned} \end{equation*} }
%
\begin{equation} \begin{aligned}
U_0 [W, Z] - U_2 [W, Z] = J_0(Z) + 2 \sum_{m=1}^{\infty} \left( 
\frac{\mathit{V}t - z}{\mathit{V}t + z} \right)^m J_{2m} (Z)
\end{aligned} \end{equation}

%%%%%%%%%%%%%%%%%%%%%%%%%%%%%%%%%%%%%%%%%%%%%%%%%%%%%%%%%%%%%%%%%%%%%%%%%%%%%%%
\section{Інтегродиференціальні властивості}

Функція Ломмеля типова для нестаціонарних задач. В \cite[ст. 41]{Borisov1991} 
приведено корисні інтегродиференціальні.
%
\begin{equation} \begin{aligned}
\int \limits_{\xi}^{\tau} ds e^{-i \gamma s} J_0(\sqrt{s^2 - \xi^2 }) = 
\frac{e^{-i \gamma \tau}}{\sqrt{\gamma^2 - 1}} \left( U_1(W_+,Z) + \right. \\ 
\left. + i U_2(W_+,Z) - U_1(W_-,Z) - i U_2(W_-,Z) \right)
\end{aligned} \end{equation}

Тут $ W_\pm = (\gamma \pm \sqrt{\gamma^2 - 1}) (\tau - \xi) $ a 
$ Z = \sqrt{\tau^2 - \xi^2} $. Також для використання цієї формули повинна
виконуватись умова $ \tau - \xi > 0 $.
%
\textcolor{red}{ \begin{equation}
\left. \begin{array}{c}
U_{2n} (W_+, Z) = U_{2n} (W_-, Z) \\
U_{2n+1} (W_+, Z) = - U_{2n+1} (W_-, Z)
\end{array} \right| n \in \Z
\end{equation} }
%
\begin{equation} 
\partder{}{Z} U_n (W,Z) = - \frac{Z}{W} U_{n+1} (W,Z)
\end{equation}
%
\begin{equation}
2 \partder{}{W} U_n (W,Z) = U_{n-1} (W,Z) + 
\left( \frac{Z}{W} \right)^2 U_{n+1} (W,Z)
\end{equation}

%%%%%%%%%%%%%%%%%%%%%%%%%%%%%%%%%%%%%%%%%%%%%%%%%%%%%%%%%%%%%%%%%%%%%%%%%%%%%%%
\section{Інтеграл 3}

\begin{equation} 
I_3 = R \int \limits_{0}^{\infty} \frac{d \nu}{\rho \nu} 
J_1(\nu \rho) J_1(\nu R) (U_0[ W_-, Z ] - U_2[ W_-, Z ])
\end{equation}
%
\begin{equation} \label{eq:intergal3}
I_3 = I_1 - 2 R \int_{0}^{\infty} \frac{d \nu}{\nu \rho} 
J_1(\nu \rho) J_1(\nu R) U_2[ W_-, Z ]
\end{equation}
%
На осі випромінювання, тобто при $ \rho = 0 $. 
%
\textcolor{lightgray}{ \begin{equation*} \begin{aligned}
\left. 2 R \int_{0}^{\infty} \frac{d \nu}{\nu \rho} 
J_1(\nu \rho) J_1(\nu R) \sum_{m=1}^{\infty} \left( 
\frac{ct - z}{ct + z} \right)^m J_{2m} (Z) 
\right|_{\rho = 0} = \\ = R \frac{ct - z}{ct + z} \int_{0}^{\infty} 
d \nu J_1(\nu R) J_2 (\nu \sqrt{c^2t^2 + z^2}) + \\ 
+ R \left( \frac{ct - z}{ct + z} \right)^2 
\int_{0}^{\infty} d \nu J_1(\nu R) J_4 (\nu \sqrt{c^2t^2 + z^2}) + ...
\end{aligned} \end{equation*} }
%
\begin{equation*} \begin{aligned}
I_3 = I_1 + R \sum_{m=1}^{\infty} \left( \frac{ct - z}{ct + z} \right)^m 
\int_{0}^{\infty} d \nu J_1(\nu R) J_{2m} (\nu \sqrt{c^2t^2 + z^2})
\end{aligned} \end{equation*}
%
За допомогою табличного інтегралу 2.12.31.1 з 
\cite[ст. 209]{SpecFunc1983}, що має вид:
%
\begin{equation*} \begin{aligned}
\int_0^\infty J_\mu (bx) J_\nu (cx) dx = A_{\mu,\nu}^1
\end{aligned} \end{equation*}
%
\begin{equation*} \begin{aligned}
A_{\mu,\nu}^1 \left( 0 < c < b \right) = b^{-\nu-1} c^{\nu} 
\Gamma \left[ \begin{array}{l} 
(\nu+\mu+1)/2 \\ (\mu-\nu-1)/2 + 1, \nu + 1 \end{array} \right] \cdot
\\ \cdot F \left(  \frac{\nu+\mu+1}{2}, \frac{\nu-\mu+1}{2}; 
\nu+1; \frac{c^2}{b^2} \right)
\end{aligned} \end{equation*}
%
\begin{equation*} \begin{aligned}
A_{\mu,\nu}^1 \left( 0 < b < c \right) = b^{\mu} c^{-\mu-1} 
\Gamma \left[ \begin{array}{l} 
(\nu+\mu+1)/2 \\ (\nu-\mu-1)/2 + 1, \mu + 1 \end{array} \right] \cdot
\\ \cdot F \left(  \frac{1+\mu+\nu}{2}, \frac{1+\mu-\nu}{2}; 
\mu+1; \frac{b^2}{c^2} \right)
\end{aligned} \end{equation*}
%
\textcolor{lightgray}{ \begin{equation*} \begin{aligned}
\begin{array}{lr} \forall a,k \in \Z: &
(a)_k = \frac{\Gamma(a+k)}{\Gamma(a)} = \frac{(a+k-1)!}{(a-1)!} \end{array}
\end{aligned} \end{equation*} }
%
Тут через $ F $ означено гіпергеометричну функцію Гаусса.
%
\textcolor{lightgray}{ \begin{equation*} \begin{aligned}
F (a_1, a_2; b_1; z) = \sum_{k=0}^\infty 
\frac{(a_1)_k (a_2)_k}{(b_1)_k} \frac{z^k}{k!} 
\end{aligned} \end{equation*} }
%
\begin{equation*} \begin{aligned}
F (a_1, a_2; b_1; z) = \sum_{k=0}^\infty 
\frac{\Gamma(a_1+k) \Gamma(a_2+k) \Gamma(b_1)}
{\Gamma(a_1) \Gamma(a_2) \Gamma(b_1+k)} \frac{z^k}{k!}
\end{aligned} \end{equation*}
%
Таким чином, $ I_3 $ можна записати в явному виді для $ \rho = 0 $.
%
\textcolor{lightgray}{ \begin{equation*} \begin{aligned}
b = R, c = \sqrt{c^2t^2-z^2} \\
\mu = 1, \nu = \{ 2m \}_{m=1}^{\infty}
\end{aligned} \end{equation*} }
%
\textcolor{lightgray}{ \begin{equation*} \begin{aligned}
\left. A_{1,2m}^1 \right|^{\tau < R} =
\frac{(c^2t^2-z^2)^m}{R^{2m+1}}
\Gamma \left[ \begin{array}{l} 1+m \\ 1-m, 2m + 1 \end{array} \right]
F \left( m+1, m; 2m+1; \frac{c^2t^2-z^2}{R^2} \right)
\end{aligned} \end{equation*} }
%
\textcolor{lightgray}{ \begin{equation*} \begin{aligned}
\left. A_{1,2m}^1 \right|^{\tau < R} = \frac{(c^2t^2-z^2)^m}{R^{2m+1}}
\frac{\Gamma(1+m)}{\Gamma(1-m) \Gamma(2m+1)}
F \left( m+1, m; 2m+1; \frac{c^2t^2-z^2}{R^2} \right)
\end{aligned} \end{equation*} }
%
\textcolor{lightgray}{ \begin{equation*} \begin{aligned}
\left. A_{1,2m}^1 \right|^{\tau < R} = \frac{1}{R}
\frac{\Gamma(1+m)}{\Gamma(1-m) \Gamma(2m+1)} \cdot \\
\cdot \sum_{k=0}^\infty \frac{1}{k!} 
\frac{ \Gamma(m+k+1) \Gamma(m+k) \Gamma(2m+1) }
{ \Gamma(m+1) \Gamma(m) \Gamma(2m+k+1) }
\left( \frac{c^2t^2-z^2}{R^2} \right)^{k+m}
\end{aligned} \end{equation*} }
%
\begin{equation*} \begin{aligned}
\left. A_{1,2m}^1 \right|^{c^2t^2 - z^2 < R} \simeq
\frac{1}{ \Gamma(1-m) \Gamma(m) } = 
\frac{\sin(m \pi)}{\pi} = 0, \forall m \in \Z
\end{aligned} \end{equation*}
%
\textcolor{lightgray}{ \begin{equation*} \begin{aligned}
\left. A_{1,2m}^1 \right|^{\tau > R} = \frac{R}{c^2t^2-z^2}
\Gamma \left[ \begin{array}{l} m+1 \\ m, 2 \end{array} \right]
F \left( 1+m, 1-m; 2; \frac{R^2}{c^2t^2-z^2} \right)
\end{aligned} \end{equation*} }
%
\textcolor{lightgray}{ \begin{equation*} \begin{aligned}
\left. A_{1,2m}^1 \right|^{\tau > R} = \frac{R}{c^2t^2-z^2}
\frac{\Gamma(m+1)}{\Gamma(m) \Gamma(2)}
F \left( 1+m, 1-m; 2; \frac{R^2}{c^2t^2-z^2} \right)
\end{aligned} \end{equation*} }
%
\textcolor{lightgray}{ \begin{equation*} \begin{aligned}
\left. A_{1,2m}^1 \right|^{\tau > R} = \frac{mR}{c^2t^2-z^2}
\sum_{k=0}^\infty  \frac{\left( 1-m^2 \right)^k}{2^k k!} 
\left( \frac{R^2}{c^2t^2-z^2} \right)^{k}
\end{aligned} \end{equation*} }
%
\textcolor{lightgray}{ \begin{equation*} \begin{aligned}
\left. A_{1,2m}^1 \right|^{\tau > R} = \frac{m}{R}
\sum_{k=0}^\infty  \frac{\left( 1-m^2 \right)^k}{2^k k!} 
\left( \frac{R^2}{c^2t^2-z^2} \right)^{k+1}
\end{aligned} \end{equation*} }
%
\begin{equation}
\left. I_3 \right|^{\rho=0} = \left. I_1 \right|^{\rho=0} +
\sum_{m=1}^{\infty} \left( \frac{ct - z}{ct + z} \right)^m A_{1,2m}^1
\end{equation}
%
\begin{equation}
\left. A_{1,2m}^1 \right|^{\tau > R} = m
\sum_{k=0}^\infty  \frac{\left( 1-m^2 \right)^k}{2^k k!} 
\left( \frac{R^2}{c^2t^2-z^2} \right)^{k+1}
\end{equation}
%
\begin{equation} 
\left. A_{1,2m}^1 \right|^{\tau < R} = \frac{C_{2m-1}^{m-1}}{2}
\sum_{k=0}^\infty \frac{1}{k!} \left( \frac{m^2+m}{2m+1} \right)^k
\left( \frac{c^2t^2-z^2}{R^2} \right)^{k+m}
\end{equation}

%%%%%%%%%%%%%%%%%%%%%%%%%%%%%%%%%%%%%%%%%%%%%%%%%%%%%%%%%%%%%%%%%%%%%%%%%%%%%%%
\section{Інтеграл 4}

\begin{equation}
I_4 = R \int \limits_{0}^{\infty} d \nu J_0(\nu \rho) J_1(\nu R) 
(U_0[ W_-, Z ] - U_2[ W_-, Z ])
\end{equation}
%
\begin{equation}
I_4 = I_2 - 2 R \int \limits_{0}^{\infty} d \nu 
J_0(\nu \rho) J_1(\nu R) U_2[ W_-, Z ]
\end{equation}
%
По аналогії з \eqref{eq:intergal3}, використаємо формулу 2.12.31.1 з 
\cite[ст. 209]{SpecFunc1983} для пошуку значення інтегралу при $ \rho = 0 $.
%
\textcolor{lightgray}{ \begin{equation*} \begin{aligned}
I_4 = I_2 - 2 R \int \limits_{0}^{\infty} d \nu J_1(\nu R) U_2[ W_-, Z ]
\end{aligned} \end{equation*} }
%
\textcolor{lightgray}{ \begin{equation*} \begin{aligned}
I_4 = I_2 - 2 R \sum_{m=1}^{\infty} \left( \frac{ct - z}{ct + z} \right)^m 
\int_{0}^{\infty} d \nu J_1(\nu R) J_{2m} (\nu \sqrt{c^2t^2 + z^2})
\end{aligned} \end{equation*} }
%
\begin{equation}
\left. I_4 \right|^{\rho=0} = \left. I_2 \right|^{\rho=0} -
2 \sum_{m=1}^{\infty} \left( \frac{ct - z}{ct + z} \right)^m A_{1,2m}^1
\end{equation}

%%%%%%%%%%%%%%%%%%%%%%%%%%%%%%%%%%%%%%%%%%%%%%%%%%%%%%%%%%%%%%%%%%%%%%%%%%%%%%%
\section{Інтеграл 5}

\begin{equation}
I_5 = i R \int \limits_{0}^{\infty} d \nu J_1 \left( \nu R \right) 
J_1 \left( \nu \rho \right)
U_1 \left[ - i \nu \left( ct - z \right), \nu \sqrt{c^2t^2 - z^2} \right]
\end{equation}
