\documentclass{vakthesis}

\usepackage[T2A]{fontenc}
\usepackage[utf8]{inputenc}
\usepackage[english,russian,ukrainian]{babel}

\usepackage[intlimits]{amsmath}
\allowdisplaybreaks
\usepackage{amsthm}
\usepackage{amssymb}

% Таблиці зі стовпчиками, що розтягуються
\usepackage{tabularx}

\usepackage{geometry}
\geometry{hmargin={30mm,15mm},lines=29,vcentering}

\usepackage{eqparbox} 
\usepackage[x11names]{xcolor} 

% нумкрация теорем/лемм
\theoremstyle{plain}
\newtheorem{theorem}{Теорема}[chapter]
\newtheorem{lemma}{Лема}[chapter]
\newtheorem{corollary}{Наслідок}[chapter]
\theoremstyle{definition}
\newtheorem{definition}{Означення}[chapter]
\newtheorem{example}{Приклад}[chapter]
\theoremstyle{remark}
\newtheorem{remark}{Зауваження}[chapter]

\DeclareMathOperator{\Rea}{Re}
\DeclareMathOperator{\Ima}{Im}
\DeclareMathOperator{\sinc}{sinc} % \sinc t = \frac{\sin t}{t}
% \DeclareMathOperator{\min}{min}

\newcommand{\N}{\mathbb{N}}
\newcommand{\Z}{\mathbb{Z}}
\newcommand{\Q}{\mathbb{Q}}
\newcommand{\R}{\mathbb{R}}

\newcommand{\rot}{\mathop{\mathrm{rot}}\nolimits} % rot{A}

\newcommand{\crossprod}[2]{ \left[  #1 \times #2 \right] } % вектор произвед
\newcommand{\dotprod}[2]{ \left<  #1 \cdot #2 \right> } % скалрное произвед
\newcommand{\triple}[3]{ \left<  #1 , #2 , #3 \right> } % скалрное произвед

\newcommand{\vect}[1]{ \overrightarrow{\mathbf #1} } % bold and with arrow
\newcommand{\func}[2]{ #1 \left( #2 \right) } % функциональная зависимость

\newcommand{\partder}[2]{ \frac{\partial #1}{\partial #2}}
\newcommand{\derivat}[2]{ \frac{d #1}{d #2}}

\newcommand{\mybibappendix}{
	
	\begin{center} 
		\textit{\textbf{Наукові праці у наукових фахових виданнях України:}}
	\end{center}
	
	\newcounter{ItemsInMyWriting}
	
	\begin{enumerate}
		
		\item Dumin O.M., Tretyakov O.A., \textbf{Akhmedov R.D.}, Dumina O.O. 
		Evolutionary Approach for the Problem of Electromagnetic Fiead 
		Propagation Through Nonlinear Medium // Вісник Харківського національного 
		університету імені В.Н. Каразіна, Серія ``Радіофізика та електроніка''. 
		2015. випуск 24. С. 23--28.
		
		\textit{Внесок здобувача: Аналітична робота по доказу та виведенню 
			математичних співвідношень. Аналіз отриманих результатів.}
		
		\item Думін О. М., \textbf{Ахмедов Р. Д.}, Міжмодове перетворення нестаціонарного 
		електромагнітного поля в нелінійному необмеженому середовищі // Вісник 
		Харківського національного університету імені В.Н. Каразіна, Серія 
		``Радіофізика та електроніка''. 2017, випуск 26. С. 42--47.
		
		\textit{Внесок здобувача: Отримання модового розкладу поля у відкритому 
			просторі методом еволюційних рівнянь. Статистична обробка отриманих результатів. }
		
		\item Думін О. М., \textbf{Ахмедов Р. Д.}, Випромінювання та розповсюдження 
		електромагнітного снаряду в нелінійному середовищі // Вісник 
		Харківського національного університету імені В.Н. Каразіна, Серія 
		``Радіофізика та електроніка''. 2017, випуск 27. С. 37--42.
		
		\textit{Внесок здобувача: Застосування теорії збурень для врахування 
			нелінійних складових поляризації}
		
		\item Думін О., \textbf{Ахмедов Р.}, Черкасов Д., Імпульсне випромінювання 
		антени з круговою апертурою в ближній зоні // Вісник Харківського 
		національного університету імені В. Н. Каразіна. Серія ``Радіофізика та 
		електроніка''. 2018. випуск 28. C. 30--33.
		
		\textit{Внесок здобувача: Підготовка графічних матеріалів до публікації. 
			Аналітична робота над математичним апаратом методу еволюційних рівнянь.}
		
		\item Думін О.М., \textbf{Ахмедов Р.Д.}, Черкасов Д.В., Поширення імпульсної 
		електромагнітної хвилі в керрівському середовищі // Вісник Харківського 
		національного університету імені В.Н. Каразіна. ``Радіофізика та 
		електроніка''. 2018. Вип. 29. С.11--16.
		
		\textit{Внесок здобувача: Розв'язання системи рівнянь Максвела з урахуванням
			нелінійних властивостей середовища для неоднорідності у вигляді плаского 
			диску з електричним струмом.}
		
		\item \textbf{Ахмедов Р.Д.}, Виокремлення корисної інформації з 
		надширокосмугової хвилі у ближній зоні випромінювання. ``Технология и 
		конструирование в электронной аппаратуре'', 2020, No 3-4, с. 3—10. DOI: 
		http://dx.doi.org/10.15222/TKEA2020.3-4.03
		
		\textit{Внесок здобувача: Розробка авторської методики виділення корисної 
			інформації з імпульсної надширокосмугової електромагнітної хвилі, проведення 
			числових симуляцій процесу випромінювання-поширювання-приймання імпульсів з
			урахуванням розробленої методики.}
		
		\setcounter{ItemsInMyWriting}{\value{enumi}}
	\end{enumerate}
	
%	\begin{center}
%		\textit{\textbf{Патенти:}}
%	\end{center}
%		
%	\begin{enumerate}
%		\setcounter{enumi}{\value{ItemsInMyWriting}}
%		
%		\item \textbf{Ахмедов Р. Д.}, Спосіб виділення корисної інформації з 
%		надширокосмугових (НШС) електромагнітних хвиль // Український інститут 
%		інтелектуальної власності. Київ. 2020.
%		
%		\textit{Внесок здобувача: Розроблено методику виділення корисної інформації 
%			з нестаціонарних імпульсних електромагнітних хвиль. Проведено порівняльну 
%			характеристику різних.}
%		
%		\setcounter{ItemsInMyWriting}{\value{enumi}}
%	\end{enumerate}
		
	\begin{center} 
		\textit{\textbf{Наукові праці у фахових виданнях, що входять до 
				міжнародних наукометричних баз:}}
	\end{center}
	
	\begin{enumerate}
		\setcounter{enumi}{\value{ItemsInMyWriting}}
		
		\item \textbf{Akhmedov R.}, Dumin O., Katrich V., Impulse radiation of antenna 
		with circular aperture // Telecommunications and Radio Engineering. Kharkiv. 
		2018. Vol. 77. P. 1767--1784.
		
		\textit{Внесок здобувача: Розв'язання задачі випромінювання імпульсу довільної
			геометричної форми лінзовою імпульсною антеною з круговою апертурою. Аналітична
			робота по отриманню перехідної функції для ближньої зони, як явної функції 
			від просторових координат та часу.}
		
		\setcounter{ItemsInMyWriting}{\value{enumi}}
	\end{enumerate}
		
	% \begin{center} 
	% \textit{\textbf{Наукові праці у фахових закордонних виданнях:}}
	% \end{center}
	
	\begin{center} 
		\textit{\textbf{Наукові праці апробаційного характеру (тези доповідей на 
				наукових конференціях) за темою дисертації:}}
	\end{center}
	
	\begin{enumerate}
		\setcounter{enumi}{\value{ItemsInMyWriting}}
		
		\item Dumin O.M., Katrich V.A., \textbf{Akhmedov R.D.}, Tretyakov O.A., 
		Dumina O.O., Evolutionary Approach for the Problems of Transient 
		Electromagnetic Field Propagation in Nonlinear Medium // 15th International 
		Conference on Mathematical Methods in Electromagnetic Theory (MMET).
		Dnipropetrovsk. 2014.
		
		\item Dumin O.M., Tretyakov O.A., \textbf{Akhmedov R.D.}, Stadnik Yu.B., 
		Katrich V.A., Dumina, O.O., Modal Basis Method for Propagation of 
		Transient Electromagnetic Fields in Nonlinear Medium // Proc. 7th 
		International Conference on Ultrawideband and Ultrashort Impulse Signals 
		(UWBUSIS). Kharkiv. 2014.
		
		\item Dumin O.M., Tretyakov O.A., \textbf{Akhmedov R.D.}, and Dumina O.O., 
		Transient Electromagnetic Field Propagation through Nonlinear Medium in 
		time domain // International Conference on Antenna Theory and Techniques, 
		21 -- 24 April, 2015. Kharkiv. 2015.
		
		\item Dumin O.M., \textbf{Akhmedov R.D.}, Dumina O.O., Propagation of 
		Transient Field Radiated from Plane Disk in Nonlinear Medium // 
		Ultrawideband and Ultrashort Impulse Signals, 5--11 September 2016. 
		Odessa. 2016.
		
		\item Dumin O., \textbf{Akhmedov R.}, Dumina O., Transient Field 
		Radiation of Plane Disk into Nonlinear Medium // Radio Electronics and 
		Info Communications, 11--16 September 2016. Kiev. 2016.
		
		\item Dumin O., \textbf{Akhmedov R.}, Katrich V., Dumina O., Transient 
		Radiation of Circle with Uniform Current Distribution // 2017 IEEE First 
		Ukraine Conference on Electrical and Computer Engineering (UKRCON), 
		May 29 -- June 2 2017. Kiev. 2017.
		
		\item \textbf{Akhmedov R.}, Dumin O., Ultrashort Impulse Radiation from 
		Plane Disk with Uniform Current Distribution // Ultrawideband and 
		Ultrashort Impulse Signals, 4--7 September 2018. Odessa. 2018.
		
		\item Dumin O., \textbf{Akhmedov R.}, Dumina O., Cherkasov D., Near Zone 
		of Plane Disk with Uniform Transient Current Distribution // 2017 IEEE 2nd 
		Ukraine Conference on Electrical and Computer Engineering (UKRCON), 
		June 2 -- June 9 2019. Lviv. 2019.
		
		\item Dumin O., \textbf{Akhmedov R.}, Katrich V., Cherkasov D., 
		Impulse Electromagnetic Wave Propagation in Kerr Medium // 2019 XXIVth 
		International Seminar/Workshop on Direct and Inverse Problems of 
		Electromagnetic and Acoustic Wave Theory (DIPED). Lviv. 2019.
		
		\item  \textbf{R. Akhmedov}, Neural Radio in DS-UWB IoT Applications // 2020 
		IEEE Ukrainian Microwave Week (UkrMW), Kharkiv, Ukraine, 2020, pp. 1073-1078, 
		doi: 10.1109/UkrMW49653.2020.9252611.
		
		\setcounter{ItemsInMyWriting}{\value{enumi}}
	\end{enumerate}
	
} % Локальние определения

%\includeonly{ch1,bib,app3,app1}

% информация о подключенных пакетах в логах
%\listfiles

\begin{document}

\title{Випромінювання нестаціонарних полів та їх розповсюдження в нелінійному 
просторі}

\author{Ахмедов Ролан Джавадович}

\supervisor{Думін Олександр Миколайович}
{кандидат фізико-математичних наук, доцент}

\speciality{01.04.03}

\udc{511.72} % Індекс за УДК

\institution
{Харкивській національний університет імені В. Н. Каразіна}
{Харків}

\date{2017}

\maketitle % титулка

\tableofcontents % содержание

%\chapter*{Вступ}

%%%%%%%%%%%%%%%%%%%%%%%%%%%%%%%%%%%%%%%%%%%%%%%%%%%%%%%%%%%%%%%%%%%%%%%%%%%%%%%
\paragraph{Актуальність теми}

Час та частота - це абстракції, що описують одне явище природи - зміна
енергетичних взаємодій в системі. Для макроскопічної електродинаміки, зокрема,
існують підходи, що застосовують обидві абстракції. Вибір абстракції, тобто 
підходу, визначаються характером задачі, що вирішується. Історично склалось, що
більш широке розповсюдження отримала частотно-орієнтована методологія, яка в 
повній мірі виправдала себе. Однак новітні технології все частіше потребують 
розв'язків, які методи частотної області надати не зможуть. Наприклад, 
розв'язання задач розповсюдження та випромінювання в нестаціонарних 
неоднорідних середовищах, що характеризуються нелінійними та анізотропними 
ефектами. Це відновило інтерес до методів часової області.

%%%%%%%%%%%%%%%%%%%%%%%%%%%%%%%%%%%%%%%%%%%%%%%%%%%%%%%%%%%%%%%%%%%%%%%%%%%%%%%
\paragraph{Зв'язок роботи з науковими програмами}

Erasmus+

%%%%%%%%%%%%%%%%%%%%%%%%%%%%%%%%%%%%%%%%%%%%%%%%%%%%%%%%%%%%%%%%%%%%%%%%%%%%%%%
\paragraph{Мета та задача дослідження}

%%%%%%%%%%%%%%%%%%%%%%%%%%%%%%%%%%%%%%%%%%%%%%%%%%%%%%%%%%%%%%%%%%%%%%%%%%%%%%%
\paragraph{Методи дослідження}

\textcolor{red}{
Мотивуючись цим, вибираємо часовий метод для аналізу імпульсного випромінювання,
а саме метод еволюційних рівнянь. В якості основи для метода є вилучення 
поперечних компонент поля методом Рімана-Вольтера, також відомого у вітчизняній 
літературі як метод еволюційних рівнянь. Це дозволяє звести задачу розв'язання 
системи рівнянь Максвела до розв'язання диференціального рівняння другого 
порядку в часних похідних. У випадку вакуумного середовища це рівняння 
зводиться до рівняння Клейна-Гордона.}

В данній роботі використано метод теорії збуджень для побудови рекурентного
ітеративного методу врахування нелінійності.

Напрям пошуку солітоноподібних розвя'зків задачі випромінювання спрямовано 
згідно досліджень, що передбачають їх появу в нелінійний оптиці. Той факт,
що за основні ідеї та положення нелінійної оптики близькі до положень 
нелінійної радіофізики в теоретичномц плані дозволяє говорити про правильний 
напрям досліджень.

\textcolor{red}{Рекурентні нейронні мережі}

\textcolor{red}{Метод перехідної функції}

%%%%%%%%%%%%%%%%%%%%%%%%%%%%%%%%%%%%%%%%%%%%%%%%%%%%%%%%%%%%%%%%%%%%%%%%%%%%%%%
\paragraph{Наукова новизна отриманих результатів}

\textcolor{red}{Розв'язок декількох задач випромінювання в часовій області для 
лінійного та нелінійного простору}

\textcolor{red}{Адаптація методу еволюційних рівнянь для чисельного розв'язку
та побудова відповідного програмного комплексу, що розповсюджується під 
ліцензією GPL, як один з проектів GNU спів-товариства}

\textcolor{red}{Авторський метод синтезу протоколів передачі інформації}

%%%%%%%%%%%%%%%%%%%%%%%%%%%%%%%%%%%%%%%%%%%%%%%%%%%%%%%%%%%%%%%%%%%%%%%%%%%%%%%
\paragraph{Практичне значення отриманих результатів}

\begin{enumerate} 
	\item Шведський проект по аналізу властивостей надпровідникових матеріалів
	\item Передача інформації імпульсами на велику відстань
	\item Анаітичний розв'язок для лінзевих антен збуджених TEM рупором
	\item Оцінка нелінійних явищ що супроводжують випромінювання LIRA-и
	\item Новий підхід до прийому надширокосмугового сигналу без АЦП
\end{enumerate} 

%%%%%%%%%%%%%%%%%%%%%%%%%%%%%%%%%%%%%%%%%%%%%%%%%%%%%%%%%%%%%%%%%%%%%%%%%%%%%%%
\paragraph{Особистий вклад дисертанта}

Аналітичний розв'язок задачі плаского диску у всіх точках
Розв'язок задачі плаского диску слабкого нелінійного простору
Авторський метод синтезу протоколів передачі інформації

%%%%%%%%%%%%%%%%%%%%%%%%%%%%%%%%%%%%%%%%%%%%%%%%%%%%%%%%%%%%%%%%%%%%%%%%%%%%%%%
\paragraph{Апробація результатів дослідження}

\begin{enumerate} 
	\item Erasmus+
	\item Upsala Univ.
	\item Young scientist award UWBUSIS 2018
\end{enumerate} 

%%%%%%%%%%%%%%%%%%%%%%%%%%%%%%%%%%%%%%%%%%%%%%%%%%%%%%%%%%%%%%%%%%%%%%%%%%%%%%%
\paragraph{Публікації}

%%%%%%%%%%%%%%%%%%%%%%%%%%%%%%%%%%%%%%%%%%%%%%%%%%%%%%%%%%%%%%%%%%%%%%%%%%%%%%%
\paragraph{Зміст роботи}

%%%%%%%%%%%%%%%%%%%%%%%%%%%%%%%%%%%%%%%%%%%%%%%%%%%%%%%%%%%%%%%%%%%%%%%%%%%%%%%
\paragraph{Подяка}

\chapter{Метод еволюційних рівнянь}
\label{ch:evolution}

%%%%%%%%%%%%%%%%%%%%%%%%%%%%%%%%%%%%%%%%%%%%%%%%%%%%%%%%%%%%%%%%%%%%%%%%%%%%%%
\section{Матеріальні рівняння середовища}

Електромагнітні властивості середовища можна математично описати шляхом
визначення векторів електричної $ \vect{D} $ та магнітної $ \vect{B} $ 
індукції за допомогою математичних рівнянь.

\begin{equation} \label{eq:MInduct}
\vect{D} = \epsilon_0 \vect{E} + \func{\vect{P}}{\vect{E},\vect{H}}
\end{equation}

\begin{equation} \label{eq:EInduct} 
\vect{B} = \mu_0 \vect{H} + \mu_0 \func{\vect{M}}{\vect{E},\vect{H}}
\end{equation}

Нелінійне середовище характеризується нелінійною залежністю поляризації
$ \vect{P} $ і намагніченості $ \vect{M} $ від векторів напруження 
електромагнітного поля. 

\textcolor{red}{Коли Р залежить від Н? Чи можна її не враховувати 
надалі. Приклади.}

\textcolor{red}{Яка фізична остова лежить у відмінності розмірностей доданих?}

В загальному випадку вектор поляризації має довільній вид та залежать від 
магнітної та електричної складової поля, а у лінійному випадку має вид
$ \epsilon \vect{E} $. Відносна діелектрична проникність середовища 
$ \epsilon $ взагалі є матриця, кожний з елементів якої, залежить від 
повного переліку незалежних координат та часу. Розглянемо шарувате середовище, 
як середу для розповсюдження і припустимо що фронт хвильового пакету проходить 
через шари під прямим кутом, тоді $ \epsilon $ є скалярна функція, що залежить 
лише від поздовжньої координати та часу. Аналогічні міркування можна провести 
і для вектора намагніченості.

\textcolor{red}{Уточнити чи не є $ \epsilon $ тензором для нелінійних 
компонент.}

Для задач слабкої нелінійності оптичної фізики використовується ряд Тейлора, 
так як при \textcolor{red}{не надто сильних полях} вклад більших степенів,
дійсно, мінімізується за рахунок невеликого відхилення від лінійної
функції \textcolor{red}{[джерело]}.

\begin{equation} \label{eq:polar}
\vect{P} = \epsilon_0 \left( \epsilon - 1 \right) \vect{E} + 
\vect{P^\prime}  = \epsilon_0 \left( \epsilon - 1 \right)
\vect{E} + \sum\limits_{i=2}^\infty  {\chi^e}_i \vect{E}^i 
\end{equation}

\textcolor{lightgray}{ \begin{equation*} \begin{aligned}
\vect{D} = \epsilon_0 \epsilon \vect{E} + \vect{P^\prime}
\end{aligned} \end{equation*} }

\begin{equation} \label{eq:magnit}
\vect{M} = \left( \mu - 1 \right) \vect{H} + 
\vect{M^\prime} = \left( \mu - 1 \right)
\vect{H} + \sum\limits_{i=2}^\infty  {\chi^m}_i \vect{H}^i 
\end{equation}

\textcolor{lightgray}{ \begin{equation*} \begin{aligned}
\vect{B} = \mu_0 \mu  \vect{H} + \mu_0 \vect{M^\prime}
\end{aligned} \end{equation*} }

Тут перший додаток має особливий фізичний смисл -- це лінійна складова поля та 
складова з найбільшим абсолютнім значенням коефіцієнту при векторі 
напруженості. Фізично, коефіцієнт є відносною проникністю середовища для 
відповідного степеню поля. Всі додатки крім першого це нелінійні складові поля,
кожен з яких має свій фізичний смисл.  Позначмо суму нелінійних складових 
векторів поляризації та намагніченості $ \vect{P^\prime} $ та 
$ \vect{M^\prime} $ відповідно.

\textcolor{red}{Смисл перших 5и додатків (таблиця).}

%%%%%%%%%%%%%%%%%%%%%%%%%%%%%%%%%%%%%%%%%%%%%%%%%%%%%%%%%%%%%%%%%%%%%%%%%%%%%%
\section{Рівняння Максвела}

\textcolor{red}{Закон Ампера}
\begin{equation} \label{eq:AmpereLow}
\crossprod{\nabla}{\vect{H}} = 
\frac{\partial \vect{D}}{\partial t} + \vect{J^\sigma} + \vect{J^e}
\end{equation}

\textcolor{red}{Закон индукції Фарадея}
\begin{equation} \label{eq:FaradayInduction}
-\crossprod{\nabla}{\vect{E}} =
\frac{\partial \vect{B}}{\partial t} + \vect{J^{h}}
\end{equation}

\textcolor{red}{Теорема Гаусса}
\begin{equation} \label{eq:GaussTheorem}
\dotprod{\nabla}{\vect{D}} = \rho^\sigma + \rho^e
\end{equation}

\textcolor{red}{Теорема Гаусса для магнітного поля}
\begin{equation} \label{eq:GaussMagnetic}
\dotprod{\nabla}{\vect{B}} = \rho^h
\end{equation}

%%%%%%%%%%%%%%%%%%%%%%%%%%%%%%%%%%%%%%%%%%%%%%%%%%%%%%%%%%%%%%%%%%%%%%%%%%%%%%
\subsection{Узагальнене джерело поля для задач випромінювання}

Додатки з нелінійними складовими векторів поляризації та намагніченості мають
розмірність густин струму, відповідно. Введемо узагальнений електричний 
$ \vect{J} $ та $ \vect{I} $ магнітній струми таким чином, щоб ці додатки 
не заважали майбутнім міркуванням. Ця дія відповідає фізичному змісту цих 
додатків та не порушує математичної консеквентності, що буде обумовлено далі.

\begin{equation*}
\vect{J} = \partder{\vect{P^\prime}}{t} + 
\vect{J^\sigma} + \vect{J^e}
\end{equation*}

\begin{equation*}
\vect{I} = \mu_0 \partder{\vect{M^\prime}}{t} + \vect{J^h}
\end{equation*}

Підставляючи поляризацію \eqref{eq:polar} і намагніченість 
\eqref{eq:magnit} до матеріальних рівнянь \eqref{eq:EInduct} и 
\eqref{eq:MInduct} з наступною підставковою в роторні рівняння Максвелла
\eqref{eq:AmpereLow} и \eqref{eq:FaradayInduction} отримаємо наступне: 

\textcolor{lightgray}{ \begin{equation*} \begin{aligned}
\crossprod{\nabla}{\vect{H}} = \epsilon_0 \partder{}{t} \left[ 
\vect{E} + \left( \epsilon - 1 \right) \vect{E} \right] + 
\partder{\vect{P^\prime}}{t} + \vect{J^\sigma} + \vect{J^e}= \\
= \epsilon_0 \partder{}{t} \left( \epsilon \vect{E} \right) +
\partder{\vect{P^\prime}}{t} + \vect{J^\sigma} + \vect{J^e} = \epsilon_0 \left( \partder{\epsilon}{t} 
\vect{E} + \epsilon \partder{\vect{E}}{t} \right) + 
\partder{\vect{P^\prime}}{t} + \vect{J^\sigma} + \vect{J^e}
\end{aligned} \end{equation*} }

\begin{equation} \label{eq:rotHfromE}
\crossprod{\nabla}{\vect{H}} = 
\epsilon_0 \partder{}{t} \left( \epsilon \vect{E} \right) + \vect{J}
\end{equation}

\begin{equation} \label{eq:rotEfromH} 
- \crossprod{\nabla}{\vect{E}} = 
\mu_0 \partder{}{t} \left( \mu \vect{H} \right) + \vect{I}
\end{equation}

Схожа ситуація і для джерел що представлена зарядами. Нехай наступні вирази
опишуть узагальнену електричну $ \varrho $ (ро) та магнітну $ g $ густини 
заряду.

\begin{equation*}
\varrho = \rho^\sigma + \rho^e - \dotprod{\nabla}{\vect{P^\prime}}
\end{equation*}

\begin{equation*}
g = \rho^h - \mu_0 \dotprod{\nabla}{\vect{M^\prime}}
\end{equation*}

Підставляючи поляризацію та намагніченість до відповідних формулювань теореми 
Гаусса отримаємо її вигляд для задачі слабкої нелінійності в анізотропному 
середовищі.

\textcolor{lightgray}{ \begin{equation*} \begin{aligned}
\dotprod{\nabla}{ \left( \epsilon_0 \epsilon \vect{E} + 
\vect{P^\prime} \right) } = \rho^\sigma + \rho^e \\
\dotprod{\nabla}{ \epsilon_0 \epsilon \vect{E} } = \rho^\sigma + \rho^e -
\dotprod{\nabla}{ \vect{P^\prime} }
\end{aligned} \end{equation*} }

\begin{equation} \label{eq:divE} 
\epsilon_0 \dotprod{\nabla}{ \epsilon \vect{E} } = \varrho
\end{equation}

\begin{equation} \label{eq:divH}
\mu_0 \dotprod{\nabla}{ \mu \vect{H} } = g
\end{equation}

%%%%%%%%%%%%%%%%%%%%%%%%%%%%%%%%%%%%%%%%%%%%%%%%%%%%%%%%%%%%%%%%%%%%%%%%%%%%%%
\subsection{Відокремлення поздовжних компонент поля}

Диференціальні рівняння першого порядку \eqref{eq:divE}, \eqref{eq:divH} та 
векторні другого \eqref{eq:rotHfromE}, \eqref{eq:rotEfromH} формують систему 
рівнянь Максвелла відносно невідомих векторних величин $ \vect{E} $ і 
$ \vect{H} $.

Для спрощення цієї системи пропонується використати метод розділення змінних
Фур'є. Аналогічно до методу функції Гріна з класичної електродинаміки, 
спрощення відбувається шляхом зменшення кількості невідомих 
\textcolor{red}{Джерело}, вилучаючи їх з рівняння. 

Метод Функції Гріна як і будь-який метод частотної області, вибирає саме час, 
як змінну для виключення, обмежуючи себе розгляданням квазі-стаціонарних 
процесів. Метод еволюційних рівнянь, в свою чергу, пропонує виключення 
просторової змінної. З трьох просторових координат можна виділити одну -- вісь 
розповсюдження поля. 

Виключення саме цієї просторової залежності зумовлено тісним зв'язком 
координати розповсюдження з координатою часу через принцип причинності. Його 
сутність в термінології спеціальної теорії відносності полягає в тому, що дві 
події можуть бути причинно зв'язані одна з одної тоді, і тільки коли, інтервал 
між ними часоподібний, що напряму слідує з того, що ніяка взаємодія не може 
розповсюджуватись швидше за світло. \cite[ст. 22]{LandauII}. В 
електродинамічному сенсі це означає, що поле не може розповсюдитись далі у 
вільному просторі, ніж може пройти світло за той самий час та по тій самій осі 
випромінювання. Математично це можна записати, як $ ct - z > 0 $, де $ z $
поздовжна просторова координата розповсюдження, а $ c = 2,998 \cdot 10^8 $ м/с 
-- фундаментальна константа, швидкість світла в вакуумі.

В рівнянні \eqref{eq:rotHfromE} відокремимо векторну компоненту 
$ \vect{z_0} $. Користуючись визначенням векторного добутку лінійної 
комбінації векторів отримаємо два незалежні рівняння.

\textcolor{lightgray}{ \begin{equation*} \begin{aligned}
\rot{\vect{A}} = \crossprod{\nabla}{\vect{A}} = \crossprod
{\left( \nabla_\perp + \vect{z_0} \partder{}{z} \right)}
{\left( \vect{A_\perp} + \vect{z_0} A_z \right)} = \\
\crossprod{\nabla_\perp}{\vect{A_\perp}} + 
\crossprod{\nabla_\perp}{\vect{z_0} A_z}} +
\crossprod{\vect{z_0} \partder{}{z}}{\vect{A_\perp}} +
\crossprod{\vect{z_0} \partder{}{z}}{\vect{z_0} A_z}}

\end{aligned} \end{equation*} }

\begin{equation} \label{eq:rotEt} 
a + b = 1
\end{equation}

\begin{equation} \label{eq:rotEz}
a - b = 1
\end{equation}

 
%%%%%%%%%%%%%%%%%%%%%%%%%%%%%%%%%%%%%%%%%%%%%%%%%%%%%%%%%%%%%%%%%%%%%%%%%%%%%%
\subsection{Уравнения Максвелла в матричной форме}

%%%%%%%%%%%%%%%%%%%%%%%%%%%%%%%%%%%%%%%%%%%%%%%%%%%%%%%%%%%%%%%%%%%%%%%%%%%%%%
\section{Построение модового базиса}


%%%%%%%%%%%%%%%%%%%%%%%%%%%%%%%%%%%%%%%%%%%%%%%%%%%%%%%%%%%%%%%%%%%%%%%%%%%%%%
\section{Эволюционные уравнения}


%%%%%%%%%%%%%%%%%%%%%%%%%%%%%%%%%%%%%%%%%%%%%%%%%%%%%%%%%%%%%%%%%%%%%%%%%%%%%%
\section{Итеративный учет слабой нелинейности}
	% Метод эволюционных уравнений
%\chapter{Излучение плоского диска}
\label{ch:pdisk}

\section{Постановка задачи}
\subsection{Среда распространения и начальные условия}
\subsection{Эффект Керра}
\subsection{Геометрия источника поля}

\section{Линейные коэффициенты}

\section{Линейное поле}

\section{Нелинейные коэффициенты}

\section{Нелинейная поправка}
	% Излучение плоскаго диска

%\chapter*{Висновки}

\begin{enumerate}
%
\item Побудовано аналітичне розв'язання у вигляді кусково визначеної функції для 
задачі випромінювання круглої апертури при нестаціонарному збуджені у вигляді 
прямокутної функції. Розв'язок отримано без наближення дальної зони та визначено 
для всіх точок спостереження в кожен момент часу. Використання моделі круглої 
апертури, як моделі антен типу LIRA перевірено на експерементальних даних в 
окремих точках та на даних отриманих методом FDTD з комерційного електромагнітного 
симулятора CST Studio.
%
\item Отримане розв'язання задачі випромінювання плаского диску при збуджені у 
вигляді функції Хевісайда в лінійному наближенні має чітку просторово-часову 
зональність та ілюструє твердження Фарадея, що випромінює не антена, а простір 
довколі неї. Отримані області випромінювання наступають послідовно для довільної 
точки спостереженя. Остання за часом настання область $ S_3 $ відповідає 
стаціонарному (усталеному) процесу випромінювання, коли всі точки апертури 
поєднані зі спостерегічем за принципом причинності. Настанню усталеного процесу 
передує область деякого транзитивного процесу $ S_2 $, поки поле від всієї 
апертури не досягне спостерігача. Найпершою для спостерігача просторово-часовою 
областю випромінювання в прожекторній зоні круглої апертури настає область 
електромагнітного снаряду $ S_1 $, де з хвилі у ТЕМ рупора формується ТЕ хвиля 
у вільному просторі.
%
\item 
%
\end{enumerate}
%


%\begin{bibset}{Список використаних джерел}
\bibliographystyle{acm}
% Для сортування літератури за алфавітом використовуйте
%\bibliographystyle{gost71s}
\bibliography{../my,../import}
%\end{bibset}
%GATHER{xampl-mybib.bib}

%\begin{bibset}[a]{Список публікацій автора}
%\bibliographystyle{acm}
%\bibliography{mybib}
%\end{bibset}


\appendix
\chapter{Властивості добудків векторів}
\label{ch:vector}

%%%%%%%%%%%%%%%%%%%%%%%%%%%%%%%%%%%%%%%%%%%%%%%%%%%%%%%%%%%%%%%%%%%%%%%%%%%%%%%
\section{Скалярний добуток}

Визначення:

\begin{equation*}
\dotprod{\vect{A}}{\vect{B}} = \sum_{i} A_i B_i
\end{equation*}

Комутативність скалярного добуду:

\begin{equation*}
\dotprod{\vect{A}}{\vect{B}} = \dotprod{\vect{B}}{\vect{A}}
\end{equation*}

Асоціативність множення на скаляр:

\begin{equation*}
\dotprod{c \vect{A}}{\vect{B}} = \dotprod{\vect{A}}{ c \vect{B}} =
c \dotprod{\vect{A}}{\vect{B}}
\end{equation*}

Дистрибутивність додавання:

\begin{equation*}
\dotprod{\vect{A}}{\left( \vect{B} + \vect{C} \right)} = 
\dotprod{\vect{A}}{\vect{B}} + \dotprod{\vect{A}}{\vect{C}} 
\end{equation*}

%%%%%%%%%%%%%%%%%%%%%%%%%%%%%%%%%%%%%%%%%%%%%%%%%%%%%%%%%%%%%%%%%%%%%%%%%%%%%%%
\section{Змішаний добуток}

Визначення:

\begin{equation*}
\triple{\vect{A}}{\vect{B}}{\vect{C}} = 
\dotprod{\vect{A}}{\crossprod{\vect{B}}{\vect{C}}} =
\dotprod{\crossprod{\vect{A}}{\vect{B}}}{\vect{C}}
\end{equation*}

Властивість кососиметричності:

\begin{equation*}
- \triple{\vect{A}}{\vect{B}}{\vect{C}} = 
\triple{\vect{A}}{\vect{C}}{\vect{B}} =
\triple{\vect{B}}{\vect{A}}{\vect{C}} = 
\triple{\vect{C}}{\vect{B}}{\vect{A}}
\end{equation*}

%%%%%%%%%%%%%%%%%%%%%%%%%%%%%%%%%%%%%%%%%%%%%%%%%%%%%%%%%%%%%%%%%%%%%%%%%%%%%%%
\section{Векторний добуток}

Властивість самомноження:

\begin{equation*} 
\crossprod{\vect{A}}{\vect{A}} = 0
\end{equation*}

Антікомутативність векторного добуду:

\begin{equation*} 
- \crossprod{\vect{A}}{\vect{B}} = \crossprod{\vect{B}}{\vect{A}}
\end{equation*}

Асоціативність множення на константу:

\begin{equation*} 
\crossprod{c \vect{A}}{\vect{B}} = \crossprod{\vect{A}}{c \vect{B}} =
c \crossprod{\vect{A}}{\vect{B}}
\end{equation*}

Дистрибутивність додавання:

\begin{equation*} 
\crossprod{\vect{A}}{ \left( \vect{B} + \vect{C} \right) } = 
\crossprod{\vect{A}}{\vect{B}} + \crossprod{\vect{A}}{\vect{C}}
\end{equation*}

Тотожність Лагранжа:

\begin{equation*}
\crossprod{\vect{A}}{\crossprod{\vect{B}}{\vect{C}}} =
\dotprod{\vect{B}}{\dotprod{\vect{A}}{\vect{C}}} - 
\dotprod{\vect{C}}{\dotprod{\vect{A}}{\vect{B}}}
\end{equation*}

Тотожність Якобі:

\begin{equation*}
\crossprod{\vect{A}}{\crossprod{\vect{B}}{\vect{C}}} =
\crossprod{\vect{B}}{\crossprod{\vect{A}}{\vect{C}}} +
\crossprod{\crossprod{\vect{A}}{\vect{B}}}{\vect{C}} 
\end{equation*}
	% Властивості векторних операторів
\chapter{Властивості функції Бесселя першого роду}
\label{ch:bessel}

\section{Визначення на лінійні властивості}

\begin{equation}
J_{-n} \left( z \right) = \left( -1 \right)^n J_n \left( z \right)
\end{equation}

\begin{equation}
J_{n+1} \left( z \right) + J_{n-1} \left( z \right) = 
\frac{2n}{z} J_n \left( z \right)
\end{equation}

\section{Інтегропохідні властивості}

\begin{equation}
2 \derivat{}{z} J_n \left( z \right) = 
J_{n-1} \left( z \right) - J_{n+1} \left( z \right) 
\end{equation}

\begin{equation}
\derivat{}{z} J_n \left( z \right) = 
J_{n-1} \left( z \right) - \frac{n}{z} J_{n} \left( z \right) 
\end{equation}

\begin{equation}
\derivat{}{z} J_n \left( z \right) = 
\frac{n}{z} J_{n} \left( z \right) - J_{n+1} \left( z \right) 
\end{equation}

\begin{equation}
\derivat{}{z} \frac{ J_n \left( z \right) }{ z^n }  = 
- \frac{ J_{n+1} \left( z \right) }{ z^n }
\end{equation}

\begin{equation}
\derivat{}{z} \left( z^n J_n \left( z \right) \right)  = 
z^n J_{n-1} \left( z \right)
\end{equation}

\section{Інтеграл 1}

\begin{equation*}
I_1 = \int\limits_{0}^{\infty} \frac{d\nu}{\nu} 
J_1 \left( \nu R \right) J_1 \left( \nu \rho \right) 
J_0 \left( \nu \sqrt{c^2 t^2 - z^2} \right)
\end{equation*}

Інтеграли такого виду зустрічаються в \cite[ст. 398]{Watson1922}.
\begin{equation} \begin{aligned} \label{eq:intJJJtable}
\int\limits_{0}^{\infty} \frac{d t}{t^{\lambda + \nu}} 
J_\mu \left( at \right) J_\nu \left( bt \right) J_\nu \left( ct \right) =
\frac{ \left( bc/2 \right) ^\nu }
{ \Gamma \left( \nu + 1/2 \right) \Gamma \left( 1/2 \right) } \cdot \\
\cdot \int\limits_{0}^{\infty} \int\limits_{0}^{\pi}
\frac{J_\mu \left( at \right) J_\nu \left( \omega t \right)}
{\omega^\nu t^\lambda} \sin^{2\nu}{\phi} d\phi dt, \\
\omega = \sqrt{b^2 + c^2 - 2bc \cos \phi} \\
\Re \left( \nu \right) > - \frac{1}{2};
\Re \left( \mu + \nu + 2 \right) > \Re \left( \lambda + 1 \right) > 0
\end{aligned} \end{equation}
%
\textcolor{lightgray}{ \begin{equation*} \begin{aligned}
a = \sqrt{c^2 t^2 - z^2}; b = R; c = \rho; \lambda = 0 \\
\nu = 1; \mu = 0; \omega = \sqrt{R^2 + \rho^2 - 2 \rho R \cos \phi} \\
\int\limits_{0}^{\infty} \frac{d\nu}{\nu} 
J_1 \left( \nu R \right) J_1 \left( \nu \rho \right) 
J_0 \left( \nu \sqrt{c^2 t^2 - z^2} \right) = 
\frac{\rho R}{ 2 \Gamma \left( 3/2 \right) \Gamma \left( 1/2 \right) } \cdot \\
\int\limits_{0}^{\pi} 
\frac{\sin^2{\phi}}{\sqrt{R^2 + \rho^2 - 2 \rho R \cos \phi}}
\int\limits_{0}^{\infty} d \nu J_1 \left( \nu \omega \right) 
J_0 \left( \nu \sqrt{c^2 t^2 - z^2} \right) d \phi
\end{aligned} \end{equation*} }
%
\textcolor{lightgray}{ \begin{equation*} \begin{aligned}
\Gamma \left( 3/2 \right) \Gamma \left( 1/2 \right) = 
\frac{\sqrt{\pi}}{2} \cdot \sqrt{\pi} = \frac{\pi}{2} 
\end{aligned} \end{equation*} }
%
\textcolor{lightgray}{ \begin{equation*} \begin{aligned}
I_1 = \frac{\rho R}{\pi} \int\limits_{0}^{\pi} 
\frac{\sin^2{\phi}}{\sqrt{R^2 + \rho^2 - 2 \rho R \cos \phi}}
\int\limits_{0}^{\infty} d \nu J_1 \left( \nu \omega \right) 
J_0 \left( \nu \sqrt{c^2 t^2 - z^2} \right) d \phi
\end{aligned} \end{equation*} }

Використання формули \eqref{eq:intJJJtable} дозволяє спростити $ I_1 $ до 
інтегралу по двом функціям Бесселя в ядрі замість трьох. Використаємо наступну 
формулу з \cite{Golubovic2013} для пошуку рішення нового інтегралу. 
%
\begin{equation} \begin{aligned} \label{eq:intJJtable}
\int\limits_{0}^{\infty} d \nu
J_n \left( a \nu \right) J_{n-1} \left( b \nu \right) = \begin{cases} 
b^{n-1} / a^n , 0 < b < a \\
1 / 2 b , 0 < a = b \\
0 , 0 < a < b
\end{cases} 
\end{aligned} \end{equation}
%
\textcolor{lightgray}{ \begin{equation*} \begin{aligned}
\int\limits_{0}^{\infty} d \nu J_1 \left( \nu \omega \right) 
J_0 \left( \nu \sqrt{c^2 t^2 - z^2} \right) = \begin{cases}
\left( R^2 + \rho^2 - 2 \rho R \cos \phi \right)^{-1/2}, 0 < b < a \\
\frac{1}{2} \left( c^2 t^2 - z^2 \right)^{-1/2}, 0 < a = b \\
0 , 0 < a < b
\end{cases} 
\end{aligned} \end{equation*} }

Згідно з умовами інтегрування з \eqref{eq:intJJtable}, повинно виконуватись 
співвідношення $ 0 < b \leq a $ для того, щоб значення інтегралу $ I_1 $ 
існувало та було б відмінне від нуля. Застосовуючи цю умову в рамках поставленої 
задачі бачимо її фізичність: відсутність протиріч з принципом причинності. 
Доцільно виписати діапазон значень $ \phi $ користуючись областю значень 
інтегралу з \eqref{eq:intJJJtable} та умови інтегрування \eqref{eq:intJJtable}.
%
\textcolor{lightgray}{ \begin{equation*} \begin{aligned}
\sqrt{R^2 + \rho^2 - 2 \rho R \cos \phi} \geq \sqrt{c^2 t^2 - z^2} \\
R^2 + \rho^2 - 2 \rho R \cos \phi \geq c^2 t^2 - z^2 \\
\cos \phi \leq \frac{R^2 + \rho^2}{2 \rho R} - \frac{c^2 t^2 - z^2}{2 \rho R} \\
\phi \leq \arccos \left( \frac{\rho^2 + R^2}{2 \rho R} - 
\frac{c^2 t^2 - z^2}{2 \rho R} \right), 0 \leq \phi \leq \pi
\end{aligned} \end{equation*} }
%
\textcolor{red}{ \begin{equation*}
0 \leq \phi \leq \arccos \left( \frac{\rho^2 + R^2}{2 \rho R} - 
\frac{c^2 t^2 - z^2}{2 \rho R} \right)
\end{equation*} }

Зазначимо, що тригонометрична функція $ \arccos $ завжди менша за $ \pi $,
тому остання нерівність не протирічить тому, що $ \phi \leq \pi $.

Очевидно що застосування \eqref{eq:intJJtable} накладає умови на співвідношення
просторових координат а часу. Користуючись областю значень 
$ -1 \leq \cos \phi \leq 1 $, запишемо систему що визначає світловий конус 
\cite[ст. 22]{LandauII} для компоненти поля, що мітить даний інтеграл. 
\textcolor{red}{ Так як значення інтегралу при від'ємних значеннях $ \cos \phi $ 
рівні нулю. Обмежимо область значень знизу: $ 0 \leq \cos \phi \leq 1 $ }. 
%
\textcolor{lightgray}{ \begin{equation*} \begin{aligned}
\begin{cases}
\frac{\rho^2 + R^2}{2 \rho R} - \frac{c^2 t^2 - z^2}{2 \rho R} \leq 1 \\
\frac{\rho^2 + R^2}{2 \rho R} - \frac{c^2 t^2 - z^2}{2 \rho R} \geq - 1
\end{cases}
\begin{cases}
\rho^2 + R^2 - c^2 t^2 + z^2 \leq 2 \rho R \\
\rho^2 + R^2 - c^2 t^2 + z^2 \geq - 2 \rho R
\end{cases} 
\end{aligned} \end{equation*} }
%
\textcolor{lightgray}{ \begin{equation*} \begin{aligned}
\begin{cases}
0 \leq \left( \rho - R \right)^2 \leq c^2 t^2 - z^2 \\ 
\left( \rho + R \right)^2 \geq c^2 t^2 - z^2 \geq 0
\end{cases} 
\end{aligned} \end{equation*} }
%
\textcolor{red}{ \begin{equation*} \begin{aligned}
\rho^2 + R^2 \geq c^2 t^2 - z^2 \geq \left( \rho - R \right)^2
\end{aligned}  \end{equation*} }
%
\begin{equation*}
\left( \rho + R \right)^2 \geq c^2 t^2 - z^2 \geq \left( \rho - R \right)^2
\end{equation*}
%
\textcolor{lightgray}{ \begin{equation*} \begin{aligned}
I_1 = \frac{\rho R}{\pi} \int\limits_{0}^{\pi} 
\frac{\sin^2{\phi}}{\sqrt{R^2 + \rho^2 - 2 \rho R \cos \phi}}
\int\limits_{0}^{\infty} d \nu J_1 \left( \nu \omega \right) 
J_0 \left( \nu \sqrt{c^2 t^2 - z^2} \right) d \phi = \\
= \frac{\rho R}{\pi} \int\limits_{0}^{\psi} 
\frac{\sin^2{\phi}}{\sqrt{R^2 + \rho^2 - 2 \rho R \cos \phi}}
\frac{1}{\sqrt{R^2 + \rho^2 - 2 \rho R \cos \phi}} d \phi = \\
= \frac{\rho R}{\pi} \int\limits_{0}^{\psi}
\frac{\sin^2{\phi}}{R^2 + \rho^2 - 2 \rho R \cos \phi} d \phi = 
\frac{\rho}{\pi R} \int\limits_{0}^{\psi}
\frac{\sin^2{\phi}}{1 + \frac{\rho^2}{R^2} - \frac{2 \rho}{R} \cos \phi} d \phi
\end{aligned} \end{equation*} }

Застосовуючи формули \eqref{eq:intJJtable} і \eqref{eq:intJJJtable} приведемо 
початкову форму $ I_1 $ до наступного виду.
%
\begin{equation*} \begin{aligned}
I_1 = \frac{\rho}{\pi R} \int\limits_{0}^{\psi}
\frac{\sin^2{\phi}}{1 + \frac{\rho^2}{R^2} - 
\frac{2 \rho}{R} \cos \phi} d \phi \\
\psi = \arccos \left( \frac{\rho^2 + R^2}{2 \rho R} - 
\frac{c^2 t^2 - z^2}{2 \rho R} \right)
\end{aligned} \end{equation*}

\textcolor{red} {Зауважимо, що для $ \psi = 0 $ підінтегральна функція не 
відповідає \eqref{eq:intJJtable} та і значення інтегралу в цій точці виходить не 
визначене. Але, згідно властивостей інтегралів Рімана, значення інтегралу в одній 
точці не впливає на значення означеного інтегралу}
%
\textcolor{lightgray}{ \begin{equation*} \begin{aligned}
\int \frac{\sin^2{\phi}}{a + b \cos \phi} d \phi = 
\int \frac{1 - \cos^2{\phi}}{a + b \cos \phi} d \phi = 
\int\frac{d \phi}{a + b \cos \phi}  -
\int \frac{\cos^2{\phi}}{a + b \cos \phi} d \phi = \\
= \int \frac{d \phi}{a + b \cos \phi}  - 
\int \frac{\cos^2{\phi}}{a + b \cos \phi} d \phi -
\frac{a}{b} \int \frac{\cos \phi}{a + b \cos \phi} d \phi + \\
+ \frac{a}{b} \int \frac{\cos \phi}{a + b \cos \phi} d \phi = 
\int \frac{d \phi}{a - b \cos \phi} +
\frac{a}{b} \int \frac{\cos \phi}{a + b \cos \phi} d \phi - \\
- \int \frac{\cos^2{\phi} + \frac{a}{b} \cos \phi} {a + b \cos \phi} d \phi =
\int \frac{d \phi}{a + b \cos \phi} + 
\frac{a}{b} \int\limits_{0}^{\psi} \frac{\cos \phi}{a + b \cos \phi} d \phi -
\end{aligned} \end{equation*} }
%
\textcolor{lightgray}{ \begin{equation*} \begin{aligned}
- \frac{1}{b} \int \frac{\cos \phi + a/b} {a/b +  \cos \phi} \cos \phi d \phi = 
\int \frac{d \phi}{a + b \cos \phi} + 
\frac{a}{b} \int \frac{\cos \phi}{a + b \cos \phi} d \phi - \\
- \frac{1}{b} \int \cos \phi d \phi = \int \frac{d \phi}{a + b \cos \phi} - 
\frac{1}{b} \int \cos \phi d \phi + \frac{a}{b^2} \int
\frac{a - a + b \cos \phi}{a + b \cos \phi} d \phi = \\ 
= \int\frac{d \phi}{a + b \cos \phi} - \frac{1}{b} \int \cos \phi d \phi +
\frac{a}{b^2} \int \frac{a + b \cos \phi}{a + b \cos \phi} d \phi - \\ 
- \frac{a^2}{b^2} \int \frac{d \phi}{a + b \cos \phi} = 
\left( 1 - \frac{a^2}{b^2} \right) \int\frac{d \phi}{a + b \cos \phi} - 
\frac{1}{b} \int \cos \phi d \phi + \frac{a}{b^2} \int \phi =
\end{aligned} \end{equation*} }
%
\textcolor{lightgray}{ \begin{equation*} \begin{aligned}
= \left( 1 - \frac{a^2}{b^2} \right)
\int\limits_{0}^{\psi} \frac{d \phi}{a + b \cos \phi} -
\frac{\sin \psi - \sin 0}{b} + a \frac{\psi}{b^2} = \\
= \left( 1 - \frac{a^2}{b^2} \right)
\int\limits_{0}^{\psi} \frac{d \phi}{a + b \cos \phi} +
\frac{\sin \psi}{b} + a \frac{\psi}{b^2}
\end{aligned} \end{equation*} }
%
\textcolor{lightgray}{ \begin{equation*} \begin{aligned}
a = 1 + \frac{\rho^2}{R^2}; b = - \frac{2 \rho}{ R } \\
\frac{\pi R}{\rho} I_1 = \left( 1 - \frac{a^2}{b^2} \right)
\int\limits_{0}^{\psi} \frac{d \phi}{a + b \cos \phi} +
\frac{\sin \psi}{b} + a \frac{\psi}{b^2} = \\
= \left( 1 - \left( \frac{1 + \frac{\rho^2}{R^2}} 
{ \frac{2 \rho}{R} } \right)^2 \right) 
\int \limits_{0}^{\psi} \frac{d \phi}{1 + \frac{\rho^2}{R^2} -  
\frac{2 \rho}{R} \cos \phi} -
\frac{\sin \psi}{\frac{2 \rho}{ R }} + \left( 1 + \frac{\rho^2}{R^2} \right) 
\frac{\psi}{\left( \frac{\rho^2}{R^2} \right)^2} = \\
= \left( R^2 - \left( \frac{R^2 + \rho^2} 
{2 \rho} \right)^2 \right) 
\int \limits_{0}^{\psi} \frac{d \phi}{R^2 + \rho^2 - 2 \rho R \cos \phi} -
\frac{R}{2 \rho} \sin \psi + \frac{R^2}{\rho^2} 
\left( \frac{R^2}{\rho^2} + 1 \right) \psi  
\end{aligned} \end{equation*} }
%
\textcolor{lightgray}{ \begin{equation*} \begin{aligned}
\frac{4 \rho^2}{4 \rho^2} R^2 - \left( \frac{R^2 + \rho^2}{2 \rho} \right)^2 =
\frac{4 \rho^2 R^2 - R^4 - 2 \rho^2 R^2 - \rho^4}{4 \rho^2} =
- \frac{\left( \rho^2 - R^2 \right)^2}{4 \rho^2} 
\end{aligned} \end{equation*} }
%
\textcolor{lightgray}{ \begin{equation*} \begin{aligned}
\frac{\pi R}{\rho} I_1 = 
- \frac{\left( \rho^2 - R^2 \right)^2}{4 \rho^2} 
\int \limits_{0}^{\psi} \frac{d \phi}{R^2 + \rho^2 - 2 \rho R \cos \phi} - \\
- \frac{R}{2 \rho} \sin \psi + \frac{R^2}{\rho^2} 
\left( \frac{R^2}{\rho^2} + 1 \right) \psi 
\end{aligned} \end{equation*} }

Тригонометричними перетвореннями зведемо поточний вид $ I_1 $ до табличного 
інтегралу.
%
\begin{equation*} \begin{aligned}
I_1 = - \frac{\left( \rho^2 - R^2 \right)^2}{4 \pi \rho R} 
\int \limits_{0}^{\psi} \frac{d \phi}{R^2 + \rho^2 - 2 \rho R \cos \phi} - 
\frac{\sin \psi}{2 \pi} + \frac{R}{\rho} 
\left( \frac{R^2}{\rho^2} + 1 \right) \frac{\psi}{\pi}
\end{aligned} \end{equation*}

Таблична формула для неозначеного випадку інтегралу може буде знайдена в 
\cite[ст. 181]{ElementFunc1983}.
%
\begin{equation} \label{eq:caseTableIntegral}
\int \frac{d x}{a + b \cos{x}} = \begin{cases}
\frac{2}{\sqrt{a^2-b^2}} \arctan \frac{\sqrt{a^2-b^2} \tan \frac{x}{2}}
{a + b}, a^2 > b^2 \\
\frac{1}{\sqrt{b^2-a^2}} \ln 
\frac{\sqrt{b^2-a^2} \tan \frac{x}{2} + a + b}
{\sqrt{b^2-a^2} \tan \frac{x}{2} - a - b}, a^2 < b^2
\end{cases}
\end{equation}

Помітимо, що застосування формули змушує нас розглядати поле прожекторної зони
($ \rho < R $) окремо від поля в іншому простору ($ \rho > R $), через умову 
співвідношення $ a $ та $ b $. Отримаємо значення інтегралу $ I_1 $ для 
\textcolor{red}{області Френеля} в явному вигляді.
%
\textcolor{lightgray}{ \begin{equation*} \begin{aligned}
a^2 > b^2  \Rightarrow  
\left( R^2 + \rho^2 \right)^2 > 4 \rho^2 R^2 \\
R^4 + 2 \rho^2 R^2 + \rho^4 - 4 \rho^2 R^2 > 0 \Rightarrow 
\left( \rho^2 - R^2 \right)^2 > 0 \\
\rho > R
\end{aligned} \end{equation*} }
%
\textcolor{lightgray}{ Далі знадобиться: }
%
\textcolor{lightgray}{ \begin{equation*} \begin{aligned}
a^2 - b^2 = - \left( b^2 - a^2 \right) = 
R^4 + 2 \rho^2 R^2 + \rho^4 - 4 \rho^2 R^2 = \left( \rho^2 - R^2 \right)^2 \\
\lim_{\alpha \to 0} \tan{\alpha} = 0 \Rightarrow
\lim_{\alpha \to 0} \arctan \left( a \tan{\alpha} \right) = 0
\end{aligned} \end{equation*} }
%
\textcolor{lightgray}{ \begin{equation*} \begin{aligned}
\int \limits_{0}^{\psi} \frac{d \phi}{R^2 + \rho^2 - 2 \rho R \cos \phi} =
\left. \begin{cases}
\frac{2}{\rho^2 - R^2} \arctan \left( \frac{\rho^2 - R^2}
{\left( \rho - R \right)^2} \tan \frac{\phi}{2} \right), \rho > R \\
- \frac{1}{\rho^2 - R^2} \ln
\frac{- \left( \rho^2 - R^2 \right) \tan \frac{\phi}{2} + \left( \rho - R \right)^2}
{- \left( \rho^2 - R^2 \right) \tan \frac{\phi}{2} - \left( \rho - R \right)^2}
, \rho < R
\end{cases} \right|_{0}^{\psi} = \\
= \frac{1}{\rho^2 - R^2} \left. \begin{cases} 
2 \arctan \left( \frac{\rho + R}{\rho - R} \tan \frac{\phi}{2} \right) \\
- \ln \left| \frac{\tan \frac{\phi}{2} - \frac{\rho - R}{\rho + R}} 
{\tan \frac{\phi}{2} + \frac{\rho - R}{\rho + R}} \right|
\end{cases} \right|_{0}^{\psi} = 
\frac{1}{\rho^2 - R^2} \begin{cases} 
2 \arctan \left( \frac{\rho + R}{\rho - R} \tan \frac{\psi}{2} \right) \\
- 0 + \ln \left| \frac{\tan \frac{\psi}{2} - \frac{\rho - R}{\rho + R}}
{\tan \frac{\psi}{2} + \frac{\rho - R}{\rho + R}} \right|
\end{cases}
\end{aligned} \end{equation*} }
%
\begin{equation*} \begin{aligned}
I_1 \left( \rho > R \right) = 
\frac{R^2}{\rho^2} \frac{\rho^2 + R^2}{\pi \rho R} \psi - 
\frac{\sin \psi}{2 \pi} - \frac{\rho^2 - R^2}{2 \pi \rho R} 
\arctan \left( \frac{\rho + R}{\rho - R} \tan \frac{\psi}{2} \right)
\end{aligned} \end{equation*}

\textcolor{red}{ Застосуємо \eqref{eq:caseTableIntegral} для наявних лімітів 
користуючись формулою Ньютона та побачимо, що при аргументі $ \phi = 0 $ інтеграл 
стає неозначеним. Це наслідок неправильного використання формули 
\eqref{eq:intJJtable}. Прирівнявши $ \cos \phi $ до нуля побачимо, що рівняння 
світлового конуса починає описувати область просторово подібних інтервалів. Це 
пояснює чому інтеграл дає ірраціональні значення: це просторово подібний інтервал в 
термінології СТВ. Для того щоб нейтралізувати помилку внесену застосуванням 
інтеграла \eqref{eq:intJJtable} розглянемо $ \ln |f(r,t)| $ замість 
$ \ln f(r,t) $.} Тепер випишемо функціональну залежність $ I_1 $ для прожекторної 
зони.
%
\begin{equation*} \begin{aligned}
I_1 \left( 0 < \rho < R \right) = 
\frac{R^2}{\rho^2} \frac{\rho^2 + R^2}{\pi \rho R} \psi - 
\frac{\sin \psi}{2 \pi} - \frac{\rho^2 - R^2}{4 \pi \rho R} 
\ln \left| \frac{\tan \frac{\psi}{2} - \frac{\rho - R}{\rho + R}}
{\tan \frac{\psi}{2} + \frac{\rho - R}{\rho + R}} \right|
\end{aligned} \end{equation*}

Що стосовно поля при $ \rho = R $, тобто $ a^2 = b^2 $, то очевидно, що обидва 
випадки однаково підходять.
%
\begin{equation*} \begin{aligned}
\left. I_1 \right|_{\rho = R} = 
\left. I_1 \left(0 < \rho < R \right) \right|_{\rho = R} =
\left. I_1 \left( \rho > R \right) \right|_{\rho = R}
\end{aligned} \end{equation*}

На останок, спростимо тригонометричні вирази, що містять $ \psi $. Розглянемо 
$ \psi = \arccos f(r,t) $, де $ f(r,t) $ задовільна функція координат. 
Тоді $ f(r,t) = \cos \psi $. Зазначимо, що з означення відомо, що 
$ \psi \in \left[ 0, \pi \right] $, тому $ \sin \psi \geq 0 $. Таким чином:
%
\begin{equation*} \begin{aligned}
\sin \psi = \sqrt{1 - \cos^2{\psi}} = \sqrt{1 - f^2(r,t)}
\end{aligned} \end{equation*}

Згадуючи введене означення для $ \psi $ зашипимо, що
%
\textcolor{lightgray}{ \begin{equation*} \begin{aligned}
\psi = \arccos \left( \frac{\rho^2 + R^2}{2 \rho R} - 
\frac{c^2 t^2 - z^2}{2 \rho R} \right)
\end{aligned} \end{equation*} }
%
\textcolor{lightgray}{ \begin{equation*} \begin{aligned}
\sin \psi = \sqrt{1 - \left( \frac{\rho^2 + R^2}{2 \rho R} - 
\frac{c^2 t^2 - z^2}{2 \rho R} \right)^2} = 
\sqrt{1 - \frac{\left( \rho^2 + R^2 - c^2 t^2 + z^2 \right)^2}{4 \rho^2 R^2} } = \\
= \sqrt{\frac{4 \rho^2 R^2}{4 \rho^2 R^2} - 
\frac{\left( \rho^2 + R^2 - c^2 t^2 + z^2 \right)^2}{4 \rho^2 R^2} } =
\sqrt{\frac{4 \rho^2 R^2 - \left( \rho^2 + R^2 - c^2 t^2 + z^2 \right)^2}
{4 \rho^2 R^2}} = \\
= \frac{1}{2 \rho R} \sqrt{4 \rho^2 R^2 - \left( \rho^2 + R^2 \right)^2 +
2 \left( \rho^2 + R^2 \right) \left( c^2 t^2 - z^2 \right) - 
\left( c^2 t^2 - z^2 \right)^2} = \\
= \frac{1}{2 \rho R} \sqrt{- \left( \rho^2 - R^2 \right)^2 +
2 \left( \rho^2 + R^2 \right) \left( c^2 t^2 - z^2 \right) - 
\left( c^2 t^2 - z^2 \right)^2} = \\
= \frac{c^2 t^2 - z^2}{2 \rho R} \sqrt{2 \frac{\rho^2 + R^2 }{c^2 t^2 - z^2} - 
\left( \frac{\rho^2 - R^2 }{c^2 t^2 - z^2} \right)^2 - 1}
\end{aligned} \end{equation*} }
%
\textcolor{lightgray}{ \begin{equation*} \begin{aligned}
\tan \frac{\psi}{2} = \pm \sqrt{ \frac{1 - \cos \psi}{1 + \cos \psi} } = 
\sqrt{ \frac{1- \frac{\rho^2 + R^2}{2 \rho R} + \frac{c^2 t^2 - z^2}{2 \rho R}}
{1 + \frac{\rho^2 + R^2}{2 \rho R} - \frac{c^2 t^2 - z^2}{2 \rho R}} } =
\sqrt{ \frac{c^2t^2 - z^2 - \left( \rho - R \right)^2}
{\left( \rho + R \right)^2 - \left( c^2t^2 - z^2 \right)} }
\end{aligned} \end{equation*} }

\section{Інтеграл 2}
	% Властивості функциії Бесселя
%\chapter{Свойства функции Ломмеля}
\label{ch:lommel}
	% Властивості функциії Ломмеля

\end{document}
