\chapter{Задача синтизу протоколу передачі інформації}
\label{ch:neuron}

%%%%%%%%%%%%%%%%%%%%%%%%%%%%%%%%%%%%%%%%%%%%%%%%%%%%%%%%%%%%%%%%%%%%%%%%%%%%%%%
\section{Вплив нелінійності на імпульсний сигнал}

%%%%%%%%%%%%%%%%%%%%%%%%%%%%%%%%%%%%%%%%%%%%%%%%%%%%%%%%%%%%%%%%%%%%%%%%%%%%%%%
\section{Основні засоби підвищення кількості переданої інформації}

%%%%%%%%%%%%%%%%%%%%%%%%%%%%%%%%%%%%%%%%%%%%%%%%%%%%%%%%%%%%%%%%%%%%%%%%%%%%%%%
\section{Детекція сигналу за допомогою штучного інтелекту}

Процес передачі інформації в закодованому виді можна поділити на два етапи:
задача випромінювання та задача прийому. При використанні надширокосмугових 
протоколів канального рівня, станом на сьогодні, задача випромінювання вирішується 
ефективно. Задача прийому, в свою чергу, гальмується за рахунок 

%%%%%%%%%%%%%%%%%%%%%%%%%%%%%%%%%%%%%%%%%%%%%%%%%%%%%%%%%%%%%%%%%%%%%%%%%%%%%%%
\section{Задача визначення кута за формую імпульсу}

%%%%%%%%%%%%%%%%%%%%%%%%%%%%%%%%%%%%%%%%%%%%%%%%%%%%%%%%%%%%%%%%%%%%%%%%%%%%%%%
\section{Задача классифікації імпульсу в лінійному просторі}

%%%%%%%%%%%%%%%%%%%%%%%%%%%%%%%%%%%%%%%%%%%%%%%%%%%%%%%%%%%%%%%%%%%%%%%%%%%%%%%
\section{Задача классифікації імпульсу в нелінійному просторі}

%%%%%%%%%%%%%%%%%%%%%%%%%%%%%%%%%%%%%%%%%%%%%%%%%%%%%%%%%%%%%%%%%%%%%%%%%%%%%%%
\section{Задача клвсифікації в умовах детермінованого шуму}

%%%%%%%%%%%%%%%%%%%%%%%%%%%%%%%%%%%%%%%%%%%%%%%%%%%%%%%%%%%%%%%%%%%%%%%%%%%%%%%
\section{Нілінійність, як засіб підвищення кількості переданої інформації}

%\section{Методологія написання коду та вимоги до програмного продукту}
%\section{Програмний дизаін та архітектура коду}
%\section{Тестування програмного забеспечення та вілідація чисельних розв'язків}
%\section{Чисельні методи розрахунку імпульсних полів}
%\section{Імпульсна антенна, як копмонент мереживного обряднання компьютера}
